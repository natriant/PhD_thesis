In 2018, two prototype Crab Cavities (CCs) were installed in the SPS to be tested for the first time with proton beams. One of the operational issues that needed to be addressed concerned the expected emittance growth due to noise in their RF control system. A theoretical model had already been developed
and validated by tracking simulations~\cite{PhysRevSTAB.18.101001}. Based on those studies a dedicated experiment was performed to benchmark the models with experimental data and to confirm the analytical predictions. In particular, the idea was to inject various noise levels in the CC RF system and record
 the emittance evolution. In this chapter, the measurement results from the experiment are presented and discussed. 

 \section{Experimental Setup} % from IPAC paper

The measurements in the SPS were performed with four low intensity ($\sim 3 \cdot 10^{10}$\, ppb) bunches at 270\,GeV. Only one CC was used, providing a vertical kick to the beam. The linear chromaticity, $Q^\prime$, of the machine was corrected to small positive values ($\sim$\,1-2) in both transverse planes to minimise emittance growth from other sources~\cite{Antoniou:2649815}. 
The Landau octupoles were switched off; however, a residual non-linearity was present in the machine mainly due to multipole components in the dipole magnets~\cite{Carlà:2664976, Alekou:2640326}. Some of the relevant SPS parameters during the experiment are listed in Table~\ref{tab:SPS_MD_params}. 

\begin{table}[!hbt]
    \centering
    \caption{SPS parameters during the 2018 MD studies.}
    \begin{tabular}{lc}
        \toprule
        \textbf{Parameters} & \textbf{Values}\\
        \midrule
           $E_b$  & 270\,GeV   \\ %[3pt]
           $\frev$  & 43.375\,kHz  \\ %[3pt]
           $\nu_x, \nu_y$    & 26.13, 26.18  \\ %[3pt]
            $\nu_s$ & 0.0051   \\
            $\VRF$, $\fRF$ & 5\,MV, 200\,MHz \\
            $\beta_{x, \text{CC}}$, $\beta_{y,\text{CC}}$ &  30.31\,m, 73.82\,m \\
            $\VCC$, $\fCC$ & 1\,MV, 400\,MHz \\
       \bottomrule
    \end{tabular}
    \label{tab:SPS_MD_params}
 \end{table}