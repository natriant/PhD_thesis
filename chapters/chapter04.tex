In 2018, two prototype Crab Cavities (CCs) were installed in the SPS to be tested for the first time with proton beams. One of the operational issues that needed to be addressed concerned the expected emittance growth due to noise in their RF control system. A theoretical model had already been developed
and validated by tracking simulations~\cite{PhysRevSTAB.18.101001}. Based on those studies a dedicated experiment was performed to benchmark the models with experimental data and to confirm the analytical predictions. In particular, the idea was to inject various noise levels in the CC RF system and record
 the emittance evolution. In this chapter, the measurement results from the experiment are presented and discussed. This chapter is adapted from the the studies published in Ref.~\cite{Triantafyllou}

 \section{Experimental Setup} % from IPAC paper

Several experimental studies have been performed (2010-2017) to identify the optimal conditions for the emittance growth studies with CCs in the SPS~\cite{Calaga:1451286, Antoniou:2649815}. Based on these preparatory studies, the measurements in the SPS were performed with four low intensity ($\sim 3 \cdot 10^{10}$\, ppb) bunches at 270\,GeV. To minimise the emittance growth from other sources~\cite{Antoniou:2649815} the first order chromaticity, $Q^\prime$, of the machine was corrected to small positive values ($\sim$\,1-2) in both the horizontal and the vertical planes. During the measurements the Landau octupoles were switched off. It should be note, though, that a residual non-linearity was present in the machine mainly due to multipole components in the dipole magnets~\cite{Carlà:2664976, Alekou:2640326}. Only one CC was used, providing a vertical kick to the beam. Even though the emittance growth is a single bunch effect four bunches were used to reduce the statistical uncertainty of the measurements. The distance between the bunhces was 524 ns. An overview of the relevant SPS parameters during the experiment is given in Table~\ref{tab:SPS_MD_params}. 


\begin{table}[!hbt]
    \centering
    \caption{SPS parameters during the 2018 MD studies.}
    \begin{tabular}{lc}
        \toprule
        \textbf{Parameters} & \textbf{Values}\\
        \midrule
           $E_b$  & 270\,GeV   \\ %[3pt]
           $\frev$  & 43.375\,kHz  \\ %[3pt]
           $\nu_x, \nu_y$    & 26.13, 26.18  \\ %[3pt]
            $\nu_s$ & 0.0051   \\
            $\VRF$, $\fRF$ & 5\,MV, 200\,MHz \\
            $\beta_{x, \text{CC}}$, $\beta_{y,\text{CC}}$ &  30.31\,m, 73.82\,m \\
            $\VCC$, $\fCC$ & 1\,MV, 400\,MHz \\
       \bottomrule
    \end{tabular}
    \label{tab:SPS_MD_params}
 \end{table}

 \subsection{Injected RF noise} 