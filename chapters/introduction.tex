Particle accelerators were first developed in the early 20$^\mathrm{th}$ century as a tool for high-energy physics research. By accelerating the particles to high energy, they allow us to investigate the subatomic structure of the world and to study the properties of the elementary particles and the fundamental forces. Through the years significant technological progress has been achieved resulting in higher energies and greatly enhanced performance of the machines. Additionally, various types of accelerators have been developed (cyclotrons, linacs, synchrotrons, etc) using different types of particles (hadrons or leptons), and their use was also expanded in other fields such as medicine and industrial research. 
% Brief history of particle accelerators: https://cds.cern.ch/record/261062/files/p1_2.pdf
% Summary for accelerators: https://www.energy.gov/articles/how-particle-accelerators-work#:~:text=There%20are%20two%20primary%20roles,charged%20particles%20for%20medical%20treatment.
\gls{foobar} is another strange animal

\section{The CERN accelerator complex}

CERN (European Organisation of Nuclear Research), located on the Franco-Swiss border near Geneva, is at the forefront of the accelerator physics research as it operates an extensive network of accelerators, illustrated in Fig.~\ref{fig:cern_accelerator_complex}, including the well-known Large Hadron Collider (LHC)~\cite{Brüning:782076}.

The LHC is a circular machine, 27\, km long, built about 100\,m underground and is currently the largest and most powerful accelerator in the world. It accelerates and collides two counter-rotating beams of protons or ions (circulating in two different rings) at the four main experiments, which are located around the LHC ring, namely ATLAS, CMS, ALICE and LHCb~\cite{ATLAS:2008xda, CMS:2008xjf, ALICE:2008ngc, Alves:1129809}. The highlight of CERN and of the LHC operation up to now was the discovery of the Higgs boson in 2012 from ATLAS~\cite{ATLAS_Higgs} and CMS~\cite{CMS_Higgs}, from proton collisions at 3.5\,TeV (center-of-mass energy of 7\,TeV), which was a milestone for the validation of the standard model. % Importance of higgs discovery: https://home.cern/resources/faqs/cern-and-higgs-boson

The beams used by the LHC are produced and gradually accelerated by the injector chain, which is a sequence of smaller machines boosting the energy of the beam. In particular, through the chain of Linac4, Proton Synchrotron Booster (PSB), Proton Synchrotron (PS) and the Super Proton Synchrotron (SPS) the beam is accelerated up to 450 GeV before injecting into the LHC. In the LHC they are accelerated up to the collision energy of 6.8\,TeV (center-of-mass energy of 13.6\,TeV). It should be noted, that LHC delivered collisions with center-of-mass energy of 7\,TeV during Run 1 (2010-2013) which was increased to 13\,TeV for the Run 2 (2015-2018), and reached 13.6\,TeV in Run 3 (2022-present). 
% actually from april 2012 till the end of run 1 it deleveled 8 TeV. 

\begin{figure}[!h] %https://cds.cern.ch/images/CERN-GRAPHICS-2022-001-1
    \centering         
    \includegraphics[width=1\textwidth]{images/introduction/cern_accelerator_complex.png}
        \caption{Schematic view of the CERN accelerator complex. The different colors correspond to the different machines. The year of commissioning and the type of particles used in each one of them are also indicated along with the circumference for the circular machines. The image is courtesy of CERN~\cite{Lopienska:2800984}.}
        \label{fig:cern_accelerator_complex}
 \end{figure}

 It is worth mentioning that not only protons but also lead ions are accelerated in the LHC, starting their journey from Linac3 and Low Energy Ion Ring (LEIR) and then in PS and SPS as proton beams.

 Finally, the accelerators in the injector chain not only prepare the beam for the LHC but also provide beams to various other facilities and experiments at lower energies. Examples are the Anti-proton Decelerator (AD) which studies antimatter, the Online Isotope Mass Separator (ISOLDE) which studies the properties of the atomic nuclei using radioactive beams, and the Advanced Proton Driven Plasma Wakefield Acceleration Experiment (AWAKE) which investigates particle acceleration by proton-driven plasma wakefields.

 \subsection{The CERN Super Proton Synchrotron}\label{subsec:cern_sps}
 % sps history: https://be-dep-op-sps.web.cern.ch/history
 The research described in this thesis was conducted at the Super Proton Synchrotron (SPS) and some additional information about this machine is provided here. The SPS (shown with light blue color in Fig.~\ref{fig:cern_accelerator_complex}) was first commissioned in 1967 and has a circumference of 6.9\,km. The SPS was originally built to provide beams for the fixed target experiments.  It also used to operate as a proton-antiproton collider ($\mathrm{Sp\bar{p}S}$) and later on as an injector for the Large Electron Positron collider (LEP).% while it also provided beams for fixed-target experiments (e.g. in the North Area).
 Even though the SPS can accelerate various particle types (protons, antiprotons, electrons, and heavy ions) the following information will concern its operation with proton beams which is the topic of the research presented in this thesis.

 Currently, the SPS is the second biggest accelerator at CERN and it can accelerate protons up to 450\,GeV. Due to its past use as a collider, it can also operate as a storage ring. This operational mode is called "coast" and was used for the majority of the experimental studies presented in this thesis. During coast, the bunches circulate in the machine for long periods at constant energy. The highest energy at which SPS can operate in coast is 270\,GeV due to limited cooling of the magnets to transfer away the heat when operating at high energy and consequently at large currents for long periods.

 %After that they are injected in the LHC. Hoewever, as it used to be a collider it can operate as a storage ring, it is called coast (des ti exeis grapsei.) and will be used for the studies in this thesis.
 

\section{The High-Luminosity LHC project and Crab Cavities}
%LHC is the flagship of the CERN accelerator complex and is the at the cutting edge of the accelerator and high energy physics research with almost ten thousand scientists working for and with it. In order to extend its discovery potential LHC will undergo a major upgrade in the coming years. 
% Hl-lhc book: p.2 hl-lhc is the top priority of the european strategy of particle physics. 
The High-Luminosity LHC project (HL-LHC)~\cite{HL_LHC_yellow_report, Brning2015} is the upgrade of the LHC machine, which will extend its potential for discoveries. In particular, it aims to increase the instantaneous luminosity by a factor of 5 beyond the current operational values and the integrated luminosity by a factor of 10. 

The luminosity, along with the energy, is a key parameter defining the performance of a collider as it is a measure of the collision rate. The instantaneous luminosity is obtained as~\cite{luminosity}:

\begin{equation}\label{eq:luminosity_inst}
    L = \frac{n_b f_\mathrm{rev}N_1 N_2}{4 \pi \sigma_x \sigma_y} \frac{1}{\sqrt{1+\left ( \frac{\sigma_z}{\sigma_\mathrm{xing}} \frac{\theta_c}{2} \right )^2}},
\end{equation}

where $f_{\mathrm{rev}}$ is the revolution frequency (the number of times per second a particle performs a turn in the accelerator), $n_b$ is the number of colliding bunch pairs, $N_{1,2}$ is the number of particles per bunch, $\sigma_{x,y}$ is the transverse beam size at the interaction point, $\sigma_z$ the rms bunch length of the colliding bunches, $\sigma_{\mathrm{xing}}$ the transverse beam size in the crossing plane and $\theta_c$ is the full crossing angle between the colliding beams. % Andy's comment: use "number of particles per bunch" instead of "bunch intensity" which is slightly ambiguous.
 The crossing angle, is often introduced between the bunches in a collider to reduce parasitic collisions and get rid of the remnants after the collision. For reference, in the LHC, the crossing angle is on the order of magnitude $10^{-4}$ radians. % hundreds of micro radians, see sofia's thesis Table 4.4 for example. https://cds.cern.ch/record/2743602/files/CERN-THESIS-2020-169.pdf

The integrated luminosity is the one that ultimately defines the performance of the machine as it provides the total number of recorderd events. It depends both on the instantaneous luminosity and on the machine availability. The integrated luminosity, is expressed as~\cite{HL_LHC_yellow_report}:
\begin{equation}\label{eq:integrated_luminosity}
    L_\mathrm{int} \equiv \int_{\Delta t} L dt,
\end{equation}

where $L$ is the instantaneous luminosity as defined in Eq.~\eqref{eq:luminosity_inst}.
% for more details and specific values check sofias thesis.
% and: https://accelconf.web.cern.ch/ipac2015/papers/thxb2.pdf

HL-LHC aims to achive instantaneous luminosity of $L \sim 5 \cdot 10^{34} \ \mathrm{cm^{-2} s^{-1}}$ and an increase on the integrated luminosity from 300 $\mathrm{fb^{-1}}$ to 3000 $\mathrm{fb^{-1}}$ over its lifetime of 10-12 years and considering 160 days of operation per year~\cite{Brunning_Rossi}. % 160 days. p.. 80 in the reference
% The integrated luminosity for both lhc and hl-lhc is over 10-12 years.

% 160 days of physics events per year:https://cds.cern.ch/record/2031207/files/198-201_Fessia.pdf

% Source v2: https://accelconf.web.cern.ch/ipac2019/papers/mopgw095.pdf

%\normalsize{\textbf{Crab Cavities}}\\
\subsection{Crab cavities}\label{subsec:CC_intro}
To achieve its luminosity goals, the HL-LHC will employ numerous innovative technologies. The Crab Cavity technology (will be denoted as $\CC$ in this thesis)~\cite{Calaga:2673544} is one of the key componenets of the project as it will be employed to restore the luminosity reduction caused by the crossing angle, $\theta_c$ (see Eq.~\eqref{eq:luminosity_inst}).

A crab cavity is an RF cavity which provides a transverse, sinusoidal like, kick to the particles depending on their longitudinal position within the bunch. A graphical visualisation of the kick is shown in Fig.~\ref{fig:cc_simple_kick}. It can be seen that the head (leading part) and the tail (trailing part) of the bunch receive opposite deflection while the particles at the center remain unaffected.

\begin{figure}[!h] % at the directory of ipac22
    \centering         
    \includegraphics[width=0.5\textwidth]{images/introduction/sin_CC_kick_LHC_beams.drawio.png}
        \caption{Visualisation of the CC kick (green line) on the bunch particles (blue dots). The bunch here appears much smaller than the CC wavelength which means that only the linear part of the kick affects the bunch. This will be the case for the HL-LHC scenario.}
        \label{fig:cc_simple_kick}
 \end{figure}

The $\CC$s will be installed in the two main interaction points of LHC, ATLAS and CMS. According to the plan, $\CC$s will be installed on each ring and on each side of the interaction points (eight in total). This is displayed in Fig.~\ref{fig:LHC_layout_CCs} with the red (ATLAS) and orange (CMS) markers. The reason why $\CC$s are needed on each side of the IP is discussed in the following paragraphs (local vs global scheme).

\begin{figure}[!h] % made at diagrams.net saved at google docs
    \centering         
    \includegraphics[width=0.8\textwidth]{images/introduction/LHC_layout_CCs.png}
        \caption{Layout of the LHC and the SPS. The CC location for the HL-LHC configuration is marked. Two CCs (one per ring) will be installed on each side of ATLAS (red) and CMS (orange). Two protoype CCs were also installed in the SPS (magenta) in 2018, to be tested before their installation in LHC. The layout can be found in~\cite{LHC_SPS_layout}.}% and was modified inspired by Ref.~\cite{LHC_SPS_layout_v2}.}
        \label{fig:LHC_layout_CCs}
 \end{figure}

In this configuration, the bunches receive the transverse deflection from the first pair of $\CC$s just before reaching the interaction point. This results in a rotation of the bunch, which mitigates the crossing angle and restores the head-on collisions. The deflection is cancelled once the bunches reach the second pair of $\CC$s which are symmetrically placed at the opposite side of the interaction point. The collision of the bunches in the presence of the $\CC$s is illustrated in Fig.~\ref{fig:crossing_with_and_without_CCs}~\cite{Verdú-Andrés:2263119}.
% 90 deg between CCs and IP. https://cds.cern.ch/record/2263119/files/10.1016_j.nuclphysbps.2015.09.025.pdf

\begin{figure}[!ht]
    \centering
    \begin{subfigure}[t]{0.45\textwidth}
        \centering
        \includegraphics[width=1\textwidth]{images/introduction/no_crab_crossing.png}
        \caption{Crossing without CCs}
        %\label{fig:add_label_here}
    \end{subfigure}
    \hfill
    \begin{subfigure}[t]{0.45\textwidth}
        \centering
        \includegraphics[width=1\textwidth]{images/introduction/crab_crossing.png}
        \caption{Crossing with CCs}
        %\label{fig:add_label_here}
    \end{subfigure}
    \hfill
     \caption{Collision with and without the use of CCs~\cite{Verdú-Andrés:2263119}. CCs restore the overlap between the bunches recovering the luminosity reduction caused by the crossing angle, $\theta_c$. The blue and red colors indicate two bunches in the different rings.} 
     \label{fig:crossing_with_and_without_CCs}
 \end{figure}


% Local vs global CC scheme: https://slideplayer.com/slide/10896302/ 
The above scheme, with $\CC$s before and after the interaction point, is called the local crabbing scheme. An alternative scheme, named the global crabbing scheme, was also under discussion in the first stages of the project. In such as scheme, the closed orbit distortion that is caused by the fact that the head and the tail of the bunch are kicked in opposite directions propagates around the ring~\cite{Brning2015}.%, resulting in transverse bunch oscillations~\cite{Brning2015}. T
 This scheme is cost-efficient compared to the local scheme as it only requires two CCs. However, it introduces significant constraints on the betatron phase advance between the interaction points and the CCs. The constraints are enhanced by the fact that the bunch crossing in ATLAS takes place in the vertical plane while in CMS in the horizontal. To this end, the local $\CC$ scheme was chosen for the HL-LHC project. % uncrabbing.
%HL-LHC book p.138: The transverse kick introduced by this cavity, different  for  the  head  and  the  tail  of  each  bunch,  is  equivalent  to  a  closed  orbit  distortion, i.e. head and tail would follow their individual closed orbit around the ring, their tilt wobbling around the unperturbed closed orbit of the bunch center.

In order to accommodate the crossing in both transverse planes two $\CC$ designs have been developed: the Double-Quarter Wave (DQW) and the RF dipole (RFD), which provide vertical and horizontal deflection respectively. Information on their design can be found in~\cite{Zanoni:2288282, DeSilva:2288607, Xiao:1992565, Verdú-Andrés:2113440}.

% short summary about KEKB in sylivias paper: https://cds.cern.ch/record/2263119/files/10.1016_j.nuclphysbps.2015.09.025.pdf
Crab cavities have already been successfully used in the KEKB collider~\cite{Toge:475260} in Japan, during 2007-2010, with lepton beams ($e^{+} - e^{-}$)~\cite{CC_KEKB_4440798, Funakoshi:1955812, oide:pac07-mozaki01}. However, there are significant differences in the beam dynamics between leptons and hadrons (HL-LHC case). One of the most crucial points is that noise induced emittance growth is not an issue of concern for lepton beams as they experience emittance damping due to synchrotron radiation. For proton beams, the synchrotron radiation damping is much weaker so that the beam degradation due to emittance growth eventually can result in loss of luminosity. % Andy's comment: "emittance damping" would be better than "beam-size damping". 

%the impact of errors (e.g. RF noise) which leads to beam degradation~\cite{Calaga:2773279, Alekou:2696109}. This is not an issue of concern for lepton beams as they nevertheless experience emittance damping due to synchrotron radiation.

% other crabbing errors: phase and amplitude rf noise, wakefields (Calaga)
% other difference between letptons and protons (protons: much longer bunches)

% Potential questions to answer for the CC test.
%https://indico.cern.ch/event/800428/attachments/1804664/2945632/CrabCavity_BE_Seminar.pdf
As the $\CC$s have never been used with protons before, two prototype superconducting $\CC$s were installed in the SPS (Fig.~\ref{fig:LHC_layout_CCs}, magenta markers) to test the technical systems, to validate their operation with proton beams and to identify and address potential issues before their installation in LHC. The SPS  provides an ideal test bed for these studies as it allows testing under conditions that are closer to those in HL-LHC than any other machine. In particular, the SPS operates with proton beams, can run in storage-ring mode, and in terms of the energy reach is second only to LHC. The two $\CC$s that were installed in SPS~\cite{Zanoni:2017} were both fabricated at CERN and of the DQW type (like the ones that will be used in ATLAS interaction point in HL-LHC).


\section{Motivation, objectives and thesis outline}\label{sec:motivation_outline}

As mentioned above, one of the main concerns regarding the $\CC$ operation with protons is the emittance growth due to noise in their RF system leading to luminosity loss. For the HL-LHC, the target values regarding the luminosity loss and emittance growth are very tight. In particular the maximum allowed luminosity loss due to $\CC$ RF noise induced emittance growth is targeted at just 1$\%$ during a physics fill, which corresponds to an $\CC$ RF noise induced emittance growth of 2 $\mathrm{\%/h}$~\cite{MedinaMedrano:2301928, CC_lumi_limits_philippe, CC_lumi_limits_ilias}. To this end, a good understanding and characterization of the emittance growth mechanism is crucial for the HL-LHC project.

For reference, a physics fill is the time period during which the beams are successfully injected in the LHC at the desired conditions, they are accelerated at the desired energy and they are kept in the machine for consecutive collisions. After some hours, due to beam degradation the beams are dumped and a new fill is prepared. A fill in the HL-LHC will last a couple of hours. 

The main objective of this thesis is to understand, characterise and evaluate the mechanism of $\CC$ RF noise-induced emittance growth including numerical and experimental studies. The studies presented in this thesis were conducted for the SPS machine since it has proton crabbing operational experience and allows direct comparison of predictions from models and experimental data. It should be emphasised that the CC tests in SPS constitute the first experimental beam dynamics studies with $\CC$s and proton beams. The results and the understanding obtained from this research are essential for the HL-LHC, in order to predict the long-term emittance and to define limits on the acceptable noise levels for the $\CC$s.
%and for the design of the crab cavity HL-LHC Low-Level RF system.
% the outcome is that the dedicated damper is necessary. possibility to integrate at the ADT.

This thesis reports research that was carried out between 2018 and 2022, based at CERN and is structured as follows:

Chapter~\ref{Ch:theory} presents the basics of accelerator beam dynamics focusing on the concepts that are relevant for understanding the studies presented in this thesis. In particular, definitions are given for single-particle beam dynamics, addressing both transverse and longitudinal motion. Furthermore, the collective effects are introduced focusing on the effect of wakefields. A brief discussion on optics models for accelerators is also provided. Finally, the two simulation codes used in this thesis for macroparticle tracking, PYHEADTAIL, and Sixtracklib, are described.
% see sondre's thesis.

The available theoretical model for predicting the emittance growth driven by Crab Cavity RF noise, developed by T.~Mastoridis and P.~Baudrenghien~\cite{PhysRevSTAB.18.101001}, is described in Chapter~\ref{Ch:CC_noise_theory}. This model will be referred to as Mastoridis--Baudrenghien model or theory throughout the thesis. The modeling of the noise effects in the simulations is also discussed. Last, a short reference to the experiment with crab cavities at KEKB in Japan is also made. 

Chapter~\ref{Ch:2018_analyisis} is devoted to the results of the first experimental studies of the emittance growth from $\CC$ RF noise in the SPS. The experimental configuration and procedure are reported and the artificial noise injected in the $\CC$ RF system for the measurements is discussed in detail. Subsequently, the emittance growth measurements are presented along with the measured bunch length and intensity evolution. Last the measured emittance growth rates are compared with the predictions from the Mastoridis--Baudrenghien theory. It was found that the measured growth rates were systematically a factor of 4 on average lower than the predictions. 

Various possible factors were investigated as a possible explanations for this discrepancy. These studies are described in Chapter~\ref{Ch:investigating_discrepancy}. Initially, parametric studies based on the theoretical model studied the sensitivity of the $\CC$ RF noise-induced emittance growth to possible uncertainties in the CC voltage amplitude and bunch length. The theory was also benchmarked with different simulation codes: PyHEADTAIL and Sixtracklib. The sensitivity of the emittance evolution on the non-linearities of the SPS machine (which is not included in the Mastoridis--Baudrenghien theory) was also tested. Finally, studies were performed to exclude the possibility that the discrepancy is not a result of the z-dependent orbit shift induced by the $\CC$ kick or the actual noise spectra applied on the $\CC$s. However, none of these factors could explain the discrepancy.

Finally, simulations including the SPS transverse impedance model (not included in the Mastoridis--Baudrenghien theory) showed a significant impact on the emittance growth. Chapter~\ref{Ch:suppression_impedance} discusses the investigation and characterisation of the phenomenon of the emittance growth suppression from the beam coupling impedance as observed in simulations with PyHEADTAIL. It was shown that the suppression is related to the dipole motion which is excited by the $\CC$ RF phase noise.

Chapter~\ref{Ch:experimental_CC_2022}, presents the results from the experimental studies that took place in the SPS in 2022. The objective of this experimental campaign was to validate the mechanism of the suppression of the $\CC$ RF noise-induced emittance growth from the beam coupling impedance as observed in simulations (described in Chapter~\ref{Ch:suppression_impedance}). This would also confirm that this impedance-induced effect is the reason for the discrepancy observed in the 2018 experiment between the measurements and the Mastoridis--Baudrenghien theoretical predictions. Despite the challenging conditions of the studies, the experiments of 2022 successfully confirmed the emittance growth suppression mechanism.%, despite the very challenging conditions of the studies.

%presents the results from the second round of emittance growth measurements with $\CC$s in SPS that took place in 2022. The objective of these experimental studies was to validate the mechanism of the suppression of the $\CC$ RF noise-induced emittance growth from the beam coupling impedance as observed in simulations (described in Chapter 7). This would also confirm that this impedance-induced effect is the reason for the discrepancy observed in the 2018 experiment between the measurements and the theoretical predictions. The experiments of 2022 successfully confirmed the suppression mechanism, despite the very challenging conditions of the studies.

Last, Chapter~\ref{Ch:conclusions} summarises the conclusions of the thesis. The project is viewed from a broad persepective highlighting its importance for the HL-LHC project. %Potential follow up studies are proposed.