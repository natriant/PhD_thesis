This is the introduction of my PhD thesis.
\newpage
1. General about standard model (Sofia, Schenk)
2. Role of CERN
3. Role of my thesis.

\section{The CERN accelerator complex}



 


\section{Crab Cavities for High-Luminosity LHC}


\section{Project objectives and thesis outline}

\section{General parameters of the studies}
% Maybe this should be mentioned at the end of project objectives and thesis outline.
% The following paragraph is taken from the APR Ch.1.3 
 The studies presented in this thesis were performed for the nominal SPS optics for the LHC filling which are called Q26 optics as the higher integer part of the tune in both planes is 26. These optics can be found in the official CERN repository~\cite{SPS_optics_repo} and will be referred to as the SPS model in this thesis. The values of the optics parameters in what follows correspond to the model values unless stated otherwise.



\textbf{Probably put to abstract}
In 2018, two prototype Crab Cavities ($\CC$s) were installed in the SPS to be tested for the first time with proton beams. A series of dedicated machine developemnt studies was carried out in order to validate their working principle and answer various beam dynamic questions. One of the operational issues that needed to be addressed concerned the expected emittance growth due to noise in their RF system, which is the main subject this thesis.  As mentioned in chapter~\ref{Ch:CC_noise_theory} a theoretical model had already been developed and validated by trackig simulations~\cite{PhysRevSTAB.18.101001}. 
As a part of the first experiemental campaign with $\CC$s in SPS a dedicated experiment was conducted to benchmark these models with experimental data and confirm the analytical predictions. The objective of this chapter is to provide an overview of the machine setup for the $\CC$ experiements and introduce the instruments and methods used for measuring the beam parameters of interest for the emittance growth studies.