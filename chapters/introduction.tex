This is the introduction of my PhD thesis.
\newpage
1. General about standard model (Sofia, Schenk)
2. Role of CERN
3. Role of my thesis.

\section{The CERN accelerator complex}



 


\section{Crab Cavities for High-Luminosity LHC}


\section{Project objectives and thesis outline}

\section{General parameters of the studies}
% Maybe this should be mentioned at the end of project objectives and thesis outline.
% The following paragraphs are taken from the APR Ch.1.3 
\subsection{SPS optics}
 The studies presented in this thesis were performed for the nominal SPS optics for the LHC filling which are called Q26 optics as the higher integer part of the tune in both planes is 26. 

 \normalsize{\textbf{SPS nominal model}\\
 The model for the Q26 optics can be found in the official CERN repository~\cite{SPS_optics_repo} and will be referred to as the nominal SPS model in this thesis. The values of the optics parameters in what follows correspond to the model values unless stated otherwise.

 \normalsize{\textbf{SPS non-linear model}\\
The nominal SPS model includes only the nonlinear fields produced by the chromatic sextupoles. However, one of the most important sources of non-linearities in SPS are the odd multipole components of its main dipole magnets. For some of the studies presented in this thesis their impact on the beam dynamics must be studied and therefore they should be included in the nominal model. 

The multipole error of the SPS main dipoles are unfortunately not available from magnetic measurements. On this ground a non-linear optics model of the SPS has been established with beam-based measurements of the chromatic detuning over a range of momentum deviation~\cite{Carlà:2664976, Alekou:2640326}.  The optics model was obtained by assigning systematic multipole components to the main lattice magnets, in the nominal model of SPS, in order to reproduce the tune variation with themomentum deviation as it was measured in the real machine. The calculations were performed with MAD-X [].

The values of the multipole components up to seventh order obtained from this methodare given in Table  3.1 where, (bA3,bB3) (bA5,bB5) and (bA7,bB7) stand for the sextupolar, decapolarand decatetrapolar mutipoles respectively. Note that different values have been obtained foreach of the two different kinds of SPS main dipoles (MBA and MBB) which are marked withthe indices A and B respectively.

\begin{table}[!hbt]
    \begin{centering}
   \caption{Parameters for computing the CC voltage from the example HT monitor measurements discussed in this chapter.}
	\begin{tabu} to \textwidth {X[c,m] X[0.01c,m] X[0.01c,m] X[0.01c,m]}
		&&& \\[-6mm]
		\toprule \toprule
		\multicolumn{2}{l}{\textbf{Multipole}} &
		\multicolumn{2}{c}{\textbf{Value}} \\
		\bottomrule
      \multicolumn{2}{l}{test 1}& \multicolumn{2}{c}{49.19\,m} \\
      \multicolumn{2}{l}{test 2 & \multicolumn{2}{c}{15.68 $\times$ 2$\mathrm{\pi}$} \\
      \multicolumn{2}{l}{Beta function at the $\CC$1, $\beta_{y, CC1}$}& \multicolumn{2}{c}{76.07\,m} \\
      \multicolumn{2}{l}{Phase advance between the start$^{\ast}$ of the lattice and $\CC$1, $\psi_{y, CC1}$} & \multicolumn{2}{c}{23.9 $\times$ 2$\mathrm{\pi}$} \\
      \multicolumn{2}{l}{Vertical betatron tune, $\Qy$} & \multicolumn{2}{c}{26.13} \\
      \multicolumn{2}{l}{Beam energy, $\symE$} & \multicolumn{2}{c}{26\,GeV} \\
      \bottomrule
	\end{tabu}
   \label{tab:SPS_HT_CC}
   \end{centering}
\end{table}


\textbf{Probably put to abstract}
In 2018, two prototype Crab Cavities ($\CC$s) were installed in the SPS to be tested for the first time with proton beams. A series of dedicated machine developemnt studies was carried out in order to validate their working principle and answer various beam dynamic questions. One of the operational issues that needed to be addressed concerned the expected emittance growth due to noise in their RF system, which is the main subject this thesis.  As mentioned in chapter~\ref{Ch:CC_noise_theory} a theoretical model had already been developed and validated by trackig simulations~\cite{PhysRevSTAB.18.101001}. 
As a part of the first experiemental campaign with $\CC$s in SPS a dedicated experiment was conducted to benchmark these models with experimental data and confirm the analytical predictions. The objective of this chapter is to provide an overview of the machine setup for the $\CC$ experiements and introduce the instruments and methods used for measuring the beam parameters of interest for the emittance growth studies.