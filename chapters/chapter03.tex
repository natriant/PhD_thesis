\section{Noise}\label{sec:noise_definition}
In the particle accelerators a major issue of concern is the presence of noise. Examples of noise sources are: ripples in the power converter ripples, flactuation of the magnetic fields, ground motion and various instruments in the accelerator stracture such as the transverse damper of the Crab Cavities.




From the various noise sources that are present in a particle accelerator dipole noise source and CC with mainly focusing on the secong.


In this thesis, the term "noise" refers to a sequence of random kicks (stochastic process) that affect the particles within a bunch by changing their transverse momentum each turn as follows: %The noise can be modeled as a single kick per particle per turn.  (if you use this sentence re-phrase it first).
\begin{equation}\label{eq:external_noise}
    u^\prime_1 =  u^\prime_0 + \Delta u^\prime,
\end{equation}
where $u=(x,y)$ denoting the horizontal and vertical plane and $\Delta u^\prime$ is the change of the momentum due to the noise kick. 

The term "noise" is also limited to external noise for the studies of this thesis. External noise means that it is independent of the beam itself.


Most noise sources produce noise that is constant across a bunch, being proportional to the constant. This type of noise will be referred to as rigid or dipolar noise. The shape of crab cavity amplitude noise, on the
other hand, is proportional to th
e, which excites all particles in the bunch equally




- Lebedev
- 
- KEKB, (references) why are they different (philippe email exchange)

2. Noise, reference to noise studies ... to kek. mention why they are different.\\


For a uniform noise spectrum across the betatron tune distribution, the emittance growth resulting from amplitude noise can be estimated from:
\begin{equation}\label{eq:dey_an}
    \frac{d\epsilon}{dt}  = \betaCC \left( \frac{e\VCC\frev}{2E_b}\right)^2 \!\! C_{\Delta A} (\sigma_{\phi}) \!\! \sum_{k=-\infty}^{+\infty} S_{\Delta A}[(k \pm \bar{\nu}_b \pm \bar{\nu}_s)\frev].
\end{equation}
For phase noise, the emittance growth can be estimated from:
\begin{equation}\label{eq:dey_pn}
    \frac{d\epsilon}{dt}  = \betaCC \left( \frac{e\VCC\frev}{2E_b}\right)^2 C_{\Delta \phi} (\sigma_{\phi}) \sum_{k=-\infty}^{+\infty} S_{\Delta \phi}[(k \pm \bar{\nu}_b)\frev].
\end{equation}
 In these formulas, $\betaCC$ is the beta function at the location of the CC, $\VCC$ the CC voltage, $\frev$ the revolution frequency of the beam, $E_b$ the beam energy, and $\bar{\nu}_b$ and $\bar{\nu}_s$ the mean of the betatron and synchrotron tune distribution. $S_{\Delta A}$ and $S_{\Delta \phi}$ are the power spectral densities (PSD)~\cite{b_papoulis1991probability} of the noise at all the betatron and synchrobetatron (for the amplitude noise case) sidebands. %In SI units, $S_{\Delta A}$ and $S_{\Delta \phi}$ are measured in Hz$^{-1}$ and rad$^2$Hz$^{-1}$, respectively. 
 $C_{\Delta A}$ and $C_{\Delta \phi}$ are correction terms to account for the bunch length:
\begin{align}
C_{\Delta A}(\sigma_{\phi}) = ~& e^{-\sigma_{\phi}^2}\sum_{l=0}^{+\infty} I_{2l+1}(\sigma_{\phi}^2),\\
C_{\Delta \phi}(\sigma_{\phi}) = ~& e^{-\sigma_{\phi}^2} \left[I_0(\sigma_{\phi}^2) + 2 \sum_{l=1}^{+\infty} I_{2l}(\sigma_{\phi}^2) \right],
\end{align}
with $\sigma_{\phi}$ the rms bunch length (in radians) with respect to the CC frequency $\fCC$, and $I_n(x)$ the modified Bessel function of the first kind. 


You need to described how they are modeled in pyheadtail.
% https://docs.google.com/presentation/d/1Jv0Es99utlZSSg25_9oplI53A5MdPDfJSjsSneRXaV8/edit#slide=id.gbb99f3cf0a_0_65
$\symc$
$\symE$


\section{Modelling of phase and amplitude noise in simulations}

As follows from the analysis in Ref.~\cite{PhysRevSTAB.18.101001} the phase and amplitude noise kicks can be treated separately, and be modeled as the following kicks on the momentum: % at the location of the CC; s=s_CC. Eq.7 in ref~\cite{PhysRevSTAB.18.101001} but for not normalised co-oridnates.
% Presetnation: https://docs.google.com/presentation/d/1Jv0Es99utlZSSg25_9oplI53A5MdPDfJSjsSneRXaV8/edit#slide=id.gbb99f3cf0a_0_65
\begin{equation}\label{eq:amplitude_noise_kick}
  \textbf{\textrm{Amplitude noise:}} \  u^\prime_1 =  u^\prime_0 + A \sin{\left (  \frac{2 \pi \fCC}{c \beta_0}z   \right )},
\end{equation}
\begin{equation}\label{eq:phase_noise_kick}
    \textbf{\textrm{Phase noise:}} \ u^\prime_1 =  u^\prime_0 + A \cos{\left (  \frac{2 \pi \fCC}{c \beta_0}z   \right )},
\end{equation}

where $u^\prime$, with $u=(x,y)$, is the normalised transverse momentum and $z$ the longitudinal co-ordinate as defined in Eq.~\eqref{eq:particle_coordinates}, $\fCC$ is the $\CC$ frequency, $c$ is the speed of light and $\beta_0$ the relativistic $\beta$. Finally, $A=\VCC /(E_b \cdot \Delta A )$ or $A=\VCC /(E_b \cdot \Delta \phi )$ the scaling factor for amplitude or phase noise respectively.