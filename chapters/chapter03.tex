Write introduction here and explain what you discuss in each section.
You need the frist section to introduce the dipole noise as you will apply it in the simulations.
\section{Noise}\label{sec:noise_definition}
%External noise means that it is independent of the beam itself.
In particle accelerators, a major issue of concern is the presence of external noise as it leads to transverse emittance growth, particle losses and limits the beam lifetime. Examples of noise sources are ripples in the power converter ripples which lead to fluctuations of the magnetic fields, ground motion, and various instruments in the accelerator structure such as the transverse damper and the Crab Cavities. % sofias's thesies introduction of chatper 5.
%Ripples in the power supply voltage are converted into current ripples, depending on the magnet's impedance, which eventually leads to magnetic field perturbations through the magnet's transfer function (vacuum chamber, beam screen). sofia's thesis p.95

From the various noise sources that are present in an accelerator, this thesis focuses on the dipolar noise and mainly on the CC noise. Dipolar noise is the one produced by the majority of the noise sources and is constant along the bunch, i.e. all the particles are affected the same way. % this noise is also referred to as rigid noise 
On the other hand, the way the CC noise affects the particles depends on their longitudinal position within the bunch (more details in the following paragraphs).

%Therefore, in the following the term "noise" refers to a sequence of random kicks (stochastic process) that affect the particles within a bunch by changing their transverse momentum each turn as follows.
Past studies~\cite{Lebedev:248620, Lebedev:248622, PhysRevSTAB.18.101001} have dealt with this type of noise in the context of the induced emittance growth theoretically and in simulations. It has been shown, that the noise can be modelled as a sequence of random kicks (stochastic process) that affect the particles within a bunch by changing their transverse momentum each turn as follows:
\begin{equation}\label{eq:external_noise}
    u^\prime_1 =  u^\prime_0 + \Delta u^\prime,
\end{equation}
where $u=(x,y)$ denoting the horizontal and vertical plane and $\Delta u^\prime$ is the change of the momentum due to the noise in units of rad. In this thesis the term "noise" refers to the above mentioned stochastic process.
% This approach will be used in the following. In th
% if it is dipole noise it is oftern written: Delta u = thera

\section{Crab Cavity noise and emittance growth}\label{eq:CC_noise_intro}
As already mentioned in the Introduction (Section~\ref{sec:motivation_outline}) the presence of noise in the $\CC$ low-level RF system is an issue of major concern for the HL-LHC project as it results to transverse emittance growth and subsequently in loss of luminosity. To this end, in 2015, P.~Baudrenghien and T.~Mastoridis developed a theoretical model~\cite{PhysRevSTAB.18.101001} which predicts this transverse emittance growth induced by $\CC$ noise focusing on the HL-LHC scenario. In particular, the model assumes a hadron machine, zero synchrotron radation damping, long bunches (in the order of cm), and white RF noise.




\section{Theoretical model}\label{eq:CC_emit_growth_formulas}

%\section{Theoretical model for Crab Cavity noise induced-emittance growth}\label{eq:CC_emit_growth_formulas}

- KEKB, (references) why are they different (philippe email exchange) they are reffered here as there are the only other studies that exist for cc noise even though they are not of used. due to different particles and different nature of noise


2. Noise, reference to noise studies ... to kek. mention why they are different.\\


For a uniform noise spectrum across the betatron tune distribution, the emittance growth resulting from amplitude noise can be estimated from:
\begin{equation}\label{eq:dey_an}
    \frac{d\epsilon}{dt}  = \betaCC \left( \frac{e\VCC\frev}{2E_b}\right)^2 \!\! C_{\Delta A} (\sigma_{\phi}) \!\! \sum_{k=-\infty}^{+\infty} S_{\Delta A}[(k \pm \bar{\nu}_b \pm \bar{\nu}_s)\frev].
\end{equation}
For phase noise, the emittance growth can be estimated from:
\begin{equation}\label{eq:dey_pn}
    \frac{d\epsilon}{dt}  = \betaCC \left( \frac{e\VCC\frev}{2E_b}\right)^2 C_{\Delta \phi} (\sigma_{\phi}) \sum_{k=-\infty}^{+\infty} S_{\Delta \phi}[(k \pm \bar{\nu}_b)\frev].
\end{equation}
 In these formulas, $\betaCC$ is the beta function at the location of the CC, $\VCC$ the CC voltage, $\frev$ the revolution frequency of the beam, $E_b$ the beam energy, and $\bar{\nu}_b$ and $\bar{\nu}_s$ the mean of the betatron and synchrotron tune distribution. $S_{\Delta A}$ and $S_{\Delta \phi}$ are the power spectral densities (PSD)~\cite{b_papoulis1991probability} of the noise at all the betatron and synchrobetatron (for the amplitude noise case) sidebands. %In SI units, $S_{\Delta A}$ and $S_{\Delta \phi}$ are measured in Hz$^{-1}$ and rad$^2$Hz$^{-1}$, respectively. 
 $C_{\Delta A}$ and $C_{\Delta \phi}$ are correction terms to account for the bunch length:
\begin{align}
C_{\Delta A}(\sigma_{\phi}) = ~& e^{-\sigma_{\phi}^2}\sum_{l=0}^{+\infty} I_{2l+1}(\sigma_{\phi}^2),\\
C_{\Delta \phi}(\sigma_{\phi}) = ~& e^{-\sigma_{\phi}^2} \left[I_0(\sigma_{\phi}^2) + 2 \sum_{l=1}^{+\infty} I_{2l}(\sigma_{\phi}^2) \right],
\end{align}
with $\sigma_{\phi}$ the rms bunch length (in radians) with respect to the CC frequency $\fCC$, and $I_n(x)$ the modified Bessel function of the first kind. 


You need to described how they are modeled in pyheadtail.
% https://docs.google.com/presentation/d/1Jv0Es99utlZSSg25_9oplI53A5MdPDfJSjsSneRXaV8/edit#slide=id.gbb99f3cf0a_0_65
$\symc$
$\symE$


\section{Modelling of phase and amplitude noise in simulations}

As follows from the analysis in Ref.~\cite{PhysRevSTAB.18.101001} the phase and amplitude noise kicks can be treated separately, and be modeled as the following kicks on the momentum: % at the location of the CC; s=s_CC. Eq.7 in ref~\cite{PhysRevSTAB.18.101001} but for not normalised co-oridnates.
% Presetnation: https://docs.google.com/presentation/d/1Jv0Es99utlZSSg25_9oplI53A5MdPDfJSjsSneRXaV8/edit#slide=id.gbb99f3cf0a_0_65
\begin{equation}\label{eq:amplitude_noise_kick}
  \textbf{\textrm{Amplitude noise:}} \  u^\prime_1 =  u^\prime_0 + A \sin{\left (  \frac{2 \pi \fCC}{c \beta_0}z   \right )},
\end{equation}
\begin{equation}\label{eq:phase_noise_kick}
    \textbf{\textrm{Phase noise:}} \ u^\prime_1 =  u^\prime_0 + A \cos{\left (  \frac{2 \pi \fCC}{c \beta_0}z   \right )},
\end{equation}

where $u^\prime$, with $u=(x,y)$, is the normalised transverse momentum and $z$ the longitudinal co-ordinate as defined in Eq.~\eqref{eq:particle_coordinates}, $\fCC$ is the $\CC$ frequency, $c$ is the speed of light and $\beta_0$ the relativistic $\beta$. Finally, $A=\VCC /(E_b \cdot \Delta A )$ or $A=\VCC /(E_b \cdot \Delta \phi )$ the scaling factor for amplitude or phase noise respectively.