% This first paragraph is slightly modified from the phd description: /eos/user/n/natriant/Documents/education/cern/doctoral/DOCT_PhD v2.docx

The work presented in this thesis addressed the emittance growth driven by noise in the $\CC$ RF system, which is anticipated to limit the performance of the HL-LHC. The validity of the Mastoridis--Baudrenghien model, which predicts the $\CC$ RF noise-induced emittance growth was benchmarked against experimental data and its limitations were identified. Based on tracking simulations and experimental measurements it was shown that the beam transverse impedance can have a significant impact on the noise-induced emittance growth. 

The studies presented in this thesis were conducted at the CERN SPS, where two prototype $\CC$s were installed in 2018, to allow for tests with proton beams before their installation in the LHC.

The first beam dynamic studies with $\CC$s and proton beams took place in 2018 in the SPS and are presented in this thesis. By analysing the emittance growth measurements it was found that $\CC$ RF noise-induced emittance growth was a factor four on average lower than predicted from the Mastoridis--Baudrenghien model. Follow-up studies excluded the possibility that the observed discrepancy was a result of some error in the analysis of the experimental data.


%The HL-LHC project is the designed uprade of the LHC mahine aiming at about a five-fold increase of the yearly luminosity production with proton beams compared to the present LHC operation. The HL-LHC will employ $\CC$s in the two main experiments (ATLAS and CMS) to restore the luminosity reduction caused by the crossing angle. A particular challenge about the $\CC$s' operation is associated with the fact they are expected to induce undesired transverse emittance growth due to noise in their RF control system, and therefore loss of luminosity. Given the very tight HL-LHC target values for luminosity loss, a solid understanding of the emittance growth due to $\CC$ RF noise is essential. The scope of this thesis has been to understand, characterise, and evaluate the mechanism of the $\CC$ RF noise-induced emittance growth for proton beams through tracking simulations and experimental studies. 

%In 2015, T.~Mastoridis and P.~Baudrenghien developed an analytical model for predicting the emittance growth induced by noise in the $\CC$ RF system which is used for setting the specifications for the noise limits on $\CC$s in HL-LHC. The model was benchmarked with tracking simulations, however, in order to gain confidence in its predictions, benchmarking with experimental measurements was required. 

%Two prototype $\CC$s have been installed in the SPS in 2018, to allow for tests with proton beams prior to their installation in the LHC. Even though the SPS and the LHC are very different machines (discussed in the folllowing section), the objective of the SPS tests was to test the validity of the Mastoridis--Baudrenghien model and identify possibile limitations. During the first experimental campaign with $\CC$s in the SPS in 2018, which is analysed in the thesis, it was found that the measured $\CC$ RF noise-induced emittance was a factor four on average lower than predicted from the available theoretical model of Mastoridis--Baudrenghien~\cite{PhysRevSTAB.18.101001}.  Follow-up studies excluded the possibility that the observed discrepancy was a result of some error in the analysis of the experimental data.



%The studies presented in this thesis, were conducted in the SPS machine where two prototype $\CC$s were installed in 2018, to allow for tests with proton beams prior to their installation in the LHC.


%To extend the physics reach of the main experiments of the Large Hadron Collider (LHC) at CERN, the machine will undergo a major upgrade during the coming years. This upgrade, namely the High-Luminosity LHC (HL-LHC) project, aims at about a five-fold increase of the yearly luminosity production with proton beams compared to the present LHC operation. For these ambitious luminosity goals, the precise control and minimization of beam degradation (such as losses and emittance growth) is critical. A particular challenge is associated with the fact that HL-LHC will employ Crab Cavities to compensate the crossing angle of the colliding bunches, and hence to restore the head-on collisions, in the two main experiments (ATLAS and CMS). However, these Crab Cavities are expected to result in undesired transverse emittance growth due to noise in their RF control system, and therefore loss of luminosity. %  by residual noise on the beam introduced by the crab cavity RF control. 
%Given the very tight HL-LHC target values for luminosity loss and emittance growth from the Crab Cavities (see Introduction chapter) a solid understanding of the emittance growth due to Crab Cavity RF noise is essential. 


%In 2015, T.~Mastoridis and P.~Baudrenghien developed an analytical model for predicting the emittance growth induced by noise in the $\CC$ RF system which is used for setting the specificationsfor the noise limits on $\CC$s in HL-LHC. The model was benchmarked with tracking simulations, however, in order to gain confidence in its predictions, benchmarking with experimental measurements was required. During the first experimental campaign with $\CC$s in the SPS in 2018, it was found that the predictions from the Mastoridis--Baudrenghien model overestimated the measured emittance growth by a factor of four on average. Detailed follow-up studies excluded the possibility that the observed discrepancy was a result of some error in the analysis of the experimental data.

%The HL-LHC will employ $\CC$s in the two main experiments (ATLAS and CMS) to restore the luminosity reduction caused by the crossing angle. A particular challenge about the $\CC$s' operation is associated with the fact they are expected to induce undesired transverse emittance growth due to noise in their RF control system, and therefore loss of luminosity. The scope of this thesis has been to understand, characterise, and evaluate the mechanism of the $\CC$ RF noise-induced emittance growth for proton beams through tracking simulations and experimental studies. A new mechanism of emittance growth suppression resulting from the beam tranvserse impedance was identified.


%Two prototype crab cavities have been installed in the SPS machine and were tested with a proton beam in 2018, prior to the installation in the LHC. A particular challenge about their operation is associated with the fact that the Crab Cavities are expected to induce undesired transverse emittance growth due to noise in their RF control system. This thesis investigated the mechanism of the emittance growth due to Crab Cavity RF noise with tracking simulations and experimental measurements.



%To extend the physics reach of the main experiments of the Large Hadron Collider (LHC) at CERN, the machine will undergo a major upgrade during the coming years. This upgrade, namely the High-Luminosity LHC (HL-LHC) project, aims at about a five-fold increase of the yearly luminosity production with proton beams compared to the present LHC operation. For these ambitious luminosity goals, the precise control and minimization of beam degradation (such as losses and emittance growth) is critical. A particular challenge is associated with the fact that HL-LHC will employ Crab Cavities to compensate the crossing angle of the colliding bunches, and hence to restore the head-on collisions, in the two main experiments (ATLAS and CMS). However, these Crab Cavities are expected to result in undesired transverse emittance growth due to noise in their RF control system, and therefore loss of luminosity. %  by residual noise on the beam introduced by the crab cavity RF control. 
%Given the very tight HL-LHC target values for luminosity loss and emittance growth from the Crab Cavities (see Introduction chapter)
%a solid understanding of the emittance growth due to Crab Cavity RF noise is essential. 

%To study the transverse emittance growth induced in the presence of the Crab Cavities, two prototype Crab Cavities were installed in the SPS in 2018, for experimental tests. The first round of analysis of the experimental studies indicated that the measured emittance growth was a factor 2-3 lower than predicted from the available theoretical model of T.~Mastoridis and P.~Baudrenghien~\cite{PhysRevSTAB.18.101001}. This PhD thesis set out to to understand the results of experimental tests of Crab Cavities in the SPS. Understanding the results is crucial for gaining confidence in the predictions of the theoretical models and the specifications set for the noise limits on Crab Cavities in HL-LHC (which depend on the theoretical model).

%The first step was to revisit thoroughly the experimental data from the SPS tests of 2018 in order to identify any misinterpretations that could explain the observed discrepancy. These studies are presented in Chapter~\ref{Ch:2018_analyisis}. The outcome of this analysis was that the machine configuration and the measurement method could not explain the observed discrepancy. However, these studies provided a clear understanding of the operational aspects of the Crab Cavities in the SPS and of beam-based measurements of the Crab Cavity voltage which were not documented before. Furthermore, an automated procedure for the calibration of the Crab Cavities during operation was established and will be used in future experiments (and was already used in experiments in 2022). Moreover, the quality of the  Gaussian fit on the transverse beam profiles used to obtain the emittance values was verified and the impact of the measurement errors was included in the computation of the emittance growth rates. Open questions, like a possible relation of the transverse emittance growth to the longitudinal beam evolution or the beam intensity, were addressed. Finally, an interesting finding was that the four bunches used in the experiment had different emittance growth rates and only one of them was longitudinally stable. To this end, it was decided that future experimental studies would be performed with a single bunch. Focusing the analysis on the stable bunch the difference between its measured emittance growth rates and the ones predicted from the analytical model was then found to be even larger than had been estimated from the initial analysis at the time of the experiments. Up to a factor of five lower growth rates were observed in the measurements as compared to the theoretical prediction. 

%To study the transverse emittance growth induced in the presence of the Crab Cavities, two prototype Crab Cavities were installed in the SPS in 2018, for experimental tests. The analysis of the experimental data from 2018 indicated that the measured emittance growth was a factor 4 on average lower than predicted from the available theoretical model of Mastoridis--Baudrenghien~\cite{PhysRevSTAB.18.101001}. Detailed follow-up studies excluded the possibility that the observed discrepancy was a result of some error in the analysis of the experimental data.

%Understanding this observation is crucial for gaining confidence in the predictions of the model and the specifications set for the noise limits on Crab Cavities in HL-LHC.% (which depend on the theoretical model).


%The rest of the studies presented in this PhD thesis set out to understand the results of the 2018 experimental tests of Crab Cavities in the SPS. Understanding the results is crucial for gaining confidence in the predictions of the theoretical models and the specifications set for the noise limits on Crab Cavities in HL-LHC (which depend on the theoretical model).


%Chapter~\ref{Ch:2018_analyisis} also provides a thorough examination of the experimental data from the SPS tests of 2018 to identify any misinterpretations that could explain the observed discrepancy. In particular, the quality of the  Gaussian fit on the transverse beam profiles used to obtain the emittance values was verified and the impact of the measurement errors was included in the computation of the emittance growth rates. Open questions, like a possible relation of the transverse emittance growth to the longitudinal beam evolution or the beam intensity, were addressed. The outcome of this analysis was that the machine configuration and the measurement method could not explain the observed discrepancy. However, an interesting finding was that the four bunches used in the experiment had different emittance growth rates and only one of them was longitudinally stable. To this end, it was decided that the investigation of the discrepancy should focus on the experimental data from the first bunch and that future experimental studies would be performed with a single bunch.


% From ipac presentation and conclusions of Chapter 6
%In the following years, 2019-2020, a significant effort with theoretical and simulation studies was made to investigate possible explanations for the discrepancy. These studies are summarised in Chapter~\ref{Ch:investigating_discrepancy}. The sensitivity of the emittance growth rates to possible uncertainties on the measured amplitude of the Crab Cavity voltage and bunch length was tested. Furthermore, the Mastoridis--Baudrenghien theoretical model was benchmarked against different simulation codes, PyHEADTAIL and Sixtracklib, for conditions close to the experimental configuration of 2018. In PyHEADTAIL the emittance growth driven by Crab Cavity RF noise was simulated using a simple representation of the SPS consisting of one transfer map and an interaction point where the noise from the Crab Cavities was modeled as kicks on the angle co-ordinates of the particles. The same studies were repeated with Sixtracklib but the tracking was performed including the detailed optics with the non-linearities of the SPS lattice, the z-dependent orbit shift from the CC kick and the measured noise spectra from the experiments of 2018. It was found that the emittance growth driven by Crab Cavity noise simulated with PyHEADTAIL was in excellent agreement with the Sixtracklib results and with the predictions of the Mastoridis-Baudrenghien theoretical model. It was concluded that there is no sensitivity of the noise-induced emittance growth to the above mentioned parameters.  The use of PyHEADTAIL, for the rest of the studies presented in the thesis, was considered appropriate, since it appears that it includes the necessary beam dynamics for studying this phenomenon.


%It was found that the transverse beam impedance (not included in the Mastoridis--Baudrenghien model) has a significant impact on the transverse emittance growth driven by Crab Cavity RF noise. In particular, PyHEADTAIL simulations using the most updated SPS impedance model demonstrated that the emittance growth is suppressed once the detuning induced by the impedance moves the coherent tune outside of the incoherent tune spectrum. It was also shown that the mechanism of the emittance growth suppression is purely to the rigid or dipole bunch motion (head-tail mode 0) of the beam. For the experimental configuration of the SPS Crab Cavity tests in 2018 it was found that this mechanism results in an emittance growth suppression of about a factor four which is very close to the observed discrepancy . This suggested that the damping effects from the impedance might explain the observed discrepancy between measurements and theoretical predictions.


%n particular, PyHEADTAIL simulations using the most updated SPS impedance model demonstrated that the decoherence and thus the emittance growth is suppressed once the detuning induced by the impedance moves the coherent tune outside of the incoherent tune spectrum. % A similar effect of decoherence suppression was studied in the past in the context of beam-beam modes. 
%It was also shown that the decoherence suppression is related purely to the rigid or dipole bunch motion (head-tail mode 0) of the beam which is excited by the Crab Cavity RF phase noise kicks. For the experimental configuration of the SPS Crab Cavity tests in 2018 it was found that this mechanism results in an emittance growth suppression by a factor up to about 4 which is very close to the observed discrepancy. This suggested that the damping effects from the impedance might explain the discrepancy between the measured and theoretically estimated emittance growth rates. The PyHEADTAIL simulations with the SPS transverse impedance model also revealed that the suppression of the emittance growth as a result of the transverse impedance depends on the amplitude-dependent betatron tune spread. This dependence was tested experimentally in the SPS in 2022.


Tracking simulations with PyHEADTAIL revealed that the transverse beam impedance (not included in the Mastoridis--Baudrenghien model) affects the transverse emittance growth induced by $\CC$ RF noise and may therefore explain the experimental observations. In particular, PyHEADTAIL simulations including the accurate SPS impedance model demonstrated that the noise-induced emittance growth is suppressed by about a factor four for the 2018 experimental conditions. Detailed simulation studies were conducted to characterise this newly observed effect. It was identified that the emittance growth suppression is related purely to the rigid (or dipole) bunch motion (head-tail mode 0) of the beam, which is induced by the $\CC$ RF phase noise. It was also demonstrated that the decoherence and thus the emittance growth is suppressed once the detuning induced by the impedance moves the coherent tune outside of the incoherent tune spectrum. Finally, it was shown that this emittance growth suppression mechanism depends on the amplitude-dependent betatron tune spread and that for large enough tune spread values the emittance growth predicted by the Mastoridis--Baudrenghien model can be restored.


%In particular, simulations with the most updated SPS impedance model demonstrated that the decoherence and thus the emittance growth is suppressed once the detuning induced by the impedance moves the coherent tune outside of the incoherent tune spectrum. The simulations also revealed that the suppression of the emittance growth as a result of the transverse impedance depends on the amplitude-dependent tune shift. Furthermore, it was shown that the emittance growth suppression is related purely to the rigid or dipole bunch motion (head-tail mode 0) of the beam. In the presence of Crab Cavity RF phase noise (which is associated with head-tail mode 0) for the experimental configuration of the SPS Crab Cavity tests in 2018, it was found that the above-described mechanism results in an emittance growth suppression by a factor up to about 4. This is very close to the observed discrepancy and hence suggested that the impedance effects might explain the discrepancy between the measured and theoretically estimated emittance growth rates. %The PyHEADTAIL simulations with the SPS transverse impedance model also revealed that the suppression of the emittance growth as a result of the transverse impedance depends on the amplitude-dependent tune shift. This dependence was tested experimentally in the SPS in 2022.

%It was found that the transverse beam impedance (not included in the Mastoridis--Baudrenghien model nor the simulations so far) has a significant impact on the transverse emittance growth induced by Crab Cavity RF noise. In particular, PyHEADTAIL simulations using the the most updated SPS impedance model demonstrated that the decoherence and thus the emittance growth is suppressed once the detuning induced by the impedance moves the coherent tune outside of the incoherent tune spectrum. A similar effect of decoherence suppression was studied in the past in the context of beam-beam modes. Detailed investigations have shown that the decoherence suppression is related purely to the rigid or dipole bunch motion (head-tail mode 0) of the beam. In the case of Crab Cavity induced noise kicks, this dipolar excitation is provided by the phase noise kicks. For the experimental configuration of the SPS Crab Cavity tests in 2018 it was found that this mechanism results in an emittance growth suppression by a factor up to about 4 which is very close to the observed discrepancy. It is concluded, that this result suggests that the impedance effects might explain the discrepancy between the measured and theoretically estimated emittance growth rates. The PyHEADTAIL simulations with the SPS transverse impedance model also suggested that the suppression of the emittance growth as a result of the transverse impedance depends on the amplitude-dependent tune shift. This dependence was tested experimentally in the SPS in 2022.


An additional campaign took place in the SPS in 2022 and confirmed experimentally for the first time the suggested emittance growth suppression mechanism from impedance induced effects. In particular, the emittance growth driven primarily by $\CC$ RF phase noise was measured as a function of different values of amplitude-dependent betatron tune spread. The measurements were found to be in very good qualitative agreement with the expectations from PyHEADTAIL simulations including the SPS transverse impedance model for very similar machine and beam conditions. Some uncertainty on the quantitative agreement was observed, however, it is within the uncertainties expected from the experimental setup and the instruments used for the measurements. The emittance growth measured in the presence of large betatron tune spread was found to be very close to the predictions of the Mastoridis--Baudrenghien model gaining confidence in its validity.



%was measured as a function of different values of amplitude-dependent betatron tune spread. The experiment yielded results in very good qualitative agreement with the expectations from PyHEADTAIL simulations including the SPS transverse impedance model for very similar machine and beam conditions. Thus, the measurements of 2022 provided a clear experimental validation of the emittance growth suppression mechanism from the beam transverse impedance.



%The main strategy was to measure the emittance growth driven primarily by Crab Cavity RF phase noise as a function of different values of amplitude-dependent betatron tune spread, aiming to reproduce the dependence revealed from the simulations. The experiment yielded results in very good qualitative agreement with the expectations from the PyHEADTAIL simulations including the SPS transverse impedance model for very similar machine and beam conditions.  This result provided a clear experimental validation of the emittance growth suppression mechanism from the beam transverse impedance.  Some uncertainty on the quantitative agreement was observed. However, it is within the uncertainties expected from the experimental setup and the instruments used for the measurements. The emittance growth measured in the presence of large betatron tune spread (which is expected to mitigate the emittance growth suppression mechanism), was found to be very close to the predictions of the Mastoridis--Baudrenghien model gaining confidence in its validity.

%The emittance growth measured in the presence of large betatron tune spread, was found to be very close to the predictions of the Mastoridis--Baudrenghien model (as expected from the simulations), gaining confidence in its validity.

%The measured emittance growth driven by Crab Cavity RF phase noise demonstrated a clear dependence on the amplitude-dependent tune spread which was in very good qualitative agreement with the results from PyHEADTAIL simulations including the SPS transverse impedance model for very similar machine and beam conditions. This result provided a clear experimental validation of the emittance growth suppression mechanism from the beam transverse impedance. Some uncertainty on the quantitative agreement was observed. However, it is within the uncertainties expected from the experimental setup and the instruments used for the measurements, as well as from other effects, like space charge, that are not included in the simulations and might play a role in the overall behavior. The emittance growth measured in the presence of large octupole detuning, was found to be very close to the predictions of the Mastoridis--Baudrenghien model (as expected from the simulations), gaining confidence in its validity.


%Five different experiments were performed with the emittance growth driven by Crab Cavity RF phase noise, pure dipole noise, and Crab Cavity RF amplitude noise. The measurements from all of the experiments support the hypothesis that the impedance provides a damping mechanism. In particular, the measured emittance growth driven by Crab Cavity RF phase noise and a pure dipolar noise source demonstrated a clear dependence on the amplitude-dependent tune spread (introduced by the Landau octupoles in SPS) as expected from PyHEADTAIL simulations including the SPS transverse impedance model. These results confirm the suppression mechanism from the beam transverse impedance and offer an explanation for the experimental observations made in 2018. Furthermore, a direct comparison of the emittance growth rates was made between the results from the experiment with Crab Cavity RF phase noise and PyHEADTAIL simulations for very similar machine and beam conditions. The qualitative agreement demonstrated between experimental data and simulations was very good. Some uncertainty on the quantitative agreement was observed. However, this is within the uncertainties expected from the experimental setup and the instruments used for the measurements, as well as from other effects, like space charge, that are not included in the simulations and might play a role in the overall behavior. The emittance growth measured in the presence of large octupole detuning, was found to be very close to the predictions of Mastoridis--Baudrenghien model (as expected from the simulations), gaining confidence in its validity. Finally, the emittance growth driven by Crab Cavity amplitude noise, which is associated with head-tail mode $\pm$1 and hence is not subject to the emittance growth suppression mechanism, was indeed measured to be very close to the predictions of Mastoridis--Baudrenghien.


%Achieving a good understanding of the 2018 results is essential for developing confidence in the theoretical model and its predictions for the HL-LHC. 
%The results of the experimental measurements with Crab Cavity noise that took place in 2022 are presented in Chapter~\ref{Ch:experimental_CC_2022}. Despite the very limited available machine time, the results demonstrated a clear dependence of the measured emittance growth, driven primarily of Crab Cavity RF phase noise, on the octupole strength, which supports the hypothesis that the impedance provides a damping mechanism, and may offer an explanation for the observations made in 2018.

%impedance and amplitude detuning on emittance growth from Crab Cavity phase noise. The objective was to validate experimentally the suggested suppression mechanism of the emittance growth by the beam transverse impedance, by reproducing the dependence on amplitude-dependent tune spread (introduced by the Landau octupoles) predicted by PyHEADTAIL simulations. 
%Achieving a good understanding of the 2018 results is essential for developing confidence in the theoretical model and its predictions for the HL-LHC. 
%The results of the experimental measurements with Crab Cavity noise that took place in 2022 are presented in Chapter~\ref{Ch:experimental_CC_2022}. Despite the very limited available machine time, the results demonstrated a clear dependence of the measured emittance growth, driven primarily of Crab Cavity RF phase noise, on the octupole strength, which supports the hypothesis that the impedance provides a damping mechanism, and may offer an explanation for the observations made in 2018.

%However, quantitative agreement between theory (taking the impedance into account) and experimental results has not yet been demonstrated. Possible additional factors, such as space charge, have been identified and further studies, including simulations and measurements, are foreseen to investigate the quantitative agreement. An additional experiment in the SPS with emittance growth driven by a pure dipolar source (beam transverse damper), supported the results from the Crab Cavity experiment since it also successfully reproduced the dependence of the suppression on the amplitude-dependent tune shift. This improves the confidence in the current understanding of the experimental results with Crab Cavities.


%To conclude, this thesis presents the first experimental beam dynamic studies with $\CC$s and proton beams. It allowed us to identify in simulations and observe experimentally for the first time the effect of the emittance growth suppression from the beam transverse impedance. The analysis and the results presented here improved the understanding of the $\CC$ RF noise effects that are anticipated in the operation of the HL-LHC.


%it constitutes the first investigation and experimental validation of the suppression mechanism of the Crab Cavity RF phase noise or dipolar noise-induced emittance growth by the beam transverse impedance. The studies provided a significant step forward in the understanding of the mechanism of the transverse emittance growth driven by noise in the Crab Cavity RF systems which directly impacts the HL-LHC performance. This project gained confidence in the validity of the Mastoridis--Baudrenghien model which is used for defining limits on the acceptable noise levels for the HL-LHC Crab Cavities.


%To summarise, this thesis addressed issues of significant importance for HL-LHC as it demonstrates the first experimental beam dynamic studies with Crab Cavities and proton beams. Additionally, it constitutes the first investigation and experimental validation of the suppression mechanism of the Crab Cavity RF phase noise or dipolar noise-induced emittance growth by the beam transverse impedance. The studies provided a significant step forward in the understanding of the mechanism of the transverse emittance growth driven by noise in the Crab Cavity RF systems which directly impacts the HL-LHC performance. The project also confirmed that the relevant beam dynamic effects are sufficiently well understood and that the Mastoridis--Baudrenghien model 




%The identification of the emittance growth suppression from the impedance is a significant step forward in the understanding of the mechanism of the transverse emittance growth driven by noise in the Crab Cavity RF systems which directly impacts the HL-LHC performance.

%The project also provides a strong starting point for future additional studies. Specifically, additional measurements are planned in the SPS to refine the experimental observations, by obtaining more data points showing emittance growth rate as a function of amplitude-dependent tune shift. Furthermore, the interplay of the dipolar noise or Crab Cavity RF phase noise with the beam transverse impedance and the space charge needs to be checked to make another step towards the full understanding of the emittance growth suppression mechanism. Finally, dedicated experiments are foreseen to study the emittance growth in the presence of amplitude noise which is related to head-tail modes $\pm$1 which are not damped by the impedance (for the slightly positive chromaticity of the experimental configuration). Another interesting study would be to investigate the impact of linear chromaticity on the suppression factor and the possible transverse instabilities.


%\textbf{Implications for HL-LHC}\\
\section{Implications for HL-LHC}
Since the main motivation of these studies was the use of the $\CC$s in the HL-LHC, the implications of these results on the HL-LHC project are discussed. For reference, Table~\ref{tab:sps_vs_lhc} summarises some of the main machine and beam design parameters for the SPS and HL-LHC machines.

%\section{Implications for HL-LHC}
%In this section, the implications of the results of this thesis for the HL-LHC project are discussed.


%The studies reported in this thesis investigated the emittance growth due to noise in the Crab Cavity RF system experimentally and in simulations for the SPS machine configuration. However, the main motivation of these studies was the planned use of the Crab Cavities in the HL-LHC. To this end, it is appropriate to discuss here the implications of the results of this thesis for the HL-LHC project. 


%The studies reported in this thesis investigated the emittance growth due to noise in the Crab Cavity RF system experimentally and in simulations for one SPS machine configuration. However, the main motivation of these studies was the planned use of the Crab Cavities in the HL-LHC and the importance of the effect of the emittance growth suppression for the HL-LHC. To this end, it is considered appropriate to discuss here the implications of the results of this thesis to the HL-LHC project. It is also explained why simulation studies for the HL-LHC case were not strongly motivated in the context of this thesis.

%The studies in the SPS machine are different from the HL-LHC case. Some of the main machine and beam design parameters are summarised in Table~\ref{tab:sps_vs_lhc} for the SPS and HL-LHC. Some of the most important differences (in the context of emittance growth and impedance interaction) between the two machines are the beam energy, the intensity, the bunch length, the type of the crabbing scheme, the frequency of the main RF system, and of course the presence or not of beam-beam interactions and collisions. Therefore, as stated in the Inroduction, the pure objective of the SPS studies was to validate the predictions of the Mastoridis--Baudrenghien model, in terms of Crab Cavity RF noise-induced emittance growth. Then, one should be able to extrapolate the theoretical predictions to the HL-LHC case. 
%  It appears that the results obtained in the SPS cannot be directly applied, in a straightforward way, to HL-LHC. 

%For reference, Table~\ref{tab:sps_vs_lhc} summarises some of the main machine and beam design parameters for the SPS and HL-LHC machines. Important differences (in the context of emittance growth and impedance interaction) between the two machines are the beam energy, the intensity, the bunch length, the type of the crabbing scheme, the frequency of the main RF system, and of course the presence or not of beam-beam interactions and collisions. %To this end, the objective of the SPS studies was to test the validity of the Mastoridis--Baudrenghien model and identify possible limitations.

\begin{table}[!hbt]
	\centering
  \caption{Overview of the design parameters for the SPS and HL-LHC~\cite{HL_LHC_yellow_report}. The listed values for the SPS correspond to its operation as a storage ring for studying the long-term emittance evolution. The listed values for the HL-LHC case are for beams at collision energy.}
	\begin{tabu} to \textwidth { X[c,m] X[c,m] X[c,m]}
		&& \\[-6mm]
		\toprule \toprule
     \multicolumn{1}{l}{\textbf{Parameter}} &\multicolumn{1}{c}{\textbf{SPS}} & \multicolumn{1}{c}{\textbf{HL-LHC}}\\
      \midrule
      \multicolumn{1}{l}{Circumference, $C_0$}  & \multicolumn{1}{c}{6.9\,km} & \multicolumn{1}{c}{26.7\,km}  \\
      
      \multicolumn{1}{l}{Beam energy, $E_b$}  & \multicolumn{1}{c}{270\,GeV} & \multicolumn{1}{c}{7\,TeV (per beam)} \\

      \multicolumn{1}{l}{Rms bunch length, $\sigma_z$}  & \multicolumn{1}{c}{12-16\,cm} & \multicolumn{1}{c}{7.55\,cm}  \\
    \multicolumn{1}{l}{Frequency of main RF system, $f_\mathrm{RF}$}  & \multicolumn{1}{c}{400\,MHz} & \multicolumn{1}{c}{200\,MHz}  \\

      \multicolumn{1}{l}{Number of bunches}  & \multicolumn{1}{c}{1} & \multicolumn{1}{c}{2808 (per ring)} \\ 
      \multicolumn{1}{l}{Intensity, $N_b$}  & \multicolumn{1}{c}{$3 \times 10^{10}$ protons/bunch} & \multicolumn{1}{c}{$2.2 \times 10^{11}$ protons/bunch}\\
      \multicolumn{1}{l}{Crab Cavity scheme} & \multicolumn{1}{c}{Global} & \multicolumn{1}{c}{Local} \\
      \multicolumn{1}{l}{Beam-beam interaction} & \multicolumn{1}{c}{No} & \multicolumn{1}{c}{Yes} \\
      \multicolumn{1}{l}{Interaction points} & \multicolumn{1}{c}{No}  & \multicolumn{1}{c}{Yes} \\
      \multicolumn{1}{l}{Crossing angle} & \multicolumn{1}{c}{No}  & \multicolumn{1}{c}{Yes} \\
      \arrayrulecolor{black}\bottomrule
	\end{tabu}
   \label{tab:sps_vs_lhc}
\end{table}


\textbf{Regarding the suppression mechanism from the beam coupling impedance}\\
The mechanism of the emittance growth suppression from impedance (or beam-beam effects) due to the separation of the coherent modes from the incoherent spectrum is not expected to appear for the HL-LHC operational configuration for the following two reasons:
\begin{itemize}
   \item Past studies for the LHC and HL-LHC operational conditions, which feature complex bunch train structures, multiple interaction points with asymmetric phase advance, and non-zero chromaticity have shown that the coherent modes (dominated by the beam-beam interactions) are expected to lie inside the incoherent spectrum~\cite{Pieloni:1259906, Buffat:2712068}.
   \item In the HL-LHC operational scenarios, the transverse feedback (ADT~\cite{lhc_adt_info_presentation}) is switched on. Briefly, the ADT measures the bunch-by-bunch beam position every turn and tries to maintain zero centroid oscillations. For the foreseen gain values of this device, the damping time is much faster than the damping time from impedance in the potential (though unlikely) case where the coherent modes would emerge from the incoherent spectrum. This is supported by simulation results, which include noise (not $\CC$ RF noise), beam-beam interactions, wakefields, and the transverse feedback~\cite{Buffat:2712068}.
\end{itemize}

For these reasons, the simulation studies for the HL-LHC case were not strongly motivated in the context of this thesis.

\textbf{Regarding the validity of the Mastoridis--Baudrenghien model}\\
The experimental studies in the SPS during 2022 showed that for configurations where the coherent mode lies inside the incoherent betatron tune spread (like in the HL-LHC scenario) the transverse emittance growth was measured very close to the predictions of Mastoridis--Baudrenghien model. Therefore, the work presented in this thesis gained confidence in the predictions of the model for the HL-LHC, and it can be used for defining limits on the acceptable noise levels for the HL-LHC $\CC$s.



%An uncertainty up to 25$\%$ was observed between measurements and predictions which is, as discussed, within the uncertainty from the measurement method. The work presented in this thesis gained confidence in the predictions of the model for the HL-LHC configuration. 


%Furthermore, in the HL-LHC operational scenarios, the main source for suppressing emittance growth is the transverse feedback (which is also known as ADT~\cite{lhc_adt_info_presentation}). For the foreseen gain values for this device, the damping time is much faster than the damping time from impedance in the potential (though unlikely) case where the coherent modes would emerge from the incoherent spectrum. This is supported by simulation results, which include noise, beam-beam interactions, wakefields, and the transverse feedback~\cite{Buffat:2712068}. % The plots are not shown. Xavier has run the simulations but the results are discussed at the bottom of  p.5
%To this end, it is concluded that the mechanism of the emittance growth suppression from impedance (or beam-beam effects) due to the separation of the coherent modes from the incoherent spectrum does not appear for the HL-LHC operational configuration. 

%The experimental observation of the effect of the suppression from the impedance in the SPS was feasible due to its operation with very small incoherent betatron tune spread. The possibility of applying this mechanism in other machines with similar conditions could be investigated. % pathological case.


%To this end, two conclusions are made: 
%\begin{itemize}
 %   \item First, that the mechanism of the emittance growth suppression from impedance (or beam-beam effects) due to the separation of the coherent modes from the incoherent spectrum does not appear for the HL-LHC operational configuration. The experimental observation of the effect of the suppression from the impedance in the SPS was possible due to its operation with very small incoherent betatron tune spread. The possibility of applying this mechanism in other machines with similar conditions could be investigated. % pathological case.
  %  \item Second, that the results of the studies presented in this thesis do not challenge the available models~\cite{van_kamper_presentation_xavier_theory} which predict the emittance growth for the HL-LHC from the external noise sources that have been identified for LHC~\cite{Buffat:2712068} (Crab Cavities are not considered).
%\end{itemize}

% Note if needed: Simulation results with damper in the SPS: https://docs.google.com/presentation/d/1PkF-Z6i2lv9MrFDFm9FR2t_BFJdJQpFUEMNZ-u99J34/edit#slide=id.gc14df4516e_0_61

%\textbf{Theoretical predictions for emittance growth driven by Crab Cavity RF noise}\\

%However, the results presented in this thesis, and in particular the experimental results of 2022, with Crab Cavity RF phase noise (see Figs.~\ref{fig:cc_md_2022_overview_plots_klod_scan} and~\ref{fig:cc_md_2022_measurement_vs_pyheadtail_simualtion}) challenge the predictions of the analytical models for emittance growth driven by Crab Cavity RF noise. As shown in Figs.~\ref{fig:H_V_emit_growth_background_subtracted_octupole_scan} and~\ref{fig:cc_md_2022_measurement_vs_pyheadtail_simualtion} the measured Crab Cavity RF noise emittance growth might be larger than predicted. The data points acquired during the 2022 measurements are not sufficient to conclude on this issue since it is not clear yet if the saturation of the suppression mechanism (which is expected for strong octupole settings) was reached. This is planned to be addressed in the foreseen experiments with Crab Cavities in SPS. This is highlighting the need for an effective transverse feedback system which is discussed in the next paragraph.



\textbf{Plans for mitigating the emittance growth driven by Crab Cavity RF noise in the HL-LHC}\\
In this paragraph, the current plans for mitigating the emittance growth driven by $\CC$ RF noise in the HL-LHC are discussed for the completeness of the thesis. Recall that the emittance growth suppression mechanism from the beam transverse impedance will not appear in the HL-LHC configuration.

%The contribution of this thesis' work to the folllowing discussion is that it gained confidence the Mastoridis--Baudrenghien model for Crab Cavity RF noise-induced emittance growth in the HL-LHC. Second, they highlighted the necessity for an effective feedback system that will suppress this emittance growth since the emittance growth suppression mechanism from the beam transverse impedance cannot be used in the HL-LHC configuration. 

%During the past years, numerous studies have been performed in order to predict and mitigate the emittance growth from noise in the RF system of the HL-LHC Crab Cavities. The main points are discussed here for the completeness of the thesis.
%The results presented in this thesis, and in particular the experimental results of 2022, with Crab Cavity RF phase noise gained confidence on the predictions of the Mastoridis--Baudrenghien model which are used for the HL-LHC estimates. These HL-LHC estimates~\cite{cc_noise_hl_lhc_estimates}, show that the expected emittance growth from noise present in the Crab Cavity RF system is about 15.4$\%/h$. In the presence of the transverse damper, which is also known as ADT~\cite{lhc_adt_info_presentation}, (assuming damping time of 10 turns) this emittance growth is reduced to about 5.3$\%$/h. These rates correspond to the emittance growth from both amplitude and phase noise. 

The HL-LHC estimates~\cite{cc_noise_hl_lhc_estimates} (based on the Mastoridis--Baudrenghien model), show that the expected emittance growth from noise present in the $\CC$ RF system is about 15.4$\%/h$. In the presence of the transverse damper (assuming damping time of 10 turns) this emittance growth is reduced to about 5.3$\%$/h. Note that these rates correspond to the emittance growth from both amplitude and phase noise. 


However, the target value for emittance growth induced by $\CC$ RF noise for the HL-LHC is 2$\%/h$~\cite{MedinaMedrano:2301928, CC_lumi_limits_philippe, CC_lumi_limits_ilias}. It becomes clear that an additional reduction of 3$\%/h$ is required to meet the target value of the HL-LHC. Reducing the noise floor of the $\CC$ is technologically very challenging as it lies well below the noise floor of the main RF cavities~\cite{cc_noise_hl_lhc_estimates}. %p.2

%For the above-mentioned reasons, it is clear that unfortunately the suppression mechanism from the beam transverse impedance cannot be used to mitigate the expected emittance growth from the Crab Cavity RF. This highlights the necessity for an effective feedback system that will suppress the noise effects from the Crab Cavities. 

Therefore, the further reduction from $\sim 5\%/h$ to 2$\%/h$ could come from a proposed feedback system that uses transverse beam measurements. This system has already been proposed in 2019 but its necessity is underlined, since the identified emittance growth suppression mechanism from the beam transverse impedance cannot be used in the HL-LHC configuration. To provide some information, this $\CC$ feedback system would use an already existing pickup (the same as the transverse damper, ADT) and it could act on both amplitude and phase noise.  Furthermore, the transverse damper and the feedback could be used together, to provide a more effective reduction of the emittance growth, but one should keep in mind that the result is not additive. This system is described in detail in~\cite{Baudrenghien:2665950} and it is still under construction. % \textcolor{red}{To be confirmed}. % Is it the same that Elias mention in the follwing presentation?https://indico.cern.ch/event/1197424/contributions/5035904/attachments/2507124/4308131/HLLHCCollaborationMeeting_Uppsala_19-09-2022_EM.pdf
Another alternative solution would be to operate HL-LHC with slightly flat optics (with different horizontal and vertical $\beta^{\ast}$ values\footnote{The $\beta^{\ast}$ is often used to refer to the beta function at an interaction point.})~\cite{elias_run4_op}. %p.18 
This configuration allows smaller beta functions in the crabbing plane at the locations of the $\CC$s and thus the impact from the noise present in their RF system is smaller. To summarise, there is a lot of challenging and critical work currently in progress to reach the required $2\%/h$ emittance growth rate from $\CC$ RF noise.

%For the above-mentioned reasons, it is clear that unfortunately the suppression mechanism from the beam transverse impedance cannot be used to mitigate the expected emittance growth from the Crab Cavity RF. This highlights the necessity for an effective feedback system that will suppress the noise effects from the Crab Cavities. 

%The contribution of the studies presented in this thesis to the above discussion is that first they gained confidence in the predictions of the Mastoridis--Baudrenghien model for Crab Cavity RF noise-induced emittance growth in the HL-LHC. Second, they highlighted the necessity for an effective feedback system that will suppress this emittance growth since the emittance growth suppression mechanism from the beam transverse impedance cannot be used in the HL-LHC configuration. 


%\textbf{Conclusions}\\
%To conclude, the emittance growth suppression mechanism from the beam transvserse impedance which was identified and validated through the work presented in this thesis will not appear in the HL-LHC configuration, for which it is expected that the coherent modes lie within the incoherent beatron spectrum. Furthermore, the presence of the transverse feedback, in the HL-LHC operational scenario outweighs any potential damping mechanism from impedance due the faster damping time of the former. 

%The experimental results of 2022, support the predictions for the emittance growth in the presence of Crab Cavity noise. This, accompanied with the fact that the current predictions for the emittance growth from Crab Cavity RF noise are almost double than required stress out the need for an effective feedback system on the Crab Cavities. As discussed, there is is lot of challenging and critical work currently in progress to reach the required $2\%/h$ emittance growth rate from Crab Cavity RF noise.

