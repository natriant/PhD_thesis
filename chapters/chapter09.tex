
% This first paragraph is slightly modified from the phd description: /eos/user/n/natriant/Documents/education/cern/doctoral/DOCT_PhD v2.docx
To extend the physics reach of the main experiments of the Large Hadron Collider (LHC) at CERN, the machine will undergo a major upgrade during the coming years. This upgrade, namely the High-Luminosity LHC (HL-LHC) project, aims at about a five-fold increase of the yearly luminosity production with proton beams comparing to the present LHC operation. Reaching these ambitious luminosity goals the precise control and minimization of beam degradation (such as losses and emittance growth) is critical. A particular challenge is associated with the fact that HL-LHC will employ Crab Cavities to compensate the crossing angle of the colliding bunches, ande hence to restore the head-on collisions, in the two main experiments (ATLAS and CMS). However, these Crab Cavities are expected to result in undesired transverse emittance growth due to noise in their RF control system, and therefore loss of luminosity. %  by residual noise on the beam introduced by the crab cavity RF control. 
Given to the very tight HL-LHC target values for luminosity loss and emittance growth from the Crab Cavities (the values are given in the Introduction chapter)
a solid understanding of the emittance growth due to Crab Cavity RF noise is essential. 

To study the transverse emittance growth induced in the presence of the Crab Cavities, two prototype Crab Cavities have been installed in the SPS in 2018 to be tested. From the first round of analysis, the measured
emittance growth was found to be a factor 2-3 lower than predicted from the available theoretical model of T.~Mastoridis and P.~Baudrenghien~\cite{PhysRevSTAB.18.101001}. This PhD thesis set out to to understand the results of experimental tests of Crab Cavities in the SPS. Understanding the results is crucial for gaining confidence in the predictions of the theoretical models and the specifications set for the noise limits on Crab Cavities in HL-LHC which are computed through it.

The first action was to revisit thoroughly the experimental data from the SPS tests of 2018 in order to identify any misinterpretations that could explain the observed discrepancy. These studies are presented in Chapters~\ref{Ch:CC_set_up} and~\ref{Ch:2018_analyisis}. The outcome of this analysis was that the machine configuration and the measurement method could not explain the observed discrepancy. However, these studies provided a clear understanding of the operational aspects of the Crab Cavities in the SPS and of beam-based measurements of the CC voltage which were not documented before. Furthermore, an automated procedure for the calibration of the Crab Cavities during operation was established and will be used in future experiments (actually it was already used in experiments in 2022). Moreover, the quality of the  Gaussian fit on the transverse beam profiles used to obtain the emittance values was verified and the impact of the measurement errors was included in the computation of the emittance growth rates. Open questions, like a possible relation of the transverse emittance growth to the longitudinal beam evolution or the beam intensity, were addressed. Last, an interesting finding was that the four bunches used in the experiment had different emittance growth rates and only one of them was longitudinally stable. To this end, it was decided that future experimental studies would be performed with a single bunch. Focusing the analysis on the stable bunch the difference between its measured emittance growth rates and the ones predicted from the analytical model was then found to be even bigger. Up to a factor of five, lower growth rates were observed in the measurements as compared to the theoretical prediction. 

% From ipac presentation and conclusions of Chapter 6
In the following years, 2019-2020, a big effort with theoretical and simulation studies took place to investigate possible explanations for the discrepancy. These studies are summarised in Chapter~\ref{Ch:investigating_discrepancy}. The sensitivity of the emittance growth rates in possible uncertainties on the measured Crab Cavity voltage and the rms bunch length was tested. Furthermore, the theoretical model of T.~Mastoridis and P.~Baudrenghien was benchmarked against different simulation codes, PyHEADTAIL and Sixtracklib, for conditions close to the experimental configuration of 2018. In PyHEADTAIL the emittance growth driven by CC RF noise was simulated using a simple representation of the SPS consisting of one transfer map and an interaction point where the noise from the Crab Cavities was modeled as kicks on the angle co-ordinates of the particles. The same studies were repeated with Sixtracklib but the tracking was performed using the detailed optics of the SPS, real Crab Cavity element, the measured noise spectra, and also including the non-linearities of the SPS lattice. It was found that none of the above-mentioned points could explain the discrepancy. An important conclusion that was drawn from these studies was that the emittance growth driven by Crab Cavity noise simulated with PyHEADTAIL was in excellent agreement with the Sixtracklib results. This means that there is no sensitivity of the noise-induced emittance growth to the detailed optics of the machine, to the orbit shift from a real Crab Cavity element, or to the non-linearities of the SPS lattice. The use of PyHEADTAIL, for the rest of the studies presented in the thesis, was considered appropriate, since it appears that it includes the necessary beam dynamics for studying this phenomenon.

Finally, a second round of PyHEADTAIL simualtions showed that the transverse beam impedance (not included in the theory nor the simulations so far) has a significant impact on the transverse emittance growth for the SPS experimental conditions. These studies are discussed in Chaper~\ref{Ch:suppression_impedance}. In particular, the simulations using the the most up to date impedance model of the SPS demonstrated that the decoherence and thus the emittance growth is suppressed once the detuning induced by the impedance moves the coherent tune outside of the incoherent tune spectrum. A conceptually similar effect of decoherence suppression was studied in the past in the context of beam-beam modes. Detailed investigations have shown that the decoherence suppression is related purely to the rigid or dipole bunch motion (headtail mode 0) of the beam. In the case of Crab Cavity induced noise kicks, this dipolar excitation is provided by the phase noise kicks. For the experimental configuration of the SPS Crab Cavity tests in 2018 it was found that this mechanism results in an emittance growth suppression by a factor up to about 4 which is very close to the observed discrepancy. It is concluded, that this result suggests that the impedance effects might explain the discrepancy between the measured and theoretically estimated emittance growth rates. The PyHEADTAIL simualtions with the SPS transverse impedance model also suggested that the suppression of the emittance growth as a result of the transverse impedance depends on the amplitude-dependent tune shift. This dependence was tested experimentally in the SPS in 2022.

Another experimental campaign took place in SPS in 2022 to investigate the effects of impedance and amplitude detuning on emittance growth from CC phase noise. The objective was to validate experimentally the suggested suppression mechanism of the emittance growth by the beam transverse impedance by reproducing its dependence on amplitude-dependent tune spread (introduced by the Landau octupoles). Achieving a good understanding of the 2018 results is essential for developing confidence in the theoretical model and its predictions for the HL-LHC. The results of the experimental measurements with Crab Cavity noise that took place in 2022 are presented in Chapter~\ref{Ch:experimental_CC_2022}. Despite the very limited available machine time, the results demonstrated a clear dependence of the measured emittance growth on the octupole strength, which confirms the damping mechanism from the impedance and seems to explain the experimental observations of 2018. The quantitative agreement is not yet achieved. Possible factors, such as space charge, have been identified and further studies, including simulations and measurements, are foreseen to investigate the quantitative agreement. An additional, experiment in the SPS with emittance growth driven by a pure dipolar source (beam transverse damper), supported the results from the Crab Cavity experiment since it also successfully reproduced the dependence of the suppression on the tune spread. This enhanced the condfidence on the understanding of the experimental results with Crab Cavities.

To summarise, this PhD thesis is of fundamental importance as it demonstrates the first experimental beam dynamic studies with Crab Cavities and proton beams. Additionally, it constitutes the first investigation and experimental validation of the suppression mechanism of the Crab Cavity RF phase noise or dipolar noise-induced emittance growth by the beam transverse impedance. The identification of the suppression from the impedance is a breakthrough in the understanding of the mechanism of the transverse emittance growth driven by noise in the Crab Cavity RF systems which directly impacts the HL-LHC performance.

The project also provides a strong starting point for future additional studies.
Specifically, additional measurements are planned in the SPS to refine the experimental observations, by obtaining more data points on the plots of the emittance growth as a function of the detuning from octupoles. Furthermore, the interplay of the dipolar noise or Crab Cavity RF phase noise with the beam transverse impedance and the space charge needs to be evaluated to make another step towards the full understanding of the emittance growth suppression mechanism. Last, dedicated experiments are foreseen to study the emittance growth in the presence of amplitude noise which is related to headtail modes $\pm$1 which are not damped by the impedance (for the slightly positive chromaticity of the experimental configuration). Another interesting study would be to investigate the impact of linear chromaticity on the suppression factor and the possible transverse instabilities.

\section{Implications of the results for the HL-LHC}
This PhD project studied extensively the emittance growth due to noise in the Crab Cavity RF system experimentally and in simulations for one SPS machine configuration. However, the main motivation of these studies was the planned use of the Crab Cavities in the HL-LHC and the criticalness of the effect of the emittance growth suppression for the HL-LHC. To this end, it is considered appropriate to discuss here the implications of the results of this thesis to the HL-LHC project. Additionally, it will be explained why simualtion studies for the HL-LHC case were not strongly motivated.

The studies in the SPS machine are very different from the HL-LHC case. Some of the main machine and beam design parameters are summarised in Table~\ref{tab:sps_vs_lhc} for the SPS and HL-LHC. Some of the most important differences (in the context of emittance growth and impedance interaction) between the two machines are the beam energy, the intensity, the bunch length, the type of the crabbing scheme, the frequency of the main RF system, and of course the presence or not of beam-beam interactions and collisions. Therefore, as stated in the Inroduction, the pure objective of the SPS studies was to validate the predictions of the available theoretical models, in terms of Crab Cavity RF noise-induced emittance growth. Then, one should be able to extrapolate the predictions to the HL-LHC case. It appears that the results obtained in the SPS are not transposable in a straightforward way to the HL-LHC.

\begin{table}[!hbt]
	\centering
   \caption{Overview of the design parameters for the SPS and HL-LHC~\cite{HL_LHC_yellow_report}. The listed values for the SPS correspond to its operation as a storage ring for studying the long-term emittance evolution. The listed values for HL-LHC case are for beams at collision energy.}
	\begin{tabu} to \textwidth { X[c,m] X[c,m] X[c,m]}
		&& \\[-6mm]
		\toprule \toprule
      \multicolumn{1}{l}{\textbf{Parameter}} &\multicolumn{1}{c}{\textbf{SPS}} & \multicolumn{1}{c}{\textbf{HL-LHC}}\\
      \midrule
      \multicolumn{1}{l}{Circumference, $C_0$}  & \multicolumn{1}{c}{6.9\,km} & \multicolumn{1}{c}{26.7\,km}  \\
      
      \multicolumn{1}{l}{Beam energy, $E_b$}  & \multicolumn{1}{c}{270\,GeV} & \multicolumn{1}{c}{7\,TeV (per beam)} \\

      \multicolumn{1}{l}{Rms bunch length, $\sigma_z$}  & \multicolumn{1}{c}{12-16\,cm} & \multicolumn{1}{c}{7.55\,cm}  \\
      \multicolumn{1}{l}{Frequency of main RF system, $f_\mathrm{RF}$}  & \multicolumn{1}{c}{400\,MHz} & \multicolumn{1}{c}{200\,MHz}  \\

      \multicolumn{1}{l}{Number of bunches}  & \multicolumn{1}{c}{1} & \multicolumn{1}{c}{2808 (per ring)} \\ 
      \multicolumn{1}{l}{Intensity, $N_b$}  & \multicolumn{1}{c}{$3 \times 10^{10}$ protons/bunch} & \multicolumn{1}{c}{$2.2 \times 10^{11}$ protons/bunch}\\
      \multicolumn{1}{l}{Crab Cavity scheme} & \multicolumn{1}{c}{Global} & \multicolumn{1}{c}{Local} \\
      \multicolumn{1}{l}{Beam-beam interaction} & \multicolumn{1}{c}{No} & \multicolumn{1}{c}{Yes} \\
      \multicolumn{1}{l}{Interaction points} & \multicolumn{1}{c}{No}  & \multicolumn{1}{c}{Yes} \\
      \multicolumn{1}{l}{Crossing angle} & \multicolumn{1}{c}{No}  & \multicolumn{1}{c}{Yes} \\
      \arrayrulecolor{black}\bottomrule
	\end{tabu}
   \label{tab:sps_vs_lhc}
\end{table}


\textbf{Regarding the suppression mechanism from the beam coupling impedance}\\
Past studies for the LHC and HL-LHC operational conditions which feature complex bunch train structures, multiple interaction points with assymetric phase advance between them in each beam and chromaticity has shown that the coherent modes (dominated by the beam-beam interactions) are expected to lie inside the incoherent spectrum~\cite{Pieloni:1259906, Buffat:2712068}.

Furthermore, in the HL-LHC operational scenarios the main source for suppressing emittance growth is the transverse feedback (which is also known as ADT~\cite{lhc_adt_info_presentation}). For the forseen gain values for this device, the damping time is much faster than the damping time from impedance in the potential case where the coheren modes would emerge from the incoherent spectrum (not-realistic). This is supported by sumulation results, which include noise, the beam-beam interactions, wakefields and the transverse feedback~\cite{Buffat:2712068}. % The plots are not shown. Xavier has run the simulaitons but the results are discussed in the bottom of  p.5



- BB tune spread, multiple interaction points
-  damper is faster. simulations by xavier at the same paper.
- Add link in comments for your study in the sps.

\textbf{Necessity for alternative mitigation methods}\\
The HL-LHC estimates~\cite{cc_noise_hl_lhc_estimates}, show that the expected emittance growth from noise present in the Crab Cavity RF system is about 15.4$\%/h$. In the presence of the transverse damper, which is also known as ADT~\cite{lhc_adt_info_presentation}, (assuming 10 turns of damping time) this emittance growth is reduced to about 5.3$\%/h$. These rates correspond to the emittance growth from both amplitude and phase noise. It becomes clear that an additional reduction of 3$\%/h$ is required to meet the target value of the HL-LHC (2$\%/h$).

Reducing the noise floor of the instrument is technologically very challenging as it is already improved and lies well below the noise floor of the main RF cavities~\cite{cc_noise_hl_lhc_estimates}. %p.2

For the above-mentioned reasons, it is clear that unfortunately the suppression mechanism from the beam transverse impedance cannot be used to mitigate the expected emittance growth from the Crab Cavity RF. This highlights the necessity for an effective feedback system that will suppress the noise effects from the Crab Cavities. 

Therefore, the further reduction from $\sim 5\%/h$ to 2$\%/h$ should come from a proposed feedback system that uses transverse beam measurements. This system has already been proposed in 2019 but its necessity is highlighted in the results presented in this thesis. To provide some information, this Crab Cavity feedback system would use an already existing pickup (the same with the transverse damper, ADT) and it could act on both amplitude and phase noise.  Furthermore, the transverse damper and the feedback could be used together, to provide a more effective reduction of the emittance growth, but one should keep in mind that the result is not additive. This system is described in detail in  Ref.~\cite{Baudrenghien:2665950} and it is still under construction. \textcolor{red}{To be confirmed}. % Is it the same that Elias mention in the follwing presentation?https://indico.cern.ch/event/1197424/contributions/5035904/attachments/2507124/4308131/HLLHCCollaborationMeeting_Uppsala_19-09-2022_EM.pdf

Another alternative solution would be to operate HL-LHC with slightly flat optics (with different horizontal and vertical $\beta^{\ast}$ values \footnote{$\beta^{\ast}$ is often used to refer to the beta function at an interaction point.})~\cite{elias_run4_op}. %p.18 
This configuration results, to smaller beta functions at the location of the Crab Cavities and thus the impact from the noise present in their RF system is smaller.  \textcolor{red}{To be confirmed.}

To conclude, there is lot of challenging and critical work in progress to reach the required $2\%/h$ emittance growth rate from Crab Cavity RF noise.

\textbf{The model of themis and philippe is challenged}\\
- e. The expected CC phase and amplitude noise induce a transverse
emittance growth larger than expected before  (from elias modify.)
- We are not sure if we reached the saturaiton. 
- maybe now the measurements overestimate the theory
- more data in sps to validat .

\textbf{Conclusions}\\
- simulations , tjeoreu and experimental data, agree that for the HL-lHc case the tune is inside the incoherent spectrum.
- which is basically already studied in the presence of the damper. damper is faster. it wil be used any way. it is still not enough.
- bigger problems on investigating the effect of the impedance on hl-lhc.

\newpage

The experiments in SPS are conduced with a single bunch, a global Crab Cavity scheme, much lower inte





This PhD project, was performed in view of (and supported by) the HL-LHC project. However, all of the experimental and numerical studies presented in this thesis were conducted for the SPS machine. In this paragraph, how these studies can be reflect to the HL-LHC case is discussed.



we should be able to “extrapolate” the predictions for  the hl-lhc case. 


The work presented in this thesis, was performed as a part of the HL-LHC project and the findings provide a crucial step forward on the understanding of the Crab Cavity noise effects which impact the HL-LHC performance. 


- what is planned next. Maybe mention MDs septemeber. 
- need of damper is essential for HL-LHC.

comment on damper on 30 months report from Hannes.\\
- do I have any simualtions with damper?
- also in my part I mention it.

% Following lines form t30 months report.
The identification of the emittance growth suppression from the beam coupling impedance is a breakthrough in resolving the above-mentioned discrepancy between emittance growth predicted by existing theoretical models and beam observations. Her work on the quantitative understanding of the experiment at the SPS will be essential for defining limits of acceptable phase noise in the design of the LLRF for the crab cavities in the HL-LHC. The last steps to complete her PhD project are well defined and she is on track.








To appear badass write the conclusions from IPAC presentati


\newpage
- The use of damper is an appropriate configuration as it provides dipolar noise kicks in the beam and as shown in Chapter 7 the suppression mechanism is related ot the dipole motion. 
- More machine time. no cryo module and team needed. only two people..
- Less uncertainties from CC setup.




\section{Towards HL-LHC case}
\subsection{Damper}
Comment on the 200 MHz

I might make a comment:
% From 30 months progress report doctoral: cernbox/Documents/education/cern/doctoral/30months_progress_report
Detailed studies with PyHEADTAIL showed that the emittance growth suppression can reach up to a factor of 2-4 for crab cavity phase noise when the bunch length is longer than the cavity wavelength, which was the case for the SPS experiments. However, for short bunch lengths, which will be the case in the HL-LHC, the suppression factor was found to be similar to the case of pure dipolar noise reaching a factor of 10. The scope of this project is also to characterize the impact of the crab cavities in the long-term emittance evolution in HL-LHC. Thus, the interplay of the transverse impedance and the transverse feedback system, which is planned to suppress the crab cavity noise induced emittance growth in HL-LHC, started being investigated. The first simulation results indicate that the emittance suppression from the feedback system is not affected by the impedance, due to the much faster damping time of the former.


