
% This first paragraph is slightly modified from the phd description: /eos/user/n/natriant/Documents/education/cern/doctoral/DOCT_PhD v2.docx
To extend the physics reach of the main experiments of the Large Hadron Collider (LHC) at CERN, the machine will undergo a major upgrade during the coming years. This upgrade, namely the High-Luminosity LHC (HL-LHC) project, aims at about a five-fold increase of the yearly luminosity production with proton beams comparing to the present LHC operation. Reaching these ambitious luminosity goals the precise control and minimization of beam degradation (such as losses and emittance growth) is critical. A particular challenge is associated with the fact that HL-LHC will employ Crab Cavities to compensate the crossing angle of the colliding bunches, ande hence to restore the head-on collisions, in the two main experiments (ATLAS and CMS). However, these Crab Cavities are expected to result in undesired transverse emittance growth due to noise in their RF control system, and therefore loss of luminosity. %  by residual noise on the beam introduced by the crab cavity RF control. 
Given to the very tight HL-LHC target values for luminosity loss and emittance growth from the Crab Cavities (the values are given in the Introduction chapter)
a solid understanding of the emittance growth due to Crab Cavity RF noise is essential. 

To study the transverse emittance growth induced in the presence of the Crab Cavities, two prototype Crab Cavities have been installed in the SPS in 2018 to be tested. From the first round of analysis, the measured
emittance growth was found to be a factor 2-3 lower than predicted from the available analytical and computational models. This PhD thesis set out to to understand the results of experimental tests of Crab Cavities in the SPS. Understanding the results is crucial for gaining confidence in the predictions of the theoretical models and the specifications set for the noise limits on Crab Cavities in HL-LHC which are computed through it.

The first action was to revisit thoroughly the experimental data from the SPS tests of 2018 in order to identify any misinterpretations that could explain the observed discrepancy. These studies are presented in Chapters~\ref{Ch:CC_set_up} and~\ref{Ch:2018_analyisis}. The outcome of this analysis was that the machine configuration and the measurement method could not explain the observed discrepancy. However, these studies provided a clear understanding of the operational aspects of the Crab Cavities in the SPS and of beam-based measurements of the CC voltage which were not documented before. Furthermore, an automated procedure for the calibration of the Crab Cavities during operation was established and will be used in future experiments (actually it was already used in experiments in 2022). Moreover, the quality of the  Gaussian fit on the transverse beam profiles used to obtain the emittance values was verified and the impact of the measurement errors was included in the computation of the emittance growth rates. Open questions, like a possible relation of the transverse emittance growth to the longitudinal beam evolution or the beam intensity, were addressed. Last, an interesting finding was that the four bunches used in the experiment had different emittance growth rates and only one of them was longitudinally stable. To this end, it was decided that future experimental studies would be performed with a single bunch. Focusing the analysis on the stable bunch the difference between its measured emittance growth rates and the ones predicted from the analytical model was then found to be even bigger. Up to a factor of five, lower growth rates were observed in the measurements as compared to the theoretical prediction. 



- automated procedures that will be used in the furure experemtn. They didnt provide any explanation for the apparent discrepancy. Larger factor of 4.



-  automated procedure for cc calibrationphase offset that will be used in 2018. 
- facotr of 4 andimporvemetns int he experimental prcodeure like single bunch experement and not mulitbunc.


Natalia has worked with colleagues at CERN who performed the tests in SPS to understand important details of the experiments, and she has investigated a number of possible explanations for the apparent discrepancy with the theoretical predictions.


To appear badass write the conclusions from IPAC presentation.


The proposed PhD project concerns beam dynamics studies in preparation for High-Luminosity operation of the LHC. A particular focus will be on operation of the crab cavities in the SPS, including analysis of data taken during machine development studies. Detailed understanding of the observations will require beam dynamics simulations, including dynamic aperture and beam lifetime calculations with and without crab cavities. Other possible contributions to emittance growth in the SPS, such as power supply ripple, may also need to be considered. At a later stage, understanding gained from the studies in the SPS will be applied to simulation studies of incoherent effects in the High Luminosity LHC. The aim will be to characterise the evolution of particle distributions in the presence of crab cavities and to develop a detailed understanding of the impact of possible beam degradation mechanisms (e.g. through tune and orbit modulation induced by the crab cavity RF fields).


\newpage
- The use of damper is an appropriate configuration as it provides dipolar noise kicks in the beam and as shown in Chapter 7 the suppression mechanism is related ot the dipole motion. 
- More machine time. no cryo module and team needed. only two people..
- Less uncertainties from CC setup.




\section{Towards HL-LHC case}
\subsection{Damper}
Comment on the 200 MHz

I might make a comment:
% From 30 months progress report doctoral: cernbox/Documents/education/cern/doctoral/30months_progress_report
Detailed studies with PyHEADTAIL showed that the emittance growth suppression can reach up to a factor of 2-4 for crab cavity phase noise when the bunch length is longer than the cavity wavelength, which was the case for the SPS experiments. However, for short bunch lengths, which will be the case in the HL-LHC, the suppression factor was found to be similar to the case of pure dipolar noise reaching a factor of 10. The scope of this project is also to characterize the impact of the crab cavities in the long-term emittance evolution in HL-LHC. Thus, the interplay of the transverse impedance and the transverse feedback system, which is planned to suppress the crab cavity noise induced emittance growth in HL-LHC, started being investigated. The first simulation results indicate that the emittance suppression from the feedback system is not affected by the impedance, due to the much faster damping time of the former.


