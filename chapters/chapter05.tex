\vspace*{-1mm}

1. present brifly the parameters
2. injected rf noise
3. MD emittance growth overview. 
    - average from IN and OUT. As mentioned in CH4. vs time and vs noise level for all bunches. Not yet comparison with the theory. Probably you need to re-run this to make correctly the error propagation. 
    - 1 noise point was excluded
    - bunch length and longitudinal profiles and relative position from the wall current monitor.  unstable bunches.
    - bunch 2-3-4 longutidinally unstable.
    - intensity --> no losses.
4. bunch 1 comparison with the theory. dey/dt vs noise levels plots. Factor of 4-5 difference. 

Par 2: Section 5.1 .. blah blah

\section{Experimental procedure}
\begin{sloppypar} % to fix \hbox too wide
The beam and machine conditions for the emittance growth studies were discussed extensively in Chapter~\ref{Ch:2018_setup} and are listed in Tables~\ref{tab:machine_beam_param_2018} and~\ref{tab:SPS_CC_main}. In principle the measurements were performed with four bunches at 270\,GeV with low intensity (3 $\times \mathrm{10^{10}}$ ppb) with linear chromaticity corrected to to $\sim$ 1. Only $\CC2$ was used, providing a vertical kick to the beam.
\end{sloppypar} % to fix \hbox too wide


\normalsize{\textbf{Injected RF noise}}\\
 In order to characterize the CC noise induced emittance growth, artificial noise was injected into their LLRF system and the bunch evolution was recorded for about 20-40 minutes. The injected noise was a mixture of amplitude and phase noise up to 10 kHz, overlapping and primarily exciting the first betatron sideband at $\sim 8$ kHz. The phase noise was always dominant. 
 
 - Show 1 examples of AN and 1 of position, measured with a spectrum analyzer [Ref]

 - effective noise
 - Comparison to the theroeical --> mention further explanation is provided ip 

 \section{Comparison with the theory}\label{sec:MD2018_vs_theory}