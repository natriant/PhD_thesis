\vspace*{-1mm}

1. present brifly the parameters
2. injected rf noise
3. MD emittance growth overview. 
    - average from IN and OUT. As mentioned in CH4. vs time and vs noise level for all bunches. Not yet comparison with the theory. Probably you need to re-run this to make correctly the error propagation. 
    - bunch length and longitudinal profiles and relative position from the wall current monitor.  unstable bunches.
    - bunch 2-3-4 longutidinally unstable.
    - intensity --> no losses.
4. bunch 1 comparison with the theory. dey/dt vs noise levels plots. Factor of 4-5 difference. 


\section{Experimental procedure}

Mention again briefly the experiemntal conditions, 1 vertical CC, 4 bunches, intensity 3e10 and chromaticity, Table 4. 


 \section{Injected RF noise} 

 In order to characterize the CC noise induced emittance growth, controlled noise was injected into their LLRF system and the evolution of the bunch was recorded for about 20-40 minutes. The injected noise was a mixture of amplitude and phase noise up to 10 KHz, overlapping and primarily exciting the fisrt betatron sideband at $\sim 8$ kHz. The phase noise was always dominant. 
 
 - Show 1 examples of AN and 1 of position, measured with a spectrum analyzer [Ref]

 - effective noise
 - Comparison to the theroeical --> mention further explanation is provided ip 

 \section{Comparison with the theory}\label{sec:MD2018_vs_theory}