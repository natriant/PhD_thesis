\vspace*{-1mm}
The theoretical model for the transverse emittance growth caused by amplitude and phase noise in a CC was introduced in Chapter~\ref{Ch:CC_noise_theory}. On September 5, 2018, a dedicated experiment was conducted in the SPS to benchmark this model against experimental data and confirm the analytical predictions. In this chapter, the machine setup, the beam configuration and the instrumentation used for the emittance growth studies with $\CC$s in the SPS are presented.


\begin{sloppypar} % to fix \hbox too wide
The chapter is structured as follows: Section~\ref{sec:SPS_setup}
describes the experimental machine configuration. Thereafter, Section~\ref{sec:CC_SPS_setup} elaborates on the installation and the operational aspects of the CCs in the SPS. In Sections~\ref{sec:HT_info}-~\ref{sec:ABWLM_WallCurrentMonitor} the instruments used for the parameters of interest (see Chapter~\ref{Ch:CC_noise_theory}, Eq.~\eqref{eq:dey_an} and Eq.~\eqref{eq:dey_pn})  i.e. $\CC$ voltage, emittance and bunch length are discussed, including the post processing methods where it was performed by the author. The results of the experiment are presented separately in Chapter~\ref{Ch:2018_analyisis}.
\end{sloppypar}


\section{Machine and beam configuration}\label{sec:SPS_setup}
For studying the long-term emittance evolution a special mode of operation was set up in the SPS which is called "coast" (in other machines, it is referred to as storage ring mode) with bunched beams. In this mode, the bunches circulate in the machine at constant energy for long periods, from a few minutes up to several hours, similar to the HL-LHC case.

To make sure that the SPS can be used as a testbed for the emittance growth studies with $\CC$s an extensive preparatory campaign was carried out through 2012-2017~\cite{Calaga:1451286, Alekou_CC_coast_prep_2016, Antoniou:2649815}. The primary concern was the emittance growth that was observed in the machine from other sources than injected noise and will be referred to as the natural emittance growth in this thesis. The natural emittance growth needs to be well characterized and be kept sufficiently small in order to distinguish and understand the contribution from the $\CC$ noise. From these studies, it was concluded that the optimal coast setup is at high energies, with low chromaticity and bunches of low intensity as it minimises the natural emittance growth~\cite{Antoniou:2649815}. The highest energy for which the SPS could operate in "coast" was 270\,GeV and thus the experiments were performed at this energy. That limitation was introduced due to the rms power deposited in its magnets when operating at high energy for long period of time. Moreover, as the natural emittance growth was found to be a single bunch effect four bunches were used. That choice was made to reduce the statistical uncertainty of the measurements but not to increase the beam intensity.

During the experiment the Landau octupoles were switched off. Nevetherless, a residual non-linearity was present in the machine mainly due to multipole components in the dipole magnets~\cite{Carlà:2664976, Alekou:2640326}. Last, the transverse feedback system was switched off. The main machine and beam parameters used in the experiment of 2018 are listed in Table~\ref{tab:machine_beam_param_2018}. It should be noted, that no measurements of chromaticity are available from the day of the experiment. However it was ensured that the chromaticity was corrected to small positive values. 

\begin{table}[!hbt]
	\centering
   \caption{Main machine and beam parameters for the emittance growth studies with CCs in SPS in 2018.}
	\begin{tabu} to \textwidth {X[c,m] X[0.5c,m] X[0.5c,m] X[0.01c,m]}
		&&& \\[-6mm]
		\toprule \toprule
		\multicolumn{2}{l}{\textbf{Parameter}} &
		\multicolumn{2}{c}{\textbf{Value}} \\
		\bottomrule
      \multicolumn{2}{l}{Beam energy, $\symE$} & \multicolumn{2}{c}{270\,GeV} \\
      \multicolumn{2}{l}{Revolution frequency, $\frev$}  & \multicolumn{2}{c}{43.375\,kHz} \\
      \multicolumn{2}{l}{Number of proton per bunch, $\Nb$} & \multicolumn{2}{c}{3 $\times 10^{10}$ p/b} \\
      \multicolumn{2}{l}{Number of bunches}  & \multicolumn{2}{c}{4} \\
      \multicolumn{2}{l}{Bunch spacing}  & \multicolumn{2}{c}{524\,ns} \\
      \multicolumn{2}{l}{Main RF frequency, $\fRF$}  & \multicolumn{2}{c}{200.39\,MHz} \\ %200.3945
      \multicolumn{2}{l}{Main RF voltage, $\VRF$}  & \multicolumn{2}{c}{3.8\,MV} \\
      \multicolumn{2}{l}{Horizontal / Vertical betatron tune, $\Qx$ / $\Qy$}  & \multicolumn{2}{c}{26.13 / 26.18} \\
      \multicolumn{2}{l}{Horizontal / Vertical first order chromaticity, $\Qpx$ / $\Qpy$}  & \multicolumn{2}{c}{ $\sim$ 1.0 / $\sim$ 1.0} \\
      \multicolumn{2}{l}{Synchrotron tune, $\Qs$}  & \multicolumn{2}{c}{0.0051} \\
      \bottomrule
	\end{tabu}
   \label{tab:machine_beam_param_2018}
\end{table}

\section{Crab Cavities in the SPS}\label{sec:CC_SPS_setup}

For the SPS tests two prototype $\CC$s of the Double Quarter Wave (DQW) type, $\CC$1 and $\CC2$, were fabricated by CERN and were assembled in the same cryomodule, shown in Fig.~\ref{fig:DQW_cryomodule}~\cite{Zanoni:2017}. For its istallation For its installation an available space was found at the SPS Long Straight Section 6 (SPS-LSS6) zone. As this section is also used for the extraction of the beam to the LHC, the cryomodule was placed on a mobile transfer table~\cite{Calaga:2649807} which  moved the cryomodule in the beamline for the $\CC$ tests and out of it for the usual SPS operation without breaking the vaccum. For the noise induced emittance growth studies only $\CC$2 was used. Nevertheless, the parameters for both $\CC$s are shown in Table~\ref{tab:SPS_CC_main} for completeness. In case someone wants to repeat the study, the location of the $\CC$s along the SPS ring is also indicated.

\begin{figure}[h]
   \centering         
   \includegraphics[width=0.8\textwidth]{images/Ch4/CC_cryomodule.png}
       \caption{Cross section view of the CC cryomodule~\cite{Zanoni:2017}. It has a total length of 3\,m~\cite{Baudrenghien:1520896} and at its core there are the two DQW cavities, which are illustrated with light green color.}
       \label{fig:DQW_cryomodule}
\end{figure}
% Couldn't find information on height, weight etc. Any idead where to look for spces?

% Photos from the cryomodule installation day: 1. https://home.cern/news/news/engineering/crabs-settled-tunnel (maybe good to show the scale?)
% 2. https://science.osti.gov/-/media/np/pdf/research/NP-Accel-RD-PI-Meeting/2019/Wu_2019FOA_PImeeting_crabcavity_Rev9.pdf?la=en&hash=B29283EDB3C704EC3EFD9DA23CA8DEA22076C88B

% QF.10010  is the start of the SPS lattice --> https://layout.cern.ch/reports/mad?fileType=STANDARD&machineId=2180065&versionId=34464591 (access with sshuttle)
% The actual start of the lattice at 0 m is: %BEGI.10010. which is a marker (not a real element).

% Optics parameters for CC and diagnostics in 2018: https://cernbox.cern.ch/index.php/apps/files/?dir=/__myshares/SPS_MDs_2018%20(id%3A271128)/SPSMeasurementTools&#editor

\subsection{Operational considerations}

For the beam tests with the $\CC$ in the SPS the approach regarding the energy ramp and the adjustment of the phasing with the main RF system needed to be evaluated and they are briefly discussed here.

\normalsize{\textbf{Energy ramp}}\\
SPS recieves the proton beam at 26\,GeV from the PS. It was found that the ramp to higher energies could not be performed with the $\CC$ on, as the beam was getting lost while crossing one of the vertical betatron sidebands due to resonant excitation~\cite{Rama_Paris_persentation}. Therefore, it was established that the acceleration has to be performed with the $\CC$ off and its voltage must be set up only after the energy of interest has been achieved. It is worth noting that this approach will also be used in the HL-LHC.

\normalsize{\textbf{Crab Cavity - main RF synchronisation}}\\
It was important to ensure that during the "coast" the beam will epxeriene the same kick from the $\CC$ each turn. In other words the SPS main RF system operating at 200\,MHz needed to be synchronous with the $\CC$ operating at 400\,MHz. Due to the larger bandwidth of the SPS main RF system the $\CC$ was used as a master. Therefore the $\CC$ was operating at a fixed frequency and phase, while the main accelerating cavities were adjusted to the exact half of the $\CC$ frequency (see values at Tables~\ref{tab:machine_beam_param_2018} and~\ref{tab:SPS_CC_main}) and were re-phased so that they become synchronous with the crabbing signal. For the studies at 270\,GeV the synchronisation took place at the end of the ramp to the coast energy and shortly after the cavity was switched on~\cite{BE_seminar}.

% Steps for synchronisation: 1) SPS main RF was set at the exact half of the main RF. 2) The phase was adjusted. 3) The frequency needed to be re-adjusted slightly again such is it becomes the exact half of it.
% Nice explanation of the synchronisatio: https://journals.aps.org/prab/pdf/10.1103/PhysRevAccelBeams.24.062001
% For studies at the injection energy of 26\,GeV this synchronisation took place shortly after the injection. 


\normalsize{\textbf{Estimation of the amplitude and the uncertainty of the measurement}}\\
It is clear from Fig.~\ref{fig:VCC_from_HT_monitor_measurement} that the reconstructed $\CC$ voltage, $\VCC(t)$, is not centered around zero. This voltage offset is not a systematic error as it is not in the same direction and has a different value for each measurement. In this case, the amplitude of the signal is defined as half the peak to peak amplitude, $\VCC = V_{CC, p-p}/2$. Note that the peak to peak amplitude, $V_{CC, p-p}$, is the difference between the maximum positive and negative peaks of the signal. The uncertainty in the amplitude is defined as the offset that needs to be added or subtracted to the signal, such as it is centered around zero. In other words, the uncertainty, $\Delta \VCC$, is defined as half the sum of the maximum positive and negative peaks of the signal. Figure~\ref{fig:VCC_amplitude_and_uncertainty} shows the peak to peak amplitude, $V_{CC, p-p}$, the amplitude of the signal, $\VCC$, and the uncertainty, $\Delta \VCC$, for the reconstructed $\CC$ voltage, $\VCC(t)$.

\begin{figure}[!h]
   \centering         
   \includegraphics[width=0.65\textwidth]{images/Ch4/Vcc_amplitude_uncertainty.pdf}
       \caption{Illustration of the CC amplitude voltage, $V_{CC}$, defined as half the peak to peak amplitude, $V_{CC, p-p}$ (blue) and of its uncertainty, $\Delta V_{CC}$, defined as the offset needed for the signal to be zero-centered. Here, $V_{CC}$=0.96\,MV and $\Delta V_{CC}$=0.15\,MV.}
       \label{fig:VCC_amplitude_and_uncertainty}
   \end{figure}

\normalsize{\textbf{Reconstruction of crabbing}}\\
Additionally, the measurements from the HT monitor were used for reconstructing the crabbing and representating schematically the beam projection {\color{red}in the transverse plane}. The technique for reconstructing the crabbing was developed at CERN in 2018 and was extensively used throught the experimental campaign with $\CC$s since (together with the calibrated voltage) it gives a straightforward estimate of the applied $\CC$ kick, as illustrated in Fig.~\ref{fig:crabbing_reconstruction_HT_monitor}. 

To obtain this schematic representation, which is practically a density plot, of the effect of the CC kick on the beam one needs to multiply the measured longitudinal profile, mean of the sum signals acquired after the synchronisation, with the measured intra-bunch offset, mean of the differene signals acquired after the synchronisation. An example of this is shown in Fig.~\ref{fig:HT_baseline_correction_crabbing_mm}. For the transverse plane a gaussian distribution is considered with $\sigma$ obtained from the wire scanner (addressed in more detail in the following section). The color code of Fig.~\ref{fig:crabbing_reconstruction_HT_monitor} is normalised to the maximum intensity within the bunch.


\section{SPS Wire Scanners}\label{sec:sps_ws}
The SPS is equipped with Wire Scanners (WS) to measure the transverse beam emittance. The SPS WS system is described in detail in Ref.~\cite{BOSSER1985475, Berrig:1972478}. For the SPS tests, the emittance was measured with WS both for the horizontal and vertical plane (BWS.51995.H and BWS.41677.V respectively).

The working principle is shown in Fig.~\ref{fig:SPS_WS_ROT}. A thin wire rapidly moves across the proton beam and a shower of secondary particles is generated. The signal from the secondary particles is then detected by a system of scintillator and photomultiplier (PM) detectors outside of the beam pipe. By measuring the PM current as a function of wire position over multiple turns the transverse beam profile is reconstructed. An example of a vertical profile is shown in Fig.~\ref{fig:WS_example_V_profile}.

\begin{figure}[!h]
   \centering         
   \includegraphics[width=0.8\textwidth]{images/Ch4/Wire_scanner.png}
       \caption{Sketch of the SPS rotational wire scanners~\cite{Berrig:1972478}. The wire moves across the proton beam generating secondary particles which are then detecting by a scintillator and a photomultiplier. From the measured photomultiplier current the beam profile is reconstructed.}
       \label{fig:SPS_WS_ROT}
\end{figure}


\normalsize{\textbf{Fitting of transverse profiles}}\\
To obtain the beam size, $\sigma$, the standard procedure of fitting measured data to a model is followed~\cite{gaus_fit_least_squares}. In particular, the transverse profiles from each scan are fitted with the following four-parameter gaussian function:

\begin{equation}\label{eq:4p_gauss}
   f(x) = k + A e^{-\frac{(x-\mu)^2}{2 \sigma^2}},
\end{equation}

where $k$ is the signal offset of the PM, $A$ is the signal amplitude, $\mu$ is the mean of the Gaussian distribution and $\sigma$ its standard deviation. A non-linear least square minimization is used to fit the gaussian function to the measured data and obtain the optimal values for the parameters ($A, k, \mu, \sigma$). The standard error of the parameters' estimates is given by the square root of the diagonal of their covariance matrix.~\cite{gaus_fit_least_squares}. The uncertainty of the measured beam size, $\Delta \sigma$, is defined as the standard error of the $\sigma$ parameter. The optimal parameters' values and their covariance matrix are computed here using the $\mathrm{scipy.curve \_ fit}$ \cite{scipy_curve_fit} function of Python programming language. 
% Curve fitting explanation: https://www.youtube.com/watch?v=Jl-Ye38qkRc&ab_channel=BrantCarlson


% Location where the figure was produced: /eos/user/n/natriant/2020/WS_analysis
\begin{figure}[!h]
   \centering         
   \includegraphics[width=0.6\textwidth]{images/Ch4/SPS.BWS.41677.V_ROT_2018-09-05 15_45_01.33500_raw_and_fit.png}
       \caption{Vertical beam profile obtained from the BWS.41677.V instrument. The measured data points (light blue) are fitted with a four parameter Gaussian (orange) to obtain the beam size. The calculated emittance is also shown.}
       \label{fig:WS_example_V_profile}
\end{figure}
   

The general formula for computing the normalised beam emittance from the beam size, $\sigma$ is given by:
\begin{equation}\label{eq:emittance_from_WS}
   \centering
   \epsilon = \frac{\sigma^2}{\beta_{WS}} \betarel \gammarel ,
\end{equation}

where $\sigma$ is the rms beam size, $\beta_{WS}$ the beta function at the WS location and $\betarel, \gammarel$ the relativistic parameters.


Note that, in the 2018 SPS operational configuration, the dispersion was small at the WSs location and thus its contribution to the beam size was considered to be negligible \footnote{The dispersion at BWS.51995.H location in 2018 was $\Dx$= -15\,mm. At 270\,GeV, the energy spread, $\delta$, is of the order of $\mathrm{10^{-4}}$. Thus, from Eq.~\eqref{eq:emit_from_beam_size} the horizontal normalised emittance from the dispersion is expected at the order of $\mathrm{10^{-6} \ \mu m}$. Comparing to the observed beam size during the CC tests of a few microns the dispersion is negligible. \color{red}{The measured $\Dx, \Dy$ were found to be very small and thus their contribution is also considered negligible. The plan is to perform some measurments in 2022 to get a feeling of their values at the location of the wire scanners}}. For the $\CC$ studies at 270\,GeV beam energy, $\betarel \gammarel$ equals 287.8 and the beta functions were 81.5\,m and 62.96\,m at the locations of the horizontal and vertical WS respectively. The uncertainty on the beta functions at the location of the WS, $\Delta \beta_{WS}$, is 5$\%$ in both planes, which represents the rms beta-beating in the SPS~\cite{SPS-beta-beating-Rogelio}.
% Computations from dispersive contribution: /eos/user/n/natriant/Project_thesis/material/Ch4: 1st experimental campaign in SPS/dispersive_contribution_in_the_emittance.pdf

Assuming that the relativistic parameters are free of error, the uncertainty of the computed emittance, $\Delta \epsilon$, depends on the uncertainty of the measured beam size, $\Delta \sigma$ and of the beta function at the location of the WS, $\Delta \beta_{WS}$, as follows:

\begin{equation}\label{eq:emittance_from_WS_uncertainty}
   \centering
   \Delta \epsilon = \sqrt{\left ( \frac{\partial \epsilon}{\partial \sigma}\right )^2 \Delta \sigma^2 + \left ( \frac{\partial \epsilon}{\partial \beta_{WS}} \right )^2 \Delta \beta_{WS}^2} = \epsilon  \sqrt{\left ( \frac{2 \Delta \sigma}{\sigma}\right )^2 + \left ( \frac{ \Delta \beta_{WS}}{\beta_{WS}} \right )^2} .
\end{equation}
% Propagation of uncertainty in paper: /eos/user/n/natriant/Project_thesis/material/Ch4: 1st experimental campaign in SPS/propagation_of_uncertainty_emittance.pdf

\normalsize{\textbf{Further considerations}}\\
It is worth noting here that during each measurement with the WS the beam profile is actually acquired twice as the wire crosses the beam in the forward direction (IN scan) and then in the reverse direction (OUT scan). For the 2018 measurements the emittance values obtained from IN and OUT scans, $\epsilon_\mathrm{IN} \pm \Delta \epsilon_\mathrm{IN}$ and $\epsilon_\mathrm{OUT} \pm \Delta \epsilon_\mathrm{OUT}$, were found to be very similar. In the analysis of the 2018 measurements, the average emittance from the two scans, $\epsilon_\mathrm{avg} = \langle \epsilon_\mathrm{IN}, \epsilon_\mathrm{OUT}\rangle$, is used. The uncertainty in the average, $\Delta \epsilon_\mathrm{avg, 1}$, is given by (see Appendix ...): 
\begin{equation}\label{eq:uncertainty_mean_ws}
   \Delta \epsilon_\mathrm{avg, 1} = \frac{\mid \epsilon_\mathrm{IN} - \epsilon_\mathrm{OUT} \mid}{2 \sqrt{2}}.
\end{equation}
The propagated uncertainty from the measurement errors, $\Delta \epsilon_\mathrm{IN}$ and $\Delta \epsilon_\mathrm{OUT}$, is given by:
\begin{equation}\label{eq:propagated_uncertainty_ws}
   \Delta \epsilon_\mathrm{avg, 2} = \frac{1}{2}\sqrt{ \Delta \epsilon_\mathrm{IN}^2 + \Delta \epsilon_\mathrm{OUT}^2}.
\end{equation}
Considering that $\Delta \epsilon_\mathrm{avg, 1}$ and $\Delta \epsilon_\mathrm{avg, 2}$ are independent, the combined uncertainty in the average, $\Delta \epsilon_\mathrm{avg}$, is given by:

\begin{equation}\label{eq:combined_uncertainty_ws}
   \Delta \epsilon_\mathrm{avg} = \sqrt{\Delta \epsilon_\mathrm{avg, 1} ^2 + \Delta \epsilon_\mathrm{avg, 2} ^2}.
\end{equation}

Finally, some emittance increase is expected during each wire scan, due to multiple Coulomb scattering. This effect has been extensively studied in Ref.~\cite{Roncarolo:1481835}. For the rotational SPS WS and the energy of 270\,GeV, at which the $\CC$ experiments were performed the expected emittance growth from the WS is expected to be small. However, a conservative number of scans were carried, $\sim$ 20 scans per bunch and per plane during $\sim$ 1 hour, in order to minimise the contribution from this effect.

\section{ABWLM and Wall Current monitor}\label{sec:ABWLM_WallCurrentMonitor}

\begin{sloppypar}
The bunch length was measured with two different instruments the ABWLM (A for RF, Beam, Wideband, Longitudinal, Measurement)~\cite{ABWLM} and the Wall Current monitor~\cite{Papotti:1124099}. The ABWLM measures the longitudinal profiles from which the bunch length is computed by performing a gaussian fit. The Wall Current monitor acquirest not just the longitudinal profiles but also the longitudinal beam position relative to the monitor i.e. the beam arrival with respect to the reference. The bunch length is estimated by computing the full width half maximum of the profiles and then using it to estimate the sigma of a gaussian distribution. No further details on the operation of these instruments are discussed here as the offline analysis was not performed by the author.
\end{sloppypar}


% beam position relative to monitor
