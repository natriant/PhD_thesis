\vspace*{-1mm}
In Chapter~\ref{Ch:CC_noise_theory} the theoretical model for the transverse emittance growth caused by amplitude and phase noise in a $\CC$ was discussed. On September 5, 2018, a dedicated experiment was conducted in the SPS to benchmark this model against experimental data and confirm the analytical predictions. In particular, the idea was to inject artificial noise in the $\CC$ RF system and compare the measured emittance growth rates with the the theoretically computed ones. In this chapter the measurement results from the SPS are presented and discussed. The work published in Ref.~\cite{Triantafyllou:2021khx} is the basis of this chapter.


Section ... describes the machine setup and the beam configuration for the emittance growth measurements. This includes a summary of the preparatory studies conducted in the previous years. In Section ...


\section{Experimental configuration and procedure}\label{sec:exp_setup_2018}
This section gives an overview of the experimental setup and the procedure that was followed. First, it briefly discusses the preparatory studies that were performed during 2012-2017~\cite{Calaga:1451286, Alekou_CC_coast_prep_2016, Antoniou:2649815}, explaining the choice of the intensity and energy values for which the emittance growth measurements were conducted. Furthermore, it presents in detail the rest of the beam and machine conditions during the experiment. Last, the experimental procedure is explained.

%\normalsize{\textbf{Preparatory experimental studies}}\\
\subsection{Preparatory experimental studies}\label{sec:preparatory_studies_for_2018_MD}
For studying the long-term emittance evolution a special mode of operation was set up in the SPS which is called "coast" (in other machines, it is referred to as storage ring mode) with bunched beams. In this mode, the bunches circulate in the machine at constant energy for long periods, from a few minutes up to several hours, similar to the HL-LHC case.\setlength{\parskip}{2ex} %set it manually here, as for some reason it was not displayed properly in the pdf.

To make sure that the SPS can be used as a testbed for the emittance growth studies with $\CC$s an extensive preparatory campaign was carried out through 2012-2017~\cite{Calaga:1451286, Alekou_CC_coast_prep_2016, Antoniou:2649815}. The primary concern was the emittance growth that was observed in the machine from other sources than injected noise and will be referred to as the natural emittance growth in this thesis. The natural emittance growth needs to be well characterized and be kept sufficiently small in order to distinguish and understand the contribution from the $\CC$ noise. 

From these studies, it was concluded that the optimal coast setup is at high energies, with low chromaticity and bunches of low intensity as it minimises the natural emittance growth~\cite{Antoniou:2649815}. The highest energy for which the SPS could operate in "coast" was 270\,GeV and thus the experiments were performed at this energy. That limitation was introduced due to the rms power deposited in its magnets when operating at high energy for long period of time. Moreover, as the natural emittance growth was found to be a single bunch effect four bunches were used. That choice was made to reduce the statistical uncertainty of the measurements but not to increase the beam intensity.

%\normalsize{\textbf{Machine and beam configuration}}\\
\subsection{Machine and beam configuration}
During the experiment the Landau octupoles were switched off. Nevetherless, a residual non-linearity was present in the machine mainly due to multipole components in the dipole magnets~\cite{Carlà:2664976, Alekou:2640326}. The transverse feedback system was also switched off. Unfortunately, no measurements of chromaticity are available from the day of the experiment. However it was ensured that the chromaticity was corrected to small positive values. 

Last, only one $\CC$, $\CC 2$, was used for simplicity and it operated at 1\,MV. This value was validated with the HT monitor (post-processing procedure desribed in Chapter~\ref{Ch:CC_set_up}). Unfortunately, only one beam based measurement of the $\CC$ voltage is available whcih is displayed in Fig.~\ref{fig:crabbing_sin_fit_MD5}. It is clear that the measured value of voltage amplitude, $\VCC$ = 0.99 $\pm$ 0.04\,MV , is in good agreement with the requested one. It should be noted, that due to the beam energy of 270\,GeV the crabbing is less visible than the example discussed in Chapter~\ref{Ch:CC_set_up} (see Fig.~\ref{fig:crabbing_sin_fit_MD2}) for 26\,GeV. Therefore, here the part of the signal that is used for the fit is the one for which the normalised sum signal (blck dashed line) is above 0.2 (instead of 0.4 that was the condition for the case of 26\,GeV)

\begin{figure}[!h]
   \centering         
   \includegraphics[width=0.7\textwidth]{images/Ch5/HT_VCC_callibration_20180905_135033_sin_fit_fixed_freq.png}
       \caption{Demonstration of the sinusoidal fit on the HT monitor reading in order to obtain the CC parameters as described in Section~\ref{sec:CC_voltage_meas}. The fit results, are given in the yellow box. The measured voltage amplitude, $V_{CC}$, was found to be 0.99\,MV while its uncertainty, $\Delta V_{CC}$, was measured at 0.04\,MV. The measured voltage value agrees well with the requested value of 1\,MV.}
       \label{fig:crabbing_sin_fit_MD5}
\end{figure}


The main machine and beam parameters used for the emittance growth measurements in 2018 are listed in Table~\ref{tab:machine_beam_param_2018}. % the parameters shown are the relatives one for Eq.(3.1) and (3.2) --> theoretical formulas for the emittance growht.


\begin{table}[!hbt]
	\begin{minipage}{\textwidth}
      \begin{centering}
   \caption{Main machine and beam parameters for the emittance growth studies with CCs in SPS in 2018.}
	\begin{tabu} to \textwidth {X[c,m] X[0.5c,m] X[0.5c,m] X[0.01c,m]}
		&&& \\[-6mm]
		\toprule \toprule
		\multicolumn{2}{l}{\textbf{Parameter}} &
		\multicolumn{2}{c}{\textbf{Value}} \\
		\bottomrule
      \multicolumn{2}{l}{Beam energy, $\symE$} & \multicolumn{2}{c}{270\,GeV} \\
      \multicolumn{2}{l}{Revolution frequency, $\frev$}  & \multicolumn{2}{c}{43.375\,kHz} \\
      \multicolumn{2}{l}{Main RF voltage / frequency,  $\VRF$ / $\fRF$}  & \multicolumn{2}{c}{3.8\,MV / 200.39\,MHz} \\ %200.3945
      \multicolumn{2}{l}{Horizontal / Vertical betatron tune, $\Qx$ / $\Qy$}  & \multicolumn{2}{c}{26.13 / 26.18} \\
      \multicolumn{2}{l}{Horizontal / Vertical first order chromaticity, $\Qpx$ / $\Qpy$}  & \multicolumn{2}{c}{ $\sim$ 1.0 / $\sim$ 1.0} \\
      \multicolumn{2}{l}{Synchrotron tune, $\Qs$}  & \multicolumn{2}{c}{0.0051} \\
      \multicolumn{2}{l}{$\CC 2$ voltage / frequency, $\VCC$ / $\fCC$}  & \multicolumn{2}{c}{1\,MV / 400.78\,MHz} \\
      \multicolumn{2}{l}{Number of protons per bunch, $\Nb$} & \multicolumn{2}{c}{3 $\times 10^{10}$ p/b$^\ast$} \\
      \multicolumn{2}{l}{Number of bunches}  & \multicolumn{2}{c}{4} \\
      \multicolumn{2}{l}{Bunch spacing}  & \multicolumn{2}{c}{524\,ns} \\
      \multicolumn{2}{l}{Rms bunch length, $\sigmat$}  & \multicolumn{2}{c}{1.8 \,ns$^\ast$}\\
      \multicolumn{2}{l}{Horizontal / Vertical normalised emittance, $\emitx$ / $\emity$}  & \multicolumn{2}{c}{2\,$\mathrm{\mu m}$ / 2\,$\mathrm{\mu m^\ast}$}\\
      \multicolumn{2}{l}{Horizontal / Vertical rms tune spread, $\Dqxrms$ / $\Dqyrms$}  & \multicolumn{2}{c}{2\,ns / 2\,ns$^\dagger$}\\
      \bottomrule
	\end{tabu}
   \label{tab:machine_beam_param_2018}
   \end{centering} \footnotesize{$^\ast$ The value correspond to the requested intial value at the start of each coast. The measured evolution of the parameter through the experiment is presented in the Sections~\ref{sec:emit_growth_meas_2018} and~\ref{sec:bunch_length_intensity_meas_2018}.\\$^\dagger$ Here the rms betatron tune spread includes only the contribution from the amplitude detuning introduced by the multiple components in the SPS dipole magnets. The calulcations for the listed values can be found in Appendix~\ref{app:detuning_with_amplitude}.}
   \end{minipage}
\end{table}


%\normalsize{\textbf{Experimental procedure}}\\
\subsection{Experimental procedure}\label{sec:experimental_procedure_2018}
The experiment took place on September 5, 2018, and was given a total time window of about 6 hours (start:$\sim$10:30, end:$\sim$17:00). In order to characterize the CC noise induced emittance growth, different levels of controlled noise were injected into its LLRF system and the bunch evolution was recorded for about 20-40 minutes (for each noise setting). The experiment was conducted over three coasts, since a new beam was injected every time the quality of the beam was seen to be degraded e.g. very large beam size. In the following, the diffrent noise settings will be denoted as "Coast$N$-Setting$M$", where $N$ stands for the coast number and $M$ for the different noise levels applied in each coast in chronological order. After the experiment, the measured growth rates would be compared with the theoretically expected values from the model described in Chapter~\ref{Ch:CC_noise_theory}.
%The dedicated experiment to study the emittance growth induced by noise in the $\CC$ RF sytsem took place on September 5, 2018, and was given a total time window of about 16 hours (start:$\sim$10:30, end:$\sim$17:00).

\section{Injected RF noise}\label{sec:injected_RF_noise}
The noise injected in the $\CC$ RF system was a mixture of amplitude and phase noise up to 10\,kHz, overlapping and primarily exciting the first betatron sideband at $\sim 8$\,kHz. The phase noise was always dominant. The noise levels were measured with a spectrum analyzer E5052B~\cite{E5052B_insight} and are expressed as $10\log_{10}\mathcal{L}(f)$\,[dBc/Hz]. The relation between the measrued noise levels and the PSDs in Eq.~\eqref{eq:dey_an} and Eq.~\eqref{eq:dey_pn} is given by $S_\Delta = 2\mathcal{L}(f)$, with $S_{\Delta A}$ in\,1/Hz and $S_{\Delta \phi}$ in $\mathrm{rad^2/Hz}$. This relation is extensivly discussed in Appendix~\ref{ch:app_B} and specifically in~\ref{app:Measured_noise}}. Figure~\ref{fig:example_PN_and_AN_coast1_setting2} displays an example measurement of amplitude (left) and phase (right) noise acquired during the experiment.

 % Loaction for creating the figure: /eos/user/n/natriant/2018/CC_MD_2018_summary/measured_psd
 \begin{figure}[!ht]
   \centering
   \begin{subfigure}[t]{0.45\textwidth}
       \centering
       \includegraphics[width=1\textwidth]{images/Ch5/Measured_spectrum_MD5_Coast1-Setting2-AN.csv_no_psd.png}
       %\caption{$y=\sin(2 \pi f t),\ f=50$ Hz}
       %\label{fig:add_label_here}
   \end{subfigure}
   \hfill
   \begin{subfigure}[t]{0.45\textwidth}
       \centering
       \includegraphics[width=1\textwidth]{images/Ch5/Measured_spectrum_MD5_Coast1-Setting2-PN.csv_no_psd.png}
       %\caption{Discrete Fourier transform}
       %\label{fig:add_label_here}
   \end{subfigure}
   \hfill
    \caption{Example amplitude (left) and phase (right) noise spectra measured with a spectrum analyzer E5052B~\cite{E5052B_insight} during the emittance growth studies with CCs in SPS. The noise spreads-out up to 10\,kHz (grey dashed line) overlapping the first betatron sideband at $\sim$8\,kHz (green dashed line). The spikes at high frequencies correspond to the harmonics of the revolution frequency and are a result of the bunch crossing.} % bunch passage
    \label{fig:example_PN_and_AN_coast1_setting2}
\end{figure}

\subsubsection*{PSD values of interest}
As already discussed in Chapter~\ref{Ch:CC_noise_theory} the noise induced emittance growth depends on the noise power at the beteatron and synchrobetatron sidebands for the phase and amplitude noise respectively (see Eq.~\eqref{eq:dey_pn} and Eq.~\eqref{eq:dey_an}). Therefore, the noise power values of interest for this thesis are the ones at the first betatron $f_b = 0.18 \times \frev$ = 7.82\,kHz and at the synchrobetatron sidebands at $f_b \pm \Qs \times \frev  = f_b \pm  \sim 220$\,kHz. 


However, it can be clearly seen from Fig.~\ref{fig:example_PN_and_AN_coast1_setting2} that the measured noise spectra are noisy: random changes in amplitude are observed from point to point within the signal. To this end, the PSD value at the first betatron sideband, $f_b$, is determined as the average of the PSD values over a frequency range of $\pm$ 500\,Hz around it, while its uncertainty is considered to be the standard deviation over that range. In the following, it is assumed for simplicity that the PSD at the synchrobetatron sidebands equals the PSD at the first betaton sideband as they lie very close to each other. At this point, it should be mentioned that the validity of these assumptions was tested with numerical simulations which used the measured spectra (Chapter~\ref{Ch:investigating_discrepancy}).

The emittance growth measurements were performed with seven different noise levels. The values of the phase and amplitude noise for each setting are are listed in Table~\ref{tab:noise_settings_2018}. 

\subsubsection*{Effective phase noise}
In order to make a meaningful comparison between the different levels of noise, the concept of effective phase noise is introduced: this is the phase noise level that would lead to the same emittance growth as that from both phase and amplitude noise according to the theoretical model (Chapter~\ref{Ch:CC_noise_theory}). 

For the calulcation of the effective phase noise the averaged bunch length for each case is used (bunch length measurements at Section~\ref{sec:bunch_length_intensity_meas_2018}). The uncertainty on the effective phase noise is computed following the standard procedure of the propagation of the uncertainty (Appendix~\ref{app:uncertainty_propagation}). \textcolor{blue}{Do I need to show all the calulcations?.} The calculated effective phase noise values for the experimental conditions are also listed in Table~\ref{tab:noise_settings_2018}. The shown values correspond to the results using the parameters of the first bunch. However, the difference between the values for the other bunches is very small and is also within the displayed uncertainties.The noise levels mentioned in the following analysis of the experimental data correspond to the calculated effective phase noise. 
% Script for computing the values: /eos/user/n/natriant/2018/CC_MD_2018_summary/measured_psd/job001_cmpt_psd_at_fb.ipynb
% The resulted pickle file can be found at: /eos/user/n/natriant/2018/CC_MD_2018_summary/measured_psd/output
% Effective phase noise: /eos/user/n/natriant/2018/CC_MD_2018_summary/measured_psd/job002_cmpt_effective_PN.ipynb
% Analytical calculations of the effective phase noise and its uncertainty: /eos/user/n/natriant/2018/CC_MD_2018_summary/measured_psd/notes_on_computing_effective_pn_psd_and_uncertainties.pdf

\begin{table}[!hbt]
	\centering
   \caption{Phase and amplitude noise levels injected in the CC RF system for the emittance growth studies of 2018. The listed values correspond to the average PSD values over a frequency range of $\pm$ 500\,Hz around the first betatron sideband, $f_b$. The calculated effective phase noise for the parameters of the first bunch are also listed.}
	\begin{tabu} to \textwidth { X[c,m] X[c,m] X[c,m] X[c,m]}
		&&& \\[-6mm]
		\toprule \toprule
		\multicolumn{1}{l}{} &
		\multicolumn{3}{c}{$\mathbf{10\,\boldsymbol{\log}_{10} \mathcal{L}(f)}$ \textbf{[dBc/Hz]}} \\
		\bottomrule
      \multicolumn{1}{l}{} & 	\multicolumn{1}{c}{\textbf{Phase noise}} & \multicolumn{1}{c}{\textbf{Amplitude noise}} & 	\multicolumn{1}{c}{\textbf{Effective phase noise}} \\
      \midrule
      \multicolumn{1}{l}{Coast1-Setting1}  & \multicolumn{1}{c}{-122.6 $\pm 0.6$} & \multicolumn{1}{c}{-128.1 $\pm$ 0.6} & \multicolumn{1}{c}{-121.8 $\pm$ 0.5} \\
      
      \multicolumn{1}{l}{Coast1-Setting2}  & \multicolumn{1}{c}{-101.4 $\pm$ 0.8} & \multicolumn{1}{c}{-115.2 $\pm$ 0.6} & \multicolumn{1}{c}{-101.3 $\pm$ 0.8} \\

      \multicolumn{1}{l}{Coast2-Setting1}  & \multicolumn{1}{c}{-115.0 $\pm$ 0.8} & \multicolumn{1}{c}{-124.1 $\pm$ 0.5} & \multicolumn{1}{c}{-114.6 $\pm$ 0.7} \\

      \multicolumn{1}{l}{Coast2-Setting2}  & \multicolumn{1}{c}{-111.4 $\pm$ 0.6} & \multicolumn{1}{c}{-115.7 $\pm$ 0.4} & \multicolumn{1}{c}{-110.2 $\pm$ 0.5} \\ 

      \multicolumn{1}{l}{Coast3-Setting1}  & \multicolumn{1}{c}{-110.9 $\pm$ 0.9} & \multicolumn{1}{c}{-116.9 $\pm$ 0.4} & \multicolumn{1}{c}{-110.1 $\pm$ 0.8} \\

      \multicolumn{1}{l}{Coast3-Setting2} & \multicolumn{1}{c}{-106.4 $\pm$ 0.3} & \multicolumn{1}{c}{-112.9 $\pm$ 0.6} & \multicolumn{1}{c}{-105.8 $\pm$ 0.3}\\

      \multicolumn{1}{l}{Coast3-Setting3} & \multicolumn{1}{c}{-101.4 $\pm$ 0.7}  & \multicolumn{1}{c}{-106.9 $\pm$ 0.5} & \multicolumn{1}{c}{-100.6 $\pm$ 0.6} \\
      \arrayrulecolor{black}\bottomrule
	\end{tabu}
   \label{tab:noise_settings_2018}
\end{table}


\section{Emittance growth measurments}\label{sec:emit_growth_meas_2018}
This section presents the transverse emittance growth measurments with $\CC$ RF noise. It discusses first the measurement of the beam emittance with the SPS Wire Scanners (WS) and then it provides an overview of the emittance growth measurements for the four bunches over all the different noise settings.

\subsection{SPS Wire Scanners}\label{subsec:sps_ws}
The SPS is equipped with Wire Scanners (WS) to measure the transverse beam emittance. The SPS WS system is described in detail in Ref.~\cite{BOSSER1985475, Berrig:1972478}. For the SPS tests, the emittance was measured with WS both for the horizontal and vertical plane (BWS.51995.H and BWS.41677.V respectively).

The working principle is shown in Fig.~\ref{fig:SPS_WS_ROT}. A thin wire rapidly moves across the proton beam and a shower of secondary particles is generated. The signal from the secondary particles is then detected by a system of scintillator and photomultiplier (PM) detectors outside of the beam pipe. By measuring the PM current as a function of wire position over multiple turns the transverse beam profile is reconstructed. An example of a vertical profile is shown in Fig.~\ref{fig:WS_example_V_profile}.

\begin{figure}[!h]
   \centering         
   \includegraphics[width=0.8\textwidth]{images/Ch5/Wire_scanner.png}
       \caption{Sketch of the SPS rotational wire scanners~\cite{Berrig:1972478}. The wire moves across the proton beam generating secondary particles which are then detecting by a scintillator and a photomultiplier. From the measured photomultiplier current the beam profile is reconstructed.}
       \label{fig:SPS_WS_ROT}
\end{figure}
%\normalsize{\textbf{Fitting of transverse profiles}}\\
\subsubsection*{Fitting of transverse profiles}
Assuming gaussian beams and for $u=x,y$ being the index that respectively corresponds to the horizontal and vertical plane, the rms beam size, $\sigma_u$, is obtained following the standard procedure of least squares fitting (see Appendix~\ref{app:non_linear_fitting}). In particular, the measured beam profiles from each scan are fitted with the following four-parameter ($A, k, \mu, \sigma_u$) gaussian function:
\begin{equation}\label{eq:4p_gauss}
   f(x) = k + A e^{-\frac{(x-\mu)^2}{2 \sigma_u^2}},
\end{equation}
where $k$ is the signal offset of the PM, $A$ is the signal amplitude, $\mu$ is the mean of the Gaussian distribution and $\sigma_u$ its standard deviation. The uncertainty of the measured rms beam size, $\Delta \sigma_u$, is defined as the error of the fit of the $\sigma_u$ parameter (see Appendix~\ref{app:non_linear_fitting}).

%T A non-linear least square minimization is used to fit the gaussian function to the measured data and obtain the optimal values for the parameters ($A, k, \mu, \sigma_u$).he standard error of the parameters' estimates is given by the square root of the diagonal of their covariance matrix.~\cite{gaus_fit_least_squares}. The uncertainty of the measured beam size, $\Delta \sigma$, is defined as the standard error of the $\sigma$ parameter. The optimal parameters' values and their covariance matrix are computed here using the $\mathrm{scipy.curve \_ fit}$ \cite{scipy_curve_fit} function of Python programming language. 
% Curve fitting explanation: https://www.youtube.com/watch?v=Jl-Ye38qkRc&ab_channel=BrantCarlson

An example of the beam profile measured from the SPS WS at a specific time is shown in Fig.~\ref{fig:WS_example_V_profile} (light blue dots) along with the gaussian fit (orange line).
% Location where the figure was produced: /eos/user/n/natriant/2020/WS_analysis
\begin{figure}[!h]
   \centering         
   \includegraphics[width=0.6\textwidth]{images/Ch5/SPS.BWS.41677.V_ROT_2018-09-05 15_45_01.33500_raw_and_fit.png}
       \caption{Vertical beam profile obtained from the BWS.41677.V instrument. The measured data points (light blue) are fitted with a four parameter gaussian (orange) to obtain the beam size. The calculated emittance and its uncertainty are also shown.}
       \label{fig:WS_example_V_profile}
\end{figure}
   
%\normalsize{\textbf{Computing the normalised beam emittance}}\\
\subsubsection*{Computing the normalised beam emittance}
The formula for computing the normalised beam emittance from the beam size, $\sigma_u$ is given by:
\begin{equation}\label{eq:emittance_from_WS}
   \centering
   \epsilon_u = \frac{\sigma_u^2}{\beta_{u, WS}} \betarel \gammarel ,
\end{equation}
where $\sigma_u$ is the rms beam size, $\beta_{u, WS}$ the beta function at the WS location and $\betarel, \gammarel$ the relativistic parameters. Note that $u=x,y$ is the index that respectively corresponds to the horizontal and vertical plan.

Assuming that the relativistic parameters are free of error, the uncertainty of the computed emittance, $\Delta \epsilon_u$, depends on the uncertainty of the measured beam size, $\Delta \sigma_u$ and of the beta function at the location of the WS, $\Delta \beta_{u, WS}$, as follows:
\begin{equation}\label{eq:emittance_from_WS_uncertainty}
   \centering
   \Delta \epsilon_u = \sqrt{\left ( \frac{\partial \epsilon_u}{\partial \sigma_u}\right )^2 \Delta \sigma_u^2 + \left ( \frac{\partial \epsilon_u}{\partial \beta_{u, WS}} \right )^2 \Delta \beta_{u, WS}^2} = \epsilon_u  \sqrt{\left ( \frac{2 \Delta \sigma_u}{\sigma_u}\right )^2 + \left ( \frac{ \Delta \beta_{u, WS}}{\beta_{u, WS}} \right )^2} .
\end{equation}
% Propagation of uncertainty in paper: /eos/user/n/natriant/Project_thesis/material/Ch4: 1st experimental campaign in SPS/propagation_of_uncertainty_emittance.pdf

For the computation of the emittane values from the $\CC$ experiment of 2018, the following points were considered. First, in the 2018 SPS operational configuration, the dispersion was small at the WSs location and thus its contribution to the beam size was considered to be negligible \footnote{The dispersion at BWS.51995.H location in 2018 was $\Dx$= -15\,mm. At 270\,GeV, the energy spread, $\delta$, is of the order of $\mathrm{10^{-4}}$. Thus, from Eq.~\eqref{eq:emit_from_beam_size} the horizontal normalised emittance from the dispersion is expected at the order of $\mathrm{10^{-6} \ \mu m}$. Comparing to the observed beam size during the CC tests of a few microns the dispersion is negligible. \color{red}{The measured $\Dx, \Dy$ were found to be very small and thus their contribution is also considered negligible. The plan is to perform some measurments in 2022 to get a feeling of their values at the location of the wire scanners}}. Moreover, for the studies at 270\,GeV beam energy, $\betarel \gammarel$ equals 287.8 and the beta functions were 81.5\,m and 62.96\,m at the locations of the horizontal and vertical WS respectively. Last, the uncertainty on the beta functions at the location of the WS, $\Delta \beta_{u, WS}$, is 5$\%$ in both planes, which represents the rms beta-beating in the SPS~\cite{SPS-beta-beating-Rogelio}.
% Computations from dispersive contribution: /eos/user/n/natriant/Project_thesis/material/Ch4: 1st experimental campaign in SPS/dispersive_contribution_in_the_emittance.pdf
\subsubsection*{Further considerations}
It is worth noting here that during each measurement with the WS the beam profile is actually acquired twice as the wire crosses the beam in the forward direction (IN scan) and then in the reverse direction (OUT scan). For the 2018 measurements the emittance values obtained from IN and OUT scans, $\epsilon_\mathrm{IN} \pm \Delta \epsilon_\mathrm{IN}$ and $\epsilon_\mathrm{OUT} \pm \Delta \epsilon_\mathrm{OUT}$, were found to be very similar. In the analysis of the 2018 measurements, the average emittance from the two scans, $\epsilon_\mathrm{avg} = \langle \epsilon_\mathrm{IN}, \epsilon_\mathrm{OUT}\rangle$, is used. The uncertainty in the average, $\Delta \epsilon_\mathrm{avg, 1}$, is given by~\cite{uncertainty_in_the_mean}: 
\begin{equation}\label{eq:uncertainty_mean_ws}
   \Delta \epsilon_\mathrm{avg, 1} = \frac{\mid \epsilon_\mathrm{IN} - \epsilon_\mathrm{OUT} \mid}{2 \sqrt{2}}.
\end{equation}
The propagated uncertainty from the measurement errors, $\Delta \epsilon_\mathrm{IN}$ and $\Delta \epsilon_\mathrm{OUT}$, is given by:
\begin{equation}\label{eq:propagated_uncertainty_ws}
   \Delta \epsilon_\mathrm{avg, 2} = \frac{1}{2}\sqrt{ \Delta \epsilon_\mathrm{IN}^2 + \Delta \epsilon_\mathrm{OUT}^2}.
\end{equation}
Considering that $\Delta \epsilon_\mathrm{avg, 1}$ and $\Delta \epsilon_\mathrm{avg, 2}$ are independent, the combined uncertainty in the average, $\Delta \epsilon_\mathrm{avg}$, is given by:

\begin{equation}\label{eq:combined_uncertainty_ws}
   \Delta \epsilon_\mathrm{avg} = \sqrt{\Delta \epsilon_\mathrm{avg, 1} ^2 + \Delta \epsilon_\mathrm{avg, 2} ^2}.
\end{equation}

Finally, some emittance increase is expected during each wire scan, due to multiple Coulomb scattering. This effect has been extensively studied in Ref.~\cite{Roncarolo:1481835}. For the rotational SPS WS and the energy of 270\,GeV, at which the $\CC$ experiments were performed the expected emittance growth from the WS is expected to be between 0.0-0.2$\%$ per scan in both transverse planes. However, a conservative number of scans were carried out, $\sim$ 20 scans per bunch and per plane during $\sim$ 1 hour, in order to minimise the contribution from this effect.
% Table 5.1 0.0% for 450 GeV, 0.2% for 26 GeV

\subsection{Experimental results}\label{sec:MD5_overview}
In this section, an overview of the emittance growth measurements is presented. Figure~\ref{fig:MD5_overview_x_y} displays the bunch by bunch transverse emittance evolution throught the total duration of the experiment. The three different coasts are distinguished in this plot with the blue dashed vertical lines. The values of the effective phase noise are also displayed (see Table~\ref{tab:noise_settings_2018}), while the moments when the noise level changed are shown with the grey vertical lines. The four different colors (blue, orange, red, green) correspond to the four different bunches. For the bunches the notation "bunch $N$" will be used, where $N=\{1,2,3,4\}$ according to their position in the bunch train. The errorbars of the emittance values correspond to the uncertainty computed using Eq.~\ref{eq:combined_uncertainty_ws}. However, as they are very small comparing to the scale of the plots they are barely visible. Last, the emittance growth rates, $d\epsilon_u /dt$, for each setting and for each bunch are displayed at the bottom of each plot along with their uncertainties. The growth rates are obtained following the standard procedure of weighted least squares fitting (see Appendix~\ref{app:non_linear_fitting}). In particular, the measured beam profiles from each scan are fitted with the following polynomial:
% fit with numpy polyfit
\begin{equation}\label{eq:polynimial_for_linear_fit}
   p(x) = c_0 + d\epsilon_u /dt \times t
\end{equation}
where $t$ is the time in seconds, $d\epsilon_u /dt$ the growth rate in meters per second and $c_0$ the constant offset in meters. The uncertainties of the growth rates correspond to the error of the fit (see Appendix~\ref{app:non_linear_fitting}). 

% Analysis and figures: /eos/user/n/natriant/2018/CC_MD_2018_summary/emitGrowth_ws
% Merge png online free: https://products.aspose.app/pdf/merger/png-to-png
\begin{sidewaysfigure}
   \centering
   \includegraphics[width=1.0\textwidth]{images/Ch5/MD5_overview_x_y.png}
   \caption{Bunch by bunch horizontal (top) and vertical (bottom) emittance evolution during the experiment on September, 15, 2018. The four different colors indicate the different bunches. The different applied noise levels are also shown while the moments when the noise level changed are inficated with the grey vertical dashed lines. The emittance growth rates along with their uncertainties for the seven different noise settings are displayed at the legend at the bottom of the plots.}
   \label{fig:MD5_overview_x_y}
\end{sidewaysfigure}
   
\subsubsection*{First observations and comments}
Figure~\ref{fig:MD5_overview_x_y} demonstrates a clear emittance growth in the vertical plane which is expected due to the vertical $\CC$. However, the $\CC$ noise is observed to induce growth also in the horizontal emittance as a result of residual coupling in the machine. Thus, the total emittance growth given by $d\epsilon_x/dt + d\epsilon_y/dt$ should be considered in the following. That was confirmed by PyHEADTAIL simulations~\cite{Baudrenghien_HL-LHC19v6} in the presence of $\CC$ RF noise and transverse coupling. 

%Option A)Furthermore, the phase and amplitude noise levels for $Coast1-Setting1$ were found to be below noise floor of the instrument and this the data from that case will not be included in the following. % e-mail from Themis (27/April/2021. I think the noise floor was -120 dBc/Hz. Not sure. If that sentence is not enough we need to check

Furthermore, both the phase and amplitude noise levels for Coast$1$-Setting$1$ were found to be below noise floor of the instrument. Therefore, the transverse emittance growth observed during that case is a result of other sources (natural emittance growth, see Section~\ref{sec:preparatory_studies_for_2018_MD} ) and will be considered as the background growth rate in the analysis below. \textcolor{blue}{Is this apporach ok? instead of using 0.45 and 0.55 um/h for the y and x planes respectively.}

\subsubsection*{Summary plot}
Figure~\ref{fig:MD5_summary_plot} provides a clearer view of the measruments presented in Fig.~\ref{fig:MD5_overview_x_y}. It displays the measured emittance growth rates for each one of the four bunches for the different levels of injected noise. The horizontal error bars correspond to the uncertainty of the effective phase noise (see Section~\ref{sec:injected_RF_noise}) while the veritcal ones correspond to the uncertainty of the total transverse emittance growth calculated from the uncertainties of the horizontal and vertical growth rates following the standard procedure of the propagation of the uncertainty (Appendix~\ref{app:uncertainty_propagation}).

\begin{figure}[!h]
   \centering         
   \includegraphics[width=0.6\textwidth]{images/Ch5/MD5_summary_plot_no_backg_subtraction.png}
       \caption{Summary plot of the emittance growth study with CC noise in 2018. The transverse emittance growth rate, for the four bunches, is shown as a function of the different levels of applied noise.}
       \label{fig:MD5_summary_plot}
\end{figure}

From the plot it becomes clear that the measured emittance growth was different for the four different bunches. Furthermore, the frist bunch (blue) had systematically the smallest growth rate.

An attempt to understand these observations will be made in the following section by investigating a possible correlation between the transverse emittance growth and the beam evolution in the longitudinal plane.

\section{Bunch length and intensity measurements}\label{sec:bunch_length_intensity_meas_2018}
The measurements of the bunch length and intensity that took place in parallel with the emittance growth measurements are presented in this section. The goal is to get a more complete insight of the experimental conditions and possibly explain the different emittance growth rates observed for the four bunches which was discussed in the previous section. Initially, a short introduction on the instrumenst used for the measurements is provided. After that, the evolution of the longitudinal plane and of the intensity is analaysed and discussed.

\subsection{ABWLM and Wall Current Monitor}\label{sec:ABWLM_WallCurrentMonitor}
The bunch length was measured with two different instruments the ABWLM (A for RF, B for Beam, W for Wideband, L for Longitudinal, M for Measurement)~\cite{ABWLM} and the Wall Current Monitor~\cite{Papotti:1124099}. Both ABWLM and Wall Current Monitor acquire the longitudinal bunch profiles, while ABWLM is much faster than the Wall Current Monitor. %I concluded this from the data. If I need to give specific values I will remove it.
In the ABWLM case the bunch length is obtained by performing a gaussian fit on the acquired profiles. Only the calculated bunch length values are available but not the profiles themselves. For the case of the Wall Current Monitor the bunch length is estimated by computing the full width half maximum of the profiles and then using it to estimate the sigma of a gaussian distribution. The longitudinal profiles and the calculated bunch lengths are available for each acquisition. Furthermore, the Wall Current Monitor provides additional information on the relative bunch position with respect to the center of the RF bucket, which will also be used in tha following analysis. No further details on the operation of these instruments are discussed here as the offline analysis was not performed by the author. % bunch position not very clear.
% longitudinal profiles for the ABLWM don't exist.


\textcolor{blue}{How is the intensity calculated?}

\subsection{Bunch length measurements}\label{subsec:bunch_length_meas_2018}
The bunch length measurements that took place during the $\CC$ noise induced emittance growth studies are shown in the bottom plot in Fig.~\ref{fig:MD5_overview_x_y_sigma_t}. The small markers correspond to the data acquried with the ABWLM while the bigger markers to the data acquired with the Wall Current Monitor. The two upper plots contain the transverse emittance growth as discussed in Section~\ref{sec:MD5_overview}. This is for easier comparison of the beam evolution in the transverse and in the longitudinal plane. The color code corresponds to the four different bunches.

% Script for plotting: /eos/user/n/natriant/2018/CC_MD_2018_summary/ABWLM/job001_abwlm_bunch_length_analysis_plot_overview.ipynb
% Merging with https://photo333.com/merge-png.php
\begin{sidewaysfigure}
   \centering
   \includegraphics[width=1.0\textwidth]{images/Ch5/MD5_overview_x_no_legendMD5_overview_y_no_legendMD5_overview_4sigma_t_no_title_with_wall_current_monitor.png}
   \caption{Evolution of the beam in transverse and longitudinal planes during the CC noise induced emittance growth experiment. Top: Horizontal emittance growth measured with the SPS WS. Middle: Vertical emittance growt measured with the SPS WS. Bottom: Bunch length evolution measured with the ABWLM (small markers) and the Wall Current Monitor (bigger markers).}
   \label{fig:MD5_overview_x_y_sigma_t}
\end{sidewaysfigure}
% dont mention about the scaling with the wall current monitor, subtracted offset: -0.18 ns.
% UPDATE 24March:20222. The scaling factor is actually 1/1.11 but I am not gonna correct it now. Is more or less the same. However, if you update the plot again, fix it. 

Four main observations can be made. First, the plot demonstrates a very good agreement between the ABLWM and the Wall Current monitor. Second, an approximately bunch length increase of $\sim$9\,$\%$/h is observed for bunch 1 (blue) in all the three coasts. This rate, which is computed from the ABWLM data, is similar to the blow-up observed in the SPS for similar machine conditions~\cite{Alekou_CC_coast_prep_2016}. Third, the bunch length increase for the last three bunches (2, 3, and 4) is larger compared to bunch 1. However, bunches 2, 3, and 4 seem to be longitudinally unstable as sudden jumps appear in their bunch length evolution and this could explain the faster bunch length increase. Last, no correlation is observed between the bunch length evolution and the change of noise level. In order to validate that bunches 2, 3, and 4 are unstable ,the longitudinal profiles acquired with the Wall Current Monitor are studied in the next paragraph. 
%cmpt bunch length increase for bunch 1 in units of %/h.

\subsection{Longitunal profile measurements}\label{subsec:long_profiles_meas_2018}
Two example longitudinal profile acquisitions from the Wall Current Monitor are discussed here as they can provide further insight on the sudden jumps observed in the bunch length values for bunches 2, 3, and 4. The selected acquisitions correspond to the moments where the sudden jumps are performed in the second and third coast and are shown in Fig~\ref{fig:long_profiles_and_rel_bunch_position_2018}. The relative bunch position with respect to the center of the RF bucket of each bunch for an acquisition period of 7\,ms is also illustrated in the bottom plots of Fig.~\ref{fig:long_profiles_and_rel_bunch_position_2018} for completeness. 


\begin{figure}[!ht]
   \centering
   \begin{subfigure}[t]{0.42\textwidth}
       \centering
       \includegraphics[width=1\textwidth]{./images/Ch5/bunchProfiles_MD_115.png}
       %\caption{Phase noise spectrum measured with a spectrum analyzer E5052B, in units dBc/Hz.}
       %\label{fig:coast1_setting2_a}
   \end{subfigure}
   \hfill
   \begin{subfigure}[t]{0.42\textwidth}
       \centering
       \includegraphics[width=1 \textwidth]{./images/Ch5/bunchProfiles_MD_126.png}
       %\caption{Measured phase noise spectrum in  units rad$^2$/Hz.}
       %\label{fig:coast1_setting2_b}
   \end{subfigure}
   \hfill
   \begin{subfigure}[t]{0.42\textwidth}
       \centering
       \includegraphics[width=1\textwidth]{./images/Ch5/bunchPosition_MD_115.png}
       %\caption{Linear interpolation of the measured noise spectrum. %sampled every $\Delta f = f_{rev}/N$. Here $f_{rev}$=43.45 [kHz] and N=$10^5$ turns are used.}
       
       %\label{fig:coast1_setting2_c}
   \end{subfigure}
   \hfill
   \begin{subfigure}[t]{0.42\textwidth}
       \centering
       \includegraphics[width=1\textwidth]{./images/Ch5/bunchPosition_MD_126.png}
       %\caption{Positive spectral components of the two-sided power spectrum $S_\phi$.}
       %\label{fig:coast1_setting2_d}
   \end{subfigure}
   \hfill
   \caption{Longitudinal profiles (top) and relative bunch position with respect to the center of the RF bucket (bottom) acquired with the Wall Current Monitor. The acquisitions correspond to the times when the sudden jumps in the bunch length evolution are observed (see Fig.~\ref{fig:MD5_overview_x_y_sigma_t}).}
   \label{fig:long_profiles_and_rel_bunch_position_2018}
\end{figure}
% Each acquisition, last 7 ms? And then we compute the average? Is that right?
% The profiles, are averaged over a period of 7 ms. More details here: https://docs.google.com/presentation/d/1wqgJK7uDFDGoX8gyx2-JhnO8mZT51N1VH1RtNL-jhWA/edit?usp=sharing
% Hannes: The relative bunch position is with respect to the center of the bucket of each bunch, as obtained from the revolution frequency clock I would say. The revolution frequency clock is the system that keeps track of the revolution frequency of the beam

From Fig.~\ref{fig:MD5_overview_x_y_sigma_t}, it becomes clear that bunches 2,3, and 4 (orange, green, and red) are longitudinally unstable. This is believed to be due to the
fact that the phase loop was sampling only the first bunch
because of the large bunch spacing of 525\,ns~\cite{Argyropoulos_unstable_bunches_2018}. For this
reason, the following analysis is focused only on bunch 1,
which was not affected by the instability. However, in the next paragraph the intensity measurements for all the four bunches are exceptionally illustrated.

\subsection{Intensity measurements}\label{subsec:intensity_2018_meas}
The bunch by bunch intensity measurements that were performed along the experiment with artificial $\CC$ noise are displayed in Fig.~\ref{fig:MD5_overview_intensity}. In particular the intensity values normalised with the intial value are shown for each bunch. The four different bunches are indicated with the four different colors. The acquisitions from both the ABWLM and the Wall Current Monitor are illustrated with the small and bigger markers respectively.

The following observations can be made. First, there is very good agreement between the measruments from the ABLWM and the Wall Current Monitor. Second, losses of $\sim$2-4\,$\%$/h, computed from the ABLWM acquisitions, are observed for bunch 1 (blue) in all the three coasts. This rate is even smaller of what was observed in the SPS in coast studies without external noise ($\sim$10\,$\%$/h)~\cite{Alekou_CC_coast_prep_2016}. Last, more significant losses are observed for the longitudinally unstable bunches (bunch 2,3, and 4). However, this is not of concern as the last three bunches will not be included in the following analysis as discussed in the previous paragraph (~\ref{subsec:long_profiles_meas_2018}).

\begin{sidewaysfigure}
   \centering
   \includegraphics[width=1.0\textwidth]{images/Ch5/MD5_overview_intensity_with_wall_current_monitor.png}
   \caption{Intensity evolution as measured with ABLWM (smaller markers) and with the Wall Current Monitor (bigger markers) during the experiment with CC noise in 2018.}
   \label{fig:MD5_overview_intensity}
\end{sidewaysfigure}
% Scripts: /eos/user/n/natriant/2018/CC_MD_2018_summary/ABWLM/job004_abwlm_bunch_intensity_analysis_plot_overview_with_wall_current_monitor.ipynb
% ABWLM data: /eos/user/n/natriant/2018/CC_MD_2018_summary/ABWLM/output/abwlm_bunch_intensity.pkl
% Wall Current Monitor data: /eos/user/n/natriant/2018/CC_MD_2018_summary/Wall_Current_Monitor/pickled_data_from_2018

\section{Comparison of measured transverse emittance growth with the theoretical predictions}\label{sec:meas_2018_vs_theory}
This section, focuses on the main objective of the experiment which was the comparison of the measured transverse emittance growth with the expected values as computed from the theoretical model discussed in Chapter~\ref{Ch:CC_noise_theory}. As already discussed (Section~\ref{subsec:long_profiles_meas_2018}), the comparison considers only bunch 1 as the other three bunches were found to be longitudinally unstable.

Figure~\ref{fig:MD5_bunch1_theory_vs_meas} compares the the measured (blue) and the theoretically calculated (black) emittance growth rates of bunch 1 for the different noise levels. For the comparison, the background growth rate from other sources (measured during Coast$1$-Setting$1$, as discussed in Section~\ref{sec:emit_growth_meas_2018}) is subtracted from the measured values. In particular the background growth was measured 0.6\,$\mathrm{\mu m}$/h and 0.44\,$\mathrm{\mu m}$/h for the horizontal and veritcal plane respectively.

The expected emittance growth due to $\CC$ noise was estimate for all noise settings using Eq.~\eqref{eq:dey_pn}. The growth was computed for the beam energy of 270\,GeV, considering the veritcal beta function at the location of the $\CC$2 of 73.82\,m and the revolution frequency of SPS which is 43.37\,kHz. For each setting, the measured noise PSDs (i.e. effective phase noise) and the average bunch length over each observation window were used in the calculation. These values are listed in the first two columns of Table .. . 

The horizontal error bars, for both measured and calculated growths, correspond to the uncertainty of the effective phase noise values (see Table~\ref{tab:noise_settings_2018}). The vertical error bars for the measured growth are defined as the error of the linear fit on the emittance values (see Section~\ref{sec:emit_growth_meas_2018}). The veritcal error bars on the theoretically calculated rates are computed following the standard procedure of propagation of the uncertainty. It should be mentioned here that only the uncertainties on the effective phase noise ($\sim$ 13\% on average for bunch 1) are included in the error propagation. The beam energy and the revolution frequency are assumed to be free of error, while the uncertainties of the rest of the parameters: bunch length, $\CC$ votlage and beta function ($\sim 2 \%, \ 0.01 \%$, and $5 \%$ respectively) are not included as they are much smaller than those of the noise.


% 1. Compute % uncertainty on the bunch length: /eos/user/n/natriant/2018/CC_MD_2018_summary/ABWLM/job0006_cmpt_averaged_bunch_length_and_uncertainty_for_each_setting_save2pickle.ipynb
% 2. Uncertainty on the beta function is 5% from beta beating. 
% 3. Compute uncertainty on the PSD --> in rad^2/Hz NOT in dBc/Hz as this is the form that it enters the Eq.(3.2).
% 4. Uncertainty on the V_CC is 0.01/0.98*100 = 0.01%
% 5. Compute the theoretically growth rates: /eos/user/n/natriant/2018/CC_MD_2018_summary/cmpt_emit_growth_theoretical_model/job001_cmpt_theoretical_emit_growth_MD5_2018.ipynb
% Compute the unertainty on the theoretical growth rates: /eos/user/n/natriant/2018/CC_MD_2018_summary/cmpt_emit_growth_theoretical_model/job003_cmpt_uncertainty_on_theoretical_emit_grwoth_MD_2018.ipynb and the computation is done accordind to the notes in the measured_psd directory. In general the procedure is the same with computing the uncertainty of the measured effective noise.


\begin{figure}[!h]
   \centering         
   \includegraphics[width=0.6\textwidth]{images/Ch5/MD5_summary_bunch1_backg_subtracted_vs_theory.png}
       \caption{Summary plot of the emittance growth study with CC noise in 2018 focused on bunch 1 only. The measured emittance growth rate (blue) and the expected growths from the theoretical model (black) are shown as a function of the different levels of applied noise.}
       \label{fig:MD5_bunch1_theory_vs_meas}
\end{figure}


From Fig.~\ref{fig:MD5_bunch1_theory_vs_meas} it is evident that the theory systematically overestimates the measured growth rates. The averaged discrepancy over all noise levels is a factor of 4: numerical values are given in Table ..




- table with psds for bunch 1 (effective phase noise), average bunch length, measured growth vs calculated (like in IPAC).

- subtract backgorund growth rate (coast1 setting 1). mention values for bunch 1


\section{Conclusions and outlook}\label{sec:MD2018_summary}





% beam position relative to monitor
