In 2018, two prototype Crab Cavities ($\CC$s) were installed in the SPS to be tested for the first time with proton beams. One of the operational issues that needed to be addressed concerned the expected emittance growth due to noise in their RF control system. A theoretical model that describes this emittance growth had already been developed and validated by tracking simulations~\cite{PhysRevSTAB.18.101001}. Based on those studies a dedicated experiment was performed to benchmark the models with experimental data and to confirm the analytical predictions. In particular, the idea was to inject various noise levels in the $\CC$ RF system and record the emittance evolution. In this chapter, the experimental procedure, the measurement methods and results are presented and discussed.
 
The chapter is stractured as follows: Section~\ref{sec:CC_SPS_setup} describes the operational setup for the SPS $\CC$ tests and discusses the main diagnostic deployed for the derivation of the $\CC$ voltage.

blah blah ... describe sections and subsections after they are completed.

blah blah ... describe sections and subsections after they are completed.

blah blah ... describe sections and subsections after they are completed.

\section{Crab Cavities in the SPS}

For the SPS tests two prototype $\CC$s of the Double Quorter Wave (DQW) type were fabricated by CERN and were assembled into the same cryomodule~\cite{Zanoni:2017}. The cryomodule was installed in the SPS-LSS6 zone and was placed on a mobile transfer table~\cite{Garlaschè:2648553}. The table moved with high precision and without breaking the vaccum the cryomodule in the beam line for the $\CC$ tests and out of it for the usual SPS operation. For the emittance growth measurements only one of these $\CC$s was used and its main optics and desgin parameters are listed in Table~\ref{tab:SPS_CCs}. 