\section{Motivation}\label{sec:motivation_md_2022}
As discussed in the previous chapter, PyHEADTAIL simulations including the SPS impedance model suggest that the beam coupling impedance leads to an effective suppression of the $\CC$ RF phase noise induced emittance growth through the separation of the coherent tune from the incoherent spectrum. This suppression, which is related to the coherent (dipole) motion, can reach up to a factor of 4-5 for the experimental conditions of the first experimental campaign with $\CC$s that took place in the SPS in 2018, which seems to be the explanation for the experimental observations (see Section~\ref{sec:MD5_overview}).

This suppression effect has never been observed before. To this end, another experimental campaign took place in the SPS in 2022 where the main objective was to validate experimentally the above-mentioned suggested emittance growth suppression mechanism. If successful, it would constitute the first experimental investigation and validation of this effect. Moreover, achieving a good understanding of the 2018 results is essential for developing confidence in the theoretical model and its predictions for the HL-LHC.

The experimental campaign of 2022, was organised in two proof-of-concept experiments. The first experiment was carried out in the presence of phase noise in the $\CC$ RF system. The second experiment took place with a pure dipolar noise source: the beam transverse damper. This chapter reports on the preparation, the methodology, and the results of these experiments.

\section{Experiment with CC as noise source}\label{sec:cc_md_2022}
Due to the preceding PyHEADTAIL simulations which provided strong evidence that the observed discrepancy between the 2018 measurements and the theoretical predictions could be explained by the beam transverse impedance additional machine time was dedicated to emittance growth studies with $\CC$ in the SPS in 2022. The time allocated for the $\CC$ experiment was limited to about 10 hours since many different studies have to take place in the SPS during the year. Taking into consideration the time needed for the setup and the $\CC$ calibration the time available for the emittance growth measurements is reduced even more. To this end, measurement repeatability is limited and the expereimental procedure had to be carefully planed in advance.
% Also you can add that each point in the emittance growth studies needs about 30-40 min.

\subsection{Experiment preparation and procedure}

Nevertheless, a number of steps which are described below were followed.


\subsection{Calibration of CC voltage and phase offset}
\subsection{Scaling of emittance growth with noise power}
\subsection{Sensitivity to amplitude-dependent tune shift} 

\section{Experiment with beam damper as noise source}


\newpage

\section{Strategy and preparational studies}\label{sec:strategy_md_2022}
% MD planning: https://docs.google.com/presentation/d/13b4KAywwANL_hk0ttAmxjW8P0zLzrJpNUKiEGYtgD3w/edit#slide=id.p 
The experiment of 2022, was consisted of emittance growth measurements at 270\,GeV in coast mode, as in 2018. The available machine time ($\sim$ 10 hours) for the $\CC$ experiment of 2022 was split in two parts. 

For the first part, the objective was to measure the emittance growth with the same noise levels and conditions as in 2018 in order a) to reproduce the observed scaling of emittance growth (see Fig.~\ref{fig:MD5_summary_plot}) and b) to benchmark the expected suppression factor from PyHEADTAIL simulations  with the impedance model.

The objective of the second part was to investigate the effects of impedance and amplitude detuning on the emittance growth from $\CC$ phase noise. The preceding analysis of the PyHEADTAIL simulations revealed a significant sensitivity of the emittance growth suppression on amplitude-dependent tune shift (e.g. Fig.~\ref{fig:MD_2018_impedance_simulations}). This behavior can be tested experimentally in the SPS with the use of the Landau octupole families, which allow to introduce controlled detuning with amplitude. A successfully reproduction of this behavior would provide the proof-of-concept for the emittance growth suppression mechanism from the beam transverse impedance.




Tonise oti to behavior einai auto pou thes na kaneis reproduce oxi ta exact numbers. des parousiasi high schooll.


In particular, the LOD family was used which affects primarly the vertical plane. The reason behind this, is that as in 2018, in the experiment of 2022 the DQW $\CC$ module was used which provides a vertical deflection of on the beam and hence the focus was put in the vertical plane.





\section{Simple model of Xavier vs experimental results}
for dipolar nosie


The Landau octupoles current might be a limtiing factor.


If you want to describe the steps for the MD see section 3.3 David amorim.