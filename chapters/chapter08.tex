\section{Motivation}\label{sec:motivation_md_2022}
As discussed in the previous chapter, PyHEADTAIL simulations including the SPS impedance model suggest that the beam coupling impedance leads to an effective suppression of the $\CC$ RF phase noise induced emittance growth through the separation of the coherent tune from the incoherent spectrum. This suppression can reach up to a factor of 4-5 for the expereimental conditions of the first experimental campaign with $\CC$s that took place in the SPS in 2018, which seems to explain the experimental observations (see Section~\ref{sec:MD5_overview}).

To this end, another experimental campaign took place in the SPS in 2022. The main objective was to validate experimentally the above-mentioned suggested emittance growth suppression mechanism. If successfull, it would consist the first experimental investigation and validation of this effect. Moreover, achieving a good understanding of the 2018 results is essential for developing confidence in the theoretical model and its predictions for the HL-LHC.

\section{Strategy and preparational studies}\label{sec:strategy_md_2022}
% MD planning: https://docs.google.com/presentation/d/13b4KAywwANL_hk0ttAmxjW8P0zLzrJpNUKiEGYtgD3w/edit#slide=id.p 
The available machine time for the $\CC$ experiment of 2022 was split in two parts. For the first part, the objective was to measure the emittance growth with the same noise levels and conditions as in 2018 in order a) to reproduce the observed scaling of emittance growth (see Fig.~\ref{fig:MD5_summary_plot}) and b) to benchmark the expected suppression factor from PyHEADTAIL simulations  with the impedance model.

The objective of the second part was to investigate the effects of impedance and amplitude detuning on the emittance growth from $\CC$ phase noise. The preceding analysis of the PyHEADTAIL simulations revealed a significant sensitivity of the emittance growth suppression on amplitude-dependent tune shift (e.g. Fig.~\ref{fig:MD_2018_impedance_simulations}). This behavior can be tested experimentally in the SPS with the use of the Landau octupole families, which allow to introduce controlled detuning with amplitude. A successfully reproduction of this behavior would provide the proof-of-concept for the emittance growth suppression mechanism from the beam transverse impedance.





In particular, the LOD family was used which affects primarly the vertical plane. The reason behind this, is that as in 2018, in the experiment of 2022 the DQW $\CC$ module was used which provides a vertical deflection of on the beam and hence the focus was put in the vertical plane.





\section{Simple model of Xavier vs experimental results}
for dipolar nosie


The Landau octupoles current might be a limtiing factor.


If you want to describe the steps for the MD see section 3.3 David amorim.