\section{Motivation}\label{sec:motivation_md_2022}
As discussed in the previous chapter, PyHEADTAIL simulations including the SPS impedance model suggest that the beam coupling impedance leads to an effective suppression of the $\CC$ RF phase noise induced emittance growth through the separation of the coherent tune from the incoherent spectrum. This suppression, which is related to the coherent (dipole) motion, can reach up to a factor of 4-5 for the experimental conditions of the first experimental campaign with $\CC$s that took place in the SPS in 2018, which seems to be the explanation for the experimental observations (see Section~\ref{sec:MD5_overview}).

This suppression effect has never been observed before. To this end, another experimental campaign took place in the SPS in 2022 where the main objective was to validate experimentally the above-mentioned suggested emittance growth suppression mechanism. If successful, it would constitute the first experimental investigation and validation of this effect. Moreover, achieving a good understanding of the 2018 results is essential for developing confidence in the theoretical model and its predictions for the HL-LHC.

The experimental campaign of 2022, was organised in two proof-of-concept experiments. The first experiment was carried out in the presence of phase noise in the $\CC$ RF system. The second experiment took place with a pure dipolar noise source: the beam transverse damper. This chapter reports on the preparation, the methodology, and the results of these experiments.

\section{Experiment with CC as noise source}\label{sec:cc_md_2022}
Due to the preceding PyHEADTAIL simulations which provided strong evidence that the observed discrepancy between the 2018 measurements and the theoretical predictions could be explained by the beam transverse impedance additional machine time was dedicated to emittance growth studies with $\CC$ in the SPS in 2022. The time allocated for the $\CC$ experiment was limited to about 10 hours since many different studies have to take place in the SPS during the year. Taking into consideration the time needed for the setup and the $\CC$ calibration the time available for the emittance growth measurements is reduced even more. To this end, measurement repeatability is limited and the experimental procedure had to be carefully planned in advance.
% Also you can add that each point in the emittance growth studies needs about 30-40 min.

\subsection{Machine and beam configuration}\label{sec:cc_md_2022_parameters}
The emittance growth measurements in 2022 were performed in coast at 270\,GeV following the same setup as in 2018 (see Section~\ref{sec:exp_setup_2018}) and very similar machine and beam conditions. The most relevant ones are listed in Table~\ref{tab:machine_beam_param_2022}. 

The linear chromaticity was corrected to about zero units in both the horizontal and vertical planes. That was a result of miscommunication with the operator team of the SPS machine as the desired value was between 0.5 and 1.0. However, the analysis of the PyHEADTAIL simulations in Section~\ref{subsec:chroma_scan} showed that the sensitivity of the emittance growth suppression to the linear chromaticity values (for small positive values) is expected to be insignificant. 

\begin{table}[!hbt]
	\begin{minipage}{\textwidth}
      \begin{centering}
   \caption{Main machine and beam parameters for the emittance growth studies with CCs in SPS in 2022.}
	\begin{tabu} to \textwidth {X[c,m] X[0.5c,m] X[0.5c,m] X[0.01c,m]}
		&&& \\[-6mm]
		\toprule \toprule
		\multicolumn{2}{l}{\textbf{Parameter}} &
		\multicolumn{2}{c}{\textbf{Value}} \\
		\bottomrule
      \multicolumn{2}{l}{Beam energy, $\symE$} & \multicolumn{2}{c}{270\,GeV} \\
      \multicolumn{2}{l}{Main RF voltage / frequency,  $\VRF$ / $\fRF$}  & \multicolumn{2}{c}{5\,MV / 200.39\,MHz} \\ %200.3945
      \multicolumn{2}{l}{Horizontal / Vertical betatron tune, $\Qx$ / $\Qy$}  & \multicolumn{2}{c}{26.13 / 26.18} \\
      \multicolumn{2}{l}{Horizontal / Vertical first order chromaticity, $\Qpx$ / $\Qpy$}  & \multicolumn{2}{c}{ $\sim$ 0.0 / $\sim$ 0.0} \\
      \multicolumn{2}{l}{Synchrotron tune, $\Qs$}  & \multicolumn{2}{c}{0.0051} \\
      \multicolumn{2}{l}{$\CC 1$ voltage / frequency, $\VCC$ / $\fCC$}  & \multicolumn{2}{c}{1\,MV / 400.78\,MHz} \\
      \multicolumn{2}{l}{Number of protons per bunch, $\Nb$} & \multicolumn{2}{c}{3 $\times 10^{10}$ p/b$^\ast$} \\
      \multicolumn{2}{l}{Number of bunches}  & \multicolumn{2}{c}{1} \\
      \multicolumn{2}{l}{Rms bunch length, 4$\sigmat$}  & \multicolumn{2}{c}{1.83 \,ns$^\ast$}\\
      \multicolumn{2}{l}{Horizontal / Vertical normalised emittance, $\emitx$ / $\emity$}  & \multicolumn{2}{c}{2\,$\mathrm{\mu m}$ / 2\,$\mathrm{\mu m^\ast}$}\\
      \multicolumn{2}{l}{Horizontal / Vertical rms tune spread, $\Dqxrms$ / $\Dqyrms$}  & \multicolumn{2}{c}{2.02 $\times 10^{-5}$ / 2.17 $\times 10^{-5}$ $^\dagger$}\\
      \bottomrule
	\end{tabu}
   \label{tab:machine_beam_param_2022}
   \end{centering} \footnotesize{$^\ast$ The value corresponds to the requested intial value at the start of each coast. Nevertheless, the intensity drop and the bunch length increase were found to be insnificant for all the coasts. \\$^\dagger$ Here the rms betatron tune spread includes only the contribution from the detuning with amplitude present in the SPS machine. More details along with the calulcations for the listed values can be found in Appendix~\ref{app:detuning_with_amplitude}.}
   \end{minipage}
\end{table}


\textbf{CC RF noise}\\
The noise injected in the $\CC$ RF system was again a mixture of phase and amplitude noise, however the contribution of amplitude noise to the total emittance growth was found to be insignificant (about 6$\%$ \footnote{Further details on this analyisis are reported in the Appendix...}). The noise excitaion extended from DC up to 10\,kHz and thus the noise was applied on the first betatron sideband only, at $\sim$8\,kHz.
% I don't have the data to plot the spetra. Only screenshots in the logbook.

\textbf{CC1 instead of CC2}\\
In the 2022 campaign, $\CC$1 was used instead of $\CC$2 that was used in 2018. The reason behind this, is that during the phase offset scan performed for the calibration of the $\CC$ module (details on the procedure can be found in ... section in chaper 4 and  Section~\ref{subsec:cc_calibration_2022}) $\CC$2 tripped systematically. The issue is associated with the change of the RF phase but treating it would have been time consuming which was not an option due to the very limited machine time of the MD. Therfore, for the measurements in 2022 $\CC$1 was used.
% CC2 tripped: see entry of the logbook 4/5/2022, at 11:30:30 

\textbf{SPS Wire Scanners}\\
The emittance values were measured with the SPS Wire Scanners according to the procedure discussed in Section~\ref{subsec:sps_ws}. In particular, the following two devices were used for the measurements in the horizontal and vertical planes respectively: SPS.BWS.51637.H and SPS.BWS.41677.V. For both devices the data points from the second photomultiplier were used (PM2) \footnote{Each Wire Scanner device is equiped with four PMs. Each one of them provides better resolution of the amplitude signal of the secondary particles for a different regime. The choice of PM2 for the emittance growth studies in 2022 was done "online", during the experiment, by examining the obtained beam profiles.}. The beta functions at of the respective plane in their location are: 79.29\,m and  60.75\,m. 
% The values of the beta functions were obtained from MADX: https://github.com/natriant/exploring_SPS/blob/master/madx_studies/optics_new_seq_after_LS2/output/twiss_thin_elements/find_beta_functions_at_locations.ipynb

The OUT scan performed just 200\,ms after the IN scan. However, the measurements from the OUT scan appeared to have systematically larger fluctuation than in the IN scan and quite frequently there was a significant difference in the emittance values obtained from the two scans which in some cases reached up to 1\,$\mu m$. A significant effort, was done with the Wire Scanner experts during the MD trying to mitigate this effect without success due to limitations on the hardware of the current intrument \textcolor{red}{Needs to be confirmed}. Therefore, it was decided that the post-process analysis would be performed taking into account only the IN scan measurements.
% Ecident for strong difference between IN and OUT https://docs.google.com/presentation/d/1QIaQNfqVWaI8cHGGgb5eeS7c_jdUMuxLqD_poHrOtH0/edit?usp=sharing


\textbf{HT normalisation factor}\\
Measured in 2021 but should have been the same for 2022. In 2022 there was not ime to perform the calibration again.

coasts at 270 gev, cc1 more control iver cc2 preparatory studies 


\subsection{Experiment preparation and procedure}



Nevertheless, a number of steps which are described below were followed.

\subsection{Calibration of CC voltage and phase offset}\label{subsec:cc_calibration_2022}
\subsection{Results: scaling of emittance growth with noise power}
\subsection{Results: sensitivity to amplitude-dependent tune shift} 

\section{Experiment with beam damper as noise source}



\newpage

\section{Strategy and preparational studies}\label{sec:strategy_md_2022}
% MD planning: https://docs.google.com/presentation/d/13b4KAywwANL_hk0ttAmxjW8P0zLzrJpNUKiEGYtgD3w/edit#slide=id.p 
The experiment of 2022, was consisted of emittance growth measurements at 270\,GeV in coast mode, as in 2018. The available machine time ($\sim$ 10 hours) for the $\CC$ experiment of 2022 was split in two parts. 

For the first part, the objective was to measure the emittance growth with the same noise levels and conditions as in 2018 in order a) to reproduce the observed scaling of emittance growth (see Fig.~\ref{fig:MD5_summary_plot}) and b) to benchmark the expected suppression factor from PyHEADTAIL simulations  with the impedance model.

The objective of the second part was to investigate the effects of impedance and amplitude detuning on the emittance growth from $\CC$ phase noise. The preceding analysis of the PyHEADTAIL simulations revealed a significant sensitivity of the emittance growth suppression on amplitude-dependent tune shift (e.g. Fig.~\ref{fig:MD_2018_impedance_simulations}). This behavior can be tested experimentally in the SPS with the use of the Landau octupole families, which allow to introduce controlled detuning with amplitude. A successfully reproduction of this behavior would provide the proof-of-concept for the emittance growth suppression mechanism from the beam transverse impedance.




Tonise oti to behavior einai auto pou thes na kaneis reproduce oxi ta exact numbers. des parousiasi high schooll.


In particular, the LOD family was used which affects primarly the vertical plane. The reason behind this, is that as in 2018, in the experiment of 2022 the DQW $\CC$ module was used which provides a vertical deflection of on the beam and hence the focus was put in the vertical plane.





\section{Simple model of Xavier vs experimental results}
for dipolar nosie


The Landau octupoles current might be a limtiing factor.


If you want to describe the steps for the MD see section 3.3 David amorim.