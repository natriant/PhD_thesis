In Chapter~\ref{Ch:2018_analyisis} the analysis of the experiment data from the CC experiments in 2018 was presented. It was found that the available theoretical model~\cite{PhysRevSTAB.18.101001} which predicts the emittance growth in the presence of noise in the $\CC$ RF system overestimates the corresponding measurements by a factor of 4 on average. The reason behind this discrepancy needs to be understood in order to gain confidence in the predictions of the model (which is used to define the acceptable noise levels for the HL-LHC project). Therefore, this observation triggered a series of studies which are presented in this chapter.

This chapter is structured as follows: First, in Section~\ref{sec:paramteric_studies_theory} parametric studies based on the theoretical model are presented, exploring the sensitivity of the emittance growth rates to the $\CC$ voltage and rms bunch length. In the following two sections, the theory is benchmarked against two different simulation tools: PyHEADTAIL (in Section~\ref{sec:benchmark_theory_with_pyheadtail}) and Sixtracklib (in Section~\ref{sec:benchmark_theory_with_sixtracklib}). Finally, the main observations and conclusions are discussed in Section~\ref{sec:Ch6_conclusions}.


\textbf{Disclaimer:} The PyHEADTAIL and Sixtracklib simulation results that are presented in this chapter were performed at a preliminary stage of this project. In  particular they were performed before the thorough analysis of the experimental data which was described in Chapter~\ref{Ch:2018_analyisis}. To this end, the simulations were performed for beam and machine conditions similar to the ones in SPS during the tests with $\CC$s in 2018 but with some small deviations in the bunch length and the beta function at the location of the CC, the value of linear chromaticity and the synchrotron tune. However, it should be highlighted, that since the objective of these studies was to benchmark the theoretical model of T.~Mastoridis and P.~Baudrenghien against different simulation codes these small numerical deviations do not affect the quality of the study. This statement is validated by the simulation studies that will be presented in Chapter~\ref{Ch:experimental_CC_2022} which were undertaken with more rigorous parameters for the 2018 experiment.

% I think this sentence was misplaced here.
%From these plots, it becomes evident that there is an excellent agreement between the theoretically predicted vertical emittance growth and the simulation results with Sixtracklib. Ths results from Sixtracklib simulations are also in agreement with the results from PyHEADTAIL. To this end, it is concluded that taking into account the detailed optics of the SPS cannot explain the discrepancy between expected and measured emittance growth that was observed in the SPS CC tests in 2018.

\section{Parametric studies based on the theoretical model}\label{sec:paramteric_studies_theory}
% Link to presentations: https://docs.google.com/presentation/d/1wqgJK7uDFDGoX8gyx2-JhnO8mZT51N1VH1RtNL-jhWA/edit#slide=id.ga69ead34aa_0_419

The basics of the existing theoretical model which describes the emittance growth in the presence of amplitude and phase noise in $\CC$s in a storage ring have been introduced in Chapter~\ref{Ch:CC_noise_theory}. In this section, this theory is used to study the sensitivity of the noise-induced emittance growth on the $\CC$ voltage and rms bunch length. The objective of the study is to investigate if the uncertainty in the measurements of these two variables could explain the observed discrepancy of a factor $\sim$4 between measured emittance growth and the analytically predicted values (see Section~\ref{sec:meas_2018_vs_theory}).
% The study is limited in these two parameters as they are the only measured ones that could introduce some incertainty in the Eqs. of phase and amplitude noise. In reality there is also some uncertaninty on the noise levels themselves but is not treated here. Nevertheless, extended studies on the noise spectrum are presented later in this section.

The following parametric studies were performed for the experimental configuration of 2018: beam energy of 270\,GeV, vertical beta function of 73\,m (at the location of $\CC$2), and phase and amplitude noise of -111.4 and -115.7\,dBc/Hz respectively (Coast2-Setting2). The phase and amplitude noise are considered here independently, instead of the effective phase noise, due to the different dependence of the correction term (see Fig.~\ref{fig:correction_term_bunch_length}).

\subsection{Sensitivity to bunch length}\label{subsec:bunch_length_dependence}
Using Eqs.~\eqref{eq:dey_an} and~\eqref{eq:dey_pn} with the above mentioned parameters and CC voltage, $V_\mathrm{CC}$=1\,MV the normalised vertical emittance growth is computed as a function of different values of bunch length over a range from 0.001\,ns to 2.5\,ns (expressed in $4\sigma_t$). The results are illustrated in Fig.~\ref{fig:sensitivity_bunch_length_theory_bunch1}. 

A clear dependence of the vertical emittance growth on the bunch length is observed. However, there is only a strong dependence for bunch lengths within a certain range. In the regime of the measured bunch length during the $\CC$ experiment for bunch 1 ($\sim$1.6\,ns-\,2.0ns) the sensitivity to the bunch length is very small and cannot explain the factor of about 4 that was observed between measurements and theoretical predictions in SPS $\CC$ tests in 2018.

% Plotting script: Requires python 3
% https://github.com/natriant/theoreticalModel_for_emitGrowth_due2CCNoise/blob/master/cmpt_GrowthDependence_on_BunchLength_003_for_thesis.py
\begin{figure}[!h]
    \centering         
    \includegraphics[width=0.7\textwidth]{images/Ch6/dey_vs_4sigmat_Coast2-Setting2_withBunches_v2.png}
        \caption{Vertical emittance growth for different bunch length values computed using the analytical formulas Eqs.~\eqref{eq:dey_an} and~\eqref{eq:dey_pn} for the experimental configuration of 2018. The blue dot shows the average bunch length over all coasts in 2018. The blue box around it gives the upper and lower limits of its measurements.}
        \label{fig:sensitivity_bunch_length_theory_bunch1}
 \end{figure}


\subsection{Sensitivity to CC voltage}\label{subsec:bunch_length_dependence}
Here, the sensitivity of the vertical emittance growth is studied for the parameters mentioned above and bunch length of $4\sigma_t=$1.7\,ns. The vertical emittance growth is computed again analytically using Eqs.~\eqref{eq:dey_an} and~\eqref{eq:dey_pn} over a range of $\CC$ voltage values equally spaced from 0.6 to 1.3\,MV. 

Figure~\ref{fig:sensitivity_VCC_theory_bunch1} illustrates the computed vertical emittance as a function of the CC voltage. From the analysis in 2018, 
the calibration of the $\CC$ voltage (which showed~1\,MV) was not straightforward. However, from the plot it is evident that even if the actual voltage was 30$\%$ lower due to errors this would lead to just a factor of 2 lower vertical emittance growth. An error of this scale is not realistic. To this end, it is concluded that uncertainties on the beam based measurements of the CC voltage cannot explain the experimental observations of 2018, where there was a factor 4 between measured and predicted emittance growth.

% Plotting script: Requires python 3
% https://github.com/natriant/theoreticalModel_for_emitGrowth_due2CCNoise/blob/master/cmpt_GrowthDependence_on_Vcc.py
\begin{figure}[!h]
    \centering         
    \includegraphics[width=0.7\textwidth]{images/Ch6/dey_vs_Vcc_Coast2-Setting2.png}
        \caption{Vertical emittance growth for different values of CC voltage computed using the analytical formulas Eq.~\eqref{eq:dey_an} and~\eqref{eq:dey_pn} for the experimental configuration of 2018.}
        \label{fig:sensitivity_VCC_theory_bunch1}
 \end{figure}

\section{Benchmarking theory against PyHEADTAIL}\label{sec:benchmark_theory_with_pyheadtail}

% Note on HEADTAIL to PyHEADTAIL: https://www2.kek.jp/accl/legacy/seminar/file/PyHEADTAIL_PyECLOUD_2.pdf
As mentioned in Chapter~\ref{Ch:CC_noise_theory} the available analytical model which predicts the emittance growth driven by CC RF phase and amplitude noise was benchmarked against simulation results with the HEADTAIL~\cite{PhysRevSTAB.18.101001} simulation tool. In this section, the predictions of the model are benchmarked against the PyHEADTAIL simulation tool. PyHEADTAIL is the implementation of the HEADTAIL (which was written in C/C++ language) in Python, so that it can be more easily maintained and extended~\cite{pyheadtail_schenk}. Further details on the PyHEADTAIL are provided in the introductory Subsection~\ref{subsec:pyheadtail}.

The parameters used for setting up the linear transfer map, the longitudinal tracking, and the initialisation of the beam distribution are shown in Table~\ref{tab:pyheadtail_simulation_parameters_first_tests}: they are similar to the parameters in the SPS CC experiments of 2018. The accelerator ring consists of one segment, with one interaction point, where the beam receives the noise kicks from the $\CC$ every turn. In particular, at that location, the angle variable, $y^\prime$, of each particle within the bunch is updated every turn following the description of Eqs.~\eqref{eq:amplitude_noise_kick} and~\eqref{eq:phase_noise_kick} for modeling the phase and amplitude noise respectively. The vertical angle co-ordinate is updated to study the vertical emittance growth following the experiments of 2018 where the $\CC$ module that was used provided a vertical deflection to the bunches. Nevertheless, the beam dynamics are the same in the horizontal plane.

The simulations were performed for both phase and amplitude $\CC$ RF noise. The noise level was chosen to be much stronger than the levels used in the experiment of 2018 in order to observe a reasonable growth in the simulation time which was $10^5$ turns. For reference, this corresponds to about 2.5\,s in the SPS machine. Therefore, the simulations were performed for phase and amplitude noise with a power spectral density of 1.68 $\times 10^{-10} \ \mathrm{rad^2/Hz}$ or 1/Hz (for phase and amplitude noise respectively). This corresponds to a scaling factor, $A=10^{-8}$, in Eqs.~\eqref{eq:amplitude_noise_kick} and~\eqref{eq:phase_noise_kick}. 

The power spectra of the sequence of amplitude and phase noise kicks (discrete-time signal) are visualised in Fig.~\ref{fig:psd_cc_simulations_example}. The power spectral densities are computed using Eq.~\eqref{eq:Sxx_definition_discrete_normalized}.

% Plotting scripts: /eos/user/n/natriant/Project_thesis/chapter6/plot_psd_for_simulations_noise.ipynb
\begin{figure}[htp]
    \centering
    \begin{subfigure}{.45\textwidth}
        \centering
        \includegraphics[width=.95\linewidth]{images/Ch6/psd_amplitude_noise_example.png}  
    \end{subfigure}
    \begin{subfigure}{.45\textwidth}
        \centering
        \includegraphics[width=.95\linewidth]{images/Ch6/psd_phase_noise_example.png}
    \end{subfigure}
    \caption{Power sepctra of the CC amplitude (left) and phase (right) noise used in the PyHEADTAIL simulations.}
    \label{fig:psd_cc_simulations_example}
\end{figure}
% fft is performed over many random sequences to reduce the uncertanty. 

The simualtions were performed for a single bunch. The initial bunch was generated with Gaussian distributions in transvserse and longitudinal planes\footnote{The longitudinal distribution in reality is not a Gaussian but it was found that the shape of the longitudinal profile has no significant impact on the predicted emittance growth rates. These studies were performed by T.~Mastoridis and P.~Baudrenghien~\cite{Themis_philippe_personal_communication}. The simualtion studies presented in this thesis used a Gaussian longitudinal distribution following the studies presented in Ref.~\cite{PhysRevSTAB.18.101001}.}.
% Email communication on: 25Nov2020. 
% The attached files in the email can be found in /Documents/Phd_thesis_draft/Ch6 (locally on my mac).
The bunch intensity of $3\times 10^{10}$ protons was represented by $10^5$ macroparticles. 

The beta function at the location of the interaction point was chosen to be the value at the location of $\CC$1 for both horizontal and vertical planes (values are listed in Table~\ref{tab:pyheadtail_simulation_parameters_first_tests}). At the same location, the Twiss parameter alpha and the dispersion were chosen to be zero. This is a valid assumption for the studies since these parameters have no direct impact on the noise-induced emittance growth~\cite{PhysRevSTAB.18.101001}. % It can also be seen in the equations that give the growth.
This is also confirmed, with simulations with the Sixtracklib code which are presented in the following section. The Sixtracklib simulations use the detailed optics of the machine for the tracking. It will be shown that the emittance growth rates from the two simulation tools are in good agreement.

% Presentation with the comment on the vertical tune distribution: https://docs.google.com/presentation/d/1rBVKGd9aBZTslaAHf48FdJkNPRyXtMF_T9XpEBlLE4c/edit#slide=id.g8ae11a254a_0_952
The mechanism responsible for the emittance growth in the presence of noise is the spread of the betatron tunes~\cite{Lebedev:248620}. As explained earlier (in Section~\ref{sec:noise_definition}) the tune spread leads to a phase mixing of the particles within the bunch causing a decoherence of the betatron oscillations which then results in emittance growth~\cite{Lebedev:248620}. The time scale of the decoherence equals the inverse of the betatron frequencies~\cite{Lebedev:248620}: % p.151
\begin{equation}\label{eq:decoherence_time}
    \tau_\mathrm{decoh} = \frac{1}{2\pi \frev \mathrm{rms}(\Delta Q_u)},
\end{equation}
where $u=(x,y)$ indicates the horizontal or vertical plane, $\frev$ the revolution frequency, and $\mathrm{rms}(\Delta Q_u)$ the rms betatron tune spread. The latter can be computed by Eqs.~\eqref{eq:rms_amplitute_detuning_3} and~\eqref{eq:rms_amplitute_detuning_3_x} for the vertical and horizontal planes respectively.

Therefore, it becomes clear that in order to observe some emittance growth a source of tune spread must be included in the simulations. For the simulations presented here detuning with transverse amplitude is introduced as described in Section~\ref{subsec:pyheadtail}, by applying a change of the phase advance of each individual particle depending on its action variable and the detuning coefficients. Detuning in both transverse planes is thus introduced for vertical detuning coefficient $\alpha_{xx}=179.35$/m, vertical detuning coefficient $\alpha_{yy}=-30.78$/m and for cross-term coefficient $\alpha_{yx}=-441.34$/m. These coefficients were computed using MAD-X~\cite{madx} for the nominal SPS lattice (introduced in Section~\ref{sec:optics_model_designs}) and for $Q^\prime_{x,y=0.5}$. It should be noted that the detuning in the vertical plane is the value of interest since the emittance evolution will be investigated in the vertical plane. Using Eq.~\eqref{eq:rms_amplitute_detuning_3} for the above mentioned coefficients and the initial transverse actions of the bunch, it is computed that $\mathrm{rms}(\Delta Q_y)\approx 7 \times 10^{-6}$. 

For the emittance growth studies to be valid the simulation time should be much longer than the decoherence time defined in Eq.~\eqref{eq:decoherence_time}. For the above value of rms vertical tune spread, the decoherence time is computed to be: $\tau_\mathrm{decoh} \approx 0.5$\,s. Therefore, the simulation time of about 2.5\,s is reasonable for these studies.
% from MAD-X for no b3b5b7 and QpcQPy=0--> axx=170.65, ayy = -29.70 and axy = -440.87.


% For the compaction factor I am not sure but it doesn't matter
% The chromaticity value was not 5e-1 for all the simulations. For some of the cases was 5e-1, for some others was 2 etc. However, it doesn't matter.
\begin{table}[!hbt]
	\begin{minipage}{\textwidth}
      \begin{centering}
   \caption{Simulation parameters used to benchmark the theoretically predicted emittane growth in Chapter~\ref{Ch:investigating_discrepancy}.}
	\begin{tabu} to \textwidth {X[c,m] X[0.5c,m] X[0.5c,m] X[0.01c,m]}
		&&& \\[-6mm]
		\toprule \toprule
		\multicolumn{2}{l}{\textbf{Parameter}} &
		\multicolumn{2}{c}{\textbf{Value}} \\
		\bottomrule
      \multicolumn{2}{l}{Beam energy, $\symE$} & \multicolumn{2}{c}{270\,GeV} \\
      \multicolumn{2}{l}{Machine circumference, $C_0$}  & \multicolumn{2}{c}{6911.5623\,m} \\
      \multicolumn{2}{l}{Horizontal / Vertical betatron tune, $\Qx$ / $\Qy$}  & \multicolumn{2}{c}{26.13 / 26.18} \\
      %\multicolumn{2}{l}{Horizontal / Vertical linear chromaticity, $Q^\prime_x$ / $Q^\prime_y$}  & \multicolumn{2}{c}{0.5 / 0.5} \\
      \multicolumn{2}{l}{Synchrotron tune, $\Qs$}  & \multicolumn{2}{c}{0.0035}\\
      \multicolumn{2}{l}{Momentum compaction factor, $\alpha_p$}  & \multicolumn{2}{c}{1.9 $\times 10^{-3}$}\\
      \multicolumn{2}{l}{Number of bunches}  & \multicolumn{2}{c}{1} \\
      \multicolumn{2}{l}{Rms bunch length, $\sigma_z$}  & \multicolumn{2}{c}{15.5\,cm}\\
      \multicolumn{2}{l}{Horizontal / Vertical normalised emittance, $\epsilon_x$ / $\epsilon_y$}  & \multicolumn{2}{c}{2\,$\mathrm{\mu m}$ / 2\,$\mathrm{\mu m}$}\\
      \multicolumn{2}{l}{Horizontal / vertical beta function, $ \beta_{x, CC1} / \beta_{y, CC1}$}  & \multicolumn{2}{c}{29.24\,m / 76.07\,m $^\dagger$ } \\
      \bottomrule
      \multicolumn{2}{l}{Number of macroparticles, $N_\mathrm{mp}$}  & \multicolumn{2}{c}{$10^5$} \\
      \multicolumn{2}{l}{Number of turns, $N_\mathrm{turns}$}  & \multicolumn{2}{c}{$10^5$} \\
      \bottomrule
	\end{tabu}
   \label{tab:pyheadtail_simulation_parameters_first_tests}
   \end{centering}\footnotesize{\\ $^\dagger$ Model values for the Q26 optics.}
   \end{minipage}
\end{table}

The tracking was performed for $10^5$ turns and the geometric emittance was computed every 100 turns (for computational efficiency) using the statistical definition introduced in Eq.~\eqref{eq:geometric_emittance_v2}. Thereafter, the normalised emittances were obtained using Eq.~\eqref{eq:normalised_emittance}. To reduce the statistical uncertainty of the results, due to the way the noise kicks are applied, the simulation was performed for thirty different runs. The initial bunch distribution and the sequence of the uncorrelated noise kicks were randomly regenerated every run (a different seed was used in the random generator).

The PyHEADTAIL simulation results are summarised in Fig.~\ref{fig:study_pyheadtail_normalised_momentum_kicks}. The simulated vertical emittance evolution in the presence of amplitude noise (left) and phase noise (right) is plotted as a function of time. For both noise types, the theoretically predicted growth (computed using Eqs.~\eqref{eq:dey_an} and~\eqref{eq:dey_pn} for the above mentioned parameters) is shown with the red line. The dark orange and dark blue lines show the evolution of the averaged emittance values over the different runs. The shaded areas, (light orange and light blue color), depict the standard deviation of the different emittance values over the thirty runs. The emittance growth rate is obtained with a linear fit to the averaged normalised emittance values over the simulation time. The slope that corresponds to the growth rate obtained by this fit is also drawn in the plot in black color. The uncertainty on the slope of the fit is displayed in the legend.

It is worth commenting, that the growth rates here are expressed in nm/s instead of $\mathrm{\mu m/h}$ that was used for the experimental results. This is due to the time scale of the simulations.
% Slides where the plots were taken from: https://docs.google.com/presentation/d/1xK63pVRDlW1KIV2NE3UMlAL3VnjhegwjbkpC0ufSpjg/edit#slide=id.g72167b2e95_0_70
% They can also be found at: https://docs.google.com/presentation/d/1WgxpGTsyy55XtOVT1ea9u3SO9EyG5EQMWuqhLyM46mM/edit#slide=id.g727b0ea384_1_7
\begin{figure}[htp]
    \centering
    \begin{subfigure}{.45\textwidth}
        \centering
        \includegraphics[width=.95\linewidth]{images/Ch6/pyheadtail_benchmark_amplitude_noise.png}  
    \end{subfigure}
    \begin{subfigure}{.45\textwidth}
        \centering
        \includegraphics[width=.95\linewidth]{images/Ch6/pyheadtail_benchmark_phase_noise.png}
    \end{subfigure}
    \caption{Vertical emittance growth driven by CC RF amplitude noise (left) and phase noise (right) as simulated with PyHEADTAIL simulation tool for a configuration close to the experimental conditions of the SPS CC tests in 2018.}
    \label{fig:study_pyheadtail_normalised_momentum_kicks}
\end{figure}

It can be seen, that for the amplitude noise case there is excellent agreement between the theoretical prediction and the simulated growth rate. For the phase noise case, the agreement between expected and simulated emittance growth is very good. The simulated growth appears slightly lower than the theoretical predictions. However, this difference is insignificant comparing to the factor of about 4 observed between the theory and the measurements of 2018. A possible reason for the small discrepancy between simulation and theory is the very small tune spread. Later simulations with larger vertical tune spread (closer to the realistic tune spread in the SPS during the 2018 experiments (see Section~\ref{sec:first_obs_suppression})) show excellent agreement between the PyHEADTAIL results and the theoretically computed values. Nevertheless, this small difference does not affect the conclusions drawn from the results, as will be shown in the following chapters.

In general, it can be concluded that the transverse emittance growth driven by $\CC$ RF noise obtained by PyHEADTAIL simulations is in good agreement with the theoretically expected values from the model of T.~Mastoridis and P.~Baudrenghien~\cite{PhysRevSTAB.18.101001}. In the following section the theory is benchmarked against a more complete simulation tool, Sixtracklib.


\section{Benchmarking theory against Sixtracklib}\label{sec:benchmark_theory_with_sixtracklib}

summarises the results of validating the theoretical model [63] against numerical simulations with Sixtracklib [58].  The additional comparison between theory and simulation is motivated by...

This section summarises the results of validating the theoretical model~\cite{PhysRevSTAB.18.101001} against numerical simulations with Sixtracklib~\cite{sixtracklib_repo}. The additional comparison between theory and simulation is motivated by the fact that PyHEADTAIL and theory may miss some beam dynamics that could explain the discrepancy between their results and the experimental observations of 2018. Sixtracklib is a more complete simulation tool than PyHEADTAIL as the tracking simulations use the detailed optics of the machine and therefore is considered appropriate for this study. Calculations can be performed on GPU decreasing the computational time. For the simulations presented in this section the nominal SPS model for Q26 optics will be used~\cite{cern_optics_repo} as introduced in Section~\ref{sec:optics_model_designs} except if it is stated otherwise.

This section is structured as follows. First, the emittance growth in the presence of phase and amplitude noise, modeled as kicks applied to the angle co-ordinate of the particles, is simulated using PyHEADTAIL. Thereafter, the simulations for the same configuration are repeated using Sixtracklib. Then, the implementation of a $\CC$ element in the SPS lattice used for the Sixtracklib simulations is presented. Then, in the following, in the following subsections, the simulation is repeated with the noise applied on the $\CC$ element and studying the obtained growth in the presence of a local and global $\CC$ scheme. Simulations were also performed using the measured noise spectra that were injected into the CC in 2018. Finally, the sensitivity of the noise-induced emittance growth to the multipole errors of the main SPS dipoles is tested.

\subsection{Emittance growth simulations with CC noise modeled as transverse kicks on the angle co-ordinate.}\label{subsec:sixtracklib_kicks_transverse_angle}
% Link to presentation with relevant slides: https://docs.google.com/presentation/d/1rBVKGd9aBZTslaAHf48FdJkNPRyXtMF_T9XpEBlLE4c/edit#slide=id.g8a041cb470_0_448

First the emittance growth simulations that were presented in the previous section, were repeated with Sixtracklib instead of PyHEADTAIL. This basically means, that the detailed optics of SPS machine are used for the tracking instead of modelling the ring as simple transfer map with one interaction point where the $\CC$  kick is applied. The nominal SPS lattice of MAD-X is used: dipoles, quadrupoles, and chromatic sextupoles as discussed in Section~\ref{sec:optics_model_designs}. It can be found in the GitLab repository of Ref.~\cite{cern_optics_repo}. The linear chromaticity was adjusted to be in the range $1 < Q^\prime_{x,y} < 2$. The relevant machine and beam parameters are listed in Table~\ref{tab:pyheadtail_simulation_parameters_first_tests}.

The rms betatron tune spread here is slightly higher than $7 \times 10^{-6}$ that was computed in the previous section, due to the larger values of linear chromaticity. To this end, tracking for $10^{5}$ turns is reasonable also for these studies. The initial distribution of $10^5$ particles follows a gaussian both in transverse and longitudinal planes.

The emittance growth was simulated in Sixtracklib in the presence of both amplitude and phase noise. The noise kicks were implemented as kicks on the vertical angle co-ordinate of each particle following the previous PyHEADTAIL simulations. The noise kicks were applied on the beam at the location of $\CC$1. The power spectral density of the noise at the betatron frequency was $1.68\times 10^{-10}$/Hz and $\mathrm{rad^2/Hz}$ for amplitude and phase noise respectively. This corresponds to a scaling factor, $A=10^{-8}$, in Eqs.~\eqref{eq:amplitude_noise_kick} and~\eqref{eq:phase_noise_kick}. The power spectra are the same as those shown in Fig.~\ref{fig:psd_cc_simulations_example}.

The emittance and emittance growth values were computed with the same procedure followed for the previous PyHEADTAIL simulations (see Section~\ref{sec:benchmark_theory_with_pyheadtail}). Figure~\ref{fig:study_1_sixtracklib_normalised_momentum_kicks} summarizes the simulated emittance growth driven by amplitude (left) and phase (right) noise. For both types of noise, the theoretical predicted growth is shown by the red line. The dark blue and dark orange lines show the evolution of the averaged emittance values over thirty different runs. The shaded areas, shown with light blue and light orange colors, depict the standard deviation of the emittance values over the thirty different runs. The emittance growth rate is obtained with a linear fit to the averaged normalised emittance values over the simulation time. The resulting fits are shown in the plots as black lines. The uncertainty on the slope of the fit is displayed in the legend.

% Figures downlaoded from: https://docs.google.com/presentation/d/1-rGHtMoeLPykjgyyotreXqWFaaxChA1pvrco-sZpobw/edit#slide=id.g71f3275427_0_89
\begin{figure}[htp]
    \centering
    \begin{subfigure}{.45\textwidth}
        \centering
        \includegraphics[width=.95\linewidth]{images/Ch6/study_1_AN_sixtracklib_kicks.png}  
    \end{subfigure}
    \begin{subfigure}{.45\textwidth}
        \centering
        \includegraphics[width=.95\linewidth]{images/Ch6/study_1_PN_sixtracklib_kicks.png}
    \end{subfigure}
    \caption{Vertical emittance growth driven by CC RF amplitude noise (left) and phase noise (right) as simulated with Sixtracklib simulation tool for a configuration close to the experimental conditions of the SPS CC tests in 2018. The CC noise is modeled as uncorrelated kicks on the angle variables of the particle every turn following Eqs.~\eqref{eq:phase_noise_kick} and.~\eqref{eq:amplitude_noise_kick} for phase and amplitude noise respectively.}
    \label{fig:study_1_sixtracklib_normalised_momentum_kicks}
\end{figure}


From these plots, it becomes evident that there is an excellent agreement between the theoretical predicted vertical emittance growth and the simulation results with Sixtracklib. The results from Sixtracklib simulations are also in agreement with the results from PyHEADTAIL. To this end, it is concluded that taking into account the detailed optics of the SPS cannot explain the discrepancy between expected and measured emittance growth that was observed in the SPS CC tests in 2018.

Another observation, is that the spread between the emittance values obtained from the different runs is much larger in the case of phase noise (left) than in the case of amplitude noise (right). The reason for this is not yet well understood. \textcolor{red}{Further comments are needed here?} Nevertheless, this does not affect the conclusions drawn from the results presented in this thesis. 

%As it will be shown later the reason for the discrepancy is not relevant to the way the stochastic noise kicks are applied on the beam. 


% Comments: The difference on the phase noise plots might be the larger tune spread in the sixtracklib simulations due to the larger positive chromaticity. In general we are no sensitive to the tune spread value in the absence of wakefields. However, for such small values of the order to $10^{-6}$ there might be some sensitivity.


\subsection{Implementation of CC element in Sixtracklib}\label{subsec:sixtracklig_CC_implementation}
% Link to corresponding presentation:https://docs.google.com/presentation/d/15038dO7vsEUYxzCsYtQXYlpRIwtVGkA8esGtvKHwBwM/edit#slide=id.g6da2141836_0_5
Sixtracklib provides the possibility to perform tracking simulations in the presence of an actual $\CC$ element on which noise can be added, instead of modeling only the transverse momentum kicks due to phase and amplitude noise in the RF system of the cavity (as was done in PyHEADTAIL). This provides the opportunity to study the $\CC$ RF noise-induced emittance growth in a more realistic scenario and therefore it will be used for the rest of the simulations presented in this chapter.

In Sixtracklib the $\CC$ element is represented by an element (referred to as RFMultipole in the Sixtracklib documentation) that has the properties of a multipole magnet (oscillating at a specified frequency) superposed on an RF cavity.  To simulate the vertical $\CC$, the multipole field component corresponds to a skew dipole.

When a particle passes through this element, it recieves the following vertical kick on the momentum:
\begin{equation}\label{eq:CC_kick_sixtracklib_vertical}
    y^\prime_{j+1} = y^\prime_{j} + \cos{\left ( \phi_\mathrm{CC} \frac{\pi}{180} - \frac{2\pi f_\mathrm{CC}}{c} \frac{z \beta_0}{\beta^2} \right )} \frac{V_\mathrm{0,CC}}{p_0 c},
\end{equation}
where $j={0, ...., N_\mathrm{turns}}$ denotes the turn number with $N_\mathrm{turns}$ being the total number of turns that the beam passes through the element. Furthermore, $y^\prime$ is the vertical angle co-ordinate and z the longitudinal co-ordinate of each particle, $\phi_\mathrm{CC}$ the $\CC$ phase in degrees, $f_\mathrm{CC}$ is the $\CC$ frequency, $c$ is the speed of light, $\beta_0$ is the relativistic beta, $V_\mathrm{0,CC}$ is the amplitude of the $\CC$ voltage and $p_0$ the reference momentum.

Before using this element for the emittance growth simulations its implementation in Sixtracklib was tested. This check was important as Sixtracklib is a recently developed simulation tool and at the time these studies took place the use of the RFMultipole as a $\CC$ element was not well tested. The $\CC$ element installed at a location $s_0$, acts like a single dipole field error at a different location, $s_1$, around the ring. To this end, the induced orbit shift at the location $s_1$ from the $\CC$ element is benchmarked against the theoretically expected orbit shift resulting from a single dipole field error. The latter has already been discussed in the context of the reconstruction of $\CC$ voltage from the HT monitor in Section~\ref{sec:Vcc_calibration}. Equation~\eqref{eq:CC_orbit_shift_CC_element}, which is obtained from Eq.\,(1) from chapter 4.7.1 in Ref.~\cite{Chao:1490001}, gives the vertical orbit shift (in meters) from the $\CC$ kick (at the location $s_0$), at the location $s_1$ as follows:

\begin{equation}\label{eq:CC_orbit_shift_CC_element}
    \Delta y_{,s_11} = \frac{\sqrt{\beta_{y, s_1}}}{2 \sin(\pi \Qy)} \Delta y^\prime \sqrt{\beta_{y, s_0}} \cos(\pi \Qy - \mid \psi_{y, s_1} - \psi_{y, s_0} \mid),
 \end{equation}
where $\Delta y^\prime=y^\prime_{j+1}-y^\prime_{j}$, where $j$ is the number of turns for which the tracking is performed, $\Qy$ is the vertical tune, $\beta_{y, s_0}$ and $\beta_{y, s_1}$ the vertical beta function at the locations $s_0$ and $s_1$ respectively and $\mid \psi_{y, s_1} - \psi_{y, s_0} \mid)$ is the vertical phase advance in tune units between the locations $s_0$ and $s_1$.

Figure~\ref{fig:sixtracklib_CC_orbit_shift_vs_theory} compares the shift of the orbit as computed analytically using Eq.~\eqref{eq:CC_orbit_shift_CC_element} (blue) for the values reported in Table~\ref{tab:SPS_CC_WS_sixtracklib} and as obtained from Sixtracklib simulations (orange). The induced orbit shift was obtained from Sixtracklib simulations after tracking 1000 particles for 500 turns in the presence of the above-mentioned CC element (which is installed at the location of $\CC$1). The simulations took place using an initial Gaussian bunch distribution in the six-dimensional phase space. The initial normalised emittances were $\epsilon_x$=2.5\, $\mathrm{\mu m}$ and $\epsilon_x=2.5 \times 10^{-6}$\, $\mathrm{\mu m}$ for the horizontal and vertical plane respectively. The distribution was chosen to be so small in the vertical plane so there is practically no initial offset which facilitates the observation of the orbit shift. % If we had offset we would need to split in longitudinal slices and take the average position of each slice.
 The location $s_1$ was set at the start of the lattice which for this study it was considered to be the horizontal rotational Wire Scanner, BWS.51995.H. This choice was arbitrary. The most relevant simulation parameters are listed in Table~\ref{tab:SPS_CC_WS_sixtracklib}.
% Note: The induced closed orbit shown in the plot corresponds to the start of the lattice (BWSA.51995) as the particles are dumped in that location. Link to complementary slides: https://docs.google.com/presentation/d/1-w9CRX_IuH6tWqg7Mx450sDhk7hBIXyBsZ0hyumq4hw/edit#slide=id.g6d31fa0069_1_14


% Image downloaded from: https://docs.google.com/presentation/d/15038dO7vsEUYxzCsYtQXYlpRIwtVGkA8esGtvKHwBwM/edit#slide=id.g6da5555eb3_1_59
\begin{figure}[!h]
    \centering         
    \includegraphics[width=0.7\textwidth]{images/Ch6/Vcc_orbit_shfit_sixtracklib_sanity_check.png}
        \caption{Vertical orbit shift at the location of the horizontal Wire Scanner (SPS.BWS.51995.H.) induced by the CC element as computed analytically (blue) and from tracking simulations with Sixtracklib (orange).}
        \label{fig:sixtracklib_CC_orbit_shift_vs_theory}
\end{figure}

% Optics were taken from: https://github.com/natriant/rf_multipole_sanity_checks/blob/master/pysixtrack/rf_multipole_sanity_check_pysixtrack_001.ipynb
\begin{table}[!hbt]
	\begin{minipage}{\textwidth}
   \begin{centering}
   \caption{Parameters for computing the vertical orbit shift induced by the CC element (at the location $s_0$) at the location of the horizontal Wire Scanner (SPS.BWS.51995.H.), $s_1$.}
	\begin{tabu} to \textwidth {X[c,m] X[0.01c,m] X[0.01c,m] X[0.01c,m]}
		&&& \\[-6mm]
		\toprule \toprule
		\multicolumn{2}{l}{\textbf{Parameter}} &
		\multicolumn{2}{c}{\textbf{Value}} \\
		\bottomrule
      \multicolumn{2}{l}{Beta function at the Wire Scanner, $\beta_{y, s_1}$}& \multicolumn{2}{c}{27.47\,m} \\
      \multicolumn{2}{l}{Phase advance to the Wire Scanner$^{\ast}$, $\psi_{y, HT}$} & \multicolumn{2}{c}{0} \\
      \multicolumn{2}{l}{Beta function at the $\CC$1, $\beta_{y, CC1}$}& \multicolumn{2}{c}{76.07\,m} \\
      \multicolumn{2}{l}{Phase advance to the $\CC$1$^{\ast}$, $\psi_{y, CC1}$} & \multicolumn{2}{c}{4.05 $\times$ 2$\mathrm{\pi}$} \\
      \multicolumn{2}{l}{Vertical betatron tune, $\Qy$} & \multicolumn{2}{c}{26.18} \\
      \multicolumn{2}{l}{Beam energy, $\symE$} & \multicolumn{2}{c}{26\,GeV} \\
      \multicolumn{2}{l}{rms bunch length, $\sigma_z$} & \multicolumn{2}{c}{0.22\,m} \\
      \multicolumn{2}{l}{rms momentum spread, $\sigma_\delta$} & \multicolumn{2}{c}{1e-4} \\
      \multicolumn{2}{l}{CC voltage, $V_\mathrm{0,CC}$} & \multicolumn{2}{c}{3\,MV} \\
      \multicolumn{2}{l}{CC frequency, $f_\mathrm{CC}$} & \multicolumn{2}{c}{400\,MHz} \\
      \multicolumn{2}{l}{CC phase, $\phi_\mathrm{CC}$} & \multicolumn{2}{c}{90\,deg$^{\dagger}$} \\
      \bottomrule
	\end{tabu}
   \label{tab:SPS_CC_WS_sixtracklib}
   \end{centering} \footnotesize{$^\ast$ The phase advances are measured from the start of the lattice which is considered the element SPS.BWS.51995.H that is the horizontal rotational Wire Scanner. \\$^\dagger$ It was found that in the definition of the RFMultipole the phase of the cavity is shifted by 90 degrees compared to the standard (theoretical) crab cavity kick.}
   \end{minipage}
\end{table}

For the computation of the theoretical prediction Eq.~\eqref{eq:CC_orbit_shift_CC_element} was used over a range of equally spaced $z$ co-ordinates from -0.6 to 0.6\,m ($\sim 3 \sigma_z$). The $z$ co-ordinates are taken into account indirectly through $\Delta_y^\prime$ and Eq.~\eqref{eq:CC_kick_sixtracklib_vertical}.

By looking at Fig.~\ref{fig:sixtracklib_CC_orbit_shift_vs_theory} it is concluded, that using Sixtracklib the result of $\CC$ element on the orbit is as expected from the analytical calculations. % on the closed orbit

\textbf{Implementing amplitude and phase noise on the real CC element}\\
% Slides for this paragraph: https://docs.google.com/presentation/d/1rcSwhZBLZi0Js9jeFiVSLzR6HxhzljYkLLrBMahdqus/edit#slide=id.g6e85dffc41_2_143
The amplitude and phase noise in the real $\CC$ element used for the following Sixtracklib simualtions are modeled following the discussion in Chapter~\ref{Ch:CC_noise_theory} (see Eq.~\eqref{eq:CC_voltage_z}).

In particular, in the presence of amplitude and phase noise the vertical kicks on the momentum of Eq.~\eqref{eq:CC_kick_sixtracklib_vertical} are modified as follows:

\begin{equation}\label{eq:CC_kick_sixtracklib_vertical_noise}
    y^\prime_{j+1} = y^\prime_{j} + \cos{\left ( (\phi_\mathrm{CC} + \zeta_j \Delta \phi) \frac{\pi}{180} - \frac{2\pi f_\mathrm{CC}}{c} \frac{z \beta_0}{\beta^2} \right )} \frac{V_\mathrm{0,CC} (1 + \zeta_j \Delta A)}{p_0 c},
\end{equation}
where $j={0, ...., N_\mathrm{turns}}$ denotes the turn number with $N_\mathrm{turns}$ being the total number of turns that the beam passes through the element. Furthermore, $\Delta \phi$ is the deviation from the nominal phase (phase noise level), and $\Delta A$ the deviation from the nominal amplitude $V_\mathrm{0,CC}$ (amplitude noise level). The typical values that will be used in the following simulations for amplitude and phase noise respectively are: $\Delta A$=$10^{-8}$ and $\Delta \phi = 10^{-8} \frac{p_0 c}{V_\mathrm{0,CC}} \frac{180}{\pi}$.
Note that $\Delta \phi$ enters Eq.~\eqref{eq:CC_kick_sixtracklib_vertical_noise} in units of degrees.
These values result to power spectral densities at the betatron frequency of: $S_{\Delta_A}(f_b) = 1.68 \times 10^{-10}$\,1/Hz for amplitude noise and $S_{\Delta_\phi}(f_b) = 1.68 \times 10^{-10}$ $\mathrm{rad^2/Hz}$ for phase noise.
Finally, $\zeta_j$ is the $j$th element of a sample drawn from a Gaussian distribution with mean 0, standard deviation 1, and size $N_\mathrm{turns}$, such as the above kicks are uncorrelated (white noise). 
% Example script to see the scaling for the noise:/eos/user/n/natriant/sixtracklib_data/27March2020/AN/job_002_track_sixtracklib.py
% I am not sure if this sentence is correct. Note that this definition is slightly different from Eq.~\eqref{eq:CC_voltage_z} due to the slghtly different definititions used form the Sixtracklib for RFMultipole element. 



\subsection{CC noise induced emittance growth in the presence of local CC scheme}\label{subsec:local_CC_sixtracklib}

% Slides for this paragraph, v1: https://docs.google.com/presentation/d/1rcSwhZBLZi0Js9jeFiVSLzR6HxhzljYkLLrBMahdqus/edit#slide=id.g6e85dffc41_2_143
% Slides for this paragraph, v2: https://docs.google.com/presentation/d/1rBVKGd9aBZTslaAHf48FdJkNPRyXtMF_T9XpEBlLE4c/edit#slide=id.g8a041cb470_0_629

In this section, the emittance growth driven by $\CC$ RF noise is simulated with Sixtracklib in the presence of a local $\CC$ scheme. It is reminded that in the local $\CC$ scheme (see Fig.~\ref{fig:crossing_with_and_without_CCs}) two $\CC$s are used with opposite phase (to cancel out the effect of the crabbing (orbit shift discussed in the previous paragraph)).

The benchmark with the local scheme is performed as it is closer to the case that was studies right before: where the $\CC$ RF noise is modeled as noise kicks in the momentum. In both of these cases the effect of the crabbing is not transporting around the machine.

The simulation studies presented in Section~\ref{subsec:sixtracklib_kicks_transverse_angle} were repeated here but this time the real $\CC$ elements were implemented and switched ON. Both $\CC$1 and $\CC2$ operated at 1\,MV, for $\phi_\mathrm{CC1}$=90\,deg and  $\phi_\mathrm{CC2}$=270\,deg. Figure~\ref{fig:sixtracklib_CC1_vs_CC2_orbit_shift_vs_theory} visualises that the orbit shifts from two $\CC$s operating in opposite phase cancel out as expected. The results shown are obtained from Sixtracklib simulations that were conducted following the same procedure as the equivalent study presented in Section~\ref{subsec:sixtracklig_CC_implementation} (see Fig.~\ref{fig:sixtracklib_CC_orbit_shift_vs_theory}.)

% Figures is downloaded from here: https://docs.google.com/presentation/d/1OhwEA0FdMNbgeGo5cjCuIFYaPMVfzjdHlG51YuUx9WQ/edit#slide=id.p
\begin{figure}[!h]
    \centering         
    \includegraphics[width=0.7\textwidth]{images/Ch6/Vcc_orbit_shift_CC1_CC2.png}
        \caption{Vertical orbit shift at the location of the horizontal Wire Scanner (SPS.BWS.51995.H.) as obtained with Sixtracklib tracking through the nominal SPS lattice. \textit{Green:} Only CC1 operates at 1\,MV and $\phi_\mathrm{CC1}$=90\,deg. \textit{Red:} Only CC1 operates at 1\,MV and $\phi_\mathrm{CC2}$=270\,deg. \textit{Orange:} The two CCs operate at the same voltage but in opposite phase: $\phi_\mathrm{CC1}$=90\,deg and  $\phi_\mathrm{CC}2$=270\,deg. \textit{Blue:} Orbit shift computed analytically using Eq.~\eqref{eq:CC_orbit_shift_CC_element}.}
        \label{fig:sixtracklib_CC1_vs_CC2_orbit_shift_vs_theory}
\end{figure}

For the emittance growth simulations, the noise was applied only in $\CC$1. The amplitude and phase noise were treated separately and for the simulation parameters a growth of $22$\,nm/s and $24$\,nm/s is expected respectively. The simualtion results are illustrated in Fig.~\ref{fig:study_2_sixtracklib_local_cc_scheme}.

% Figures are downloaded from here: https://docs.google.com/presentation/d/1OhwEA0FdMNbgeGo5cjCuIFYaPMVfzjdHlG51YuUx9WQ/edit#slide=id.g700f4dc708_0_2
\begin{figure}[htp]
    \centering
    \begin{subfigure}{.45\textwidth}
        \centering
        \includegraphics[width=.95\linewidth]{images/Ch6/study_2_AN_sixtracklib_local_CC_scheme.png}  
    \end{subfigure}
    \begin{subfigure}{.45\textwidth}
        \centering
        \includegraphics[width=.95\linewidth]{images/Ch6/study_2_sixtracklib_PN_local_CC_scheme.png}
    \end{subfigure}
    \caption{Vertical emittance growth driven by CC RF amplitude noise (left) and phase noise (right) as simulated with Sixtracklib simulation tool for a configuration close to the experimental conditions of the SPS CC tests in 2018 but for a local CC scheme. The CC noise is applied on CC1, following Eq.~\eqref{eq:CC_kick_sixtracklib_vertical_noise}.}
    \label{fig:study_2_sixtracklib_local_cc_scheme}
\end{figure}

The simulations show an excellent agreement between the theoretically computed and simulated growth rates for both noise types. They also demonstrate that the modeling of the noise as kicks on the angle co-ordinate provides equivalent results with "real" noise which is applied through the $\CC$ element of Sixtracklib.

\subsection{CC noise induced emittance growth in the presence of global CC scheme}\label{subsec:global_CC_sixtracklib}
Here, the simualtions presented above were repeated but for a global $\CC$ scheme. It is reminded that in the global $\CC$ scheme only one CC is in operation and the closed orbit shift is present during the circulation of the bunch around the machine. This scheme is the realistic case for the experiments of 2018, where for the emittance growth measurements only one $\CC$ was used.

For the simulations presented here, $\CC$2 was switched OFF. $\CC$1 operated at 1\,MV, for $\phi_\mathrm{CC}$=90\,deg. The noise was applied as described in Eq.~\eqref{eq:CC_kick_sixtracklib_vertical_noise} for the same noise levels as before. For this study, where only one $\CC$ is switched ON, its voltage is slowly increased to the chosen value of 1\,MV, so that the new closed orbit includes the full $\CC$ kick. Without this dynamic ramping, emittance blow-up is observed during the first turns in the simulation. Previous studies have shown that using a ramp of 200 turns minimizes the blow-up (discussed in the APR - 1st year). % APR p.39 
Nevertheless, the first 200 turns, are excluded from the linear fit used to obtained the emittance growth rates. % It doesn't matter as it is just two points.

Figure~\ref{fig:study_3_sixtracklib_global_cc_scheme} illustrates the simulated emittance growth driven by amplitude (left) and phase (right) noise in the presence of a global CC scheme. Once again, an excellent agreement is observed with the theoretically expected growth and also with the simulated rates for the different configurations discussed previously in this Chapter. This behavior is expected as the $\CC$ element itself without noise cause no emittance growth % APR 1st year. 

% The figure is downloaded from the following presentation: https://docs.google.com/presentation/d/1rBVKGd9aBZTslaAHf48FdJkNPRyXtMF_T9XpEBlLE4c/edit#slide=id.g8a041cb470_0_629 
\begin{figure}[htp]
    \centering
    \begin{subfigure}{.45\textwidth}
        \centering
        \includegraphics[width=.95\linewidth]{images/Ch6/study_3_AN_sixtracklib_global_CC_scheme.png}  
    \end{subfigure}
    \begin{subfigure}{.45\textwidth}
        \centering
        \includegraphics[width=.95\linewidth]{images/Ch6/study_3_PN_sixtracklib_global_CC_scheme.png}
    \end{subfigure}
    \caption{Vertical emittance growth driven by CC RF amplitude noise (left) and phase noise (right) as simulated with Sixtracklib simulation tool for a configuration close to the experimental conditions of the SPS CC tests in 2018 but for a global CC scheme. The CC noise is applied on CC1 which operates at 1\,MV, following Eq.~\eqref{eq:CC_kick_sixtracklib_vertical_noise}, while CC2 is switched OFF.}
    \label{fig:study_3_sixtracklib_global_cc_scheme}
\end{figure}

\subsection{CC noise induced emittance growth with the measured noise spectrum}\label{subsec:global_CC_sixtracklib_noiseCoast3_setting3}
All the simualtions discussed up to now, were performed considering white noise spectrum where the sequence of the uncorrelated random noise kicks were taken from a Gaussian distribution. 

However, as discussed in Section~\ref{sec:injected_RF_noise} there are available measurements of the phase and amplitude noise that were injected in the $\CC$ RF system for the emittance growth studies in the SPS in 2018. 

Here, the simulation studies presented in the previous Subsection~\ref{subsec:fig:study_3_sixtracklib_global_cc_scheme} (with global CC scheme) were repeated but this time the emittance growth is simulated using the real noise spectrum. 

The phase and amplitude noise spectra from Coast3-Setting3 (see Fig.~\ref{fig:example_PN_and_AN_coast3_setting3}) were used since they were the strongest noise levels from all the coasts. To this end, one can ensure the observation of reasonable emittance growth in our simulation time, which for this particular set of simualtions was increased to $\sim$10\,s ($5\times 10^{5}$ turns).


% Loaction for creating the figure: /eos/user/n/natriant/2018/CC_MD_2018_summary/measured_psd
\begin{figure}[!ht]
    \centering
    \begin{subfigure}[t]{0.45\textwidth}
        \centering
        \includegraphics[width=1\textwidth]{images/Ch6/Measured_spectrum_MD5_Coast3-Setting3-AN.csv_no_psd}
        %\caption{$y=\sin(2 \pi f t),\ f=50$ Hz}
        %\label{fig:add_label_here}
    \end{subfigure}
    \hfill
    \begin{subfigure}[t]{0.45\textwidth}
        \centering
        \includegraphics[width=1\textwidth]{images/Ch6/Measured_spectrum_MD5_Coast3-Setting3-PN.csv_no_psd}
        %\caption{Discrete Fourier transform}
        %\label{fig:add_label_here}
    \end{subfigure}
    \hfill
     \caption{Example amplitude (left) and phase (right) noise spectra measured with a spectrum analyzer E5052B~\cite{E5052B_insight} during the emittance growth studies with CCs in SPS. The noise extends up to 10\,kHz (grey dashed line) overlapping the first betatron sideband at $\sim$8\,kHz (green dashed line). The spikes at high frequencies correspond to the harmonics of the revolution frequency and are a result of the bunch crossing.} % bunch passage
     \label{fig:example_PN_and_AN_coast3_setting3}
 \end{figure}

Additionally, some of the simulation parameters were refined to be closer to the experimental conditions of 2018. The parameters that were updated from the values that were listed in Table~\ref{tab:CC_pyheadtail_simulation_parameters} and described Sections~\ref{sec:benchmark_theory_with_sixtracklib} are the following: $Q_s=0.0051$ and $\sigma_z$=0.138\,m. Following the experimental configuration of 2018, $\CC$1 was switched OFF, while $\CC$2 operated at 1\,MV. The voltage was slowly ramped up to that value during the first 200 turns. For reference, the vertical beta function at the location of $\CC$2 is 73.8\,m.
% klod=1, but I am not gonna mention it here. The detailed parameters can be found in slide 14 of the following presentation: https://docs.google.com/presentation/d/1O1Hl-Mvc99OfyI2FqIY-b0PMC2HVGsibKqRZGvy7lnU/edit#slide=id.gab6b65cbda_0_352

The measured spectra of Fig.~\ref{fig:example_PN_and_AN_coast3_setting3} were converted to a discrete time series that can be used in the numerical simulations as described in Appendix~\ref{sec:measured_spectra_to_time_series}.

In Figure~\ref{fig:study_4_sixtracklib_global_cc_scheme_measured_spectra} the simualtion results where the phase and amplitude noise from Coast3-Setting3 are applied separately on $\CC$2 are summarised. Once again, the greement between the theory and the simulations is excellent for both cases.
% Actually the results are for local CC scheme. It doesnt matter.

% The figures were downloaded from the following presentation: https://docs.google.com/presentation/d/1pSArooiopFDTG-jV-_kwwcEKiWIwnaJD1By1n-IbdQw/edit#slide=id.ga3ee40b191_0_64
\begin{figure}[htp]
    \centering
    \begin{subfigure}{.45\textwidth}
        \centering
        \includegraphics[width=.95\linewidth]{images/Ch6/study_4_AN_measured_spectra_sixtracklib.png}  
    \end{subfigure}
    \begin{subfigure}{.45\textwidth}
        \centering
        \includegraphics[width=.95\linewidth]{images/Ch6/study_4_PN_measured_spectra_sixtracklib.png}
    \end{subfigure}
    \caption{Vertical emittance growth driven by CC RF amplitude noise (left) and phase noise (right) as simulated with Sixtracklib simulation tool for a configuration close to the experimental conditions of the SPS CC tests 2018. The measured phase and amplitude noise spectra from Coast3-Setting3 are used for the simulations. The CC noise is applied on CC2 which operates at 1\,MV, following Eq.~\eqref{eq:CC_kick_sixtracklib_vertical_noise}, while CC1 is switched OFF.}
    \label{fig:study_4_sixtracklib_global_cc_scheme_measured_spectra}
\end{figure}

By looking at the plots, it is evident that there is significantly less spread between the emittance values over the different runs. This is due to the fact that as discussed in Appendix~\ref{sec:measured_spectra_to_time_series}, in the time series generated by the measured spectra the random factor is included in the set of random phases which leads to much less deviation in the values than the sequence of white noise kicks where the random factor is in their amplitude. To this end, the simulation here is repeated for just 10 different runs to reduce the uncertainty.

Last, the same simulation is repeated in the presence of both amplitude and phase noise together. The results are visualised in Fig.~\ref{fig:study_4_sixtracklib_global_cc_scheme_measured_spectra_AN_and_PN} where it is clear that the agreement with the analytically predicted rates is exceptionally good. 

% The figures were downloaded from the following presentation: https://docs.google.com/presentation/d/1pSArooiopFDTG-jV-_kwwcEKiWIwnaJD1By1n-IbdQw/edit#slide=id.ga3ee40b191_0_64
\begin{figure}[!h]
    \centering         
    \includegraphics[width=0.7\textwidth]{images/Ch6/study_4_AN_PN_sxitracklib.png}
        \caption{Vertical emittance growth driven by CC RF amplitude and phase noise as simulated with Sixtracklib simulation tool for a configuration close to the experimental conditions of the SPS CC tests 2018. The measured phase and amplitude noise spectra from Coast3-Setting3 are used for the simulations. Both types of noise are applied on CC2 which operates at 1\,MV, following Eq.~\eqref{eq:CC_kick_sixtracklib_vertical_noise}, while CC1 is switched OFF.}
        \label{fig:study_4_sixtracklib_global_cc_scheme_measured_spectra_AN_and_PN}
\end{figure}


\subsection{CC noise induced emittance growth with the non-linear SPS model}\label{subsec:global_CC_sixtracklib_noiseCoast3_setting3_non_linear_sps}


The nominal SPS model includes only the nonlinear fields produced by the chromatic sextupoles. However, one of the most important sources of non-linearities in SPS are the odd multipole components of its main dipole magnets. In this subsection their impact on the beam dynamic is studies trying to explain the observed discrepancy between measured and predicted growth rates pbserved in the experiment of 2018. To this end the multipole components of the SPS main dipoles should be included in the nominal SPS model that was used up to now.

% APR Chapter 3.1
The multipole error of the SPS main dipoles are unfortunately not available from magnetic measurements. On this ground a non-linear optics model of the SPS has been established with beam-based measurements of the chromatic detuning over a range of momentum deviation~\cite{Carlà:2664976, Alekou:2640326}.  The optics model was obtained by assigning systematic multipole components to the main lattice magnets, in the nominal model of SPS, in order to reproduce the tune variation with themomentum deviation as it was measured in the real machine. The calculations were performed with MAD-X.

The values of the multipole components up to seventh order obtained from this method are given in Table~\ref{tab:sps_mult_270GeV} where, ($b_3^A, b_3^B$) ($b_5^A, b_5^B$) and ($b_7^A, b_7^B$) stand for the sextupolar, decapolar and decatetrapolar mutipoles respectively. Note that different values have been obtained foreach of the two different kinds of SPS main dipoles (MBA and MBB) which are marked withthe indices A and B respectively.

\begin{table}[ht] % table take from the APR
    \caption{Multipole errors from SPS non-linear model, at 270\,GeV.} % title of Table
    \centering % used for centering table
    \begin{tabular}{c c c c} % centered columns (4 columns)
    \hline\hline %inserts double horizontal lines
    Multipole & Value  \\ [0.5ex] % inserts table
    %heading
    \hline  % inserts single horizontal line
    $b_3^A, b_3^B$ & 8.1 $\times 10^{-4}$\,$\mathrm{m^{-2}}$, 1.1 $\times 10^{-3}$\,$\mathrm{m^{-2}}$\\ 
    $b_5^A, b_5^B$ & 9.2\,$\mathrm{m^{-4}}$, $-$10\,$\mathrm{m^{-4}}$ \\
    $b_7^A, b_7^B$ & 1.3 $\times 10^{5}$\,$\mathrm{m^{-6}}$, 1.4 $\times 10^{5}$\,$\mathrm{m^{-6}}$\\ [1ex] % [1ex] adds vertical space
    \hline %inserts single line
    \end{tabular}
    \label{tab:sps_mult_270GeV} % is used to refer this table in the text
\end{table}

These values were assigned to the main dipoles of the SPS and the simulations presented in Section~\ref{subsec:global_CC_sixtracklib_noiseCoast3_setting3} were repeated. The simulations were performed in the presence of the measured phase noise spectrum only which was dominant during the SPS experiments in 2018. The results are displayed in Fig.~\ref{fig:study_5_sixtracklib_global_cc_scheme_measured_spectra_PN_b3b5b7}. It is clear that there is a very good agreement between the theory and the simulations when the non-linear model of SPS is used.

% Figures are downloaded from the prosentation in the following link: https://docs.google.com/presentation/d/1bQtHxZm6iLaqhcPVYYrcx38qx-mTMw7L5rTj3l3xAV8/edit#slide=id.ga23fa2d1cb_0_58
% In the same presentation, one can find also santity checks that were performed for the implementation of the multiple errors.
\begin{figure}[!h]
    \centering         
    \includegraphics[width=0.7\textwidth]{images/Ch6/study_5_non_linear_sps_model_sixtracklib.png}
        \caption{Vertical emittance growth driven by CC RF phase noise as simulated with Sixtracklib simulation tool for a configuration close to the experimental conditions of the SPS CC tests 2018. The measured phase and amplitude noise spectra from Coast3-Setting3 are used for the simulations. The non-linear model (including multipole components of the main dipole magnets) of the SPS machine was used for the tracking. Both types of noise are applied on CC2 which operates at 1\,MV, following Eq.~\eqref{eq:CC_kick_sixtracklib_vertical_noise}, while CC1 is switched OFF.}
        \label{fig:study_5_sixtracklib_global_cc_scheme_measured_spectra_PN_b3b5b7}
\end{figure}

%\textbf{random multiple errros? } Like in APR.Ch.3.2.2.


%- [ ] Further studies that could be mentioned: modulating CC phase and impact of phase offset. Only if they ask me to add them. Slides 34-35 in the following presentation: https://docs.google.com/presentation/d/1wqgJK7uDFDGoX8gyx2-JhnO8mZT51N1VH1RtNL-jhWA/edit#slide=id.ga69ead34aa_0_462


\section{Conclusions and outlook}\label{sec:Ch6_conclusions}
The work presented in this chapter focused on investigating possible explanations for the discrepancy of a factor of about 4 observed between the theoretically predicted and measured emittance growth driven by $\CC$ RF noise during the experiments of 2018 in the SPS. The following points were checked:
\begin{itemize}
    \item Sensitivity to possible uncertainties in the measured parameters and in particular in the $\CC$ voltage and bunch length.
    \item Benchmarking of the theory with two different simulation codes: PyHEADTAIL and Sixtracklib.
    \item Sensitivity of the simulated emittance growth to the detailed optics and the presence of a real $\CC$ element.
    \item Sensitivity to the measured noise spectrum. 
    \item Sensitivity to the non-linearities of the SPS lattice.
\end{itemize}

All these factors were excluded as possible sources of the discrepancy since for all of these study cases the simulated emittance growth demonstrated an excellent agreement with the theoretically predicted values. It was also confirmed that the detailed optics of the SPS, its non-linearities and the crabbing induced from a real $\CC$ element have no impact on the $\CC$ RF noise-induced emittance growth. To this end, in the next chapters, the simulations will be performed with the PyHEADTAIL simulation tool which also provides the possibility of including collective effects (such as the impedance) which were not addressed yet in the context of these studies.