\section{Parametric studies based on the theoretical model}\label{sec:paramteric_studies_theory}
% Link to presentations: https://docs.google.com/presentation/d/1wqgJK7uDFDGoX8gyx2-JhnO8mZT51N1VH1RtNL-jhWA/edit#slide=id.ga69ead34aa_0_419

The basics of the existing theoretical model which describes the emittance growth in the presence of amplitude and phase noise in the $\CC$s in a synchrotron have been introduced in Chapter~\ref{Ch:CC_noise_theory}. Here, this theory is used to study the sensitivity of the noise-induced emittance growth on the $\CC$ voltage and rms bunch length. The objective of the study is to investigate if the uncertainty in the measurements of these two variables could explain the observed discrepancy of a factor $\sim$4 between measured emittance growth and the analytically predicted values (see Section~\ref{sec:meas_2018_vs_theory}).
% The study is limited in these two parameters as they are the only measured ones that could introduce some incertainty in the Eqs. of phase and amplitude noise. In reality there is also some uncertaninty on the noise levels themselves but is not treated here. Nevertheless, extended studies on the noise spectrum are presented later in this section.

The following parametric studies were performed for the experimental configuration of 2018: beam energy of 270\,GeV, vertical beta function of 73\,m (at the location of $\CC$2), and phase and amplitude noise of -111.4 and -115.7\,dBc/Hz respectively (Coast2-Setting2). The phase and amplitude noise are considered here independently, instead of the effective phase noise, due to the different dependence of the correction term (see Fig.~\ref{fig:correction_term_bunch_length}).

\subsection{Sensitivity to bunch length}\label{subsec:bunch_length_dependence}
Using Eqs.~\eqref{eq:dey_an} and~\eqref{eq:dey_pn} with the above mentioned parameters and CC votlage, $V_\mathrm{CC}$=1\,MV the normalised vertical emittance growth is computed as a function of different values of bunch lengths over a range from $10^{-3}$ to 2.5\,ns (expressed in $4\sigma_t$). The results are illustrated in Fig.~\ref{fig:sensitivity_bunch_length_theory_bunch1}. 

A clear dependence of the vertical emittance growth on the bunch length is observed. However, the dependence is strong for short bunches. In the regime of the measured bunch length during the CC experiment for bunch 1 (~1.6ns-2.0ns) the sensitivity to the bunch length is very small and cannot explain the factor of about 4 that was observed between measurements and simualtions in SPS $\CC$ tests in 2018.

% Plotting script: Requires python 3
% https://github.com/natriant/theoreticalModel_for_emitGrowth_due2CCNoise/blob/master/cmpt_GrowthDependence_on_BunchLength_003_for_thesis.py
\begin{figure}[!h]
    \centering         
    \includegraphics[width=0.7\textwidth]{images/Ch6/dey_vs_4sigmat_Coast2-Setting2_withBunches_v2.png}
        \caption{Vertical emittance growth for different bunch length values computed using the analytical formulas Eqs.~\eqref{eq:dey_an} and~\eqref{eq:dey_pn} for the experimental configuration of 2018. The blue dot shows the average rms bunch length over all coasts in 2018. The blue box around it gives the upper and lower limits of its measurements.}
        \label{fig:sensitivity_bunch_length_theory_bunch1}
 \end{figure}


\subsection{Sensitivity to CC voltage}\label{subsec:bunch_length_dependence}
Here, the sensitivity of the vertical emittance growth is studied for the parameters mentioned above and rms bunch length of $4\sigma_t=$1.7\,ns. The vertical emittance growth is computed again analytically using Eqs.~\eqref{eq:dey_an} and~\eqref{eq:dey_pn} over a range of $\CC$ voltage values equally spaced from 0.6 to 1.3\,MV. 

Figure~\ref{fig:sensitivity_VCC_theory_bunch1} illustrates the computed vertical emittance as a function of the CC voltage. From the analysis in 2018, the calibration of the CC voltage which showed 1\,MV was tricky. However, from the plot it is evident that even if the actual voltage was 30$\%$ lower due to errors this would lead to just a factor of 2 lower vertical emittance growth. An error of this scale is not realistic. To this end, it is concluded that uncertainties on the beam based measurements of the CC voltage cannot explain the expereimental observations of 2018, where there was a factor 4 betewen measured and predicted emittance growth.

% Plotting script: Requires python 3
% https://github.com/natriant/theoreticalModel_for_emitGrowth_due2CCNoise/blob/master/cmpt_GrowthDependence_on_Vcc.py
\begin{figure}[!h]
    \centering         
    \includegraphics[width=0.7\textwidth]{images/Ch6/dey_vs_Vcc_Coast2-Setting2.png}
        \caption{Vertical emittance growth for different values of CC voltage computed using the analytical formulas Eq.~\eqref{eq:dey_an} and~\eqref{eq:dey_pn} for the experimental configuration of 2018.}
        \label{fig:sensitivity_VCC_theory_bunch1}
 \end{figure}




\section{Benchmarking theory against PyHEADTAIL}\label{sec:benchmark_theory_with_pyheadtail}

\section{Benchmarking theory against Sixtracklib}\label{sec:benchmark_theory_with_sixtracklib}
This section summarises the benchmarks that were made to validate the theoretical model~\cite{PhysRevSTAB.18.101001} against numerical simulations with Sixtracklib~\cite{sixtracklib_repo} which is a non-linear single-particle tracking librart. The idea is motivated by the fact that PyHEADTAIL and theory may miss some beam dynamics that could explain the discrepancy between their results and the experimental observations of 2018. Sixtracklib is a more complete simualtion tool than PyHEADTAIL as the tracking simualtions use the detailed optics of the machine and therfore is considered appropriate for this study. Calculations can be performed on GPU decreasing the computational time. 



This section is structured as follows. First, a direct comparison between PyHEADTAIL and Sixtracklib is provided by simulating the emittance growth for the experimental conditions of 2018. In the Sixtracklib simulations, the available $\CC$ element is used instead of modeling only the transverse momentum kicks due to phase and amplitude noise in the RF system of the cavity as in PyHEADTAIL. Afterward, the sensitivity of the noise-induced emittance growth in the multipole errors of the main SPS dipoles is tested. Furthermore, the simulations are performed using the measured noise spectra of amplitude and phase noise. Last, the sensitivity of the emittance growth to a possible $\CC$ phase modulation is studied.

\subsection{Implementation of CC element in Sixtracklib}\label{subsec:sixtracklig_CC_implementation}
% Link to corresponding presentation:https://docs.google.com/presentation/d/15038dO7vsEUYxzCsYtQXYlpRIwtVGkA8esGtvKHwBwM/edit#slide=id.g6da2141836_0_5
Sixtracklib provides the possibility to perform tracking simulations in the presence of an actual $\CC$ element on which noise can be added, instead of modeling only the transverse momentum kicks due to phase and amplitude noise in the RF system of the cavity like in PyHEADTAIL.

In Sixtracklib the $\CC$ element is represented by an element (referred to as RFMultipole in the Sixtracklib documentation) that has the properties of an RF cavity and of a magnet of arbitrary order oscillating at a certain frequency. To simulate the vertical $\CC$ it is implemented as a (modulated) skew dipole.

When a particle passes through this element, it recieves the following vertical kick on the momentum:
\begin{equation}\label{eq:CC_kick_sixtracklib_vertical}
    y^\prime_{j+1} = y^\prime_{j} + \cos{\left ( \phi_\mathrm{CC} \frac{\pi}{180} - \frac{2\pi f_\mathrm{CC}}{c} \frac{z \beta_0}{\beta^2} \right )} \frac{V_\mathrm{0,CC}}{p_0 c},
\end{equation}
where $j={0, ...., N_\mathrm{turns}}$ denotes the turn number with $N_\mathrm{turns}$ being the total number of turns that the beam passes through the element. Furthermore, $y^\prime$ is the vertical angle co-ordinate and z the longitudinal co-ordinate of each particle, $\phi_\mathrm{CC}$ the $\CC$ phase in degrees, $f_\mathrm{CC}$ is the $\CC$ frequency, $c$ is the speed of light, $\beta_0$ is the relativisitc beta, $V_\mathrm{0,CC}$ is the amplitude of the $\CC$ voltage and $p_0$ the reference momentum.

The transverse kicks on the momentum for a particle that passes througfrom this element is expressed

Before using this element for the emittance growth simulations its implementation in Sixtracklib is tested. In particular, the induced orbit shift from the $\CC$ element is benchmarked against the theoretically expected orbit shift resulting from a single dipole field error. The latter, has already been discussed in the context of the reconstruction of $\CC$ voltage from the HT monitor in Section~\ref{ sec:Vcc_calibration}.


It is concluded, that using sixtracklib the result of $\CC$ element on the orbit is as expected. % on the closed orbit


\subsection{CC noise induced emittance growth in the presence of global CC scheme}\label{subsec:global_CC}


\textbf{Validating the implementation of the CC element}\\



an RFMultipole element which  has the properties of an RF-cavity and of a magnet of arbitrary order oscillating at a certain frequency. To simulate the vertical CC kick we use it as a modulating (skew) dipole.



\subsection{CC noise induced emittance growth in the presence of local CC scheme}\label{subsec:local_CC}

\newpage



ty. The
benchmarks prove an excellent agreement and demonstrate a solid understanding of the involved
dynamics. The study has been published in Ref. [34].


for the experimental configuration of 2018 

of the existing Vlasov theory for transverse coherent beam instabilities with first-order
chromaticity have been recapitulated in Section 2.2.3. Here, we extend this theory to account for the
beam dynamics effects introduced by nonlinear chromaticity. The chapter contains the work published
in Refs. [34, 76].




\section{Benchmarking with different simulation software}
Benchmarking of theory with pyheadtail (one turn map) and Sixtracklib (element by element tracking).



\subsection{PyHEADTAIL}
- real CC element
- local vs global cc scheme
- How emittance is computed. No dipsersive contribution. studies vertical plane. \\
- It was verified in Sixtracklib simulations that the result from this simplified implementation is equivalent to the simulation with the true CC kick including phase noise.\\
- Need to say why in the simulations we have excitation of only the first betatron sideband. \\
\subsection{Sixtracklib}

Remember that in chapter 6 it was demonstrated that there is no visible difference on the cc rf noise induced emittance growth if the noise kicks are modeled as kicks on the moemntum or the real rf multiple + used in a global or local scheme.


pyheadtail vs sixtracklib: %https://docs.google.com/presentation/d/1rBVKGd9aBZTslaAHf48FdJkNPRyXtMF_T9XpEBlLE4c/edit#slide=id.g8ae11a254a_0_701

\section{Sensitivity to the non-linearities of the main SPS dipoles}
Simulation studies with sixtracklib

\section{Simulations using the measured noise spectrum}
Sixtracklib PyHEADTAIL and the exact machine paramters

\section{Sensitvity studies}
1. Sensitvity to how noisy is the noise spectrum
2. On the CC voltage
3. On the different bunch lengths. 

\section{b3b5b7 multiple errors}
Contribution of the non-linearities with sixtracklib.


All these factors were excluded as possible sources of the discrepancy.