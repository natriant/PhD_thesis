\section{Benchmarking with different simulation software}
Benchmarking of theory with pyheadtail (one turn map) and Sixtracklib (element by element tracking).

\subsection{PyHEADTAIL}
- real CC element
- local vs global cc scheme
- How emittance is computed. No dipsersive contribution. studies vertical plane. 
- It was verified in Sixtracklib simulations that the result from this simplified implementation is equivalent to the simulation with the true CC kick including phase noise.
\subsection{Sixtracklib}

Remember that in chapter 6 it was demonstrated that there is no visible difference on the cc rf noise induced emittance growth if the noise kicks are modeled as kicks on the moemntum or the real rf multiple + used in a global or local scheme.


pyheadtail vs sixtracklib: %https://docs.google.com/presentation/d/1rBVKGd9aBZTslaAHf48FdJkNPRyXtMF_T9XpEBlLE4c/edit#slide=id.g8ae11a254a_0_701

\section{Sensitivity to the non-linearities of the main SPS dipoles}
Simulation studies with sixtracklib

\section{Simulations using the measured noise spectrum}
Sixtracklib PyHEADTAIL and the exact machine paramters

\section{Sensitvity studies}
1. Sensitvity to how noisy is the noise spectrum
2. On the CC voltage
3. On the different bunch lengths. 

\section{b3b5b7 multiple errors}
Contribution of the non-linearities with sixtracklib.


All these factors were excluded as possible sources of the discrepancy.