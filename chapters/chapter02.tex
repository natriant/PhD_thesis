\vspace*{-1mm}
\section{Luminosity}


    $\mathcal{L} = \frac{n_b f_\mathrm{rev}N_1 N_2}{4 \pi \sigma_x \sigma_y} \frac{1}{\sqrt{1+(\frac{\sigma_z}{\sigma_\mathrm{xing}} \frac{\alpha}{2})^2}} $


% From: https://indico.cern.ch/event/634251/contributions/2566349/attachments/1447935/2231509/impact-crossing-angle.pdf
\section{Emittance}

\textbf{Defintion 1}: The statistical emittance is expressed in terms of the beam distribution: % Wolski Eq. (4.101)

\begin{equation}\label{eq:statistical_definition_emit}
    \epsilon_x^{\mathrm{geom}} = \sqrt{\langle x^2 \rangle - \langle px^2 \rangle - \langle x px \rangle^2}.
\end{equation}
This is the geometric emittance. 
\begin{equation}\label{eq:emit_geom_norm_relation}
    \epsilon_x = \epsilon_x^{\mathrm{geom}} \betarel \gammarel.
\end{equation}


\textbf{Defintion 2}:
For a gaussian beam distribution the normalised beam emittance it applies:
\begin{equation}\label{eq:emit_from_beam_size}
    \epsilon_{x} = \frac{\sigma_x(s)^2 - \delta^2 D_x^2(s)}{\beta_x(s)} \betarel \gammarel
\end{equation}

where $\sigma_x(s)$ is the beam size, $\beta_x(s)$ is the beta function, $D_x(s)$ is the dispersion fat a specific location s along the accelerator, $\delta=\Delta p/p0$ is the momentum spread and $\betarel, \gammarel$ the relativistic parameters. Similar expression is valid for the vertical plane, with the difference that there is no dispersion.

\section{Transfer maps}
Need to mention them briefly as you refer to them at the PyHEADTAIL section.

% Courant snyder formalism of longitudinal dynamics: https://journals.aps.org/prab/pdf/10.1103/PhysRevAccelBeams.24.094001
\section{Action angle variables}
The action for the x-plane is:

\begin{equation}\label{eq:action}
    \Jx = \frac{1}{2}(x_n^2 +xp^2_n) 
\end{equation}
where 
\begin{equation}\label{normalised_y}
    x_n = \frac{x}{\sqrt{\beta_x}}, \ \ xp_n = \frac{\alpha_x x}{\sqrt{\beta_x}} + \sqrt{\beta_x}xp
\end{equation}
the normalised coordinates and $\alpha_y, \beta_y$ the twiss parameters. 
The same applies for the y-plane. 

s
The statistical geometric emittance equals the average of the actions distribution:
\begin{equation}\label{eq:geom_emit_actions}
    \emitxgeom = \langle \Jx \rangle
\end{equation}
%\begin{equation}\label{eq:geom_emit_actions}
%    \epsilon_x^{\mathrm{geom}} = \langle \Jx \rangle
%\end{equation}
The distribution of actions is an exponential distribution (further explanation needed?). Therefore, its mean equals its standard deviation. (this property is used in the appendix for the computation of the rms tune spread.)

From Eq.~\eqref{eq:action} we write:
\begin{equation}\label{eq:Jy_exp_distr}
    e^{-J/\epsilon} = e^{(-x^2/2\epsilon - px^2/2\epsilon)}
\end{equation}
From this equation it can be seen that the actions follow an exponential distribution. 

\section{Wakefields and impedances}\label{sec:wakefields_theory}

he terms dipolar and quadrupolar have been chosen as the dipolar wake function acts on the test proton like a dipole magnet (the kick is the same whatever its transverse location), whereas the quadrupolar term acts like a quadrupole magnet (the kick increases linearly with the transverse offset of the test particle). Benoits thesis p.44


\section{Tracking simulation codes}\label{sec:simualtion_codes}
In this section the two tracking simulation codes used in this thesis to study the noise-induced emittance growth are presented. Both codes are macroparticle tracking libraries that simulate the particle motion in the six-dimensional (6D) phase space $(x, p_x, y, p_y, z, \delta)$. The first code performs tracking between interaction points around a circular accelerator with the use of trasnfer matrices while the second one uses the detailed optics model of the machine.

\subsection{PyHEADTAIL}\label{subsec:pyheadtail}

PyHEADTAIL~\cite{pyheadtail_repository} is an open-source 6D macroparticle tracking code, developed at CERN, which was originally designed to study collective effects in circular machines and to be easily extrensible with custom elements. 
% https://indico.cern.ch/event/930271/contributions/3910265/attachments/2066139/3467770/30June2020_Simulations_emit_growthCC_RFnoise.pdf  
Details on its implementation and its features can be find in Refs.~\cite{pyheadtail_manual_adrian, pyheadtail_schenk}. Below the main steps of a simulation are listed: % according to the interest of this thesis.

% Thesis david: https://cds.cern.ch/record/2707064/files/CERN-THESIS-2019-272.pdf (p.45) and for the transverse matrix.

\begin{enumerate}
    \item \textbf{Machine initialisation:} The accelerator ring is splitted into a number of segments of equal length, after each of which there is an interaction point (IP). At the interaction points the macroparticles experience kicks from various accelerator components (feedback system, multipoles etc) or from collective effects such as the wakefields. The machine parameters, such as the circumference, the betatron and synchrotron tunes and the optics at the interaction points are defined. It should be noted that PyHEADTAIL uses smooth approximation which means that only a few segments are defined per turn. % which results to a very  rough optics model.
    \item \textbf{Bunch initialisation:}  A particle bunch is represented by a collection of macroparticles, each of which represents a clustered collection of physical particles. Each macroparticle is described by four transverse $(x, x^\prime, y, y^\prime)$ and two longitudinal coordinates $(z, \delta)$, a mass and an electric charge. For the studies presented in this thesis $10^{5}$ macroparticles are sufficient for an accurate represantation of the bunch, unless it is stated otherwise. There are various distributions available. In this thesis the simulations are performed using a Gaussian distribution in transverse and longitudinal planes.
    \item \textbf{Transverse tracking}: In the transverse plane, the macroparticles are transported from one interaction point to another by linear transfer matrices which take into account the optics parameters at the beginning and the end of the corresponding segment. For example, in the horizontal plane the transport of each macroparticle from IP1 to IP2 along the ring is given by:
    \begin{equation}
        \binom{x_i}{x_i^\prime}_{\mathrm{IP0}} = M \binom{x_i}{x_i^\prime}_{\mathrm{IP1}},
    \end{equation}
    where $i=1, ..., N$ with N being the number of macroparticles and the matrix $M$ is given by:
    \begin{equation}\label{eq:linear_transfer_map}
        M = \begin{pmatrix}
            \sqrt{\beta_{x, \mathrm{IP0}}} & 0 \\ 
            -\frac{\alpha_{x, \mathrm{IP0}}}{\sqrt{\beta_{x,  \mathrm{IP0} }}} & \frac{1}{\sqrt{\beta_{x, \mathrm{IP0}}}}
            \end{pmatrix} \begin{pmatrix}
                \cos(\Delta \mu_{x, \mathrm{IP0 \to IP1 }}) & \sin(\Delta \mu_{x, \mathrm{IP0 \to IP1 }})\\
                 - \sin(\Delta \mu_{x, \mathrm{IP0 \to IP1 }})  &\cos(\Delta \mu_{x, \mathrm{IP0 \to IP1 } }) 
                \end{pmatrix}  \begin{pmatrix}
                    \sqrt{\beta_{x, IP1}} & 0 \\ 
                    -\frac{\alpha_{x, IP1}}{\sqrt{\beta_{x, IP1}}} & \frac{1}{\sqrt{\beta_{x, IP1}}}
                    \end{pmatrix} 
    \end{equation}
    where $\alpha_{x, \mathrm{IP0/IP1}}$ and $\beta{x, \mathrm{IP0/IP1}}$ are the optics paramters at the interaction points IP0 and IP1 respectively and $\Delta \mu_{x, \mathrm{IP0 \to IP1 }}$ the phase advance from IP0 to IP1. As smooth approximation is used, the phase advance equals: 
    \begin{equation}\label{eq:phase_advance_smooth_approximation}
        \Delta \mu_{x, \mathrm{IP0 \to IP1 }} = Q_x \frac{L}{C},
    \end{equation}
    where $Q_x$ is the horizontal betatron tune, $C$ the machine circumference and $L$ the length of the corresponding segment. It should be noted that if no detuning source is added (see next step) the matrix $M$ is the same for all particles.
    \item \textbf{Chromaticity and detuning with transverse amplitude:} The chromaticity (up to higher orders) and amplitude detuning are implemented as a change of the phase advance of each individual particle as follows (example for the horizontal plane):
    \begin{equation}\label{eq:change_phase_advance_detunign}
        \Delta \mu_{x, i \mathrm{IP0 \to IP1 }} = \Delta \mu_{x, \mathrm{IP0 \to IP1 }} + (\xi^{1}_x \delta_i + \alpha_{xx}J_{x, i} + \alpha_{xy}J_{y,i}) \frac{\Delta \mu_{x, \mathrm{IP0 \to IP1 }}}{2\pi Q_x}, 
    \end{equation}
    where $i=1, ..., N$ with N being the number of macroparticles, $\Delta \mu_{x, \mathrm{IP0 \to IP1 }}$ is the phase advance for all macroparticles defined in the previous step, $\xi_x^{1}$ the horizontal chromaticity of first order normalised to the tune $\alpha_{xx}$ and $\alpha_{xy}$ are the detuning coefficients, while $J_x$ and $J_y$ are the horizontal and vertical actions of the macroparticle. Therefore, in the presence of detuning the elements of the $M$ matrix are different for every particle.

    \item \textbf{Longitudinal tracking:}  In PyHEADTAIL the longitudinal particle dynamics are described with the longitudinal equations of motion (ref to eq earlier.) % slide 37: https://www2.kek.jp/accl/legacy/seminar/file/PyHEADTAIL_PyECLOUD_2.pdf 
    The longitudinal coordinates are advanced once per turn after solving numerically the equations of motion~\cite{pyheadtail_schenk}. The motion can be linear or not (non-linear RF system). The studies presented in this thesis use the linear longitudinal tracking.

    \item \textbf{Trasnverse impedance effects:} In PyHEADTAIL, wakefield kicks are used to implement the effect of the transverse impedance in time domain. To improve the computational efficiency, the total impedance of the full machine is lumped in one of the interaction points along the ring and the kicks are applied on the macroparticles once per turn. Additionally, instead of computing the wakefield kicks from each particle to the rest individually, the bunch is divided in a number of longitudinal slices and the macroparticles in each slice recieve a wakefield kick generated by the preceding slices\footnote{This is valid in the ultrarelativistic scenario when no wakefield is generated in front of the bunch. The condition applies for the SPS experiments described in this thesis, performed at 270\,GeV.}~\cite{Salvant:1274254}. This is illustrated schematically in Fig.~\ref{fig:longitudinal_slicing_wakefields}. A large number of slices is required such as the wakes can be assuemed constant within the slice. A high number of macroparticles is also needed in order to avoid statistical noise effects caused by undersampling~\cite{pyheadtail_manual_adrian}. For the studies presented in 500 slices are used over a range of three rms bunch lengths in both directions from the bunch center with the bunch being represented by $10^6$ macroparticles (instead of $10^5$ required for simulations without impedance effects).
    

    \begin{figure}[!ht]
        \centering
        \begin{subfigure}[t]{0.45\textwidth}
            \centering
            \includegraphics[width=1\textwidth]{images/Ch2/before_slicing.png}
            \caption{Without slicing}
            %\label{fig:add_label_here}
        \end{subfigure}
        \hfill
        \begin{subfigure}[t]{0.45\textwidth}
            \centering
            \includegraphics[width=1\textwidth]{images/Ch2/after_slicing.png}
            \caption{With slicing}
            %\label{fig:add_label_here}
        \end{subfigure}
        \hfill
         \caption{Longitudinal bunch slicing for the implementation of wakefields kicks in PyHEADTAIL. Without the slicing technique (left) the wake kicks on the red macroparticle are generated from all the green macroparticles resulting to computationally heavy simulations. Instead, when the bunch is sliced longitudinally (right) the wake kicks on the macroparticles in the red slice $i$ are generated by the macroparticles in the green slices $j$, decreasing significantly the computation time. The figures are a courtesy of M. Schenk~\cite{pyheadtail_schenk}} % bunch passage
         \label{fig:longitudinal_slicing_wakefields}
     \end{figure}
        
    The wakefield kicks are computed using a convolution of the wake function with the moments of each particle. %p.39 michael schenk thesis
    The wake functions are available from detailed imepdance model of the machine which are obtained from a combination of theoretical computations, electromagentic simulations and can be imported in PyHEADTAIL in form of tables. More details on the SPS impedance model are provided in Section~\ref{sec:sps_impedance_model}.
    
    \item \textbf{Data acquisition:} The updated bunch coordinates after each turn are available at IP0 for post processing. Typically, $10^{5}$ turns are required for the noise-induced emittance growth simulation presented in this thesis. 
    
\end{enumerate}


%Last sentence from (LinearMap): https://github.com/PyCOMPLETE/PyHEADTAIL/blob/master/PyHEADTAIL/trackers/longitudinal_tracking.py
%or longutidanl equations of motion: slide 37 https://www2.kek.jp/accl/legacy/seminar/file/PyHEADTAIL_PyECLOUD_2.pdf

Figure~\ref{fig:pyheadtail_accelerator_model} shows a graphic representation of the accelerator model and the tracking procedure supporting the steps described above.
% Graph is created app.diagrams.net and is saved in goodle dirve.

\begin{figure}[!h]
    \centering         
    \includegraphics[width=0.6\textwidth]{images/Ch2/accelerator_model_graph_pyheadtail.png}
        \caption{Graphic represantation of the accelerator model and tracking procedure in PyHEADTAIL (inspered by the graphs in Refs.~\cite{pyheadtail_schenk, inproceedings_ibs_pyheadtail}). In this example the ring is splitted in four segments seperated by the interaction points (IPs). Wakefield and mulitple kicks are applied on the macroparticles in IP2 and IP4. The macroparticles are transported between the IPs by a linear map (which can include detuning effects) in the transverse plane.  The longitudinal coordinates are updated once per turn without being visualised in this plot.}
        \label{fig:pyheadtail_accelerator_model}
 \end{figure}


\subsection{Sixtracklib}\label{subsec:sixtracklib}
%Introduction to sixtrackib: https://indico.cern.ch/event/833895/contributions/3577803/attachments/1927226/3190636/intro_sixtracklib.pdf
% https://inspirehep.net/files/6273430c727ace3796a92d069f651ade