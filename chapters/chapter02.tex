%The definitions and notations used throughout this thesis are based on the book of A. Wolksi in Ref.~\cite{wolski2014} unless it is stated otherwise. 
In this chapter, the basic concepts of accelerator beam physics that are essential for understanding the studies presented in this thesis are introduced. A more complete description can be found in References:~\cite{wolski2014, Wiedemann:1083415, Lee:1425444}. Additionally, in the last section, the tracking simulation codes used in this work are described.

% Details p.11 Michael schenk
Synchrotrons are circular accelerators in which electric fields accelerate the particles while magnetic fields steer and focus them. The magnetic fields are not constant but they vary according to the particles' energy, allowing acceleration and operation at very high (relativistic) energies. The LHC and SPS machines at CERN are synchrotrons like many of the machines used for High Energy Physics experiments. Usually, in synchrotrons, the beams consist of multiple bunches, longitudinally spaced around the machine.  Although the bunches interact with each other, these interactions are not relevant to the studies presented later in this thesis, and will not be considered further


% A) p. 24 M. Schenk
% B) More details in: https://uspas.fnal.gov/materials/09UNM/Unit_8_Lecture_19_Limting_phenomena.pdf

%motion is the microscopic approach where the individual motion of each particle is studied. Coherent motion is the macroscopic approach where the beam is considered as a whole and the motion of the center of mass is studied.

% The section splitting is done here according to Wolski's book.
%\section{Motion of charged particles in electromagnetic fields}
\section{Electromagnetic fields in circular accelerators}\label{sec:EM_fields_intro}

%A circular accelerator uses electric and magnetic fields to steer and focus the beam of charged particles in a prescribed curved path, known as the "closed orbit". % since it repeats (closes on) itself after exactly one turn.
%It might be the case that no particle actually follows the closed orbit but the accelerator  components aim to hold the particles as close as possible to that ideal closed orbit. 
%A beam of charged particles in a circular accelerator is expected to follow a prescribed curved path, known as the "reference trajectory". It is typically chosen such as it passes from the center of all the accelerator magnets. It might be the case that no particle actually follows the reference trajectory but the electric and magnetic fields in an accelerator aim to hold the particles as close as possible to that ideal path. 


%A circular accelerator uses electric and magnetic fields aiming to steer and focus the beam of charged particles in a prescribed curved path, known as the "reference trajectory". % since it repeats (closes on) itself after exactly one turn.

The motion of a particle with charge $q$ and velocity $\mathbf{v}=(v_x, v_y, v_z)$ moving in an electric field $\mathbf{E}$ and a magnetic field $\mathbf{B}$ is influenced by the Lorentz force~\cite{Wiedemann:1083415}: 
\begin{equation}\label{eq:Lorentz_force}
 \mathbf{F}_L = q(\mathbf{E} + \mathbf{v} \times \mathbf{B}).
\end{equation}
% v is tangential to its path.

Note that in this thesis the vectors are denoted in bold font (e.g. $\mathbf{E}$ ). % For the relativistic and ultrarelativistic regime the impacrt of E and B are the same. E = cB. p.44 Wille

From Eq.~\eqref{eq:Lorentz_force} it can be seen that the change of the kinetic energy is achieved only through the interaction with the electric field (due to the cross product of the velocity and the magnetic field). The steering and the focusing of the charged particles can be achieved with both electric and magnetic fields. However, in the relativistic regime (which is the regime of interest for this thesis) the magnetic fields are typically used for guiding the particles since they become more efficient with increasing velocity of the particles~\cite{Wiedemann:1083415}.
% Emili's thesis p.4 and wiedeman p.37

The electric fields which are used for accelerating the beams are generated by radiofrequency (RF) cavities. The magnetic fields are used to steer (dipoles) and focus (quadrupoles) and apply corrections (sextupoles, octupoles, and higher-order multipoles) to the motion of the beam. %The sequence of the various electromagnetic elements around the accelerator ring is called the machine lattice. 


\subsection{Co-ordinate system for the motion of charged particles in electromagnetic fields}\label{subsec:coordinate_system}
%The individual particles do not follow the reference trajectory due to small deviations in their initial conditions: an example trajectory is shown in Fig.~\ref{fig:coordinate_system} with the blue line. T

The co-ordinate system used to describe the individual trajectories of the beam particles around the accelerator is illustrated in Fig.~\ref{fig:coordinate_system} and it is known as Frenet-Serret co-ordinate system.  It consists of the orthogonal co-ordinate system $\Sigma(s) = (\mathbf{e_x}, \mathbf{e_y}, \mathbf{e_z})$ the origin of which moves along the reference trajectory (red line), which is the ideal path of a charged particle according to the design of the accelerator without imperfections.

In this co-ordinate system, the variable $s$ denotes the distance along the reference trajectory. In accelerator physics, $s$ is usually chosen as the independent variable instead of time, $t$.  %why? p.19 https://www.bnl.gov/isd/documents/74289.pdf
Therefore, at any given location $s$ around the ring, the coordinates $(x(s), y(s), z(s))$ give the horizontal, vertical, and longitudinal position of the particle with respect to the origin of the orthogonal moving system $\Sigma$. In the following paragraphs, the dependence of the co-ordinates on the position $s$ along the ring is omitted when possible to facilitate the notation (e.g. $x(s)$ will be denoted as $x$).

\begin{figure}[!h] % at the directory of ipac22
    \centering         
    \includegraphics[width=0.8\textwidth]{images/Ch2/coordinates_particle_motion.png}
        \caption{Co-ordinate system to describe particles' motion in a synchrotron. This is a local co-ordinate system, with the origin following the reference trajectory around the accelerator.  The unit vector $\mathbf{e_z}$ is tangential to the reference trajectory at each point, $\mathbf{e_y}$ is vertical, and $\mathbf{e_x}$ is horizontal, and perpendicular to $\mathbf{e_z}$ and $\mathbf{e_y}$.}
        \label{fig:coordinate_system}
 \end{figure}


 At any point $s$ along the reference trajectory each particle is represented by the six-dimensional phase space vector $(x, x^{\prime}, y, y^{\prime}, z, \delta)$ where:

 \begin{subequations}\label{eq:particle_coordinates}
    \begin{equation}
        x^\prime = \frac{dx}{ds} = \frac{dx}{dt}\frac{dt}{ds} = \frac{v_x}{v_z} =  \frac{p_x}{p_z} \approx \frac{p_x}{p_0},
    \end{equation}    
    \begin{equation}
        y^\prime = \frac{dy}{ds} = \frac{dy}{dt}\frac{dt}{ds} = \frac{v_y}{v_z} =  \frac{p_y}{p_z} 	\approx \frac{p_y}{p_0},
    \end{equation} 
    \begin{equation}
        \delta = \frac{\Delta p}{p_0} = \frac{p-p_0}{p_0},
    \end{equation}
    \begin{equation}
        z = \betarel c (t_0 -t),
    \end{equation}
\end{subequations}

where $p_0$ and $\beta_0$ are the momentum and relativistic (scaled) velocity, respectively, of the reference particle, $t_0$ is the time at which the reference particle arrives at the location $s$ and $t$ is the time at which the individual particle arrives at the same location. The parameter $\delta$ is the relative momentum offset from the reference particle. The longitudinal parameter $z$ indicates the longitudinal offset from the reference particle at the center of the bunch. If $z>0$ ($z < 0$) the corresponding particle arrives earlier (later) than the center of the bunch at an arbitrary reference point. In the ultra-relativistic regime the momentum of the particles in the $\mathbf{e_z}$ direction is much larger than the transvserse ones and almost equals the reference momentum: $p_x, p_y \ll p_z \approx p_0$. This is why the values of $x^\prime$ and $y^\prime$ are close to $p_x/p_0$ and $p_y/p_0$, respectively.

%Furthermore, if the direction of motion of the particle has a small angle with the reference trajectory, the paraxial approximation is valid: $p_x, p_y \ll p_z = p_0$ which means that $x^\prime \approx p_x$ and  $y^\prime \approx p_y$. This approximation is valis for the studies presented in this thesis. \textcolor{red}{$p_0=1$? not clear to me} % ultra relativistic regime, is not mentioned at wolski's book. I wrote it for the p_ 0= 1. but i am not sure.
% p.149 wolski and p.12 Michael schenk

%is the momentum of the reference particle which is given by: $p_0 = \gammarel m_0 v_0$, where $m_0$ is the proton rest mass and
%It can be seen that $x^\prime$ and $y^\prime$ are basically the normalised momenta with the reference momentum $p_0$ and $\delta$ is the relative momentum offset from the reference particle. 

%Initial I have written: Furthermore, the ultra-relativistic regime, the transverse momentum/velocity of a particle is very small comparing to the longitudinal one. Therefore the paraxial approximation is used: $p_x, p_y \ll p_z = p_0 = 1$ which can be applied to simplify Eqs.~\ref{eq:particle_coordinates}. %Or see Andy's explanation for low emittance storage rings: https://cds.cern.ch/record/1982424/plots

%To summarise, the motion of the particles is treated separately in the transverse and longitudinal planes, described by the $(x, x^\prime, y, y^\prime)$ and $(z, \delta)$ co-ordinates respectively. The co-ordinates of the reference particle are $(x=0, x^\prime=0, y=0, y^\prime = 0, z=0, \delta=0 \ (p_z=p_0))$.The $(x, y, z)$ are expressed in meters, $(x^\prime, y^\prime)$ in radians, while $\delta$ is dimensionless.




\textbf{Reference particle}\\
The particle that follows the reference trajectory is called the reference particle and has momentum, $p_0$, energy $E_0$, and velocity $v_0$.  This particle is often called the synchronous particle as it always passes from the center of the RF cavities (assuming constant energy and no energy losses). For a proton, the reference momentum is defined by: $p_0$ = $\gamma_0 m_p v_0$, where $m_p$ is the proton rest mass.


% you need a field that is perpedicular to the velocity vector to the steer the particles
\textbf{Beam rigidity}\\
Assuming that a reference particle moves only under the influence of a uniform vertical dipole field $\mathbf{B_\mathrm{dip}}=(0, B_1, 0)$, it would follow a circular path of local bending radius $\rho$ which is defined by the Lorentz force (Eq.~\eqref{eq:Lorentz_force}) being equal to the centrifugal force, as follows~\cite{wolski2014}: % Eq.(3.45)

\begin{equation}\label{eq:Brho}
    e v_0 B_1 = \frac{\gammarel m_p v_0^2}{\rho} \Rightarrow B_1 \rho = \frac{\gamma_0 m_p v_0}{e} \Rightarrow B_1 \rho = \frac{p_0}{e},
\end{equation}
where $e$ and $m_p$ are the charge and rest mass of a particle respectively, $p_0$ is the reference momentum, and $\gamma_0, \beta_0$ are the relativistic gamma and beta. The quantity, $v_0$, is the velocity of the reference particle.

%In the ultra-relativistic regime which is the case in the studies presented in this thesis, the approximation $\beta_0=1$ is often used.  Reference momentum, $p_0$, is the chosen momentum for which a particle initially following the closed orbit will stay on it (assuming no deviations in its energy through the turns). The quantity, $v_0$, is the velocity of the above mentioned particle and will be referred to as reference velocity.

The quantity $B_1 \rho$, is known as the beam rigidity and is often used in accelerators as a normalisation factor and is a measure of how the charged particles resist bending by a dipolar magnetic field. %If the particle momentum is given in $\mathrm{GeV /c}$ (which are the usual units in high energy accelerators) then the unit of magnetic rigidity is $\mathrm{T \cdot m}$.

%From Eq.~\eqref{eq:Brho} it becomes clear that the magnetic rigidity is another way of stating the reference momentum, $p_0$.
% IMPORTANT!!! -->  v_0 = v_z. You explain this in the next paragraph.



%At this point, it is appropriate to introduce the concept of magnetic rigidity, $B_0 \rho$, which is often used in accelerators as a normalisation factor and is a measure of how the charged particles resist bending by a dipolar magnetic field.

%- https://uspas.fnal.gov/materials/12MSU/xverse_dynamics.pdf



%\textbf{Reference trajecotry and reference momentum}\\
% Reference trajectory and reference moomentum are purely mathematical definitions. We chose them, usually we chose them to be convinient for the discreption. 
%Here the concepts of reference momentum and reference trajectory are introduced. They both can be chosen arbitrary and are used as a reference for describing the particles motion in the magnetic fields of an accelerator. It might be the case that no particle follows the reference trajectory and has the reference momentum.

%The reference trajectory is a chosen curved path in space which acts as a reference for the alignment of the accelerator compoenents. It is defined by the dipoles and is typically chosen such as it passes from the center of all the magnets. % not always the case and in some cases on purpose
%The total length of the reference trajectory in a circular accelerator is called the circumference, $C_0$.


% Definition based on Wolski's book p.72.
% Alternative: The reference trajectroy is chosen by us, usually for convinience is chosen to be the one followed by the reference particle.

%The particle that follows this trajectory is called the reference particle and has a momentum $p_0$, an energy $E_0$, and a velocity $v_0$. This particle is often called the synchronous particle as it crosses an RF cavity always at the same phase (assuming constant speed and no losses). %https://indico.cern.ch/event/22574/contributions/475143/attachments/371243/516589/IntroductionToAccelerators.pdf slide 36
%For a proton, the reference momentum is given by: $p_0 = \gammarel m_p v_0$, where $m_p$ is the proton rest mass, and $\gamma_0 = \frac{1}{\sqrt{1-\beta_0^2}}$ is the relativistic gamma or Lorentz factor, where $\beta_0=v_0/c$ is the relativistic $\beta$ with $c$ being the speed of light. 


\textbf{Mulitpole fields}\\
In high-energy synchrotrons, the magnetic field generated by the synchrotron magnets can usually be considered in “hard edge” approximation. In this case,
the magnetic fields of various magnet types have purely transverse components and can be expressed through the following multipole expansion:

\begin{equation}\label{eq:mult_expansion} % Andy Eq.1.29 + M Schenk Eq.2.2
    B_y(x,y) + i B_x(x,y) = \sum_{n=1}^{\infty} C_n (x+i y)^{n-1},
\end{equation} % principle of superposition 
where $n$ indicates the order of the field component: $n$=1 for a dipole (steering), $n$=2 for quadrupole (focusing), $n$=3 for sextupoles (chromaticity correction), $n$=4 for octupole (error or field correction) etc, $C_n=(b_n +i a_n)$ is a complex constant which denotes the strength and orientation of the multipole field. The coefficients $b_n=\frac{1}{(n-1)!} \frac{\partial^{n-1}B_y}{\partial x^{n-1}}$ and $a_n=\frac{1}{(n-1)!} \frac{\partial^{n-1}B_x}{\partial x^{n-1}}$ denote the strength of a normal and skew (normal multipole rotated by $\pi/2(n)$) multipole respectively.% in units of $\mathrm{T/m^{n-1}}$.
% Equations for coefficients, sofia's thesis Eq.(2.29) p. 23: https://cds.cern.ch/record/2743602/files/CERN-THESIS-2020-169.pdf
% skew quadrupole is a normal quadrupole rotated by 45 deg.

%The motion of charged particles inside a circular accelerator is controlled by magnetic fields. In this thesis, the magnets are considered purely transverse elements. Their effect is therefore described with two-dimensional multipole fields, acting in the horizontal and vertical planes\footnote{Examples of three-dimensional treatment can be found in~\cite{wolski2014, Beth:889480}. However, the two-dimensional treatment is most often used in accelerator physics as it provides a good description for the majority of the magnetic elements.}. % Wolski Ch1.3, p.33

%The description of two-dimensional magnetic fields in accelerator physics is discussed using the concept of multipole expansion and is expressed as a complex quantity. The complex quantity allows to describe a two-dimensional field in $(x,y)$ space (to be compatible with the co-ordinates used for describing the particle's trajectory as discussed in the previous section). Therefore, the magnetic field around the beam is expressed as follows~\cite{wolski2014}: 
%\begin{equation}\label{eq:mult_expansion} % Andy Eq.1.29 + M Schenk Eq.2.2
%    B_y(x,y) + i B_x(x,y) = \sum_{n=1}^{\infty} C_n (x+i y)^{n-1},
%\end{equation} % principle of superposition 
%where $n$ indicates the order of the field component: $n$=1 for a dipole (steering), $n$=2 for quadrupole (focusing), $n$=3 for sextupoles (chromaticity correction), $n$=4 for octupole (error or field correction) etc. $C_n=(b_n +i a_n)$ is a complex constant which denotes the strength and orientation of the multipole field. The coefficients $b_n=\frac{1}{(n-1)!} \frac{\partial^{n-1}B_y}{\partial x^{n-1}}$ and $a_n=\frac{1}{(n-1)!} \frac{\partial^{n-1}B_x}{\partial x^{n-1}}$ denote the strength of a normal and skew (normal multipole rotated by $\pi/2(n)$) multipole respectively in units of $\mathrm{T/m^{n-1}}$.
% Equations for coefficients, sofia's thesis Eq.(2.29) p. 23: https://cds.cern.ch/record/2743602/files/CERN-THESIS-2020-169.pdf
% skew quadrupole is a normal quadrupole rotated by 45 deg.

In accelerator physics the values of the multipole strengths are often quoted normalised to the magnetic rigidity as defined in Eq.~\eqref{eq:Brho} and are denoted by:
\begin{equation}\label{eq:kn}
    k_n = \frac{b_n}{B_1 \rho},
\end{equation}
%where $k_n$ is expressed in units of $\mathrm{m^{-n}}$. 
This is the convention that will be used in this thesis. %By definition $b_1=B_1$.\textcolor{red}{Is the last sentence correct?}
By definition for a pure dipole vertical field $b_1=B_1$.%\textcolor{red}{Is the last statement correct?}

%It should be clarified here, that the notations of $B_0$ and $b_1$ are equivalent. However, in accelerator physics the magnetic rigidity is denoted as $B_0 \rho$ as a convention due to its derivation at the early stages of the analysis (see Eq.\eqref{eq:Brho}). This notation will be used throughout this thesis.\textcolor{red}{is that ok? the way i express it here?}
% confirmed because k1 = b1/B0ρ = 1/ρ



\section{Single-particle beam dynamics}\label{sec:single_particle_dynamics}
In this section, the interactions between the particles within a bunch are neglected, hence the term single-particle beam dynamics. 

%\subsection{Transverse particle motion}\label{subsec:transverse_particle_motion}
%The motion of a relativistic particle in static electromagnetic field is defined by the Lorentz force. The Hamiltonian which describes its motion equals the total energy of the system and is written as~\cite{Lee:1425444}:% Lee Eq.(2.4)
%\begin{equation}
 %   H = c \sqrt{m_p^2 c^2+ (\mathbf{p}-q\mathbf{A})}+q \Phi,
%\end{equation}
%where 

%The particle motion is described here with the Hamiltonian formalism in the co-ordinate system of Fig.~\ref{fig:coordinate_system}. The Hamiltonian which describes the motion of a relativistic particle in a static electromagnetic field depends periodically on $s$ and is given by~\cite{Lee:1425444}: %Eq.(2.14)

%\begin{equation}\label{ref:Hamiltonian_general}
 %   H = - \left ( 1 + \frac{x}{\rho} \right ) \sqrt{\left [ \frac{E}{c^2}- m^2 c^2-(p_x-eA_x)-(p_y-eA_y)^2\right]}-eA_s
%\end{equation}



\subsection{Transvserse motion}
In the transverse plane the motion is orthogonal to the reference trajectory (see Fig.~\ref{fig:coordinate_system}) and its co-ordinates are $(x, x^\prime, y, y^\prime)$. For the discussion on the transverse beam dynamics, $(u, u^\prime)$ will be used to notate either $(x, x^\prime)$ or $(y, y^\prime)$.

%\subsubsection{Linear dynamics}
%\subsubsection{Equations of motion in a linear lattice}
Here the transverse motion of a particle moving through the two-dimensional magnetic fields described in Eq.~\eqref{eq:mult_expansion} is discussed. For now, the discussion is limited only to dipolar and quadrupolar components ($n=1$ and $n=2$) of the multipole expantion which are the basic magnetic elements of a synchrotron, providing bending and focusing of the particle beam. 

As mentioned above, the particles transversely oscillate around the reference trajectory. This motion through a sequence of dipoles and quadrupoles, is called betatron motion and is described by the following equations of motion~\cite{Lee:1425444}: % also sofia's thesis Eq.(2.37)

 \begin{equation}\label{eq:transverse_eq_x_non_linear}
   x^{\prime \prime} - \frac{\rho+x}{\rho^2} = - \frac{B_y}{B_1 \rho} \frac{p_0}{p} \left (  1+ \frac{x}{\rho} \right )^2, 
 \end{equation}

\begin{equation}\label{eq:transverse_eq_y_non_linear}
    y^{\prime \prime} = \frac{B_x}{B_1 \rho} \frac{p_0}{p}  \left (  1+ \frac{x}{\rho} \right )^2, 
\end{equation}
where $B_1 \rho$ and $\rho$ the magnetic rigidity and local bending radius as defined in Eq.~\eqref{eq:Brho},  $B_y, B_x$ the transverse magnetic fields of Eq.~\eqref{eq:mult_expansion}, and $p_0$ the reference momentum.

Since the amplitudes of the betatron oscillations are usually small, linear approximations in the above equations of motion provide good representations of the beam dynamics. Expanding Eqs.~\eqref{eq:transverse_eq_x_non_linear} and~\eqref{eq:transverse_eq_y_non_linear} up to the first order in $x$ and $y$ and taking into account the possible momentum deviation, $\delta$, of particle we write~\cite{Lee:1425444}: % This linear expantion is only valid at the vincinity of the reference orbit.

\begin{equation}\label{eq:eq_of_motion_horizontal_linear}
    x^{\prime \prime} + \left (  \frac{1-\delta}{\rho^2 (1+\delta)}  + \frac{K_2(s)}{1+\delta} \right )x = \frac{\delta}{\rho (1+\delta)}, 
\end{equation}
\begin{equation}\label{eq:eq_of_motion_vertical_linear}
    y^{\prime \prime} -  \left ( \frac{K_2(s)}{1+\delta} \right ) y = 0,
\end{equation}
where $\delta = (p-p_0)/p_0$ is the relative momentum offset from the reference particle, and $K_2=B_2 /(B_1\rho)$ is the focusing function. The sign-convention here is $K_2(s)>0$ for focusing (defocusing) in the horizontal (verticla) and vertical plane.

\textbf{Solutions for on-momentum particles}\\
For on-momentum particles (with $\delta=0$) %and neglecting effects from dipole field errors % from Hannes' thesis
these linear equations of motion can be simplified even more to the equation of motion for an harmonic oscillator (but with an $s$ dependent strength $K_u(s)$), named Hill's equation~\cite{Lee:1425444}:
\begin{equation}\label{eq:Hills_equation_1}
    u^{\prime \prime}(s) + K_u(s) u(s) = 0,
\end{equation}
where $u=(x,y)$ and:
 \begin{equation}\label{eq:Hills_equation_2}
    K_u(s) = \begin{dcases}
        \frac{1}{\rho^2(s)}+K_2(s), & u=x \\
        -K_2(s), & u=y 
    \end{dcases}
\end{equation}
 It should be noted that for Eq.~\eqref{eq:Hills_equation_1} it is assumed that the motion of the particle in the horizontal and vertical planes are independent, i.e. there is no transverse coupling. %The derivation of why the Hill's equation describes the betatron oscillations for on-momentum particles can be found in~\cite{Latina_juas}. 

 % From Emilia
%For constant $K_u$ the solution of Hill's equation is~\cite{Lee:1425444}:
%\begin{equation}\label{eq:Hills_equation_2_ku_constant}
%    u(s) = \begin{dcases}
%        C_1 \cos{(\sqrt{K_u}s)}  + C_2 \sin{(\sqrt{K_u}s)} &  K_u >0 \\
%        C_1 \cosh{(\sqrt{|K_u|}s)}  + C_2 \sinh{(\sqrt{|K_u|}s)} &  K_u <0 
%    \end{dcases},
%\end{equation}
%\begin{equation}\label{eq:Hills_equation_2_ku_constant_prime}
%    u^\prime(s) = \begin{dcases}
%        -\sqrt{K_u} C_1 \sin{(\sqrt{K_u}s)}  + \sqrt{K_u} C_2 \cos{(\sqrt{K_u}s)} &  K_u >0 \\
%        \sqrt{K_u}  C_1 \sinh{(\sqrt{-K_u}s)}  + \sqrt{K_u} C_2 \cosh{(\sqrt{-K_u}s)} &  K_u <0 
%    \end{dcases},
%\end{equation}
%where $C_1$ and $C_2$ constants of integration.  

%Assuming the initial conditions for $s=0$: $x(0)=x_0$ and $x^\prime(0)=x^\prime_0$, the constants of integration are: $C_1=x_0$ and $C_2=x^\prime_0/\sqrt{K_u}$ and the solution becomes:
%\begin{equation}\label{eq:Hills_equation_2_ku_constant_solution}
%    u(s) = \begin{dcases}
%        C_1 \cos{(\sqrt{K_u}s)}  + C_2 \sin{(\sqrt{K_u}s)} &  K_u >0 \\
%        C_1 \cosh{(\sqrt{|K_u|}s)}  + C_2 \sinh{(\sqrt{|K_u|}s)} &  K_u <0 
%    \end{dcases},
%\end{equation}
%\begin{equation}\label{eq:Hills_equation_2_ku_constant_prime_solution}
%    u^\prime(s) = \begin{dcases}
%        -\sqrt{K_u} C_1 \sin{(\sqrt{K_u}s)}  + \sqrt{K_u} C_2 \cos{(\sqrt{K_u}s)} &  K_u >0 \\
%        \sqrt{K_u}  C_1 \sinh{(\sqrt{-K_u}s)}  + \sqrt{K_u} C_2 \cosh{(\sqrt{-K_u}s)} &  K_u <0 
%    \end{dcases},
%\end{equation}

%Here the transverse motion of a particle moving through the two-dimensional fields described in Eq.~\eqref{eq:mult_expansion} is discussed. For now, the discussion is limited only to dipolar and quadrupolar components ($n=1$ and $n=2$) hence the name linear dynamics\footnote{The term "linear" refers not to the fields, but to the beam dynamics (specifically, the equations describing the particle motion through the fields). In reality, dipole and quadrupole fields can also lead to "nonlinear dynamics". However, in these cases, linear approximations to the equations of motion provide good representations of the dynamics, at least in the case of the transverse motion.}. Dipoles and quadrupoles are considered the basic magnetic elements, as in the absence of magnetic errors or momentum deviations between the particles they are sufficient to create a synchrotron. 

%As mentioned above, the particles (except for a nominal reference particle) transversely oscillate around the reference trajectory. This motion, through an arbitrary periodic sequence of dipoles and quadrupoles, is called betatron motion and can be described with the following equations of motion~\cite{Lee:1425444}: % also sofia's thesis Eq.(2.37)
%\begin{equation}\label{eq:transverse_eq_x}
%    x^{\prime \prime} - \frac{\rho+x}{\rho^2} = - \frac{B_y}{B_0 \rho} \frac{p_0}{p} \left (  1+ \frac{x}{\rho} \right )^2, 
%\end{equation}

%\begin{equation}\label{eq:transverse_eq_y}
%    y^{\prime \prime} = \frac{B_y}{B_0 \rho} \frac{p_0}{p}  \left (  1+ \frac{x}{\rho} \right )^2, 
%\end{equation}
%where $s$ is the distance along the reference trajectory, $B_0 \rho$ and $\rho$ the magnetic rigidity and radius as defined in Eq.~\eqref{eq:Brho},  $B_y, B_x$ the transverse magnetic fields of Eq.~\eqref{eq:mult_expansion}, and $p_0$ the reference momentum.

%As already mentioned, Eq.~\eqref{eq:Hills_equation_1} resembles the equation of motion for a harmonic oscillator, but with an oscillation frequency that varies with position along the beamline (i.e. varies with s). 

%\textbf{Solution of Hill's equation in a circular accelerator}
%For a circular accelerator $K_u$ is periodic: $K_u(s+C_0)=K_u(s)$, where $C_0$ is the circumference of the accelerator. The general solutions of Hill's equations in this case is (using the Floquet transformation)~\cite{Lee:1425444}:
%\begin{equation}\label{eq:Hills_solution}
%    u(s) = A w(s) \cos{(\psi_u(s)+\psi_{u,0})},
%\end{equation} % Lee p. 48
%where $A$ and $\psi_{u,0}$ the integration constants and $w(s)$ and $\psi_u(s)$ are the amplitude and betatron phase functions, which are periodic functions with the same periodicity as $K_u$.
% The following from David Amorim, in p.2
For a circular accelerator $K_u$ is periodic: $K_u(s+C_0)=K_u(s)$, where $C_0$ is the circumference of the accelerator. In this case, the solution of Eq.~\eqref{eq:Hills_equation_1} resembles the one of a harmonic oscillator and can be written as~\cite{Lee:1425444}:
%  (using the Floquet transformation)~\cite{Lee:1425444}:
\begin{equation}\label{eq:solution_eq_motion_u}
    u(s) = \sqrt{2 \beta_u(s) J_u} \cos{(\psi_u(s)+ \psi_{u,0})},
\end{equation} 
\begin{equation}\label{eq:solution_eq_motion_u_prime}
    u^\prime(s) = - \sqrt{\frac{2 J_u}{\beta_u(s)}} (\sin{(\psi_u(s) + \psi_{u,0})}+\alpha_u(s)\cos{(\psi_u(s)+ \psi_{u,0})}),
\end{equation} 
where $J_u$ and $\psi_{u,0}$ are the constants of integration and are determined by the chosen initial conditions, $\alpha_u(s)=-\frac{1}{2} \beta^\prime_u(s)$ and $\beta_u(s)$ are periodic function determined by the sequence of the quadrupole magnets in the accelerator and follow the periodicity of the machine. %Note that divergence $u^\prime$ is obtained by differentiation of the parameter $u$. 
Equation~\eqref{eq:solution_eq_motion_u} shows that a stable particle undergoes trasnverse oscillations around the reference trajectory, which are referred to as betatron oscillations. Since the $J_u$ is constant, the beta function determines the maximum amplitude of the single particle trajectory at any given position $s$ in the ring.
% this highlights its phyiscal significance.
% Last sentence from here: https://indico.cern.ch/event/407247/contributions/975702/attachments/1246508/1835998/Alexandra_p.pdf
The function $\psi_u(s)$ is the betatron phase advance from the position $s=0$ to $s$ and equals:
\begin{equation}\label{eq:phase_advance_definition_with_twiss}
    \psi_u(s)= \int_{0}^{s} \frac{ds}{\beta_u(s)}.
\end{equation}

\textbf{Betatron tune}\\
The phase advance for one complete revolution around the machine divided by $2\pi$ defines the betatron tune, $Q_u$:
\begin{equation}\label{eq:betatron_tune}
    Q_u = \frac{\psi_u(s+C_0)-\psi_u(s)}{2\pi} = \frac{1}{2\pi} \oint_{C_0} \frac{ds}{\beta_u(s)},
\end{equation}
where $C_0$ is the circumference of the machine. The betatron tune represents the number of betatron oscillations that a particle performs during one full revolution around the machine.

The tune of the individual particles may vary due to effects such as the chromaticity, the detuning with their transverse amplitude, and collective forces (e.g. impedance) that will be discussed in the following paragraphs. The horizontal and vertical tune of the reference particle will be referred to as the bare tunes and define what is called the working point of the machine, $(Q_{x0}, Q_{y0})$. 

\textbf{Courant-Snyder ellipse}\\
Inserting $\cos{(\psi_u(s)+ \psi_{u,0})} = u(s)/(\sqrt{\beta_u(s) J_u})$ from Eq.~\eqref{eq:solution_eq_motion_u} to Eq.~\eqref{eq:solution_eq_motion_u_prime} it is found that:

\begin{equation}\label{eq:action_definition}
    J_u = \frac{1}{2} (\gamma_u(s) u^2(s) + 2 \alpha_u(s) u(s) u^\prime(s) + \beta_u(s) u^{\prime 2}(s)),
\end{equation}
where $\gamma_u(s)=\frac{1+\alpha_u(s)^2}{\beta_u(s)}$. Eq.~\eqref{eq:action_definition} describes an ellipse in phase space $(u, u^\prime)$ at any given position $s$ in the ring. This ellipse is known as phase space or Courant-Snyder ellipse and is illustrated in Fig.~\ref{fig:phase_space_ellipse}. The parameter $J_u$ is also known as action or the Courant-Snyder invariant. The parameters $\alpha_u(s), \beta_u(s)$, and $\gamma_u(s)$ are the Courant-Snyder or the Twiss parameters and they define the shape and the orientation of the ellipse. The center of the ellipse is the closed orbit which, in the absence of steering errors in a synchrotron, can be identified with the reference trajectory and is also shown in the plot. The area of the phase space ellipse as defined here equals: $2\pi J_u$ and remains the same at any given location $s$.

The trajectory of each individual particle turn after turn follows the phase space ellipse described by Eq.~\eqref{eq:action_definition} at any given position $s$. It is worth mentioning, that the ellipse's size is different for each particle as it depends on their individual actions, $J_u$.% i.e. their individual initial conditions. 

%The trajectory of each individual particle can be plotted in phase space $(u, u^\prime)$ at a given position $s$ in the ring turn after turn. In phase space, the particle's path is an ellipse whose shape and orientation are determined by the Twiss parameters at the position $s$. This ellipse, named phase space or Courant-Snyder ellipse, is illustrated in Fig.~\ref{fig:phase_space_ellipse} and it has an area of $2\pi J_u$. It is worth mentioning, that the ellipse's size is different for each particle as it depends on their individual actions, $J_u$ i.e. their individual initial conditions.

\begin{figure}[!h] % at the directory of ipac22
    \centering         
    \includegraphics[width=0.6\textwidth]{images/Ch2/phase_space_ellipse.png}
        \caption{Phase space co-ordinates $(u, u^\prime)$ turn by turn, for a particle moving along the ring but at a particular position $s$ which is characterised by the following twiss parameters $[\alpha_u(s), \beta_u(s), \gamma_u(s)]$.} %In the labels shown on the diagram, the dependence on the $s$ parameter has been omitted.}
        \label{fig:phase_space_ellipse}
 \end{figure}


 \textbf{Transvserse emittance}\\
 %The concept of the beam tranverse impedance is discussed here in detail since it is one of the main parameters of interst of this project. 
Up to now, the Twiss parameters were used to describe the dynamics of single particles. However, they can also describe the distribution of the particles within a bunch. The statistical average of $u^2$ over all particles at a given point $s$ along the reference trajectory, from Eq.~\eqref{eq:solution_eq_motion_u} equals to~\cite{wolski2014}:
 \begin{equation}\label{eq:statistical_average_position}
     \langle u^2(s) \rangle = 2 \beta_u(s) \langle J_u \cos^2{\psi_u(s)} \rangle.
 \end{equation}
 %\begin{equation}\label{eq:mean_u_zerp}
  %   \langle u(s) \rangle = 0,
 %\end{equation}
 Assuming that the angle and action variables are uncorrelated Eq.~\eqref{eq:statistical_average_position} becomes:
 \begin{equation}\label{eq:statistical_average_position_2}
     \langle u^2(s) \rangle = 2 \beta_u(s) \langle J_u \rangle \langle \cos^2{\psi_u(s)} \rangle.
 \end{equation}
 Considering that the angle variables are uniformly distributed from 0 to $2\pi$: % mean of a function: https://en.wikipedia.org/wiki/Mean_of_a_function
 \begin{equation}\label{eq:mean_cosine_square}
     \langle \cos^2{\psi_u(s)} \rangle = \int_{s_0}^{s_0+C_0} \cos^2{\psi_u(s)} ds =   \int_0^{2\pi} \cos^2{\psi_u(\phi)} d\phi = \frac{\pi}{2\pi} = \frac{1}{2},
 \end{equation}
 where for the integration the phase advance $\phi$ is used instead of the location $s$ along the ring for convenience. % like in Eq. (2.20)
 
 Inserting Eq.~\eqref{eq:mean_cosine_square} in Eq.~\eqref{eq:statistical_average_position} gives:
 
 \begin{equation}\label{eq:emittance_definition_1} %Eq(4.97) wolski
     \langle u^2(s) \rangle = \beta_u(s)  \epsilon^{\mathrm{geom}}_u,
 \end{equation}
 where
 \begin{equation}\label{eq:geom_emittance_action}
     \epsilon^{\mathrm{geom}}_u=\langle J_u \rangle
 \end{equation}
 is the geometric emittance of the bunch. Assuming again that the action and angle variables are uncorrelated and that the latter are uniformly distributed from 0 to $2\pi$, Eq.~\eqref{eq:solution_eq_motion_u} and Eq.~\eqref{eq:solution_eq_motion_u_prime} results to:
 \begin{equation}\label{eq:u_uprime_eq_1}
     \langle u(s) u^\prime(s) \rangle = - \alpha_u(s) \epsilon^{\mathrm{geom}}_u,
 \end{equation}
 \begin{equation}\label{eq:u_uprime_eq_2}
     \langle u^{\prime 2}(s) \rangle = \gamma_u(s) \epsilon^{\mathrm{geom}}_u.
 \end{equation}
 
 Combining the above equations, the geometric emittance is expressed in terms of the particles' distribution as~\cite{wolski2014}:
 \begin{equation}\label{eq:geometric_emittance_v2}
     \epsilon^{\mathrm{geom}}_u = \sqrt{\langle u^2(s) \rangle \langle u^{\prime 2}(s) \rangle- \langle u (s)u^{\prime}(s) \rangle ^2}
 \end{equation}
 %which, also equals the square root of the determinant of the covariance or %Sigma matrix of the particles' distribution:
 %\begin{equation}\label{eq:Sigma_matrix_particles}
 %   \Sigma = \begin{pmatrix}
 %        \langle u^2(s) \rangle & \langle u(s) u^\prime(s) \rangle \\ 
 %        \langle u(s) u^\prime(s) \rangle & \langle u^{\prime 2}(s) \rangle 
 %        \end{pmatrix}  = \begin{pmatrix}
 %            \sigma_u^2(s) & \langle u(s) u^\prime(s) \rangle \\ 
 %            \langle u(s) u^\prime(s) \rangle & \sigma_{u^{\prime 2}}(s)
 %            \end{pmatrix} 
 %\end{equation}
 
For a Gaussian distribution with $\langle u \rangle=0$, Eq.~\eqref{eq:emittance_definition_1} becomes~\cite{wolski2014}: 
\begin{equation}\label{eq:emittance_beam_size}
    \sigma_u(s) = \sqrt{\beta_u(s) \epsilon^{\mathrm{geom}}_u},
\end{equation}
where $\sigma_u$ is the transverse rms beam size.



 %The square root of the top-left element of the covariance matrix, $\sigma_u$, is defined as the rms transverse beam size. %The definition of rms and of others of the statistical analysis can be found in Appendix~\ref{ch:app_A}.
 
 %Figure~\ref{fig:phase_space_emittance} illustrates the concepts of emittance and rms beam size. It shows the phase space of a transverse Gaussian bunch along with the histograms of the $u$ (top) and $u^\prime$ (right) variables at a particular point $s$ along the ring. Each particle follows its individual ellipse (of different sizes but with the same orientation) depending on its initial conditions. The rms beam size, $\sigma_u$, and the rms normalised momentum spread, $\sigma_{u^\prime}$, are shown in the top and right histograms of Fig.\ref{fig:phase_space_emittance} with the blue vertical lines. This corresponds to the area of the ellipse enclosed in the blue line in the phase space plot and equals the rms or geometric emittance, $\epsilon^{\mathrm{geom}}_u$ , as defined in Eq.~\eqref{eq:geometric_emittance_v2}.
 
 %\begin{figure}[!h] 
 %    \centering         
 %    \includegraphics[width=0.9\textwidth]{images/Ch2/transverse_phase_space_emittance.png}
 %        \caption{Transverse phase space of a Gaussian bunch~\cite{tirsi_thesis_presentation}.}
 %        \label{fig:phase_space_emittance}
 % \end{figure}
 
 %It should be noted, that there are also other conventions to define the emittance such as the 90$\%$ emittance (green lines in Fig.~\ref{fig:phase_space_emittance}) or the 3-sigma emittance (yellow lines in Fig.~\ref{fig:phase_space_emittance}). However, here the term geometric emittance will refer to the rms geometric emittance.
 
 According to Liouville’s theorem~\cite{wolski2014}, assuming that there are no interactions between the particles and that the energy of the beam is not changing, the geometric emittance remains constant and therefore is an invariant of bunch motion (similarly to the action $J_u$ for the single-particle motion). The geometric emittance does not remain constant during acceleration. Instead, the normalized emittance defined as~\cite{wolski2014}:
 \begin{equation}\label{eq:normalised_emittance}
     \epsilon_u = \beta_0 \gamma_0 \epsilon^{\mathrm{geom}}_u
 \end{equation}
is conserved during acceleration and is often used as an alternative to the geometric emittance, especially in situations where the beam undergoes acceleration or deceleration. It is highlighted here, that throughout this thesis the term "emittance" will refer to the rms normalised emittance.
 
 It is worth noting that for the simulation studies presented in this thesis, the emittance is computed using the statistical definition introduced in Eq.~\eqref{eq:geometric_emittance_v2}. In the experimental studies, the emittance is obtained from the rms beam size at a point in the beamline where the beta function is known following Eq.~\eqref{eq:emittance_beam_size}. These two methods of computing the emittance are considered equivalent for the stdudies presented in this thesis.
 
 %. The procedure is explained in more detail in Chapter~\ref{Ch:2018_analyisis}, but here we simply note that from Eqs.~\eqref{eq:emittance_definition_1},~\eqref{eq:Sigma_matrix} and ~\eqref{eq:normalised_emittance}, we can express the emittance as:
 % For a Gaussian beam distribution the normalised beam emittance it applies:
 %\begin{equation}\label{eq:emit_from_beam_size}
 %    \epsilon_{u} = \frac{\sigma_u(s)^2}{\beta_u(s)} \beta_0 \gamma_0,
 %\end{equation}
 
 %where $\sigma_u(s)$ is the rms beam size, $\beta_u(s)$ is the beta function, at specific location $s$ along the accelerator and $\beta_0, \gamma_0$ are the relativistic parameters. 
 
 %It should be highlighted that the emittance definitions of Eq.~\eqref{eq:geometric_emittance_v2} (after normalisation with the relativistic parameters) and of Eq.~\eqref{eq:emit_from_beam_size} are equivalent. 
 
Despite Liouville's theorem, in a real accelerator there are various phenomena that change the emittance, such as~\cite{Buon:216507}: scattering by residual gas, intra-beam scattering, stochastic or electron cooling, synchrotron radiation emission, filamentation due to non-linearities of the machine, space charge and noise effects. The studies in this thesis focus on the emittance growth due to noise effects.
 

\textbf{Transfer maps and linear transfer matrix}\\
% information and introduciton to maps:https://www.researchgate.net/publication/50819226_Direct_measurement_of_resonance_driving_terms_in_the_super_proton_synchrotron_SPS_of_cern_using_beam_position_monitors
The motion of the particles through accelerator components can be represented by transfer maps. A transfer map is a set of functions that yields the final set of phase space coordinates as a function of the inital one.
The transfer maps can be found from the equations of motion, which can be obtained from Hamilton's equations. For each component, the Hamiltonian that describes its full dynamics is used. Further details on this approach can be found in~\cite{wolski2014}. % p.282 wolski 
% between them drifts.
%Furthermore, the work presented in this thesis uses the thin lens approximation which assumes that the length of all the magnetic components tends to zero.
% drift kick drift.

% useufull andy's presnetation: /Users/nataliatriantafyllou/Downloads/NonlinearDynamics1_-_Handouts.pdf
The transfer maps of the linear magnetic elements, such as dipoles and quadrupoles can be written in terms of matrices. The Twiss parameters can be used to describe the linear transport of a particle in the accelerator from the position $s_0$ to the $s_1$ using the matrix formalism as follows~\cite{Lee:1425444}: %Eq.(2.42) lee
% w and psi are also solutions. now we write it in matrix notation
%Knowing the lattice (element per element structure of the accelerator) the solutions $w_u(s)$ and $\psi_u(s)$ of the Hill's equation can also be described using a matrix formalism as follow

\begin{equation}\label{eq:matrix_formalism_intro}
   \begin{pmatrix}
    u\\ 
    u^\prime
    \end{pmatrix}_{s_1} = M_u (s_1 |  s_0) \begin{pmatrix}
    u\\ 
    u^\prime
    \end{pmatrix}_{s_0},
\end{equation}

where $u=(x,y)$. The linear transfer matrix from the position $s_0$ to the $s_1$, $M_u (s_1 | s_0)$, can be expressed in terms of the Courant-Snyder parameters as~\cite{Lee:1425444}: %Eq.(2.42) lee

\begin{equation}\label{eq:linear_transfer_matrix}
    \begin{split}
    M_u (s_1 |  s_0) &= \begin{pmatrix}
        \sqrt{\frac{\beta_u(s_1)}{\beta_u(s_0)}} (\cos{\Delta \psi_u}+\alpha_u (s_0) \sin{\Delta \psi_u}) & \sqrt{\beta_u(s_0)\beta_u(s_1)}\sin{\Delta \psi_u} \\ 
         - \frac{1+\alpha_u(s_0) \alpha_u(s_1)}{\sqrt{\beta_u(s_0) \beta_u(s_1)}} \sin{\Delta \psi_u}+ \frac{\alpha_u(s_0) - \alpha_u(s_1)}{\sqrt{\beta_u(s_0) \beta_u(s_1)}} \cos{\Delta \psi_u} & \sqrt{\frac{\beta_u(s_0)}{\beta_u(s_1)}} (\cos{\Delta \psi_u}+\alpha_u(s_1) \sin{\Delta \psi_u})
        \end{pmatrix} \\ 
        &=\begin{pmatrix}
            \sqrt{\beta_u(s_1)} & 0 \\
            -\frac{\alpha_u(s_1)}{\sqrt{\beta_u(s_1)}}& \frac{1}{\beta_u(s_1)}
            \end{pmatrix} \begin{pmatrix}
            \cos{\Delta \psi_u} & \sin{\Delta \psi_u} \\
            -\sin{\Delta \psi_u}& \cos{\Delta \psi_u}
            \end{pmatrix} \begin{pmatrix}
            \frac{1}{\sqrt{\beta_u(s_0)}} & 0 \\
            \frac{\alpha_u(s_0)}{\sqrt{\beta_u(s_0)}} & \sqrt{\beta_u(s_0)}
            \end{pmatrix},
    \end{split}
\end{equation}
where $\Delta \psi_u = \psi_u(s_1)-\psi_u(s_0)$ is the betatron phase advance between the two locations while $\alpha_u(s_i)$ and $\beta_u(s_i)$ are the Twiss parameters at the location $s_i$, where $i=(0,1)$. Tansfer matrices provide a very convenient approach to accelerator beam dynamics, and will be used extensively throughout this thesis to study the motion of the particles in the accelerator lattice.

\textbf{Off-momentum effects: dispersion}\\
Up to now, the discussion was limited to on-momentum particles $\delta=0$: their momenta equals the reference momentum, $p_0$. In a real beam, however, the momenta of the individual particles are spread around the reference momentum, $p_0$. The "momentum spread" is described by the rms momentum deviation, $\sigma_\delta$. For reference, for the studies in the SPS machine presented in this thesis, $\sigma_\delta$ is in the order of magnitude of $10^{-4}$ to $10^{-3}$. Particles with $\delta < 0$ ($\delta>0$) are deflected stronger (less) by the dipole magnets than the reference particle due to lower (higher) magnetic rigidity.

For the off-momentum particles the solutions of Eq.~\eqref{eq:eq_of_motion_horizontal_linear} have the following form:
\begin{equation}
    x(s) = x_H(s) + x_D(s),
\end{equation}
where $x_H(s)$ is solution shown in Eq.~\eqref{eq:solution_eq_motion_u} for $u=x$ and corresponds here to betatron oscillations around the on-momentum closed orbit. % i.e.~the reference trajectroy. 
The function $x_D(s)=D_x(s) \delta$ defines the off-momentum closed orbit. 

The parameter $D_x(s)$ is the dispersion function which can be expressed as:
\begin{equation}\label{eq:dispersion_function}
    D^{\prime \prime}_x(s) + K_x(s)D_x(s) = \frac{1}{\rho(s)},
\end{equation}
where $K_x(s)= \frac{1}{\rho^2(s)}+k_2(s)$ like in Eq.~\eqref{eq:Hills_equation_2}. As an example, the rms horizontal dispersion of the SPS machine is about 1.8\,m (model value). The dispersion introduces a coupling between the longitudinal and transverse planes. Here the discussion is limited to the horizontal plane since typically only vertical dipolar fields are considered in a synchrotron\footnote{A corresponding discussion can be done for the vertical plane to obtain the vertical dispersion but it is out of the scope of this thesis.}.

Particles with non-zero $\delta < 0$ travel along the accelerator performing betatron oscillations not around the reference trajectory but around a different closed orbit as illustrated in Fig.~\ref{fig:closed_orbit_Dx} which depends on their momentum deviation, $\delta$. 


%Particles with $\delta < 0$ ($\delta>0$) undergo larger (smaller) deflections from the dipole magnets than the reference particle due to lower (higher) magnetic rigidity. Therefore, they travel along the accelerator performing betatron oscillations not around the reference trajectory but around a different closed orbit as illustrated in Fig.~\ref{fig:closed_orbit_Dx} which depends on their momentum deviation, $\delta$. The change in the closed orbit with respect to the momentum deviation is called dispersion. It is evident that the dispersion introduces a coupling between the longitudinal and transverse planes.

\begin{figure}[!h] % 
    \centering         
    \includegraphics[width=0.6\textwidth]{images/Ch2/closed_orbit_dispersion.png}
        \caption{The closed orbit and the betatron oscillations around it in the presence of dispersion~\cite{Holzer_summer_students_introduction}}
        \label{fig:closed_orbit_Dx}
 \end{figure}

%Note that the discussion here is limited to the horizontal plane since ty
%The following discussion is focused on the horizontal plane due to the fact that (as stated already) only vertical dipolar fields are considered. 
% The equation of motion for the off-momentum particles has already been discussed in Eq.~\eqref{eq:linear_eq_of_motion_x}. Its solution is~\cite{Lee:1425444}:
%\begin{equation}
%    x^{\prime \prime}(s) + K_x(s)x(s) = \frac{1}{\rho(s)} \delta,
%\end{equation} %/Users/nataliatriantafyllou/PhD_projects/exploring_SPS/madx_studies/optics_new_seq_after_LS2/output/twiss_thin_elements


% 1. Definition of D: https://yannis.web.cern.ch/teaching/transverse.pdf
% 2. Definition of D prime: https://indico.cern.ch/event/847209/contributions/3558973/attachments/1932901/3201996/AccPhys_Lect7_MomEff.pdf
For Gaussian beam distributions with $\langle \delta \rangle=0$, the rms beam size defined in Eq.~\eqref{eq:emittance_beam_size} can be re-written so that it includes the dispersive contribution as follows~\cite{wolski2014}: 
\begin{equation}\label{eq:emittance_beam_size_dispersion}
    \sigma_x(s) = \sqrt{\beta_x(s) \epsilon^{\mathrm{geom}}_x+ D_x(s)^2\sigma_\delta^2}.
\end{equation}

In a real machine, vertical dispersion can be introduced by sources such as steering errors of the dipole or quadrupole magnets~\cite{Wolski_uspas}. For reference, the rms vertical dispersion in the SPS machine is measured to be about 10\,cm. Note that the above analyisis is valid also for the vertical plane.


%\textbf{Dipole field errors}\\
%Linear magner imperfection


%Additionally, the off-momentum particles receive different focusing due to gradient errors in the quadrupoles. This effect is known as chromaticity and is discussed in detail in the next section which focuses on non-linear beam dynamics.
\textbf{Off-momentum effects: chromaticity}\\
Additionally, off-momentum particles receive different focusing in the quadrupoles. This effect is known as chromaticity and is defined as the variation of the betatron tune $Q_u$ with the relative momentum deviation $\delta$. This is a result of the fact that particles with $\delta < 0$ ($\delta > 0$) are focused more (less) strongly from the quadrupoles due to their smaller (larger) magnetic rigidity. The tune shift introduced by the chromaticity for each particle, $ \Delta Q_u (\delta)= Q_u - Q_{u0}$, is: % M. Schenk p.49

%The tune shift introduced by the chromaticity is: % M. Schenk p.49

\begin{equation}\label{eq:chromatic_tune_shift_up_to_order_n}
   \Delta Q_u (\delta) = \sum_{n=1}^m \frac{1}{n!} Q_u^{(n)} \delta^n, 
\end{equation}
where:
\begin{equation}\label{eq:chroma_up_to_order_m}
    Q_u^{(n)} = \left. \frac{\partial ^n Q_u}{\partial \delta^n} \right|_{\delta=0}, n \in \mathbb{N},
 \end{equation}
denotes the chromaticity of order $n$. The studies in this thesis, are limited to the chromaticity at the first order in $\delta$ $(n=1)$ which is often called linear chromaticity. Note that the betatron tune shift of Eq.~\eqref{eq:chromatic_tune_shift_up_to_order_n} is referred to as an "incoherent" tune shift, since each particle is affected differently, depending on its individual momentum deviation. Similarly to the tune, the chromaticity is a property of the machine lattice. % the chromaticity is one number

%Large values of chromaticity can lead to instabilities and therefore to beam loss~\cite{Lee:1425444}. Sextupole magnets are typically used to control the natural chromaticity of a machine and achieve the desired values for its operation.
% Lee page 159


\textbf{Octupole magnets}\\
The octuopole plays an important role in the studies of the thesis.

The transfer map for an octupole is:
\begin{equation}\label{eq:transfer_map_octu_thin_x}
   \Delta x^\prime = -\frac{1}{6}{k_3L}(x^3-3xy^2),
\end{equation}
\begin{equation}\label{eq:transfer_map_octu_thin_y}
    \Delta y^\prime= -\frac{1}{6}{k_3L}(y^3-3yx^2),
 \end{equation}
 where $k_3L$ is the integrated strength of the octupole of length $L$. Equations~\eqref{eq:transfer_map_octu_thin_x} and~\eqref{eq:transfer_map_octu_thin_y} are valid for the thin lens approximation. %which assumes that the length of the multipole components tends to zero. 
 In that case, the rest of the co-ordinates ($x,y,z,\delta$) do not change as the particle passes through the magnet.


As described in Chapter~10 of~\cite{Stoel:2693915} the term $-\frac{1}{6}k_3Lu^3$, with $u=(x,y)$, introduces an additional phase advance in the motion of each particle, which eventually over one turn is translated to a change in their tune given by~\cite{Stoel:2693915}:
\begin{equation}\label{eq:tune_shit_octupoles}
    \Delta Q_u  = \frac{1}{16 \pi} k_3L\beta_u(s)^2 J_u .
\end{equation}
It can be seen that a different change in the tune is introduced for each particle depending on their individual action, $J_u$. Note, that the above change of the tune is a result of a single octupole around the machine at a location $s$ and beta function $\beta_u(s)$. %The discussion in~\cite{Stoel:2693915} concerns motion in one plane and interaction with a single octupole per turn. 

%Assuming that the phase advance remains evenly distributed between $0$ and $2\pi$, the average additional phase advance of a particle as introduced by an octupole element can be written as~\cite{wolski2014}:
%\begin{equation}\label{eq:phase_advance_octupole}
%    \langle \Delta \psi_u \rangle = \frac{1}{8}k_3L\beta_u(s)^2 \langle J_u \rangle.
%\end{equation}
%The phase advance of the particle remains the same.
%It can be seen that a different phase advance is introduced for each particle depending on their individual action, $J_u$.
%Equation~\eqref{eq:phase_advance_octupole} translates into a change of the %tune of each particle as:
%\begin{equation}\label{eq:tune_shit_octupoles}
%    \Delta Q_u  = \frac{1}{16 \pi} k_3L\beta_u(s)^2 \langle J_u \rangle.
%\end{equation}

% and Eq.(1.175) in emilias thesis.


This property, of providing incoherent betatron tune spread in a controlled way is used extensively in this thesis. % and therefore some further details are discussed in the following.
For reference, the octupole magnets are typically used to increase the transverse tune spread of the beam particles to avoid %resonances \footnote{Resonances in circular accelerators are a result of perturbation terms in the equation of motion once the perturbation frequency matches the frequency of the particles' oscillatory motion. The topic of resonances is out of the scope of this thesis, however, more details can be found in Chapter~16 of~\cite{Wiedemann:1083415}} and 
instability effects\footnote{Beam instabilities in an accelerator are a result of the interplay of the wakefields (introduced in Section~\ref{sec:collective_effects}) and a perturbation (e.g. noise) on equations of motion of the beam particles. Similar to the resonances their detailed study is out of the scope of this thesis, however, more details can be found in~\cite{Rumolo:1982422}.}. 


In the SPS and LHC rings, the octupoles are installed in families (called focusing and defocusing) in order to avoid the excitation of resonances. They are usually referred to as "Landau octupoles" since they are used to create a betatron tune spread that provides the mechanism of Landau damping~\cite{Herr:1982428} (to stabilise the beam).

The general formula which describes the linear betatron detuning with transverse amplitude introduced by mutliple octupoles around the machine is~\cite{Gareyte:321824}:
\begin{equation}\label{eq:DQ_with_amplitude_horizontal}
    \Delta Q_x (J_x, J_y) = 2(\alpha_{xx} J_x + \alpha_{xy}J_y),
\end{equation}
\begin{equation}\label{eq:DQ_with_amplitude_vertical}
    \Delta Q_y (J_x, J_y) = 2(\alpha_{yy} J_y + \alpha_{yx}J_x),
\end{equation}

where $\alpha_{xx}, \alpha_{yy}$ and $\alpha_{xy}=\alpha_{yx}$ are the detuning coefficients. The detuning coefficients depend on the octupoles' strength, the beta functions at their location~\cite{Gareyte:321824}. Note that the detuning with the transverse action (or amplitude) is an incoherent effect as it depends on the individual action of each particle.


%\begin{equation}\label{eq:Hills_solution}
%    u(s) = A w(s) \cos{(\psi_u(s)+\psi_{u,0})},
%\end{equation} % Lee p. 48
%where $A$ and $\psi_{u,0}$ the integration constants and $w(s)$ and $\psi_u(s)$ are the amplitude and betatron phase functions, which are periodic functions with the same periodicity as $K_u$.



%After substituting for $u(s)$ from Eq.~\eqref{eq:Hills_solution} in Eq.~\eqref{eq:Hills_equation_1}, we find after some calculation (see Appendix C.1) that the amplitude and phase functions fulfill the following equations: 
%Combination slide 27-28 of yiannis notes:  https://yannis.web.cern.ch/teaching/transverse.pdf and federicos thesis (p.11): https://cds.cern.ch/record/1481835/files/CERN-THESIS-2005-082.pdf

%\begin{equation}\label{eq:phase_functions}
%    w^{\prime \prime}_u + K_u(s) w(s) -\frac{1}{w_u(s)^3} = 0, 
%\end{equation}
%where:
%\begin{equation}\label{eq:phase_function}
%    \psi^{\prime}_u(s) = \frac{1}{w_u(s)^2} \Rightarrow \psi_u(s) = \int_{s_0}^%{s} \frac{ds}{w_u(s)^2}
%\end{equation}
%The above equations are called the betatron envelope and phase equations.
%As $K_u(s)$ is a periodic function, the two independent solutions of the second order differential Eq.~\eqref{eq:Hills_equation_1} can be expressed using the Floquet theorem~\cite{Floquet1883} as follows (see Appendix A, Sec. I.5 in Ref.~\cite{Lee:1425444}):


%\begin{equation}\label{eq:Floquets_solutions}
%    u(s) = a_u w(s)e^{i \psi_u (s)}, \ u^*(s) = a_u w(s)e^{-i \psi_u (s)},
%\end{equation}

%where $\alpha_u$ is a constant, and $w(s)$ and $\psi(s)$ are the amplitude and betatron phase functions and $u^*(s)$ is the complex conjugate of $u(s)$. 
%$K_u(s)$ is real-valued, therefore the amplitude and phase functions fulfill the following equations: % Lee Eq.(2.38), % p.250 widerman

% If you want to see how you go from Eq.(2.10) to Eq.(2.11) see also Yiannis notes (p.27-28): https://yannis.web.cern.ch/teaching/transverse.pdf

%\textbf{Courant-Snyder parameters}\\
%The Twiss or Courant Snyder parameters are defined as:
% frpom Yiannis notes: https://yannis.web.cern.ch/teaching/transverse.pdf
%\begin{subequations}\label{eq:twiss_func}
%    \begin{equation}
%        \beta_u(s) = w_u(s)^2,
%    \end{equation}    
%    \begin{equation}
%       \alpha_u(s) = -\frac{1}{2} \beta^{\prime}_u(s),
%    \end{equation} 
%    \begin{equation}
%       \gamma_u(s) = \frac{1+\alpha_u(s)^2}{\beta_u(s)},
%    \end{equation}
%\end{subequations}
%with $\beta^{\prime}_u(s) = d\beta_u(s)/ds$.

%The betatron phase advance from Eq.~\eqref{eq:phase_function} can be re-written using the beta Twiss function as:
%\begin{equation}\label{eq:phase_advance_definition_with_twiss}
%    \psi_u(s)= \int_{s_0}^{s} \frac{ds}{\beta_u(s)}.
%\end{equation}





 %\textbf{Normalised phase space}\\  %Federicos thesis 2.3.1 and sondre eq. 2.21a
 %Often in accelerator physics, it is useful to transform the transverse phase space ellipse into a normalised phase space circle. % Wiedemman p.235 
 %In the normalised phase space the co-ordinates $(u, u^\prime)$ are normalised with the twiss parameters $(\alpha_u, \beta_u)$ for the particular location $s$ around the ring, as follows~\cite{Wiedemann:1083415}: % widerman p.235, eq.8.92-8.93
 %\begin{equation}\label{eq:normalised_coordinates_un}
 %    u_N(\phi) = \frac{u(s)}{\sqrt{\beta_u(s)}},
 %\end{equation}
%\begin{equation}\label{eq:normalised_coordinates_un_prime}
 %    u^\prime_N(\phi) = \frac{d u_N}{d \phi} = \sqrt{\beta_u(s)} u^\prime(s) + \frac{\alpha_u(s)}{\beta_u(s)}u(s),
% \end{equation}
 % The normalised transformation can also be found in Yiannis notes in the link below: https://yannis.web.cern.ch/teaching/transverse.pdf
%where $\phi = \frac{\psi_u}{Q_u}$. It can be seen that the independent variable in the normalised co-ordinates is the phase advance (normalised with the betatron tune), $\phi$, instead of the location $s$ along the ring. % p.12 in the following link: https://cds.cern.ch/record/409584/files/thesis-99-070_Chapter1.pdf
%Both of the normalised co-ordinates, $(u_N, u_N^\prime)$ are expressed in units of $m^{1/2}$.

%Combining Eq.~\eqref{eq:action_definition}, Eq.~\eqref{eq:twiss_func}, Eq.~\eqref{eq:normalised_coordinates_un} and Eq.~\eqref{eq:normalised_coordinates_un_prime} the action variable can also be written as:
%\begin{equation}\label{eq:action_normalised_coordinates}
%    J_u = \frac{1}{2} (u_N^2+u_N^{\prime 2}).
%\end{equation}
% Plotting in p.4 of: http://www.toddsatogata.net/2013-USPAS/2013-01-23-NonlinearDynamicsNotes.pdf
%The phase space in normalised co-ordinates is shown in Fig.~\ref{fig:phase_space_circle_normalised}.
%\begin{figure}[!h] % at the directory of ipac22
%    \centering         
%    \includegraphics[width=0.6\textwidth]{images/Ch2/phase_space_normalised.png}
%       \caption{Normalised phase space: the trajectory shown represents the particle motion as the particle moves around the synchrotron. This is not the case in the regular (not normalised) phase space, since the shape of the phase space ellipse will change with position around the ring. The particle moves turn by turn in a circle of radius $\sqrt{2J_u}$.}
%        \label{fig:phase_space_circle_normalised}
% \end{figure}



%From Eq.~\eqref{eq:action} we write:
%\begin{equation}\label{eq:Jy_exp_distr}
%    e^{-J/\epsilon} = e^{(-x^2/2\epsilon - px^2/2\epsilon)}
%\end{equation}
%From this equation it can be seen that the actions follow an exponential distribution. 

%The dispersive contribution can be added to the transfer matrix introduced in Eq.~\eqref{eq:matrix_formalism_intro} as follows:
%\begin{equation}\label{eq:matrix_formalism_dispersion}
%    \begin{pmatrix}
%     x\\ 
%     x^\prime
%     \end{pmatrix}_{s_1} = M_x (s_1 |  s_0) \begin{pmatrix}
%     x\\ 
%    x^\prime
%     \end{pmatrix}_{s_0} + \delta \begin{pmatrix}
 %       D_x\\ 
 %       D_x^\prime
 %       \end{pmatrix}_{s_0}.
 %\end{equation}
%The transfer matrix $M$ can be re-written such as it inclyyded, alide 14 in presentation v2 below:


%\subsubsection{Non-linear dynamics}\label{subsec:non-liner_beam_dynamics}
%Up to now, only linear elements (dipoles and quadrupoles) were considered as in theory they are sufficient to create a synchrotron. However, in a real machine non-linearities are also present due to factors such as imperfections in the magnets field and alignment, particles' momentum spread, and higher order magnets (sextupoles, octupoles, etc). Here, the preceding discussion is expanded to include the non-linear beam dynamics. The discussion is limited to the two effects that are important for the work presented in this thesis: the chromaticity and the detuning with transverse amplitude.



\subsection{Longitudinal motion}
In the longitudinal plane, the motion is parallel to the reference trajectory and is described by the co-ordinates $(z, \delta)$. In the next paragraphs, the discussion is limited to the longitudinal motion in a synchrotron storage ring operating at the ultra-relativistic regime (like the machines of interest in this thesis) and only to the concepts that are needed for understanding the studies. The discussion in this section is more limited than for the transverse motion since the work presented in the thesis mostly concern transverse beam dynamics. A complete discussion can be found in Chapter 9 of~\cite{Wiedemann:1083415} and in~\cite{wolski2014}.
% it is also called synchrotrom motion

%In the next paragraphs the basic concepts used to describe the motion in the longitudinal plane which is also called synchrotronous motion are explained. In particular, the discussion is focused on the synchronous phase, the momentum compaction and phase slip factors, the concepts of phase stability, synchrotron oscillations and synchrotron tune tune. The equations of synchrotron motion are also given.

%In the longitudinal plane the motion is tangential to the reference trajectory and is described by the co-ordinates $(z, \delta)$. In the following discussion the longitudinal motion is treated independently from the transverse one. The complete approach, which describes the particles motion in the presence of coupling between the transverse and longitudinal planes can be found in Chapter~5.2 in the book of A. Wolski in Ref.~\cite{wolski2014}.

\textbf{Revolution period and frequency}\\
The time that the reference or synchronous particle needs to complete one complete revolution around the accelerator is called the revolution period, $T_\mathrm{rev}$. The revolution frequency is $\omega_\mathrm{rev} = 2\pi /T_\mathrm{rev}$ or $f_\mathrm{rev}=1/T_\mathrm{rev} =  v_0/C_0 = \beta_0 c/C_0$, where $v_0$ is the speed of the reference particle, $\beta_0$ the relativistic beta, $c$ the speed of light and $C_0$ the circumference of the accelerator.

%\textbf{Transfer map}\\
\textbf{Equations of motion}\\ % is this a transfer map?
% The RF cavities in a storage ring aim to compensate for energy losses allowing the beam to circulate in the reference trajectory.
In a synchrotron storage ring, the motion in the longitudinal plane is controlled by the RF cavities. They provide a longitudinal electric field, which can be described by:
\begin{equation}\label{eq:RF_cavity_EF}
    E_\mathrm{{RF}}(z) = E_A \sin{(2\pi f_\mathrm{RF}z + \phi_s)},
\end{equation}
where $E_A$ it the amplitude of the electric field, $f_\mathrm{RF}$ the frequency of the RF system and $\phi_s$ is the phase of the synchronous or reference particle.

The frequency of the RF field needs to be synchronous with the revolution frequency, such that the synchronous particle arrives with the same phase at the cavity again after one turn. Therefore the frequency of the RF field is a multiple of the revolution frequency: $\omega_\mathrm{RF}=h 2\pi f_\mathrm{rev}$. The parameter $h$ (number of RF cycles per revolution) is called the harmonic number and defines the number of stable synchronous particle locations in the ring~\cite{Tecker:2674860}.


%In the longitudinal the focusing (in phase) of particles are achieved by the longitudinal time-dependent electric field of the main RF cavities:
% stefania's thesis p.20 and M. schenk thesis p.20 Eq.2.29
%\begin{equation}\label{eq:RF_cavity_EF}
%    E_\mathrm{{RF}}(z) = E_A \sin{(2\pi f_\mathrm{RF}z + \phi_s)},
%\end{equation}
%where $E_A$ it the amplitude of the electric field, $\phi_\mathrm{{RF}}(t) = \omega_\mathrm{{RF}}t$ the phase of the RF system, $\omega_\mathrm{{RF}}$ the angular frequency of the RF system and $\phi_s$ is the phase of the synchronous or reference particle

The RF cavities provide a change in the energy deviation of the individual particles as a function of their longitudinal position $z$ within the bunch ($z=0$ the location of the synchronous particle and $z>0$ the head of the bunch). The equations of motion for a particle passing through a system of synchronised RF cavities located around the accelerator in the ultra-relativistic regime can be expressed as~\cite{wolski2014}:
\begin{equation}\label{eq:long_eq_motion_z}
    z^\prime = \frac{dz}{ds} = - \eta_p \delta,
\end{equation}
\begin{equation}\label{eq:long_eq_motion_delta}
    \delta^\prime = \frac{d\delta}{ds} = - \frac{qV_\mathrm{RF}}{c p_0 C} \left ( \sin{\phi_s} - \sin{\left ( \phi_s - \frac{\omega_\mathrm{RF} z}{c} \right)} \right ).
\end{equation}

Equation~\eqref{eq:long_eq_motion_z} assumes that we can average the change of the $z$ co-ordinate over the circumference of the machine. This approximation is valid for slow synchrotron motion compared to the revolution period and for small oscillation amplitude compared to the RF wavelength\cite{wolski2014}. % p.171 below Eq.(5.51)

Equation~\eqref{eq:long_eq_motion_delta} assumes that the energy deviation, $\delta_E =\frac{E-E_0}{E_0} \approx \delta$. This approximation is valid for particle with small momentum deviation and for machines operating at the ultrarelativistic regime~\cite{wolski2014}. % A. Wolski p.(5.40)


The parameter $\eta_p$ is called phase split factor and describes the relative change in the revolution frequency of a particle with respect to its momentum spread and can be written as:
\begin{equation}\label{eq:phase_slip_def}
    \eta_p \equiv \frac{\Delta f/f_\mathrm{rev}}{\delta},
\end{equation}
where $\Delta f$ is the change of the revolution frequency of each individual particle. Further details can be found in~\cite{wolski2014}.

%The physical implementations of the phase slip factor are not described here since thay are out of the scope of this thesis. However further details can be found in~\cite{wolski2014}.  % change of the effect in the dipoles --> transverse coupling.
% p.163
%This means that particles with different momentum deviation have different revolution frequencies. 

% Dispersion and phase slip factor:https://people.nscl.msu.edu/~haoy/teaching/AP_course_materials/off_momentum_orbit/off_momentum_orbit.html

\textbf{Synchrotron oscillations}\\
% A. Wolski non-linear dynamics CAS 2018 Budapers.
Following the above described equations of motion, the off-momentum particles will perform oscillations around a stable location in the longitudinal plane.  They are also known as synchrotron oscillations and are typically order of magnitudes slower than the betatron oscillations. For example, a complete synchrotron oscillation may take many ($\sim$100) turns, in contrast to the betatron oscillations (of which there are usually many complete oscillations per turn).

The possible stable locations around which the synchrotron oscillations are performed are defined by the harmonic number $h$. The set of particles that oscillate around the same stable point will be referred to as "bunch". Note that the harmonic number $h$ defines the maximum number of bunches that can be stored in the ring.

%In the longitudinal the focusing (in phase) of particles are achieved by the longitudinal time-dependent electric field of the main RF cavities:
% stefania's thesis p.20 and M. schenk thesis p.20 Eq.2.29
%\begin{equation}\label{eq:RF_cavity_EF}
%    E_\mathrm{{RF}}(z) = E_A \sin{(2\pi f_\mathrm{RF}z + \phi_s)},
%\end{equation}
%where $E_A$ it the amplitude of the electric field, $\phi_\mathrm{{RF}}(t) = \omega_\mathrm{{RF}}t$ the phase of the RF system, $\omega_\mathrm{{RF}}$ the angular frequency of the RF system and $\phi_s$ is the phase of the synchronous or reference particle. The angular frequency needs to be an integer mulitple of the revolution frequency: $ \omega_\mathrm{RF} = h \omega_\mathrm{rev}$, where $h$ is called the harmonic number. The harmonic number (number of RF cycles per revolution) defines the maximum number of bunches that can be accelerated (or stored) in the ring. % the maximum number of sunchronous particles (around them the off momentum particles (rest of the bunch),
%In a synchrotron during the energy ramp the angular frequency increases in order to follow the increasing revolution frequency.

\textbf{Synchrotron tune}\\
The synchrotron tune, $Q_s$, is defined as the number of synchrotron oscillations performed during one complete revolution around the machine and is computed as follows~\cite{wolski2014}: % Wolski Eq.(5.49) p. 170
\begin{equation}\label{eq:Qs} 
    Q_s = \frac{1}{2\pi}\sqrt{-\frac{e V_\mathrm{RF}}{c p_0} \frac{\omega_\mathrm{RF} C_0}{c} \eta_p \cos{\phi_s}}.
\end{equation}
%where $V_\mathrm{RF}$ is the voltage of the RF cavities and $\phi_s$ the phase that the synchronous particle arrive at them. 
%where $e$ the proton charge, $V_\mathrm{RF}$ the amplitude of the RF cavity voltage, $c$ the speed of light, $p_0$ the reference momentum, $\omega_\mathrm{RF}$ the angular frequency of the RF system and $\phi_s$ the synchronous phase.
%The parameter $\eta_p$ is called phase split factor and describes the relative change in the revolution frequency of a particle with respect to momentum for particles with zero momentum.



%Assuming that the synchronous or reference particle arrives at the RF cavity at phase $\phi_s$ every turn, the energy gain equals:
%\begin{equation}\label{eq:enerhy_gain_synchronous}
%    \Delta E_0 = e V_\mathrm{RF} \sin{(\phi_s)}, 
%\end{equation}
%where $V_\mathrm{RF}$ the amplitude of the RF cavity voltage. The rest of the particles will arrive at the RF cavity at phases $\phi = \phi_s \pm \delta \phi$ and they will gain or lose a different amount of energy per turns which equals: $\Delta E_p = e V_\mathrm{RF} \sin{(\phi)}$. 
 




%In particular, the non-synchronous particles oscillate around the phase of the synchronous particle performing synchrotron oscillations (similarly to the betatron oscillations in the transverse plane). It is worth mentioning, that a complete synchrotron oscillation can take many ($\sim$100) turns, in contrast to the betatron oscillations (of which there are usually many complete oscillations per turn). The equations of motion for a particle passing through a system of synchronised RF cavities located around the accelerator are~\cite{wolski2014}: % wolski p. 173 Eq.(5.53) and (5.54)
%\begin{equation}\label{eq:long_eq_motion_z}
%    z^\prime = - \eta_p \delta,
%\end{equation}
%\begin{equation}\label{eq:long_eq_motion_delta}
%    \delta^\prime = - \frac{qV_\mathrm{RF}}{c p_0 C} \left ( \sin{\phi_s} - \sin{\left ( \phi_s - \frac{\omega_\mathrm{RF} z}{c} \right)} \right ),
%\end{equation}
%where $z^\prime = dz/ds$, $\delta^\prime = d\delta/ds$, $e$ is the charge of a proton, $c$ is the speed of light, $p_0$ is the reference momentum, $C$ is the circumference of the machine, $\omega_\mathrm{RF}$ is the angular frequency of the RF system, $\phi_s$ is the phase of the synchronous particle and $\eta_p$ is the phase slip factor.

%\textbf{Dispersion effects}\\
%As discussed in the previous chapter, in the presence of dispersion a particle with a momentum offset, $\delta$, from the reference particle will have a different closed orbit of different length (see Fig.~\ref{fig:closed_orbit_Dx}). This change of the orbit length with respect to the momentum offset of each particle is described with the momentum compaction factor~\cite{emetral_juas_2018}: %p.20
%\begin{equation}\label{eq:compaction_factor}
%    \alpha_p = \frac{\Delta C /C}{\delta} = \frac{1}{C} \oint _C \frac{D_x(s)}{\rho(s)} ds,
%\end{equation} % proof hannes' thesis appendix p.169: https://cds.cern.ch/record/1644761/files/CERN-THESIS-2013-257.pdf
%where $C$ is the circumference of the accelerator and $D_x(s)$ and $\rho(s)$ are the horizontal dispersion and bending radius respectively at a given point $s$.

%With the change of the closed orbit length due to the momentum offset the revolution frequency of the particles also changes. The change of the angular frequency depending on the momentum offset is described with the phase slip factor:
%\begin{equation}\label{eq:phase_slip_factor}
 %   \eta_p = -\frac{\Delta \omega / \omega_0}{\delta} = \alpha_p - \frac{1}{\gamma^2_0} = \frac{1}{\gamma^2_\mathrm{tr}} - \frac{1}{\gamma^2_0},
%\end{equation}
%where $\omega_0$ is the angular frequency of the reference particle, $\gamma_0$ is the Lorentz factor and $\gamma_\mathrm{tr} = 1/\sqrt{\alpha_p}$ is called the transition energy. When $\gamma_0 < \gamma_\mathrm{tr} \Rightarrow \eta_p <0$ ($\gamma_0 > \gamma_\mathrm{tr} \Rightarrow \eta_p > 0$) and the machine operates below (above) transition. For the nominal optics configuration, the SPS machine always operates above transition as $\gamma_\mathrm{tr}$ = 22.8 which is smaller than the relativistic gamma even for the injection energy ($\gamma_0$ = 27.7 at 26\,GeV).
% Source for transition energy value: https://accelconf.web.cern.ch/ipac2011/papers/mops012.pdf

%\textbf{Phase stability and synchrotron oscillations}\\
%Even though the particles arrive at different times in the RF cavity, they stay in the vicinity of the reference particle thanks to the effect of longitudinal or phase focusing, which is explained by the concept of phase stability~\cite{McMillan:1945zz, Veksler_1, Veksler_2}. Its principle is illustrated in Fig.~\ref{fig:phase_focusing} for a machine operating above transition. Above transition a particle with $\delta<0$ will follow a shorter closed orbit (than the reference trajectory) and therefore it will arrive at the RF cavity slightly earlier, than the reference particle and hence it will see a larger voltage. Therefore, it will be accelerated more than the reference particle and subsequently it will need less time to complete the next revolution and, over a number of revolutions, will fall back longitudinally towards the reference particle. The situation is the opposite for a particle with $\delta<0$. 
% Inspired by Fig. 2.3 in the thesis of M. Schenk and Fig.5.15 in wille's book.
%\begin{figure}[!h] % 
%    \centering         
%    \includegraphics[width=0.6\textwidth]{images/Ch2/phase_focusing_synchrotron_motion.png}
%        \caption{Phase stability for particles in a circular accelerator which operates above transition.}
%        \label{fig:phase_focusing}
% \end{figure}

 %In particular, the non-synchronous particles oscillate around the phase of the synchronous particle performing synchrotron oscillations (similarly to the betatron oscillations in the transverse plane). It is worth mentioning, that a complete synchrotron oscillation can take many ($\sim$100) turns, in contrast to the betatron oscillations (of which there are usually many complete oscillations per turn). The equations of motion for a particle passing through a system of synchronised RF cavities located around the accelerator are~\cite{wolski2014}: % wolski p. 173 Eq.(5.53) and (5.54)
%\begin{equation}\label{eq:long_eq_motion_z}
%    z^\prime = - \eta_p \delta,
%\end{equation}
%\begin{equation}\label{eq:long_eq_motion_delta}
%    \delta^\prime = - \frac{qV_\mathrm{RF}}{c p_0 C} \left ( \sin{\phi_s} - \sin{\left ( \phi_s - \frac{\omega_\mathrm{RF} z}{c} \right)} \right ),
%\end{equation}
%where $z^\prime = dz/ds$, $\delta^\prime = d\delta/ds$, $e$ is the charge of a proton, $c$ is the speed of light, $p_0$ is the reference momentum, $C$ is the circumference of the machine, $\omega_\mathrm{RF}$ is the angular frequency of the RF system, $\phi_s$ is the phase of the synchronous particle and $\eta_p$ is the phase slip factor.


\section{Collective effects: Wakefields}\label{sec:collective_effects}
Up to now, the motion of the particles was studied neglecting the interaction between them within the bunch. Collective effects in an accelerator desribe the phenomena in which the motion of the particles depends on their interaction with each other through electromagnetic fields. Examples are beam-beam interactions, space charge effects, wakefields and intra-beam scattering~\cite{Zimmermann:2264408}. The collective effects usually become critical for high-intensity beams as they can lead to instabilities\footnote{For example a beam is unstable when one of its co-ordinates $(x, x^\prime, y, y^\prime, z, \delta)$ undergoes exponential growth. Further details on the beam instabilities can be found in~\cite{Rumolo:1982422}} which then may degrade beam quality, or lead to beam losses. In either case, the performance of the accelerator can be adversely affected. The discussion here is limited to the description of the wakefields which are relevant for the studies presented in this thesis. A complete overview of the collective effects can be found in~\cite{wolski2014, Zimmermann:2264408, Chao:collective}.
%For a more complete picture, a beam is called unstable when one of its co-ordinates $(x, x^\prime, y, y^\prime, z, \delta)$ undergoes exponential growth. Usually, this can be observed in the motion of the centroid $\langle x \rangle $
% If you want to show an example of an instability you can find a figure in Ref.~\cite{Rumolo:1982422}}

%\subsection{Wakefields and impedance}\label{subsec:wakefields}
%The discussion in this section is based in the discussion in~\cite{wolski2014, Rumolo:1982422, instabilities_rumulo_li, benoit_ipac19_impedance, Chao:collective}.


For the following discussion, it is appropriate to introduce the terms incoherent and coherent effects. Incoherent effects (microscopic approach) affect the individual particles. Any theory or model of incoherent effects has to treat the beam as a collection of a large number of individual particles, each with its own behaviour. In contrast, coherent effects can be understood in terms of their impact on the beam as a whole.%, and can be modelled by representing each bunch, for example, as a single "macroparticle", with mass and charge corresponding to the total number of particles it contains. % from Andy's corrections ---> this is useful when studying effects with multipole bunches, e.g. rigid beam instability.


\textbf{Wakefields}\\
The charged particles within a beam interact electromagnetically with their surroundings in the beam pipe such as the resistive vacuum pipe walls, the RF cavities, etc. If these structures are not smooth (presence of discontinuities) or not perfectly conducting the interaction with the charged particles will result in electromagnetic perturbations called wakefields. The wakefields act back on the beam affecting the beam dynamics.


The longitudinal and transverse wakefields can often be treated separately. In the following only the transverse components will be discussed as the focus of the thesis is on the transverse beam dynamics.

%For the study of the wakefields, it is considered that each particle acts as a source of wakefields for the rest of the particles\footnote{In studying collective effects, the terms "source particle" and "witness particle" are sometimes used for particles generating wakefields and particles affected by the wakefields, though in reality all charges act as sources of wakefields, and are affected by them} which are called witnesses. In the ultrarelativistic regime (which is also the regime of these studies) the wakefields from a source particle act only on the particles behind it, hence the term "wake". Wakefields can also act back on the charge generating them when that charge returns to a given location in a storage ring over successive turns. %This is the multi-turn effect of the wakefields. %In this thesis, only the multi-turn effect from the resistive wall of the machine will be considered as it is the only one that is found to have a significant impact on the dynamics.

 
Consider two particles of charge $q_1$ and $q_2$ moving with ultra-relativistic speed through a structure of length $L$ as shown in Fig.~\ref{fig:wakefields}. The particle of charge $q_1$ is the source particle while the witness particle\footnote{In studying collective effects, the terms "source particle" and "witness particle" are sometimes used for particles generating wakefields and particles affected by the wakefields, though in reality all charges act as sources of wakefields, and are affected by them}  of charge $q_2$ travels behind it at a constant distance $z$. $(\Delta x_1, \Delta y_1)$ are the transverse offsets of charge $q_1$, and $(\Delta x_2, \Delta y_2)$ are the transverse offsets of charge $q_2$ from the symmetric axis of the beam pipe. From the interaction of the source particle with the structure a wakefield is generated. 

\begin{figure}[!h] % 
    \centering         
    \includegraphics[width=0.6\textwidth]{images/Ch2/wakefield_example.png}
        \caption{Wakefield interaction, where the source particle (blue) affects the witness particle (yellow) travelling at a distance $z$ behind it~\cite{instabilities_rumulo_li}. $(\Delta x_1, \Delta y_1)$ and $ (\Delta x_2, \Delta y_2)$ are the transverse offsets of the source and witness particles respectively.} 
        \label{fig:wakefields}
 \end{figure}
% Figure take from: ~\cite{instabilities_rumulo_li}

The wakefields in time domain are described with the concept of wakefunctions, $W_u(z)$, where $u=(x,y)$ denotes the horizontal and vertical plane, respectively. The wakefunction can be expressed as a series of its multipole components as follows: % https://accelconf.web.cern.ch/ipac2019/talks/weypls1_talk.pdf (slide 13) 
\begin{equation}\label{eq:wakefunctions}
    W_u(\Delta u_1, \Delta u_2, z) = W^\mathrm{const}_u(z) + W^\mathrm{dip}_u(z)\Delta u_1 +  W^\mathrm{quad}_u(z)\Delta u_2 + o(\Delta u_1, \Delta u_2),
\end{equation}
where $u=(x,y)$ and $W^\mathrm{const}_u(z)$, $W^\mathrm{dip}_u(z)$, $W^\mathrm{quad}_u(z)$ are the transverse constant, dipolar, and quadrupolar wakefunctions respectively. The higher order term $ o(\Delta u_1, \Delta u_2)$ is not considered in the work presented here. % M. Schenk p.25

The dipolar and quadrupolar wakefunctions were named after the way they act on the witness particle. The dipolar wakefunction acts like a dipole magnet: its impact is the same regardless of the transverse position of the witness particle; it depends only on the position of the source particle. The quadrupolar wakefunction acts like a quadrupole magnet: its impact increases linearly with the transverse position of the witness particle (independent of the .position of the source particle). % Benoit thesis p.44

The constant term can change the closed orbit while the dipolar and quadrupolar terms can modify the tunes~\cite{benoit_ipac19_impedance}. The dipolar term is often referred to as a driving term for coherent instabilities. The quadrupolar term is often referred to as the detuning term as it modifies the betatron frequencies of individual particles. % https://accelconf.web.cern.ch/ipac2019/talks/weypls1_talk.pdf (slide 14) 
% driving term --> coherent effects.
% detuning term --> incoherent effects.

The effect of the wakefields on the witness particles can be modeled as the following kicks on the transverse normalised momentum~\cite{Schenk:2665819}: % Eq.(2.44) M. Schenk thesis
\begin{equation}\label{eq:wakefield_kicks}
    \Delta u_2^\prime  = -q_1 q_2[W^\mathrm{const}_u(z) + W^\mathrm{dip}_u(z)\Delta u_1 +  W^\mathrm{quad}_u(z)\Delta u_2],
\end{equation}
%\begin{equation}\label{eq:wakefield_kicks}
%    \Delta u_2^\prime  = \int_0^L F_u(s, z, \Delta u_1, \Delta u_2)ds = -q_1 q_2[W^\mathrm{const}_u(z) + W^\mathrm{dip}_u(z)\Delta u_1 +  W^\mathrm{quad}_u(z)\Delta u_2],
%\end{equation}
%where $F_u$ is the transverse force of the wakefield over the length $L$. 
If the structure leading to the wakefield is axially symmetric, then the constant term of the wakefunction is zero. 


Note that in the ultrarelativistic regime (which is the regime of the work presented in this thesis) the wakefields from a source particle act only on the particles behind it, hence the term "wake". Wakefields can also act back on the charge generating them when that charge returns to a given location in a storage ring over successive turns.


\textbf{Beam coupling impedance}\\
The beam coupling impedance is the frequency spectrum of the wakefields in a given component or section of the accelerator. The impedance can be obtained from the wake function through a Fourier transform~\cite{Chao:collective}: % Eq.(2.71) and Eq.(2.72) p.69-70.
\begin{equation}\label{eq:wakes_to_impedance}
    W_u(z) = - \frac{i}{ 2\pi} \int_{- \infty}^{+\infty} Z_u(\omega) e^{i \omega z/c} d\omega,
\end{equation}
\begin{equation}\label{eq:impedance_to_wakes}
    Z_u(\omega) = \frac{i}{c} \int_{- \infty}^{+\infty} W_u(z) e^{-i \omega z/c} dz,
\end{equation}
where $u=(x,y)$, $i$ is the imaginary unit and $c$ is the speed of light.


In order to study the beam dynamics effects due to wakefields, impedance models of the particle accelerators have been developed (using both measurements and electromagnetic simulations) including the contributions from the individual elements in the accelerator. Details on how an impedance model is built can be found in~\cite{benoit_ipac19_impedance}.



%The impedance model for the SPS is discussed in Chapter~\ref{Ch:suppression_impedance} while its implementation in the simulations is discussed in Section~\ref{subsec:pyheadtail}.

% Link to chao's book:https://www.slac.stanford.edu/~achao/wileybook.html
\textbf{Head-tail modes}\\
The Vlasov equation~\cite{Vlasov:426186} is often used to describe the beam motion in the presence of wakefields as it allows its mode representation (in frequency domain): the beam motion is described by a superposition of modes. %--> source: https://cds.cern.ch/record/859658/files/ab-2005-041.pdf
Solving the Vlasov equations for the coupled system between the particle motion (synchrotron and betatron) and wakefield kicks the eigenmodes and eigenfrequencies are obtained. These modes are often referred to as head-tail modes as they are related to the betatron phase shift between the head and tail of the bunch in a synchrotron. The head-tail modes can either be stable, damped, or excited; in the latter case, they evolve into instabilities. For the beam to become unstable the wakefield kicks (source of energy) need to be synchronized with the bunch motion (e.g. with chromaticity)~\cite{instabilities_rumulo_li}.
% in general the modes are a result of the coupling of betatron and synchroton motion--> is that right?

% From https://cas.web.cern.ch/sites/default/files/lectures/-27%20March%202022/Chavannes4_Instabilities.pdf
The head-tail modes can be described with the mode indices $m$ and $l$ which denote the transverse and the longitudinal (or azimuthal) mode numbers~\cite{Chao:collective}. In the following modes with $m=1$ will be considered and they are often referred to as transvserse modes~\cite{Chao:collective}. Figure~\ref{fig:azimuthal_mode} shows a graphical representation of the azimuthal modes for $m=1$.

\begin{figure}[!h] % 
    \centering         
    \includegraphics[width=0.6\textwidth]{images/Ch2/transverse_mode_l.png}
        \caption{Graphical representation of the lowest azimuthal modes in the longitudinal beam structure, for $m=1$ as the bunch oscillates transversely~\cite{Chao:collective}. } 
        \label{fig:azimuthal_mode}
 \end{figure}
 %This image is valid for low beam intensities.
 
In this thesis the terms "mode" or "head-tail mode" will refer to the azimuthal modes\footnote{The modes' radial structure in the longitudinal plane is described by their radial modes~\cite{Chao:collective}. They are not considered nor discussed here as they are out of they are beyond the scope of the work presented in this thesis.}. 

As shown in Fig.~\ref{fig:azimuthal_mode} for the head-tail mode 0 ($l=0$), the bunch moves transversely as a rigid unit or the head-tail mode 1 ($l=1$), the bunch oscillates transversely but the head and the tail of the bunch move 180$^\circ$ out of phase. %if you can rephrase it.
% from presentation: https://www.slideserve.com/mostyn/axel-2013-introduction-to-particle-accelerators-powerpoint-ppt-presentation

% All the equations are taken from: https://github.com/natriant/exploring_SPS
\textbf{Sacherer formulae and complex coherent frequency shift}\\
One of the impedance induced effects, that is relevant for the studies of the thesis, is the complex tune shift. The complex tune shift can be computed analytically based on the Vlasov formalism~\cite{Vlasov:426186} through the use of perturbation theory. Formulae expressing the results were derived by Sacherer~\cite{Sacherer:322545, Sacherer:322645}. %  Source of this sentence: second column: https://cds.cern.ch/record/2674107/files/757-3481-2-PB.pdf

The headtail modes introduce an exponential dependence on the amplitude of the bunch centroid as follows~\cite{Schenk:2665819}:
\begin{equation}\label{eq:HeadtailModesExponentialDependence}
    u(t) \propto e^{i(\Omega_{u0}^{(l)}+\Delta \Omega_u^{(l)})t} =  e^{i(\Omega_{u,0}^{(l)}+\Delta \Omega_{u, \mathrm{{re}}}^{(l)})t} e^{-\Delta \Omega_{u, \mathrm{{im}}}^{(l)} t},
\end{equation}
where $\Omega_{u0}^{(l)}$ is the real-valued, unperturbed frequency of the head-tail mode $l$, and $\Delta \Omega_u ^{(l)} = \Delta \Omega_{u, \mathrm{re}}^{(l)} + i \Delta \Omega_{u, \mathrm{im}}^{(l)}$ is the complex coherent frequency shift introduced by the beam impedance. From Eq.~\eqref{eq:HeadtailModesExponentialDependence} it can be seen that the real part $\Delta \Omega_{u, \mathrm{re}}^{(l)}$ modifies the oscillation frequency. The second term $e^{-\Delta \Omega_{u, \mathrm{im}}^{(l)}t}$ illustrates that the amplitude of the motion grows for $\Delta \Omega_{u, \mathrm{im}}^{(l)}<0$ (unstable bunch) and is damped for $\Delta \Omega_{u, \mathrm{im}}^{(l)}>0$ (stable bunch).


The complex coherent frequency shift for the mode $l$ for a bunched beam is (Chapter 6 in.~\cite{Chao:collective}): % Eq.(6.207)
\begin{equation}\label{eq:complext_tune_shift_modes_m}
    \Delta \Omega_u ^{(l)}= \Omega_{u0}^{(l)}-\omega_{u0}-l\omega_s = -\frac{1}{4\pi}\frac{\Gamma(l+1/2)}{2^l l!}\frac{N_b r_0 c^2}{\gamma_0 T_\mathrm{rev} \omega_{u0} \sigma_z} i Z_\mathrm{eff},
\end{equation}
where $\omega_{u0}$ and $\omega_s$ are the unperturbated betatron and synchrotron frequencies, $\Gamma(x)$ is the gamma function, $N_b$ is the number of particles in the bunch, $r_0=1.535 \cdot 10^{-16}$ is the clasical radius of the proton, $c$ is the speed of light, $\gamma_0$ is the relativistic gamma, $T_\mathrm{rev}$ is the revolution period, $\sigma_z$ is the rms bunch length, $i$ is the imaginary unit and $Z_\mathrm{eff}$ is the effective impedance.
% https://www.slideserve.com/mostyn/axel-2013-introduction-to-particle-accelerators-powerpoint-ppt-presentation


% Chao p.321
The effective impedance expresses the effect of the impedance defined in Eq.~\eqref{eq:impedance_to_wakes} on the beam dynamics. The effective impedance $Z_\mathrm{eff}$ can be computed as follows~\cite{Chao:collective}:
\begin{equation}\label{eq:effective_impedance}
    Z_{\perp \mathrm{eff}}^{(l)} = \frac{\sum_{p=-\infty}^{+ \infty}Z_{\perp }^{(l)}(\omega_{p}) h_l(\omega_{p}-\omega_\xi)}{\sum_{p=-\infty}^{+ \infty}h_l(\omega_{p}-\omega_\xi)},
\end{equation}
where $\omega_p = (p+Q_u)\omega_0$ is the discrete spectrum of the transverse bunch oscillations with $-\infty < p < + \infty$ for a single bunch (which is the case in the following studies) or several bunches oscillating independently and $\omega_\xi=(\xi \omega_u)/\eta_p) = Q^\prime \omega_0 / \eta_p$ is the chromatic angular frequency with $\eta_p$ being the phase slip factor.
% physical significance of effective impedance: chao p.319

Last, $h_l$, is the power spectral density (definition in Appendix~\ref{ch:app_B}) of a Gaussian bunch of $l$ azimuthial mode. $h_l$ is described by: % Chao ch6 Eq.(6.143)
\begin{equation}\label{eq:spectral_density_of_gaussian_bunch}
    h_l(\omega) = (\omega \sigma_z/c)^{2l} e^{-(\omega \sigma_z/c)^2},
\end{equation}
where $\sigma_z$ is the rms bunch length and $c$ the speed of light.

It should be highlighted that all the parameters inserted in Eq.~\eqref{eq:complext_tune_shift_modes_m},  Eq.~\eqref{eq:effective_impedance} and Eq.~\eqref{eq:spectral_density_of_gaussian_bunch} should be converted in CGS (centimetre–gram–second) units. For the conversion from SI to CGS system the following relations are useful:
\begin{equation}\label{eq:conversion_si_to_cgs}
    1 \mathrm{[\Omega]} = \frac{1}{9} \cdot 10^{-11} \mathrm{[s]/[cm]},
\end{equation}
where $\Omega$ (Ohm) is the SI unit of resistance.


%- 1 [Ohm] = (1/9)*10**(-11) [s]/[cm]
%- 1 [Ohm]/[m] = (1/9)*10**(-13) [s]/[cm]**2 

Finally, the imaginary part of the impedance, leads to a real coherent frequency shift which can be expressed in tune units as follows~\cite{PhysRevAccelBeams.23.124402}:
\begin{equation}\label{eq:real_tunu_mode_l}
    \Delta Q_u^{(l)} = \frac{\Delta \Omega_{u, \mathrm{{re}}}^{(l)}}{\omega_\mathrm{rev}},
\end{equation}
where $\Delta \Omega_{u, \mathrm{{re}}}^{(l)}$ the real part of the complex tune shift as defined in Eq.~\eqref{eq:complext_tune_shift_modes_m}.
The real part of the impedance, leads to an imaginary coherent frequency shift which is also known as the growth (+ sign) or damping (- sign) rate of the mode $l$ and is given by~\cite{PhysRevAccelBeams.23.124402}:
\begin{equation}\label{eq:imaginary_tune_mode_l}
    1/\tau_u^{(l)} = -\frac{\Delta \Omega_{u, \mathrm{{im}}}^{(l)}}{T_\mathrm{rev}},
\end{equation}
where $\Delta \Omega_{u, \mathrm{{im}}}^{(l)}$ the imaginary part of the complex tune shift as defined in Eq.~\eqref{eq:complext_tune_shift_modes_m}. %The $1/\tau_u^{(l)}$ is expressed in units of 1/turns. 

The real coherent tune shift introduced by the beam transverse impedance plays a crucial role in the studies with Crab Cavities discussed in the following.

%In principle, the mode representation and the particle representation of the beam motion are identical. To describe fully 1012particles, one needs 1012modes, and vice versa. The detailed methods of analysis in the two approachesare different-the particle representation usually is conveniently treated in the time domain, while in the mode representation the frequency domain is more convenient-but, in principle, they necessarilygive the same final results. (p.272, chao)


%The total betatron phase shift between head and tail is the physical origin of the head tail instability. The head and the tail of the bunch oscillate therefore with a phase difference, which reduces to rigid-bunch oscillations only in the limit of zero chromaticity. A new (within-bunch) mode number m = ... , −1, 0, 1, ... , also called head–tail (or azimuthal) mode number, was introduced. This mode describes the number of betatron wavelengths (with sign) per synchrotron period. When computing the eigen modes of the coupled system one can see that there is also a freuqency shift. The Vlasov formalism used for the mode representation of the beam motion. 

%When studying the imepdance-induced instabilities it is convinient to move in the mode domain: the beam motion is described by a superposition of modes. % the beam motion is described by a superposition of modes
%allows the study of the beam motion in 


%\subsection{Active feedback?}
\section{Optics models for accelerators}\label{sec:optics_model_designs}
% or optics design
For the study of beam dynamics, it is essential to know the detailed arrangement of the magnets (position and strength) in the lattice, which will be referred to as optics. The optics also provide information on the twiss parameters and phase advances along the ring.

MAD-X~\cite{madx} is a code which is used extensively for the design and simulation of the accelerators at CERN. The official optics repositories of the CERN machines can be found in~\cite{cern_optics_repo}.

\textbf{SPS optics}\\
The studies presented in this thesis are performed for the SPS optics called Q26 optics (as the integer part of the tune in both planes is 26). The model for the Q26 optics can be found in the official CERN repository~\cite{SPS_optics_repo} and will be referred to as the nominal SPS model in this thesis. The values of the optics parameters in what follows correspond to the model values unless stated otherwise.
\normalsize{\textbf{} % fixed with duck tape, DO NOT REMOVE THIS LINE. error in compiling.



\section{Tracking simulation codes}\label{sec:simualtion_codes}
In this section the two tracking simulation codes used in this thesis to study the noise-induced emittance growth are presented. Both codes are macroparticle tracking libraries that simulate the particle motion in the six-dimensional (6D) phase space $(x, x^\prime, y, y^\prime, z, \delta)$. 


%The first code (PyHEADTAIL) performs tracking between interaction points around a circular accelerator at which the paticles receive kicks from magnetic elements or from collective effects. The second code (Sixtracklib), uses the detailed optics model of the machine for the tracking and the tracking is per

%In both cases, the tracking is performed with the use of transfer matrices.
% sympectic intgration. drift-kick approach

\subsection{PyHEADTAIL}\label{subsec:pyheadtail}

PyHEADTAIL~\cite{pyheadtail_repository} is an open-source 6D macroparticle tracking code, developed at CERN, which was originally designed to study collective effects in circular machines and to be easily extensible with custom elements. 
% https://indico.cern.ch/event/930271/contributions/3910265/attachments/2066139/3467770/30June2020_Simulations_emit_growthCC_RFnoise.pdf  
Details on its implementation and its features can be find in~\cite{pyheadtail_manual_adrian, pyheadtail_schenk}. In principle, the tracking is performed between interaction points around a circular accelerator at which the paticles receive kicks from magnetic elements or from collective effects. 


Below the main steps for setting up a simulation are listed: % according to the interest of this thesis.

% Thesis david: https://cds.cern.ch/record/2707064/files/CERN-THESIS-2019-272.pdf (p.45) and for the transverse matrix.

\begin{enumerate}
    \item \textbf{Machine initialisation:} The accelerator ring is split into a number of segments of equal length, after each of which there is an interaction point\footnote{It should be clarified that this is not an "interaction point" in the sense of collision between counter-circulating beams.} (IP). At the interaction points the macroparticles experience kicks from various accelerator components (feedback system, multipoles etc) or from collective effects such as the wakefields. 
    
    The machine parameters, such as the circumference, the betatron and synchrotron tunes, the Twiss and the dispersion values at the interaction points are defined. %It should be noted that PyHEADTAIL uses smooth \mbox{approximation}: the strength $K_u$ in Hill's equation is considered constant. This means that the beta functions $\beta_u$ are constant at each segment. In that case the betatron oscillation wavelength is $2\pi \beta_u$ and the betatron tune is $Q_u=\frac{C_0}{2\pi \beta_u}$. The phase advance for each segment is $\Delta \psi_u=Q_u \frac{L}{C_0}$ where $L$ is the length of the segment and $C_0$ the machine circumference. which means that only a few segments are defined per turn. Note that only a few segments are defined per turn which results in a rough optics model.
    
    % which results to a very  rough optics model.
    \item \textbf{Bunch initialisation:}  A particle bunch is represented by a collection of macroparticles, each of which represents a clustered collection of physical particles. Each macroparticle is described by four transverse $(x, x^\prime, y, y^\prime)$ and two longitudinal coordinates $(z, \delta)$, a mass and an electric charge. For the studies presented in this thesis $10^{5}$ macroparticles are sufficient for an accurate representation of the bunch, unless it is stated otherwise. There are various distributions available. In this thesis the simulations are performed using a Gaussian distribution in both transverse and longitudinal planes.
    \item \textbf{Transverse tracking}: In the transverse plane, the macroparticles are transported from one interaction point to another 
    (e.g. grom IP0 to IP1) following the matrix formalism of Eq.~\eqref{eq:matrix_formalism_intro}. The linear transfer matrices, $M$, introduced in Eq.~\eqref{eq:linear_transfer_matrix} take into account the optics parameters at the beginning and the end of the corresponding segment.
    %In the transverse plane, the macroparticles are transported from one interaction point to another 
    %(e.g. grom IP0 to IP1) by linear transfer matrices which take into account the optics parameters at the beginning and the end of the corresponding segment following Eq.~\eqref{eq:matrix_formalism_intro}.
    % For example, in the horizontal plane the transport of each macroparticle from IP1 to IP2 along the ring is given by Eq.~\eqref{eq:matrix_formalism_intro}:
    %\begin{equation}
    %    \binom{x_i}{x_i^\prime}_{\mathrm{IP0}} = M \binom{x_i}{x_i^\prime}_%{\mathrm{IP1}},
    %\end{equation}
    %where $i=1, ..., N$ with N being the number of macroparticles and the matrix $M$ is given by:
    %\begin{equation}\label{eq:linear_transfer_map}
    %    M = \begin{pmatrix}
    %        \sqrt{\beta_{x, \mathrm{IP0}}} & 0 \\ 
    %        -\frac{\alpha_{x, \mathrm{IP0}}}{\sqrt{\beta_{x,  \mathrm%{IP0} }}} & \frac{1}{\sqrt{\beta_{x, \mathrm{IP0}}}}
   %         \end{pmatrix} \begin{pmatrix}
   %             \cos(\Delta \mu_{x, \mathrm{IP0 \to IP1 }}) & \sin(\Delta \mu_%{x, \mathrm{IP0 \to IP1 }})\\
   %              - \sin(\Delta \mu_{x, \mathrm{IP0 \to IP1 }})  &\cos(\Delta \mu_{x, \mathrm{IP0 \to IP1 } }) 
   %             \end{pmatrix}  \begin{pmatrix}
   %                 \sqrt{\beta_{x, IP1}} & 0 \\ 
   %                 -\frac{\alpha_{x, IP1}}{\sqrt{\beta_{x, IP1}}} & \frac{1}%{\sqrt{\beta_{x, IP1}}}
   %                 \end{pmatrix} 
   % \end{equation}
    %where $\alpha_{x, \mathrm{IP0/IP1}}$ and $\beta{x, \mathrm{IP0/IP1}}$ are the optics paramters at the interaction points IP0 and IP1 respectively and $\Delta \mu_{x, \mathrm{IP0 \to IP1 }}$ the phase advance from IP0 to IP1. As smooth approximation is used, the phase advance equals: 
    %The phase advance, for each segment equals:
    %\begin{equation}\label{eq:phase_advance_smooth_approximation}
    %    \Delta \psi_{u, \mathrm{IP0 \to IP1 }} = Q_u \frac{L}{C},
    %\end{equation}
    %where $Q_u$ is the transverse betatron tune, $C$ the machine circumference and $L$ the length of the corresponding segment. 
    It should be noted that if no detuning source is added (see next step) the matrix $M$ is the same for all particles.
    \item \textbf{Chromaticity and detuning with transverse amplitude:} The chromaticity (up to higher orders) and amplitude detuning are implemented as a change of the phase advance of each individual particle as follows (example for the horizontal plane):
    \begin{equation}\label{eq:change_phase_advance_detunign}
        \Delta \psi_{x, i \mathrm{IP0 \to IP1 }} = \Delta \psi_{x, \mathrm{IP0 \to IP1 }} + (Q^\prime_x \delta_i + \alpha_{xx}J_{x, i} + \alpha_{xy}J_{y,i}) \frac{\Delta \psi_{x, \mathrm{IP0 \to IP1 }}}{2\pi Q_x}, 
    \end{equation}
    where $i=1, ..., N_\mathrm{mp}$ with $N_\mathrm{mp}$ being the number of macroparticles, $\Delta \psi_{x, \mathrm{IP0 \to IP1 }}$ is the phase advance for all macroparticles defined in the previous step, $Q_x^\prime$ the linear horizontal chromaticity, $\alpha_{xx}$ and $\alpha_{xy}$ are the detuning coefficients, while $J_x$ and $J_y$ are the horizontal and vertical actions of the macroparticle. %Therefore, in the presence of detuning the elements of the $M$ matrix (introduced in Eq.~\eqref{eq:linear_transfer_matrix}) are different for every particle.

    \item \textbf{Longitudinal tracking:}  %In PyHEADTAIL the longitudinal particle dynamics are described with the longitudinal equations of motion introduced in Eq.~\eqref{eq:long_eq_motion_z} and ~\eqref{eq:long_eq_motion_delta}
    The longitudinal coordinates are updated once per turn after solving numerically the equations of motion introduced in Eq.~\eqref{eq:long_eq_motion_z} and Eq.~\eqref{eq:long_eq_motion_delta}. The motion can be linear or not (non-linear RF system). The studies presented in this thesis use the linear longitudinal tracking.
    %  The solutions of the above equation is obtained numerically~\cite{pyheadtail_schenk}. 
    % "numerically", slide 37: https://www2.kek.jp/accl/legacy/seminar/file/PyHEADTAIL_PyECLOUD_2.pdf . Symplectic integration E. Forst and W. herr

    \item \textbf{Transverse impedance effects:} In PyHEADTAIL, wakefield kicks are used to implement the effect of the transverse impedance in the time domain. To improve the computational efficiency, the total impedance of the full machine is lumped at one of the interaction points along the ring and the kicks are applied to the macroparticles once per turn. Additionally, instead of computing the wakefield kicks from each particle to the rest individually, the bunch is divided into a number of slices longitudinally and the macroparticles in each slice recieve a wakefield kick generated by the preceding slices\footnote{This is valid in the ultrarelativistic scenario when no wakefield is generated in front of the bunch. The condition applies for the SPS experiments described in this thesis, performed at 270\,GeV.}~\cite{Salvant:1274254}. This is illustrated schematically in Fig.~\ref{fig:longitudinal_slicing_wakefields}. A large number of slices is required such as the wakes can be assumed constant within the slice. A high number of macroparticles is also needed in order to avoid statistical noise effects caused by undersampling~\cite{pyheadtail_manual_adrian}. For the studies presented, 500 slices are used over a range of three rms bunch lengths in both directions from the bunch center with the bunch being represented by $10^6$ macroparticles (instead of $10^5$ required for simulations without impedance effects).
    


    \begin{figure}[!ht]
        \centering
        \begin{subfigure}[t]{0.45\textwidth}
            \centering
            \includegraphics[width=1\textwidth]{images/Ch2/before_slicing.png}
            \caption{Without slicing}
            %\label{fig:add_label_here}
        \end{subfigure}
        \hfill
        \begin{subfigure}[t]{0.45\textwidth}
            \centering
            \includegraphics[width=1\textwidth]{images/Ch2/after_slicing.png}
            \caption{With slicing}
            %\label{fig:add_label_here}
        \end{subfigure}
        \hfill
         \caption{Longitudinal bunch slicing for the implementation of wakefield kicks in PyHEADTAIL. Without the slicing technique (left) the wake kicks on the red macroparticle are generated from all the green macroparticles resulting in computationally expensive simulations. Instead, when the bunch is sliced longitudinally (right) the wake kicks on the macroparticles in the red slice $i$ are generated by the macroparticles in the green slices $j$, decreasing significantly the computation time. The figures are courtesy of M. Schenk~\cite{pyheadtail_schenk}} % bunch passage
         \label{fig:longitudinal_slicing_wakefields}
     \end{figure}
       
    Every turn, the change on the $u^\prime$ co-ordinate, where $u=(x,y)$, of all the macroparticles in slice $j$ by the effect of the wakefields generated by all slices $i$ is modeled as follows~\cite{pyheadtail_manual_adrian}:
    \begin{equation}\label{eq:wakefield_kick_pyheadtail}
        \Delta_{u,j}^\prime = - \frac{q^2}{p_0 c}\sum_{k=j+1}^{N_\mathrm{sl}}N_k W_u(k(z_i-z_j))\langle u_k \rangle,
    \end{equation}
    where $N_\mathrm{sl}$ is the number of longitudinal slices, $N_k$ and $\langle u_k \rangle$ the number of macroparticles ad the center of mass in $u$ plane in the $k^\mathrm{th}$ slice.
    
    
    The wake functions are available from the detailed imepdance model of the machine, which are obtained from a combination of theoretical computations and electromagnetic simulations and can be imported in PyHEADTAIL in form of tables~\cite{pyheadtail_manual_adrian}. Typically, the impedance model of an accelerator is provided normalised with the average value of the horizontal and vertical beta functions over the machine at the respective plane. Therefore the transverse beta functions at the location where the wakefield kicks are applied on the beam must equal these average values.% More details on the SPS impedance model are provided in Section~\ref{sec:sps_impedance_model}.
    
    \item \textbf{Data acquisition:} The updated macroparticle coordinates after each turn are available at IP0 for post-processing. Typically, $10^{5}$ turns are required for the noise-induced emittance growth simulations presented in this thesis. 
    
\end{enumerate}

%Last sentence from (LinearMap): https://github.com/PyCOMPLETE/PyHEADTAIL/blob/master/PyHEADTAIL/trackers/longitudinal_tracking.py
%or longutidanl equations of motion: slide 37 https://www2.kek.jp/accl/legacy/seminar/file/PyHEADTAIL_PyECLOUD_2.pdf

Figure~\ref{fig:pyheadtail_accelerator_model} shows a graphic representation of the accelerator model and the tracking procedure supporting the steps described above.
% Graph is created app.diagrams.net and is saved in goodle dirve.

\begin{figure}[!h]
    \centering         
    \includegraphics[width=0.6\textwidth]{images/Ch2/accelerator_model_graph_pyheadtail.png}
        \caption{Graphical representation of the accelerator model and tracking procedure in PyHEADTAIL %(inspired by the graphs in Refs.~\cite{pyheadtail_schenk, inproceedings_ibs_pyheadtail}).
        In this example the ring is split in four segments seperated by the interaction points (IPs). Wakefield and multipole kicks are applied to the macroparticles in IP2 and IP3. The macroparticles are transported between the IPs by a linear transfer map (which can include detuning effects) in the transverse plane.  The longitudinal coordinates are updated once per turn.}
        \label{fig:pyheadtail_accelerator_model}
 \end{figure}


\subsection{Sixtracklib}\label{subsec:sixtracklib}
%Introduction to sixtrackib: https://indico.cern.ch/event/833895/contributions/3577803/attachments/1927226/3190636/intro_sixtracklib.pdf
% https://inspirehep.net/files/6273430c727ace3796a92d069f651ade

Sixtracklib~\cite{sixtracklib_repo} is a library for performing single charged particle simulations developed at CERN. It simulates the motion of the particles in the six-dimensional (6D) phase space. The individual trajectories are computed taking into account the interactions with all the magnetic elements in the machine using the detailed design optics model described in Section~\ref{sec:optics_model_designs}. The particles advance from one element to the other with transfer maps which are obtained from the Hamiltonians that describe each element e.g. drift, dipoles, quadrupolar, high-order multipoles, RF cavities etc. Note that thin lens approximation is used. Simulations with Sixtracklib are time efficient as the library can run on Graphical Processing Units (GPUs). Further details on Sixtracklib implementation and usage can be found in~\cite{sixtracklib_introduction} and~\cite{Schwinzerl:2781835}. 




Since Sixtracklib code was not used as extensively as PyHEADTAIL for the studies of this thesis, no further description is provided here, but the reader is referred to the references mentioned above.

%Sixtracklib is described here in much less detail than PyHEADTAIL as its used less extensively for the studies in this thesis and additionally because the author didn't build and customize the machine and the beam dynamics effects like in PyHEADTAIL.





%Sixtractlib--> the tracking is performed from element to elemetn using hte thin lense approximation.