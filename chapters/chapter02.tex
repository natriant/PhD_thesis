%The definitions and notations used throughout this thesis are based on the book of A. Wolksi in Ref.~\cite{wolski2014} unless it is stated otherwise. 
In this chapter, the basic concepts of accelerator beam physics that are essential for understanding the studies presented here are introduced. A more complete description can be found in the books of the following references:~\cite{wolski2014, Wiedemann:1083415, Lee:1425444}. The focus is put on the concepts for synchrotrons with proton beams. Additionally, in the last section, the tracking simulation codes used in this work are described.

% Details p.11 Michael schenk
Synchrotrons are circular accelerators where the particles follow a fixed closed-loop path. In a synchrotron, electric fields accelerate the particles while magnetic fields steer and focus them. The magnetic fields are not constant but they vary according to the particles' energy, allowing acceleration and operation at very high (relativistic) energies. The LHC and SPS machines at CERN are synchrotrons like many of the machines used for High Energy Physics experiments. Usually, in synchrotrons, the beams consist of multiple bunches, longitudinally spaced around the machine.  Although the bunches interact with each other, these interactions are not relevant to the studies presented later in this thesis, and will not be considered further

Finally, at this point, it is appropriate to introduce the terms incoherent and coherent effects. Incoherent effects (microscopic approach) affect the individual particles affect individual particles.  Any theory or model of incoherent effects has to treat the beam as a collection of a large number of individual particles, each with its own behaviour. In contrast, coherent effects can be understood in terms of their impact on the beam as a whole, and can be modelled by representing each bunch, for example, as a single 'macroparticle', with mass and charge corresponding to the total number of particles it contains. % from Andy's corrections


% A) p. 24 M. Schenk
% B) More details in: https://uspas.fnal.gov/materials/09UNM/Unit_8_Lecture_19_Limting_phenomena.pdf

%motion is the microscopic approach where the individual motion of each particle is studied. Coherent motion is the macroscopic approach where the beam is considered as a whole and the motion of the center of mass is studied.

% The section splitting is done here according to Wolski's book.
\section{Motion of charged particles in electromagnetic fields}
The motion of a particle with charge $q$ and velocity $\mathbf{v}=(v_x, v_y, v_z)$ moving in an electric field $\mathbf{E}$ and a magnetic field $\mathbf{B}$ is defined by the influence of the Lorentz force: 
\begin{equation}\label{eq:Lorentz_force}
 \mathbf{F}_L = q(\mathbf{E} + \mathbf{v} \times \mathbf{B}).
\end{equation}
% v is tangential to its path.

At this point, it is appropriate to mention that in this thesis the vectors are denoted in bold font (e.g. $\mathbf{E}$ ). % For the relativistic and ultrarelativistic regime the impacrt of E and B are the same. E = cB. p.44 Wille

In synchrotrons, the electric fields, which are generated by radiofrequency (RF) cavities, are used for accelerating the beams. The magnetic fields, are used to steer (dipoles) and focus (quadrupoles) and apply corrections (sextupoles, octupoles and higher order multipoles) to the motion of the beam.

\textbf{Reference trajecotry and reference momentum}\\
% Reference trajectory and reference moomentum are purely mathematical definitions. We chose them, usually we chose them to be convinient for the discreption. 

Here the concepts of reference momentum and reference trajectory are introduced. They both can be chosen arbitrary and are used as a reference for describing the particles motion in the magnetic fields of an accelerator. It might be the case that no particle follows the reference trajectory and has the reference momentum.

The reference trajectory is a chosen curved path in space which acts as a reference for the alignment of the accelerator compoenents. It is defined by the dipoles and is typically chosen such as it passes from the center of all the magnets. % not always the case and in some cases on purpose
The total length of the reference trajectory in a circular accelerator is called the circumference, $C_0$.


% Definition based on Wolski's book p.72.
% Alternative: The reference trajectroy is chosen by us, usually for convinience is chosen to be the one followed by the reference particle.

The particle that follows this trajectory is called the reference particle and has a momentum $p_0$, an energy $E_0$, and a velocity $v_0$. This particle is often called the synchronous particle as it crosses an RF cavity always at the same phase (assuming constant speed and no losses). %https://indico.cern.ch/event/22574/contributions/475143/attachments/371243/516589/IntroductionToAccelerators.pdf slide 36
For a proton, the reference momentum is given by: $p_0 = \gammarel m_p v_0$, where $m_p$ is the proton rest mass, and $\gamma_0 = \frac{1}{\sqrt{1-\beta_0^2}}$ is the relativistic gamma or Lorentz factor, where $\beta_0=v_0/c$ is the relativistic $\beta$ with $c$ being the speed of light. 

\textbf{Mangetic rigidity}\\
At this point, it is appropriate to introduce the concept of magnetic rigidity, $B_0 \rho$, which is often used in accelerators as a normalisation factor and is a measure of how the charged particles resist bending by a dipolar magnetic field. Assuming that a proton moves only under the influence of a uniform vertical dipole field $\mathbf{B_0}=(0, B_0, 0)$, it would follow a circular path of radius $\rho$ (it will be often referred to as bending radius) which is defined by the Lorentz force (Eq.~\eqref{eq:Lorentz_force}) being equal to the centripetal force, as follows:

\begin{equation}\label{eq:Brho}
    e v_0 B_0 = \frac{\gammarel m_p v_0^2}{\rho} \Rightarrow B_0 \rho = \frac{\gamma_0 m_p v_0}{e} \Rightarrow B_0 \rho = \frac{p_0}{e},
\end{equation}
where $e$ and $m_p$ are the charge and rest mass of a proton respectively, $p_0$ is the reference momentum, and $\gamma_0, \beta_0$ the relativistic gamma and beta. In the ultra-relativistic regime which is the case in the studies presented in this thesis, the approximation $\beta_0=1$ is often used.
% IMPORTANT!!! -->  v_0 = v_z. You explain this in the next paragraph.

If the particle momentum is given in $\mathrm{GeV /c}$ (which are the usuall units in high energy accelerators) then the unit of magnetic rigidity is $\mathrm{T \cdot m}$. 

%- https://uspas.fnal.gov/materials/12MSU/xverse_dynamics.pdf

\textbf{Co-ordinate system}\\
However, the individual particles do not follow the reference trajectory due to small deviations in their initial conditions: an example trajectory is shown in Fig.~\ref{fig:coordinate_system} with the blue line. The co-ordinate system used to describe the individual trajectories of the beam particles around the accelerator is illustrated in Fig.~\ref{fig:coordinate_system} and it is known as Frenet-Serret system.  It consists of the orthogonal co-ordinate system $\Sigma(s) = (\mathbf{e_x}, \mathbf{e_y}, \mathbf{e_z})$ whose origin moves along the reference trajectory (red line) together with the beam. 

The variable $s$ denotes the distance along the reference trajectory. In accelerator physics, $s$ is usually chosen as the independent variable instead of time, $t$.  %why? p.19 https://www.bnl.gov/isd/documents/74289.pdf
Therefore, at any given location $s$ around the ring, the coordinates $(x(s), y(s), z(s))$ give the horizontal, vertical, and longitudinal position of the particle with respect to the origin of the orthogonal moving system $\Sigma$. In the following paragraphs, the dependence of the co-ordinates on the position $s$ along the ring is omitted when possible to facilitate the notation (e.g. $x(s)$ will be denoted as $x$).

\begin{figure}[!h] % at the directory of ipac22
    \centering         
    \includegraphics[width=0.8\textwidth]{images/Ch2/coordinates_particle_motion.png}
        \caption{Co-ordinate system used to describe particles motion in a synchrotron. This is a local co-ordinate system, with the origin following the reference trajectory around the accelerator.  The unit vector $\mathbf{e_z}$ is tangential to the reference trajectory at each point, $\mathbf{e_y}$ is vertical, and $\mathbf{e_x}$ is horizontal, and perpendicular to $\mathbf{e_z}$ and $\mathbf{e_y}$.}
        \label{fig:coordinate_system}
 \end{figure}


 At any point $s$ along the reference trajectory each particle is represented by the six-dimensional "phase space" vector $(x, x^{\prime}, y, y^{\prime}, z, \delta)$ where:

 \begin{subequations}\label{eq:particle_coordinates}
    \begin{equation}
        x^\prime = \frac{dx}{ds} = \frac{dx}{dt}\frac{dt}{ds} = \frac{v_x}{v_z} =  \frac{p_x}{p_z} \approx \frac{p_x}{p_0},
    \end{equation}    
    \begin{equation}
        y^\prime = \frac{dy}{ds} = \frac{dy}{dt}\frac{dt}{ds} = \frac{v_y}{v_z} =  \frac{p_y}{p_z} 	\approx \frac{p_y}{p_0},
    \end{equation} 
    \begin{equation}
        \delta = \frac{\Delta p}{p_0} = \frac{p-p_0}{p_0},
    \end{equation}
    \begin{equation}
        z = \betarel c (t_0 -t),
    \end{equation}
\end{subequations}

where $p_0$ and $beta_0$ are the momentum and relativistic (scaled) velocity, respectively, of the reference particle, $t_0$ is the time  which the reference particle arrives at the location $s$ and $t$ is the time at which the individual particle arrives at the same location. It can be seen that $\delta$ is the relative momentum offset from the reference particle. In order to avoid a possible misconception it seems appropriate to clarify here, that the longitudinal parameter $z$ indicates the longitudinal offset from the reference particle at the center of the bunch. If $z>0$ ($z < 0$) the corresponding particle arrives earlier (later) than the center of the bunch at an arbitrary reference point. Last, in the ultra-relativistic regime the momentum of the particles in the $\mathbf{e_z}$ direction is much larger than the transvserse ones and almost equals the reference momentum: $p_x, p_y \ll p_z \approx p_0$. This is why the values of $x^\prime$ and $y^\prime$ are close to $p_x/p_0$ and $p_y/p_0$, respectively.

%Furthermore, if the direction of motion of the particle has a small angle with the reference trajectory, the paraxial approximation is valid: $p_x, p_y \ll p_z = p_0$ which means that $x^\prime \approx p_x$ and  $y^\prime \approx p_y$. This approximation is valis for the studies presented in this thesis. \textcolor{red}{$p_0=1$? not clear to me} % ultra relativistic regime, is not mentioned at wolski's book. I wrote it for the p_ 0= 1. but i am not sure.
% p.149 wolski and p.12 Michael schenk

%is the momentum of the reference particle which is given by: $p_0 = \gammarel m_0 v_0$, where $m_0$ is the proton rest mass and
%It can be seen that $x^\prime$ and $y^\prime$ are basically the normalised momenta with the reference momentum $p_0$ and $\delta$ is the relative momentum offset from the reference particle. 

%Initial I have written: Furthermore, the ultra-relativistic regime, the transverse momentum/velocity of a particle is very small comparing to the longitudinal one. Therefore the paraxial approximation is used: $p_x, p_y \ll p_z = p_0 = 1$ which can be applied to simplify Eqs.~\ref{eq:particle_coordinates}. %Or see Andy's explanation for low emittance storage rings: https://cds.cern.ch/record/1982424/plots

To summarize, the motion of the particles is treated separately in the transverse and longitudinal planes where it is described with the  $(x, x^\prime, y, y^\prime)$ and $(z, \delta)$ co-ordinates respectively. As an example the co-ordinates of the reference particle are $(x=0, x^\prime=0, y=0, y^\prime = 0, z=0, \delta=0 \ (p_z=p_0))$.

Last, $(x, y, z)$ are expressed in meters, $(x^\prime, y^\prime)$ in radians while $\delta$ is dimensionless.

\section{Single-particle beam dynamics}
In this section, the interactions between the particles within a bunch are neglected, hence the term single-particle beam dynamics. 

\textbf{Two-dimensional complex fields}\\
As already discussed, the motion of the charged particles inside a circular accelerator is controlled by magnetic fields. In this thesis, the magnets are considered purely transverse elements. Their effect is therefore described with two-dimensional multipole fields, acting in the horizontal and vertical planes\footnote{Examples of three-dimensional treatment can be found in~\cite{wolski2014, Beth:889480}. However, the two-dimensional treatment is most often used in accelerator physics as it provides a good description for the majority of the magnetic elements.}. % Wolski Ch1.3, p.33

The description of two-dimensional magnetic fields in accelerator physics is discussed using the concept of multipole expansion and is expressed as a complex quantity. The complex quantity allows to describe a two-dimensional field in $(x,y)$ space (to be compatible with the co-ordinates used for describing the particle's trajectory as discussed in the previous section). Therefore, the magnetic field around the beam is expressed as follows~\cite{wolski2014}: 
\begin{equation}\label{eq:mult_expansion} % Andy Eq.1.29 + M Schenk Eq.2.2
    B_y(x,y) + i B_x(x,y) = \sum_{n=1}^{\infty} C_n (x+i y)^{n-1},
\end{equation} % principle of superposition 
where $n$ indicates the order of the field component: $n$=1 for a dipole (steering), $n$=2 for quadrupole (focusing), $n$=3 for sextupoles (chromaticity correction), $n$=4 for octupole (error or field correction) etc. $C_n=(b_n +i a_n)$ is a complex constant which denotes the strength and orientation of the multipole field. The coefficients $b_n=\frac{1}{(n-1)!} \frac{\partial^{n-1}B_y}{\partial x^{n-1}}$ and $a_n=\frac{1}{(n-1)!} \frac{\partial^{n-1}B_x}{\partial x^{n-1}}$ denote the strength of a normal and skew (normal multipole rotated by $\pi/2(n)$) multipole respectively in units of $\mathrm{T/m^{n-1}}$.
% Equations for coefficients, sofia's thesis Eq.(2.29) p. 23: https://cds.cern.ch/record/2743602/files/CERN-THESIS-2020-169.pdf
% skew quadrupole is a normal quadrupole rotated by 45 deg.

Usually, in accelerator physics the values of the multipole strengths are quoted normalised to the magnetic rigidity as defined in Eq.~\eqref{eq:Brho} and are denoted by:
\begin{equation}\label{eq:kn}
    k_n = \frac{b_n}{B_0 \rho},
\end{equation}
where $k_n$ is expressed in units of  $\mathrm{m^{-n}}$. This is the convention that will be used in this thesis.

It should be clarified here, that the notations of $B_0$ and $b_1$ are equivalent. However, in accelerator physics the magnetic rigidity is denoted as $B_0 \rho$ as a convention due to its derivation at the early stages of the analysis (see Eq.\eqref{eq:Brho}). This notation will be used throughout this thesis.
% confirmed because k1 = b1/B0ρ = 1/ρ

\subsection{Transvserse motion}
In the transverse plane the motion is orthogonal to the reference trajectory (see Fig.~\ref{fig:coordinate_system}) and its co-ordinates are $(x, x^\prime, y, y^\prime)$. For the discussion on the transverse beam dynamics, the $(x, x^\prime)$ and $(y, y^\prime)$ co-ordinates will be both described by $(u, u^\prime)$ when possible to facilitate the notation.

\subsubsection{Linear dynamics}
Here the transverse motion of a particle moving through the two-dimensional fields described in Eq.~\eqref{eq:mult_expansion} is discussed. For now, the discussion is limited only to dipolar and quadrupolar components ($n=1$ and $n=2$) which are considered the basic magnetic elements, as in the absence of magnetic errors or momentum deviations between the particles they are sufficient to create a synchrotron.

As mentioned above, the particles transversely oscillate around the reference trajectory (except for a nominal reference particle which follows it). This motion, through an arbitrary periodic sequence of dipoles and quadrupoles, is called betatron motion and can be described by the following equations of motion~\cite{Lee:1425444}: % also sofia's thesis Eq.(2.37)

 \begin{equation}\label{eq:transverse_eq_x}
   x^{\prime \prime} - \frac{\rho+x}{\rho^2} = - \frac{B_y}{B_0 \rho} \frac{p_0}{p} \left (  1+ \frac{x}{\rho} \right )^2, 
 \end{equation}

\begin{equation}\label{eq:transverse_eq_y}
    y^{\prime \prime} = \frac{B_x}{B_0 \rho} \frac{p_0}{p}  \left (  1+ \frac{x}{\rho} \right )^2, 
\end{equation}
where $B_0 \rho$ and $\rho$ the magnetic rigidity and radius as defined in Eq.~\eqref{eq:Brho},  $B_y, B_x$ the transverse magnetic fields of Eq.~\eqref{eq:mult_expansion}, and $p_0$ the reference momentum.

In the case of dipole and quadrupole fields, linear approximations to the equations of motion (equations describing the particle motion through the fields) provide good representations of the dynamics, at least in the case of the transverse motion. Hence the name linear dynamics. For the linear approximation the following assumptions are made: the transverse offset and angle of motion of the rest of the particles ($x, x^\prime, y, y^\prime$) are very small and close to the reference trajectory and only the linear terms in $x$ and $y$ of the magnetic field (described in Eq.~\eqref{eq:mult_expansion}) are taken into account, such as~\cite{Lee:1425444}: % This linear expantion is only valid at the vincinity of the reference orbit.
\begin{equation}
    B_y = B_0 + \frac{\partial By}{\partial x}x  =B_0 + b_2 x, \ \ B_x = \frac{\partial Bx}{\partial x}y  = b_2 y.  
\end{equation}
Note that the term $B_0$ doesn't appear in the horizontal magnetic field due to the fact that, as mentioned earlier, only vertical dipole fields are considered. 

Exapnding Eqs.~\eqref{eq:transverse_eq_x} and~\eqref{eq:transverse_eq_y} to the first order of $x$ and $y$ respectively, setting $\delta = (p-p_0)/p_0$, and $k_2=b_2 /(B_0\rho)$ the linear equations of motion are obtained~\cite{Lee:1425444}:
\begin{equation}\label{eq:linear_eq_of_motion_x}
    x^{\prime \prime} + \left (  \frac{1-\delta}{\rho^2 (1+\delta)}  + \frac{k_2(s)}{1+\delta} \right )x = \frac{\delta}{\rho (1+\delta)}, 
\end{equation}
\begin{equation}\label{eq:linear_eq_of_motion_y}
    y^{\prime \prime} -  \left ( \frac{k_2(s)}{1+\delta} \right ) y = 0,
\end{equation}
where the sign-convention is $k_2>0 \ (k_2<0)$ for focusing in the horizontal (vertical) plane. It is worth commenting that $1/\rho = B_0/(B_0 \rho) = b_1/(B_0 \rho)=k_1$ the normalised dipole strength.

For on-momentum particles (with $\delta=0$) \footnote{The solution of the equation of motion for off-momentum particles is discussed in the paragraph "Off-momentum effects: dispersion" later in this section.} the linear equations of motion are simplified even more to the equation for a harmonic oscillator (but with an $s$ dependent strength $K_u(s)$), named Hill's equation~\cite{Lee:1425444}:
\begin{equation}\label{eq:Hills_equation_1}
    u^{\prime \prime}(s) + K_u(s) u(s) = 0,
\end{equation}
where $u=(x,y)$ and:
 \begin{equation}\label{eq:Hills_equation_2}
    K_u(s) = \begin{dcases}
        \frac{1}{\rho^2(s)}+k_2(s), & u=x \\
        -k_2(s), & u=y 
    \end{dcases}
\end{equation}
with $k_2(s)$ being the normalised quadrupole strength. It should be noted that for Eq.~\eqref{eq:Hills_equation_1} it is assumed the motion in the horizontal plane is independent of the motion in the vertical plane, and vice-versa, i.e. the motion is uncoupled. %The derivation of why the Hill's equation describes the betatron oscillations for on-momentum particles can be found in~\cite{Latina_juas}. 

%Here the transverse motion of a particle moving through the two-dimensional fields described in Eq.~\eqref{eq:mult_expansion} is discussed. For now, the discussion is limited only to dipolar and quadrupolar components ($n=1$ and $n=2$) hence the name linear dynamics\footnote{The term "linear" refers not to the fields, but to the beam dynamics (specifically, the equations describing the particle motion through the fields). In reality, dipole and quadrupole fields can also lead to "nonlinear dynamics". However, in these cases, linear approximations to the equations of motion provide good representations of the dynamics, at least in the case of the transverse motion.}. Dipoles and quadrupoles are considered the basic magnetic elements, as in the absence of magnetic errors or momentum deviations between the particles they are sufficient to create a synchrotron. 

%As mentioned above, the particles (except for a nominal reference particle) transversely oscillate around the reference trajectory. This motion, through an arbitrary periodic sequence of dipoles and quadrupoles, is called betatron motion and can be described with the following equations of motion~\cite{Lee:1425444}: % also sofia's thesis Eq.(2.37)
%\begin{equation}\label{eq:transverse_eq_x}
%    x^{\prime \prime} - \frac{\rho+x}{\rho^2} = - \frac{B_y}{B_0 \rho} \frac{p_0}{p} \left (  1+ \frac{x}{\rho} \right )^2, 
%\end{equation}

%\begin{equation}\label{eq:transverse_eq_y}
%    y^{\prime \prime} = \frac{B_y}{B_0 \rho} \frac{p_0}{p}  \left (  1+ \frac{x}{\rho} \right )^2, 
%\end{equation}
%where $s$ is the distance along the reference trajectory, $B_0 \rho$ and $\rho$ the magnetic rigidity and radius as defined in Eq.~\eqref{eq:Brho},  $B_y, B_x$ the transverse magnetic fields of Eq.~\eqref{eq:mult_expansion}, and $p_0$ the reference momentum.

As already mentioned, Eq.~\eqref{eq:Hills_equation_1} resembles the equation of motion for a harmonic oscillator, but with an oscillation frequency that varies with position along the beamline (i.e. varies with s). For a circular accelerator $K_u$ is periodic: $K_u(s+C_0)=K_u(s)$, where $C_0$ is the periodicity of the accelerator and real-valued. The general solution of Hill's equations in this case is (using the Floquet transformation)~\cite{Lee:1425444}:
\begin{equation}\label{eq:Hills_solution}
    u(s) = A w(s) \cos{(\psi_u(s)+\psi_{u,0})},
\end{equation} % Lee p. 48
where $A$ and $\psi_{u,0}$ the integration constants and $w(s)$ and $\psi_u(s)$ are the amplitude and betatron phase functions, which are periodic functions with the same periodicity as $K_u$.

After substituting for $u(s)$ from Eq.~\eqref{eq:Hills_solution} in Eq.~\eqref{eq:Hills_equation_1}, we find after some calculation (see Appendix C.1) that the amplitude and phase functions fulfill the following equations: 
%Combination slide 27-28 of yiannis notes:  https://yannis.web.cern.ch/teaching/transverse.pdf and federicos thesis (p.11): https://cds.cern.ch/record/1481835/files/CERN-THESIS-2005-082.pdf

\begin{equation}\label{eq:phase_functions}
    w^{\prime \prime}_u + K_u(s) w(s) -\frac{1}{w_u(s)^3} = 0, 
\end{equation}
where:
\begin{equation}\label{eq:phase_function}
    \psi^{\prime}_u(s) = \frac{1}{w_u(s)^2} \Rightarrow \psi_u(s) = \int_{s_0}^{s} \frac{ds}{w_u(s)^2}
\end{equation}
The above equations are called the betatron envelope and phase equations.
%As $K_u(s)$ is a periodic function, the two independent solutions of the second order differential Eq.~\eqref{eq:Hills_equation_1} can be expressed using the Floquet theorem~\cite{Floquet1883} as follows (see Appendix A, Sec. I.5 in Ref.~\cite{Lee:1425444}):

%\begin{equation}\label{eq:Floquets_solutions}
%    u(s) = a_u w(s)e^{i \psi_u (s)}, \ u^*(s) = a_u w(s)e^{-i \psi_u (s)},
%\end{equation}

%where $\alpha_u$ is a constant, and $w(s)$ and $\psi(s)$ are the amplitude and betatron phase functions and $u^*(s)$ is the complex conjugate of $u(s)$. 
%$K_u(s)$ is real-valued, therefore the amplitude and phase functions fulfill the following equations: % Lee Eq.(2.38), % p.250 widerman

% If you want to see how you go from Eq.(2.10) to Eq.(2.11) see also Yiannis notes (p.27-28): https://yannis.web.cern.ch/teaching/transverse.pdf

\textbf{Courant-Snyder parameters}\\
At this point it is appropriate to introduce the betatron or Twiss or Courant-Snyder:% frpom Yiannis notes: https://yannis.web.cern.ch/teaching/transverse.pdf
\begin{subequations}\label{eq:twiss_func}
    \begin{equation}
        \beta_u(s) = w_u(s)^2,
    \end{equation}    
    \begin{equation}
       \alpha_u(s) = -\frac{1}{2} \beta^{\prime}_u(s),
    \end{equation} 
    \begin{equation}
       \gamma_u(s) = \frac{1+\alpha_u(s)^2}{\beta_u(s)},
    \end{equation}
\end{subequations}
with $\beta^{\prime}_u(s) = d\beta_u(s)/ds$.

The betatron phase advance from Eq.~\eqref{eq:phase_function} can be re-written using the beta twiss function as:
\begin{equation}\label{eq:phase_advance_definition_with_twiss}
    \psi_u(s)= \int_{s_0}^{s} \frac{ds}{\beta_u(s)}.
\end{equation}

The advantage of the Twiss parameters over using $w_u(s)$ and $\psi_u(s)$ is that they can be also used to describe the properties of a bunch of particles while the latter describe the motion of each particle individually. This is discussed in further details in the following paragraph "Transverse emittance". %https://en.wikipedia.org/wiki/Courant%E2%80%93Snyder_parameters

\textbf{Betatron tune}\\
Another important quantity in accelerator physics is the betatron tune of the machine, $Q_u$, which is the phase advance for one complete revolution around the machine divided by $2\pi$:
\begin{equation}\label{eq:betatron_tune}
    Q_u = \frac{\psi_u(s+C)-\psi_u(s)}{2\pi} = \frac{1}{2\pi} \oint_C \frac{ds}{\beta_u(s)},
\end{equation}
where $C$ is the circumference of the machine. As it can be seen, the tune also represents the number of betatron oscillations that a particle undergoes during one full revolution around the machine. 

The tune of the individual particles may vary due to effects such as the chromaticity, the detuning with their transverse amplitude, and collective forces (e.g. impedance) that will be discussed in the following paragraphs. The horizontal and vertical tune of the reference particle will be referred to as the bare tunes and define what is called the working point of the machine, $(Q_{x0}, Q_{y0})$. 

\textbf{Matrix formalism}\\
% w and psi are also solutions. now we write it in matrix notation
Knowing the lattice (element per element structure of the accelerator) the solutions $w_u(s)$ and $\psi_u(s)$ of the Hill's equation can also be described using a matrix formalism as follows:

\begin{equation}\label{eq:matrix_formalism_intro}
   \begin{pmatrix}
    u\\ 
    u^\prime
    \end{pmatrix}_{s_1} = M_u (s_1 |  s_0) \begin{pmatrix}
    u\\ 
    u^\prime
    \end{pmatrix}_{s_0},
\end{equation}

where $u=(x,y)$. The transfer matrix from the position $s_0$ to the $s_1$, $M_u (s_1 | s_0)$, can be expressed in terms of the Courant-Snyder parameters as~\cite{Lee:1425444}: %Eq.(2.42) lee

\begin{equation}\label{eq:linear_transfer_matrix}
    \begin{split}
    M_u (s_1 |  s_0) &= \begin{pmatrix}
        \sqrt{\frac{\beta_u(s_1)}{\beta_u(s_0)}} (\cos{\Delta \psi_u}+\alpha_u (s_0) \sin{\Delta \psi_u}) & \sqrt{\beta_u(s_0)\beta_u(s_1)}\sin{\Delta \psi_u} \\ 
         - \frac{1+\alpha_u(s_0) \alpha_u(s_1)}{\sqrt{\beta_u(s_0) \beta_u(s_1)}} \sin{\Delta \psi_u}+ \frac{\alpha_u(s_0) - \alpha_u(s_1)}{\sqrt{\beta_u(s_0) \beta_u(s_1)}} \cos{\Delta \psi_u} & \sqrt{\frac{\beta_u(s_0)}{\beta_u(s_1)}} (\cos{\Delta \psi_u}+\alpha_u(s_1) \sin{\Delta \psi_u})
        \end{pmatrix} \\ 
        &=\begin{pmatrix}
            \sqrt{\beta_u(s_1)} & 0 \\
            -\frac{\alpha_u(s_1)}{\sqrt{\beta_u(s_1)}}& \frac{1}{\beta_u(s_1)}
            \end{pmatrix} \begin{pmatrix}
            \cos{\Delta \psi_u} & \sin{\Delta \psi_u} \\
            -\sin{\Delta \psi_u}& \cos{\Delta \psi_u}
            \end{pmatrix} \begin{pmatrix}
            \frac{1}{\sqrt{\beta_u(s_0)}} & 0 \\
            \frac{\alpha_u(s_0)}{\sqrt{\beta_u(s_0)}} & \sqrt{\beta_u(s_0)}
            \end{pmatrix},
    \end{split}
\end{equation}
where $\Delta \psi_u = \psi_u(s_1)-\psi_u(s_0)$ is the betatron phase advance between the two locations while $\alpha_u(s_i)$ and $\beta_u(s_i)$ are the Courant-Snyder parameters at the location $s_i$, where $i=(0,1)$. Tansfer matrices provide a very convenient approach to accelerator beam dynamics, and will be used extensively throughout this thesis to study the motion of the particles in the accelerator lattice.

\textbf{Action-angle variables and phase space ellipse}\\
The solution of equation of motion (Eq.~\eqref{eq:Hills_equation_1}) can alternatively be expressed in action-angle co-ordinates $(J_u, \psi_u)$ as follows:
\begin{equation}\label{eq:position_action_anlge}
    u(s) = \sqrt{2 \beta_u(s) J_u} \cos{(\psi_u(s))}.
\end{equation} 
By differentiation the divergence $u^\prime$ is written as:
\begin{equation}\label{eq:divergence_action_anlge}
    u^\prime(s) = - \sqrt{\frac{2 J_u}{\beta_u(s)}} (\sin{(\psi_u(s))}+\alpha_u(s)\cos{(\psi_u(s))}),
\end{equation} 
where $\beta_u(s), \alpha_u(s)$ are the Twiss parameters as defined in Eq.~\eqref{eq:twiss_func}, $\psi_u(s)$ the betatron phase as defined in Eq.~\eqref{eq:phase_function} and $J_u$ is an integration constant which is defined by the initial conditions. % D. Amoriom p. 12

The action, $J_u$ is an invariant of the motion and can be written in terms of the Twiss parameters as: % A. Wolski p.137
\begin{equation}\label{eq:action_definition}
    J_u = \frac{1}{2} (\gamma_u(s) u^2(s) + 2 \alpha_u(s) u(s) u(s)^\prime + \beta_u(s) u^{\prime 2}(s)) = \mathrm{constant}.
\end{equation}

The trajectory of each individual particle can be plotted in phase space $(u, u^\prime)$ at a given position $s$ in the ring turn after turn. In phase space, the particle's path is an ellipse whose shape and orientation are determined by the Twiss parameters at the position $s$. This ellipse, named phase space or Courant-Snyder ellipse, is illustrated in Fig.~\ref{fig:phase_space_ellipse} and it has an area of $2\pi J_u$. It is worth mentioning, that the ellipse's size is different for each particle as it depends on their individual actions, $J_u$ i.e. their individual initial conditions. The center of the ellipse is the closed orbit which, in the absence of steering errors in a synchrotron, can be identified with the reference trajectory and is also shown in the plot.

\begin{figure}[!h] % at the directory of ipac22
    \centering         
    \includegraphics[width=0.6\textwidth]{images/Ch2/phase_space_ellipse.png}
        \caption{Phase space co-ordinates $(u, u^\prime)$ turn by turn, for a particle moving along the ring but at a particular position $s$ which is characterised by the following twiss parameters $[\alpha_u(s), \beta_u(s), \gamma_u(s)]$. In the labels shown on the diagram, the dependence on the $s$ parameter has been omitted.}
        \label{fig:phase_space_ellipse}
 \end{figure}

 \textbf{Normalised phase space}\\  %Federicos thesis 2.3.1 and sondre eq. 2.21a
 Often in accelerator physics, it is useful to transform the transverse phase space ellipse into a normalised phase space circle. % Wiedemman p.235 
 In the normalised phase space the co-ordinates $(u, u^\prime)$ are normalised with the twiss parameters $(\alpha_u, \beta_u)$ for the particular location $s$ around the ring, as follows~\cite{Wiedemann:1083415}: % widerman p.235, eq.8.92-8.93
 \begin{equation}\label{eq:normalised_coordinates_un}
     u_N(\phi) = \frac{u(s)}{\sqrt{\beta_u(s)}},
 \end{equation}
\begin{equation}\label{eq:normalised_coordinates_un_prime}
     u^\prime_N(\phi) = \frac{d u_N}{d \phi} = \sqrt{\beta_u(s)} u^\prime(s) + \frac{\alpha_u(s)}{\beta_u(s)}u(s),
 \end{equation}
 % The normalised transformation can also be found in Yiannis notes in the link below: https://yannis.web.cern.ch/teaching/transverse.pdf
where $\phi = \frac{\psi_u}{Q_u}$. It can be seen that the independent variable in the normalised co-ordinates is the phase advance (normalised with the betatron tune), $\phi$, instead of the location $s$ along the ring. % p.12 in the following link: https://cds.cern.ch/record/409584/files/thesis-99-070_Chapter1.pdf
Both of the normalised co-ordinates, $(u_N, u_N^\prime)$ are expressed in units of $m^{1/2}$.

Combining Eq.~\eqref{eq:action_definition}, Eq.~\eqref{eq:twiss_func}, Eq.~\eqref{eq:normalised_coordinates_un} and Eq.~\eqref{eq:normalised_coordinates_un_prime} the action variable can also be written as:
\begin{equation}\label{eq:action_normalised_coordinates}
    J_u = \frac{1}{2} (u_N^2+u_N^{\prime 2}).
\end{equation}
% Plotting in p.4 of: http://www.toddsatogata.net/2013-USPAS/2013-01-23-NonlinearDynamicsNotes.pdf
The phase space in normalised co-ordinates is shown in Fig.~\ref{fig:phase_space_circle_normalised}.
\begin{figure}[!h] % at the directory of ipac22
    \centering         
    \includegraphics[width=0.6\textwidth]{images/Ch2/phase_space_normalised.png}
        \caption{Normalised phase space: the trajectory shown represents the particle motion as the particle moves around the synchrotron. This is not the case in the regular (not normalised) phase space, since the shape of the phase space ellipse will change with position around the ring. The particle moves turn by turn in a circle of radius $\sqrt{2J_u}$.}
        \label{fig:phase_space_circle_normalised}
 \end{figure}

The distribution of actions is an exponential distribution which implies that its mean equals its standard deviation. This property will be used for computations in the following chapters. % Actually, this property is used in the appendix for the computation of the rms tune spread.

%From Eq.~\eqref{eq:action} we write:
%\begin{equation}\label{eq:Jy_exp_distr}
%    e^{-J/\epsilon} = e^{(-x^2/2\epsilon - px^2/2\epsilon)}
%\end{equation}
%From this equation it can be seen that the actions follow an exponential distribution. 

\textbf{Transvserse emittance}\\
Up to now, the Twiss parameters were used to describe the dynamics of single particles. However, they also describe the distribution of the particles within a bunch. The statistical average of $u^2$ over all particles at a given point $s$ along the reference trajectory, from Eq.~\eqref{eq:position_action_anlge} equals to~\cite{wolski2014}:
\begin{equation}\label{eq:statistical_average_position}
    \langle u^2(s) \rangle = 2 \beta_u(s) \langle J_u \cos^2{\psi_u(s)} \rangle.
\end{equation}
%\begin{equation}\label{eq:mean_u_zerp}
 %   \langle u(s) \rangle = 0,
%\end{equation}
Assuming that the angle and action variables are uncorrelated Eq.~\eqref{eq:statistical_average_position} becomes:
\begin{equation}\label{eq:statistical_average_position_2}
    \langle u^2(s) \rangle = 2 \beta_u(s) \langle J_u \rangle \langle \cos^2{\psi_u(s)} \rangle.
\end{equation}
Considering tha the angle variables are are uniformly distributed from 0 to $2\pi$, it is written: % mean of a function: https://en.wikipedia.org/wiki/Mean_of_a_function
\begin{equation}\label{eq:mean_cosine_square}
    \langle \cos^2{\psi_u(s)} \rangle = \int_{s_0}^{s_0+C} \cos^2{\psi_u(s)} ds =   \int_0^{2\pi} \cos^2{\psi_u(\phi)} d\phi = \frac{\pi}{2\pi} = \frac{1}{2}.
\end{equation}
where for the integration the phase advance $\phi$ is used instead of the location $s$ along the ring for convinience. % like in Eq. (2.20)

Inserting Eq.~\eqref{eq:mean_cosine_square} in Eq.~\eqref{eq:statistical_average_position} gives:

\begin{equation}\label{eq:emittance_definition_1}
    \langle u^2(s) \rangle = \beta_u(s) \epsilon_u,
\end{equation}
where
\begin{equation}\label{eq:geom_emittance_action}
    \epsilon^{\mathrm{geom}}_u=\langle J_u \rangle
\end{equation}
is the geometric emittance of the bunch. Considering the same assumption Eq.~\eqref{eq:position_action_anlge} and  Eq.~\eqref{eq:divergence_action_anlge} results to:
\begin{equation}\label{eq:u_uprime_eq_1}
    \langle u(s) u^\prime(s) \rangle = - \alpha_u(s) \epsilon^{\mathrm{geom}}_u,
\end{equation}
\begin{equation}\label{eq:u_uprime_eq_2}
    \langle u^{\prime 2}(s) \rangle = \gamma_u(s) \epsilon^{\mathrm{geom}}_u.
\end{equation}

Combining the above equations, the geometric emittance is expressed in terms of the particles' distribution as:
\begin{equation}\label{eq:geometric_emittance_v2}
    \epsilon^{\mathrm{geom}}_u = \sqrt{\langle u^2(s) \rangle \langle u^{\prime 2}(s) \rangle- \langle u (s)u^{\prime}(s) \rangle ^2}
\end{equation}
which, also equals the square root of the determinant of the covariance or Sigma matrix of the particles' distribution:
\begin{equation}\label{eq:Sigma_matrix}
    \Sigma = \begin{pmatrix}
        \langle u^2(s) \rangle & \langle u(s) u^\prime(s) \rangle \\ 
        \langle u(s) u^\prime(s) \rangle & \langle u^{\prime 2}(s) \rangle 
        \end{pmatrix}  = \begin{pmatrix}
            \sigma_u^2(s) & \langle u(s) u^\prime(s) \rangle \\ 
            \langle u(s) u^\prime(s) \rangle & \sigma_{u^{\prime 2}(s)} 
            \end{pmatrix} 
\end{equation}

The square root of the top-left element of the covariance matrix, $\sigma_u$, is defined as the rms beam size and is also a variable that is used extensively in accelerator physics. The definition of rms and of others of the statistical analysis can be found in Appendix~\ref{ch:app_A}.

Figure~\ref{fig:phase_space_emittance} illustrates the concepts of emittance and rms beam size. It shows the phase space of a transverse Gaussian bunch along with the histograms of the $u$ (top) and $u^\prime$ (right) variables at a particular point $s$ along the ring. Each particle follows its individual ellipse (of different sizes but with the same orientation) depending on its initial conditions. The rms beam size, $\sigma_u$, and the rms normalised momentum spread, $\sigma_{u^\prime}$, are shown in the top and right histograms of Fig.\ref{fig:phase_space_emittance} with the blue vertical lines. This corresponds to the area of the ellipse enclosed in the blue line in the phase space plot and equals the rms or geometric emittance, $\epsilon^{\mathrm{geom}}_u$ , as defined in Eq.~\eqref{eq:geometric_emittance_v2}.

\begin{figure}[!h] 
    \centering         
    \includegraphics[width=0.9\textwidth]{images/Ch2/transverse_phase_space_emittance.png}
        \caption{Transverse phase space of a Gaussian bunch. The figure is a courtesy of Tirsi Prebibaj~\cite{tirsi_thesis_presentation}.}
        \label{fig:phase_space_emittance}
 \end{figure}

It should be noted, that there are also other conventions to define the emittance such as the 90$\%$ emittance (green lines in Fig.~\ref{fig:phase_space_emittance}) or the 3-sigma emittance (yellow lines in Fig.~\ref{fig:phase_space_emittance}). However, here the term geometric emittance will refer to the rms geometric emittance.

According to Liouville’s theorem~\cite{wolski2014}, assuming that there are no interactions between the particles and that the energy of the beam is not changing, the geometric emittance remains constant and therefore is an invariant of bunch motion (similarly to the action $J_u$ for the single-particle motion). The geometric emittance does not remain constant during acceleration, instead, the normalised emittance is defined as:
\begin{equation}\label{eq:normalised_emittance}
    \epsilon_u = \beta_0 \gamma_0 \epsilon^{\mathrm{geom}}_u
\end{equation}
The normalised emittance is conserved during acceleration and is often used as an alternative to the geometric emittance, especially in situations where the beam undergoes acceleration or deceleration. It is highlighted here, that throughout this thesis the term "emittance" will refer to the rms normalised emittance.

It is worth noting that for the simulation studies presented in this thesis, the emittance is computed using the statistical definition introduced in Eq.~\eqref{eq:geometric_emittance_v2}. In the experimental studies, the emittance is obtained from the rms beam size at a point in the beamline where the beta function is known. The procedure is explained in more detail in Chapter~\ref{Ch:2018_analyisis}, but here we simply note that from Eqs.~\eqref{eq:emittance_definition_1},~\eqref{eq:Sigma_matrix} and ~\eqref{eq:normalised_emittance}, we can express the emittance as:
% For a Gaussian beam distribution the normalised beam emittance it applies:
\begin{equation}\label{eq:emit_from_beam_size}
    \epsilon_{u} = \frac{\sigma_u(s)^2}{\beta_u(s)} \beta_0 \gamma_0,
\end{equation}

where $\sigma_u(s)$ is the rms beam size, $\beta_u(s)$ is the beta function, at specific location $s$ along the accelerator and $\beta_0, \gamma_0$ are the relativistic parameters. 

It should be highlighted that the emittance definitions of Eq.~\eqref{eq:geometric_emittance_v2} (after normalisation with the relativistic parameters) and of Eq.~\eqref{eq:emit_from_beam_size} are equivalent.

Finally, despite Liouville's theorem, in a real accelerator there are various phenomena that change the emittance such as~\cite{Buon:216507}: scattering by residual gas, intra-beam scattering, beam-beam scattering, stochastic or electron cooling, synchrotron radiation emission, filamentation due to non-linearities of the machine, space charge and noise effects. The studies in this thesis focus on the emittance growth due to noise effects (discussed in more detail in Chapter~\ref{Ch:CC_noise_theory}).

\textbf{Off-momentum effects: dispersion}\\
Up to now, the discussion was limited to on-momentum particles $\delta=0$: their momentum equals the reference momentum, $p_0$. In a more realistic beam, however, the momenta of the individual particles are spread around the reference momentum, $p_0$. The "spread" of momenta (or simply "momentum spread") is described by the rms momentum deviation $\sigma_\delta = \sqrt{\langle \delta^2 \rangle}$. As an example, in the SPS machine, which is of interest for this thesis, $\sigma_\delta$ is in the order of magnitude of $10^{-4}$ to $10^{-3}$.

The response of the off-momentum particles to a given force, during their passage from the different magnets in an accelrator, varies with their momentum. Here their motion through the dipole magnets which leads to dispersion effects is discussed. 

Particles with $\delta < 0$ ($\delta>0$) undergo larger (smaller) deflections from the dipole magnets than the reference particle due to lower (higher) magnetic rigidity. Therefore, they travel along the accelerator performing betatron oscillations not around the reference trajectory but around a different closed orbit as illustrated in Fig.~\ref{fig:closed_orbit_Dx} which depends on their momentum deviation, $\delta$. The change in the closed orbit with respect to the momentum deviation is called dispersion. It is evident that the dispersion introduces a coupling between the longitudinal and transverse planes.

\begin{figure}[!h] % 
    \centering         
    \includegraphics[width=0.6\textwidth]{images/Ch2/closed_orbit_dispersion.png}
        \caption{The closed orbit and the betatron oscillations around it in the presence of dispersion~\cite{Holzer_summer_students_introduction}}
        \label{fig:closed_orbit_Dx}
 \end{figure}

The following discussion is focused on the horizontal plane due to the fact that (as stated already) only vertical dipolar fields are considered. 
\footnote{A corresponding discussion can be done for the vertical plane to obtain the vertical dispersion but it is out of the scope of this thesis.}. The equation of motion for the off-momentum particles has already been discussed in Eq.~\eqref{eq:linear_eq_of_motion_x}. Its solution is~\cite{Lee:1425444}:
%\begin{equation}
%    x^{\prime \prime}(s) + K_x(s)x(s) = \frac{1}{\rho(s)} \delta,
%\end{equation}
\begin{equation}
    x(s) = x_H(s) + D_x(s)\delta,
\end{equation}
where $x_H(s)$ is the homogeneous solution shown in Eq.~\eqref{eq:position_action_anlge} (and corresponding here to betatron oscillations around the off-momentum closed orbit) and $D_x(s)$ is the dispersion function which can be expressed as:
\begin{equation}\label{eq:dispersion_function}
    D^{\prime \prime}_x(s) + K_x(s)D_x(s) = \frac{1}{\rho(s)},
\end{equation}
where $K_x(s)= \frac{1}{\rho^2(s)}+k_2(s)$ like in Eq.~\eqref{eq:Hills_equation_2}. As an example, the rms horizontal dispersion of the SPS machine is about 1.8\,m (model value). %/Users/nataliatriantafyllou/PhD_projects/exploring_SPS/madx_studies/optics_new_seq_after_LS2/output/twiss_thin_elements

The dispersion function can be repersented using the matrix formalism which therefore allows the addition of the dispersive contribution in the transfer matrix introduced in Eq.~\eqref{eq:matrix_formalism_intro}. This aspect is not discussed here, as it is not relevant for the studies in this thesis (the simulation studies consider zero dispersion function in both transverse planes). However, a detailed discussion can be found in Chapter~2 of Ref.~\cite{Lee:1425444}. %Lee p.126 Eq.(2.157), also appendix in Hannes' thesis.


%The dispersive contribution can be added to the transfer matrix introduced in Eq.~\eqref{eq:matrix_formalism_intro} as follows:
%\begin{equation}\label{eq:matrix_formalism_dispersion}
%    \begin{pmatrix}
%     x\\ 
%     x^\prime
%     \end{pmatrix}_{s_1} = M_x (s_1 |  s_0) \begin{pmatrix}
%     x\\ 
%    x^\prime
%     \end{pmatrix}_{s_0} + \delta \begin{pmatrix}
 %       D_x\\ 
 %       D_x^\prime
 %       \end{pmatrix}_{s_0}.
 %\end{equation}
%The transfer matrix $M$ can be re-written such as it inclyyded, alide 14 in presentation v2 below:

% 1. Definition of D: https://yannis.web.cern.ch/teaching/transverse.pdf
% 2. Definition of D prime: https://indico.cern.ch/event/847209/contributions/3558973/attachments/1932901/3201996/AccPhys_Lect7_MomEff.pdf

The dispersion was introduced above as it affects the definition of the normalised beam emittance, which is used to compute the emittance values from experimental measurements (see Chapter~\ref{Ch:2018_analyisis}). In particular, in the presence of dispersion the normalised beam emittance (Eq.~\eqref{eq:emit_from_beam_size}) becomes:
\begin{equation}\label{eq:emit_from_beam_size_Du}
    \epsilon_{u} = \frac{\sigma_u(s)^2 - \sigma^2_\delta D_u^2(s)}{\beta_u(s)} \beta_0 \gamma_0,
\end{equation}
where $\sigma_u(s)$ is the rms beam size, $\beta_u(s)$ is the beta function, $D_u(s)$ is the dispersion fat a specific location s along the accelerator, $\sigma_\delta$ is the momentum spread and $\beta_0, \gamma_0$ the relativistic parameters. The subsrcipt $u=(x,y)$ denotes the horizontal and vertical plane.
The impact of the vertical dispersion is included here, as only the model vertical dispersion equals zero. In a real machine, vertical dispersion can be introduced by sources such as steering errors of the dipole or quadrupole magnets~\cite{Wolski_uspas}. For reference, the rms vertical dispersion in the SPS machine is measured to be about 10\,cm.


Additionally, the off-momentum particles receive different focusing due to gradient errors in the quadrupoles. This effect is known as chromaticity and is discussed in detail in the next Section~\ref{subsec:non-liner_beam_dynamics} which focuses on non-linear beam dynamics.

\subsubsection{Non-linear dynamics}\label{subsec:non-liner_beam_dynamics}
Up to now, only linear elements (dipoles and quadrupoles) were considered as in theory they are sufficient to create a synchrotron. However, in a real machine non-linearities are also present due to factors such as imperfections in the magnets field and alignment, particles' momentum spread, and higher order magnets (sextupoles, octupoles, etc). Here, the preceding discussion is expanded to include the non-linear beam dynamics. The discussion is limited to the two effects that are important for the work presented in this thesis: the chromaticity and the detuning with transverse amplitude.

\textbf{Chromaticity}\\
We define the chromaticity as the variation of the betatron tune $Q_u$ with the relative momentum deviation delta. This is a result of the fact that particles with $\delta < 0$ ($\delta > 0$) are focused more (less) strongly from the quadrupoles due to their smaller (larger) magnetic rigidity. The tune shift introduced by the chromaticity for each particle, $ \Delta Q_u (\delta)= Q_u - Q_{u0}$, is: % M. Schenk p.49

%The tune shift introduced by the chromaticity is: % M. Schenk p.49

\begin{equation}\label{eq:chromatic_tune_shift_up_to_order_n}
   \Delta Q_u (\delta) = \sum_{n=1}^m \frac{1}{n!} Q_u^{(n)} \delta^n, 
\end{equation}
where:
\begin{equation}\label{eq:chroma_up_to_order_m}
    Q_u^{(n)} = \left. \frac{\partial ^n Q_u}{\partial \delta^n} \right|_{\delta=0}, n \in \mathbb{N},
 \end{equation}
denotes the chromaticity of order $n$. The studies in this thesis, are limited to the chromaticity at the first order in $\delta$ $(n=1)$ which is often called linear chromaticity. Note that the betatron tune shift of Eq.~\eqref{eq:chromatic_tune_shift_up_to_order_n} is referred to as an "incoherent" tune shift, since each particle is affected differently, depending on its individual momentum deviation.

Large values of chromaticity can lead to instabilities and therefore to beam loss~\cite{Lee:1425444}. Sextupole magnets are typically used to control the natural chromaticity of a machine and achieve the desired values for its operation.
% Lee page 159

Similarly to the tune, the chromaticity is a property of the machine lattice. % the chromaticity is one number

\textbf{Octupoles and detuning with amplitude}\\
Octupole magnets are most often used to increase the transverse tune spread of the beam particles to avoid resonances \footnote{Resonances in circular accelerators are a result of perturbation terms in the equation of motion once the perturbation frequency matches the frequency of the particles' oscillatory motion. The topic of resonances is out of the scope of this thesis, however, more details can be found in Chapter 16 of Ref.~\cite{Wiedemann:1083415}} and instability effects  \footnote{Beam instabilities in an accelerator are a result of the interplay of the wakefields (which will be discussed in Section~\ref{subsec:wakefields}) and a perturbation (e.g. noise) on equations of motion of the beam particles. Similar to the resonances their detailed study is out of the scope of this thesis, however, more details can be found in Ref.~\cite{Rumolo:1982422}.}. This property, of providing incoherent betatron tune spread in a controlled way is used extensively in this thesis and therefore some further details are discussed in the following.



The betatron tune spread or linear detuning that is introduced by the octupoles is action-dependent in both transverse planes. In terms of the action variable it is written as follows: % Michael schenk p.
\begin{equation}\label{eq:DQ_with_amplitude_horizontal}
    \Delta Q_x (J_x, J_y) = 2(\alpha_{xx} J_x + \alpha_{xy}J_y),
\end{equation}
\begin{equation}\label{eq:DQ_with_amplitude_vertical}
    \Delta Q_y (J_x, J_y) = 2(\alpha_{yy} J_y + \alpha_{yx}J_x),
\end{equation}
where $J_x, J_y$ the transverse action as introduced in Eq.~\eqref{eq:action_normalised_coordinates}, $\alpha_{xx}, \alpha_{yy}$ and $\alpha_{xy}=\alpha_{yx}$ are the detuning coefficients with units 1/m. The detuning coefficients depend on the octupoles strength, the beta functions at their location and the magnetic rigidity~\cite{Gareyte:321824}. This detuning with the transverse action (or amplitude) is an incoherent effect as it depends on the individual action of each particle.


In the SPS and LHC rings, the octupoles are installed in families (focusing and defocusing) in order to avoid the excitation of resonances. They are usually referred to as "Landau octupoles" since they are used to create a betatron tune spread that provides the mechanism of Landau damping~\cite{Herr:1982428} (to stabilise the beam).

\subsection{Longitudinal motion}
In the longitudinal plane, the motion is parallel to the reference trajectory and is described by the co-ordinates $(z, \delta)$. In the next paragraphs, only the basic concepts that are required for the explanation of the equations of motion are discussed, as the studies in this thesis mostly concern transverse beam dynamics. However, a complete discussion can be found in Chapter 9 of Ref.~\cite{Wiedemann:1083415}.

%In the next paragraphs the basic concepts used to describe the motion in the longitudinal plane which is also called synchrotronous motion are explained. In particular, the discussion is focused on the synchronous phase, the momentum compaction and phase slip factors, the concepts of phase stability, synchrotron oscillations and synchrotron tune tune. The equations of synchrotron motion are also given.

%In the longitudinal plane the motion is tangential to the reference trajectory and is described by the co-ordinates $(z, \delta)$. In the following discussion the longitudinal motion is treated independently from the transverse one. The complete approach, which describes the particles motion in the presence of coupling between the transverse and longitudinal planes can be found in Chapter~5.2 in the book of A. Wolski in Ref.~\cite{wolski2014}.

\textbf{Synchronous phase}\\
The time that the reference particle needs to complete one complete revolution around the machine is called the revolution period, $T_\mathrm{rev}$. Its angular revolution frequency is $\omega_\mathrm{rev} = 2\pi /T_\mathrm{rev}$ in rad/s or $f_\mathrm{rev}=1/T_\mathrm{rev} =  v_0/C = \beta_0 c/C$ in Hz, where $v_0$ the speed of the reference particle, $\beta_0$ the relativistic beta, $c$ the speed of light and $C$ the circumference of the accelerator.

In the longitudinal plane the acceleration and the focusing (in phase) of particles are achieved by the longitudinal time-dependent electric field of the main RF cavities:
% stefania's thesis p.20 and M. schenk thesis p.20 Eq.2.29
\begin{equation}\label{eq:RF_cavity_EF}
    E_\mathrm{{RF}}(t) = E_A \sin{(\phi_\mathrm{{RF}} + \phi_s)},
\end{equation}
where $E_A$ it the amplitude of the electric field, $\phi_\mathrm{{RF}}(t) = \omega_\mathrm{{RF}}t$ the phase of the RF system, $\omega_\mathrm{{RF}}$ the angular frequency of the RF system and $\phi_s$ is the phase of the synchronous or reference particle. The angular frequency needs to be an integer mulitple of the revolution frequency: $ \omega_\mathrm{RF} = h \omega_\mathrm{rev}$, where $h$ is called the harmonic number. The harmonic number (number of RF cycles per revolution) defines the maximum number of bunches that can be accelerated (or stored) in the ring. % the maximum number of sunchronous particles (around them the off momentum particles (rest of the bunch),
In a synchrotron during the energy ramp the angular frequency increases in order to follow the increasing revolution frequency.

Assuming that the synchronous or reference particle arrives at the RF cavity at phase $\phi_s$ every turn, the energy gain equals:
\begin{equation}\label{eq:enerhy_gain_synchronous}
    \Delta E_0 = e V_\mathrm{RF} \sin{(\phi_s)}, 
\end{equation}
where $V_\mathrm{RF}$ the amplitude of the RF cavity voltage. The rest of the particles will arrive at the RF cavity at phases $\phi = \phi_s \pm \delta \phi$ and they will gain or lose a different amount of energy per turns which equals: $\Delta E_p = e V_\mathrm{RF} \sin{(\phi)}$. 
 
\textbf{Dispersion effects}\\
As discussed in the previous chapter, in the presence of dispersion a particle with a momentum offset, $\delta$, from the reference particle will have a different closed orbit of different length (see Fig.~\ref{fig:closed_orbit_Dx}). This change of the orbit length with respect to the momentum offset of each particle is described with the momentum compaction factor~\cite{emetral_juas_2018}: %p.20
\begin{equation}\label{eq:compaction_factor}
    \alpha_p = \frac{\Delta C /C}{\delta} = \frac{1}{C} \oint _C \frac{D_x(s)}{\rho(s)} ds,
\end{equation} % proof hannes' thesis appendix p.169: https://cds.cern.ch/record/1644761/files/CERN-THESIS-2013-257.pdf
where $C$ is the circumference of the accelerator and $D_x(s)$ and $\rho(s)$ are the horizontal dispersion and bending radius respectively at a given point $s$.

With the change of the closed orbit length due to the momentum offset the revolution frequency of the particles also changes. The change of the angular frequency depending on the momentum offset is described with the phase slip factor:
\begin{equation}\label{eq:phase_slip_factor}
    \eta_p = -\frac{\Delta \omega / \omega_0}{\delta} = \alpha_p - \frac{1}{\gamma^2_0} = \frac{1}{\gamma^2_\mathrm{tr}} - \frac{1}{\gamma^2_0},
\end{equation}
where $\omega_0$ is the angular frequency of the reference particle, $\gamma_0$ is the Lorentz factor and $\gamma_\mathrm{tr} = 1/\sqrt{\alpha_p}$ is called the transition energy. When $\gamma_0 < \gamma_\mathrm{tr} \Rightarrow \eta_p <0$ ($\gamma_0 > \gamma_\mathrm{tr} \Rightarrow \eta_p > 0$) and the machine operates below (above) transition. For the nominal optics configuration, the SPS machine always operates above transition as $\gamma_\mathrm{tr}$ = 22.8 which is smaller than the relativistic gamma even for the injection energy ($\gamma_0$ = 27.7 at 26\,GeV).
% Source for transition energy value: https://accelconf.web.cern.ch/ipac2011/papers/mops012.pdf

\textbf{Phase stability and synchrotron oscillations}\\
Even though the particles arrive at different times in the RF cavity, they stay in the vicinity of the reference particle thanks to the effect of longitudinal or phase focusing, which is explained by the concept of phase stability~\cite{McMillan:1945zz, Veksler_1, Veksler_2}. Its principle is illustrated in Fig.~\ref{fig:phase_focusing} for a machine operating above transition. Above transition a particle with $\delta<0$ will follow a shorter closed orbit (than the reference trajectory) and therefore it will arrive at the RF cavity slightly earlier, than the reference particle and hence it will see a larger voltage. Therefore, it will be accelerated more than the reference particle and subsequently it will need less time to complete the next revolution and, over a number of revolutions, will fall back longitudinally towards the reference particle. The situation is the opposite for a particle with $\delta<0$. 
% Inspired by Fig. 2.3 in the thesis of M. Schenk and Fig.5.15 in wille's book.
\begin{figure}[!h] % 
    \centering         
    \includegraphics[width=0.6\textwidth]{images/Ch2/phase_focusing_synchrotron_motion.png}
        \caption{Phase stability for particles in a circular accelerator which operates above transition.}
        \label{fig:phase_focusing}
 \end{figure}

 In particular, the non-synchronous particles oscillate around the phase of the synchronous particle performing synchrotron oscillations (similarly to the betatron oscillations in the transverse plane). It is worth mentioning, that a complete synchrotron oscillation can take many ($\sim$100) turns, in contrast to the betatron oscillations (of which there are usually many complete oscillations per turn). The equations of motion for a particle passing through a system of synchronised RF cavities located around the accelerator are~\cite{wolski2014}: % wolski p. 173 Eq.(5.53) and (5.54)
\begin{equation}\label{eq:long_eq_motion_z}
    z^\prime = - \eta_p \delta,
\end{equation}
\begin{equation}\label{eq:long_eq_motion_delta}
    \delta^\prime = - \frac{qV_\mathrm{RF}}{c p_0 C} \left ( \sin{\phi_s} - \sin{\left ( \phi_s - \frac{\omega_\mathrm{RF} z}{c} \right)} \right ),
\end{equation}
where $z^\prime = dz/ds$, $\delta^\prime = d\delta/ds$, $e$ is the charge of a proton, $c$ is the speed of light, $p_0$ is the reference momentum, $C$ is the circumference of the machine, $\omega_\mathrm{RF}$ is the angular frequency of the RF system, $\phi_s$ is the phase of the synchronous particle and $\eta_p$ is the phase slip factor.

The synchrotron tune, $Q_s$, is the number of synchrotron oscillations performed during one complete revolution around the machine and is computed as follows~\cite{wolski2014}: % Wolski Eq.(5.49) p. 170
\begin{equation}\label{eq:Qs} 
    Q_s = \frac{1}{2\pi}\sqrt{-\frac{e V_\mathrm{RF}}{c p_0} \frac{\omega_\mathrm{RF} C}{c} \eta_p \cos{\phi_s}}.
\end{equation}
%where $e$ the proton charge, $V_\mathrm{RF}$ the amplitude of the RF cavity voltage, $c$ the speed of light, $p_0$ the reference momentum, $\omega_\mathrm{RF}$ the angular frequency of the RF system and $\phi_s$ the synchronous phase.

\section{Collective effects}\label{sec:collective_effects}
Up to now, the motion of the particles was studied neglecting the interaction between them within the bunch. Collective effects in an accelerator desribe the phenomena in which the motion of the particles depends on their interaction with each other or with electromagnetic fields. Examples are beam-beam interactions, space charge effects, wakefields and intra-beam scattering~\cite{Zimmermann:2264408}. The collective effects usually become crucial for high-intensity beams as they can lead to instabilities \footnote{A beam is called unstable when one of its co-ordinates $(x, x^\prime, y, y^\prime, z, \delta)$ undergoes exponential growth. Further details on the beam instabilities can be found in Ref.~\cite{Rumolo:1982422}} which then may degrade beam quality, or lead to beam losses.  In either case, the performance of the accelerator can be adversely affected. The discussion here is limited in the description of the wakefields and the impedance they are relevant for the studies presented in this thesis. A complete overview of the collective effects can be found in~\cite{wolski2014, Zimmermann:2264408}.
%For a more complete picture, a beam is called unstable when one of its co-ordinates $(x, x^\prime, y, y^\prime, z, \delta)$ undergoes exponential growth. Usually, this can be observed in the motion of the centroid $\langle x \rangle $
% If you want to show an example of an instability you can find a figure in Ref.~\cite{Rumolo:1982422}}

\subsection{Wakefields and impedance}\label{subsec:wakefields}
The discussion in this section is based in the discussion in Refs.~\cite{wolski2014, Rumolo:1982422, instabilities_rumulo_li, benoit_ipac19_impedance, Chao:collective}.

\textbf{Wakefields}\\
The charged particles within a beam interact electromagnetically with their surroundings in the beam pipe such as the resistive vacuum pipe walls, the RF cavities, etc. If these structures are not smooth (presence of discontinuities) or not perfectly conducting the interaction with the charged particles will result in electromagnetic perturbations called wakefields. Studying the effects of the wakefields is crucial as they act back on the beam affecting the beam dynamics.

For the study of the wakefields, it is considered that each particle acts as a source of wakefields for the rest of the particles\footnote{In studying collective effects, the terms "source particle" and "witness particle" are sometimes used for particles generating wakefields and particles affected by the wakefields, though in reality all charges act as sources of wakefields, and are affected by them} which are called witnesses. In the ultrarelativistic regime (which is also the regime of these studies) the wakefields from a source particle act only on the particles behind it, hence the term "wake". Wakefields can also act back on the charge generating them when that charge returns to a given location in a storage ring over successive turns. This is the multi-turn effect of the wakefields. In this thesis, only the multi-turn effect from the resistive wall will be considered as it is the only one that has a significant impact on the dynamics.

The longitudinal and transverse wakefields can often be treated separately. In the following only the transverse components will be discussed as the focus of the thesis is on the transverse beam dynamics.

\textbf{Wake functions}\\
Consider two particles of charge $q_1$ and $q_2$ moving with ultrarelativistic speed through a structure of length $L$ as shown in Fig.~\ref{fig:wakefields}~\cite{instabilities_rumulo_li}. The particle of charge $q_1$ is the source particle while the witness particle of charge $q_2$ travels behind it at a constant distance $z$. $(\Delta x_1, \Delta y_1)$ are the transverse offsets of charge $q_1$, and $(\Delta x_2, \Delta y_2)$ are the transverse offsets of charge $q_2$ from the symmetric axis of the beam pipe. From the interaction of the source particle with the structure of length $L$ a wakefield is generated. 

\begin{figure}[!h] % 
    \centering         
    \includegraphics[width=0.6\textwidth]{images/Ch2/wakefield_example.png}
        \caption{Wakefield interaction, where the source particle (blue) affects the witness particle (yellow) travelling at a distance $z$ begind it~\cite{instabilities_rumulo_li}. $(\Delta x_1, \Delta y_1)$ and $ (\Delta x_2, \Delta y_2)$ are the transverse offsets of the source and witness particles respectively.} 
        \label{fig:wakefields}
 \end{figure}
% Figure take from: ~\cite{instabilities_rumulo_li}

The wakefields in time domain are described with the concept of wake functions, $W_u(z)$, where $u=(x,y)$ denotes the horizontal and vertical wakefunction. The wakefunction can be expressed as a series of its multipole compoenents as follows: % https://accelconf.web.cern.ch/ipac2019/talks/weypls1_talk.pdf (slide 13) 
\begin{equation}\label{eq:wakefunctions}
    W_u(\Delta u_1, \Delta u_2, z) = W^\mathrm{const}_u(z) + W^\mathrm{dip}_u(z)\Delta u_1 +  W^\mathrm{quad}_u(z)\Delta u_2 + o(\Delta u_1, \Delta u_2),
\end{equation}
where $u=(x,y)$ and $W^\mathrm{const}_u(z)$, $W^\mathrm{dip}_u(z)$, $W^\mathrm{quad}_u(z)$ are the transverse constant, dipolar, and quadrupolar wakefunctions respectively. Last, $ o(\Delta u_1, \Delta u_2)$ is the higher order term however only the first-order terms will be considered for the rest of the analysis. % M. Schenk p.25

The dipolar and quadrupolar wakefunctions were named after the way they act on the witness particle. The dipolar wakefunction acts like a dipole magnet: its impact is the same regardless of the transverse position of the witness particle; it depends only on the position of the source particle. The quadrupolar wakefunction acts like a quadrupole magnet: its impact increases linearly with the transverse position of the witness particle (independent of the .position of the source particle). % Benoit thesis p.44

The constant term can change the closed orbit while the dipolar and quadrupolar terms can modify the tunes~\cite{benoit_ipac19_impedance}. The dipolar term is often referred to as a driving term for coherent instabilities. The quadrupolar term is often referred to as the detuning term as it modifies the betatron frequencies of individual particles. % https://accelconf.web.cern.ch/ipac2019/talks/weypls1_talk.pdf (slide 14) 
% driving term --> coherent effects.
% detuning term --> incoherent effects.

The effect of the wakefields on the witness particles can be modeled as the following kicks on the transverse normalised momentum~\cite{Schenk:2665819}: % Eq.(2.44) M. Schenk thesis
\begin{equation}\label{eq:wakefield_kicks}
    \Delta u_2^\prime  = \int_0^L F_u(s, z, \Delta u_1, \Delta u_2)ds = -q_1 q_2[W^\mathrm{const}_u(z) + W^\mathrm{dip}_u(z)\Delta u_1 +  W^\mathrm{quad}_u(z)\Delta u_2],
\end{equation}
where $F_u$ is the transverse force of the wakefield over the length $L$. If the structure leading to the wakefield is axially symmetric, then the constant term of the wakefunction is zero. 

\textbf{Beam coupling impedance}\\
The beam coupling impedance is the frequency spectrum of the wakefields in a given component or section of the accelerator. The impedance can be obtained from the wakefunction through a Fourier transform and vice versa as shown below~\cite{Chao:collective}: % Eq.(2.71) and Eq.(2.72) p.69-70.
\begin{equation}\label{eq:wakes_to_impedance}
    W_u(z) = - \frac{i}{ 2\pi} \int_{- \infty}^{+\infty} Z_u(\omega) e^{i \omega z/c} d\omega,
\end{equation}
\begin{equation}\label{eq:impedance_to_wakes}
    Z_u(\omega) = \frac{i}{c} \int_{- \infty}^{+\infty} W_u(z) e^{-i \omega z/c} dz,
\end{equation}
where $u=(x,y)$, $i$ is the imaginary unit and $c$ is the speed of light.

In order to study the beam dynamics effects due to wakefields, impedance models of the particle accelerators have been developed. Details on how an impedance model is built can be found in Ref.~\cite{benoit_ipac19_impedance}. The impedance model for the SPS is discussed in Chapter~\ref{Ch:suppression_impedance} while its implementation in the simulations is discussed in Section~\ref{subsec:pyheadtail}.

\textbf{Head-tail modes}\\
The Vlasov equation~\cite{Vlasov:426186} is often used to describe the beam motion in the presence of wakefields as it allows its mode representation (in frequency domain): the beam motion is described by a superposition of modes. %--> source: https://cds.cern.ch/record/859658/files/ab-2005-041.pdf
Solving the Vlasov equations for the coupled system between the particle motion (synchrotron and betatron) and wakefield kicks the eigenmodes and eigenfrequencies are obtained. These modes are often referred to as headtail modes as they are related to the betatron phase shift between the head and tail of the bunch in a synchrotron. The headtail modes can either be stable, damped, or excited; in the latter case, they evolve into instabilities. For the beam to become unstable the wakefield kicks (source of energy) need to be synchronized with the bunch motion (e.g. with chromaticity)~\cite{instabilities_rumulo_li}.

% All the equations are taken from: https://github.com/natriant/exploring_SPS
\textbf{Sacherer formulae and complex frequency shift}\\
One of the impedance induced effects, that is relevant for the studies of the thesis, is the complex tune shift. The complex tune shift can be computed analytically based on the Vlasov formalism~\cite{Vlasov:426186} through the use of perturbation theory. Formulae expressing the results were derived by Sacherer~\cite{Sacherer:322545, Sacherer:322645}. %  Source of this sentence: second column: https://cds.cern.ch/record/2674107/files/757-3481-2-PB.pdf

The headtail modes introduce an exponential dependence on the amplitude of the bunch centroid as follows~\cite{Schenk:2665819}:
\begin{equation}\label{eq:HeadtailModesExponentialDependence}
    u(t) \propto e^{i(\Omega_{u,0}^{(l)}+\Delta \Omega_u^{(l)})t} =  e^{i(\Omega_{u,0}^{(l)}+\Delta \Omega_{u, \mathrm{{re}}}^{(l)})t} e^{-\Delta \Omega_{u, \mathrm{{im}}}^{(l)} t},
\end{equation}
where $\Omega_{u,0}^{(l)}$ is the real-valued, unperturbed frequency of mode $l$, and $\Delta \Omega_u ^{(l)} = \Delta \Omega_{u, \mathrm{re}}^{(l)} + i \Delta \Omega_{u, \mathrm{im}}^{(l)}$ is the complex coherent frequency shift introduced by the beam impedance. From Eq.~\eqref{eq:HeadtailModesExponentialDependence} it can be seen that the the real part $\Delta \Omega_{u, \mathrm{re}}^{(l)}$ modifies the oscillation frequency. The second term $e^{-\Delta \Omega_{u, \mathrm{im}}^{(l)}t}$ illustrates that the amplitude of the motion grows for $\Delta \Omega_{u, \mathrm{im}}^{(l)}<0$ (unstable bunch) and is damped for $\Delta \Omega_{u, \mathrm{im}}^{(l)}>0$ (stable bunch). The imaginary coherent frequency shift gives the instability growth rate as follows:
\begin{equation}\label{eq:growth_rate}
    1/\tau^{(l)}=-\Delta \Omega_{u, \mathrm{im}}^{(l)} / T_\mathrm{rev},
\end{equation}
where $T_\mathrm{rev}$ is the revolution period.
In the above equations $u=(x,y)$ denoting the horizontal and vertical frequencies respectively.

The complex frequency shift for the mode $l$ for a bunched beam is (Chapter 6 in Ref.~\cite{Chao:collective}): % Eq.(6.207)
\begin{equation}\label{eq:complext_tune_shift_modes_m}
    \Omega^{(l)}-\omega_u-l\omega_s = -\frac{1}{4\pi}\frac{\Gamma(l+1/2)}{2^l l!}\frac{N r_0 c^2}{\gamma_0 T_0 \omega_u \sigma_z} i Z_\mathrm{eff},
\end{equation}
where $\omega_u$ and $\omega_s$ are the unperturbated betatron and synchrotron frequencies, $\Gamma(x)$ is the gamma function, $N$ is the number of particles in the bunch, $r_0=1.535 \cdot 10^{-16}$ is the clasical radius of the proton, $c$ is the speed of light, $\gamma_0$ is the relativistic gamma, $T_0$ is the revolution period, $\sigma_z$ is the rms bunch length, $i$ is the imaginary unit and $Z_\mathrm{eff}$ is the effective impedance.

The effective impedance $Z_\mathrm{eff}$ is computed as follows:
\begin{equation}\label{eq:effective_impedance}
    Z_{\perp \mathrm{eff}}^{(l)} = \frac{\sum_{p=-\infty}^{+ \infty}Z_{\perp }^{(l)}(\omega_{p}) h_l(\omega_{p}-\omega_\xi)}{\sum_{p=-\infty}^{+ \infty}h_l(\omega_{p}-\omega_\xi)},
\end{equation}
where $\omega_p = (p+Q_u)\omega_0$ is the discrete spectrum of the transverse bunch oscillations with $-\infty < p < + \infty$ for a single bunch (which is the case in the following studies) or several bunches oscillating independently and $\omega_\xi=(\xi \omega_u)/\eta_p) = Q^\prime \omega_0 / \eta_p$ is the chromatic angular frequency with $\eta_p$ being the phase slip factor.

Last, $h_l$, is the power spectral density (definition in Appendix~\ref{ch:app_B}) of a Gaussian bunch of $l$ azimuthial mode. $h_l$ is described by: % Chao ch6 Eq.(6.143)
\begin{equation}\label{eq:spectral_density_of_gaussian_bunch}
    h_l(\omega) = (\omega \sigma_z/c)^{2l} e^{-(\omega \sigma_z/c)^2},
\end{equation}
where $\sigma_z$ is the rms bunch length and $c$ the speed of light.

It should be highlighted that all the parameters inserted in Eq.~\eqref{eq:complext_tune_shift_modes_m},  Eq.~\eqref{eq:effective_impedance} and Eq.~\eqref{eq:spectral_density_of_gaussian_bunch} should be converted in CGS(centimetre–gram–second) units. For the conversion from SI to CGS system the following relations are useful:
\begin{equation}\label{eq:conversion_si_to_cgs}
    1 \mathrm{[\Omega]} = \frac{1}{9} \cdot 10^{-11} \mathrm{[s]/[cm]},
\end{equation}
where $\Omega$ (Ohm) is the SI unit of resistance.

%- 1 [Ohm] = (1/9)*10**(-11) [s]/[cm]
%- 1 [Ohm]/[m] = (1/9)*10**(-13) [s]/[cm]**2 

%In principle, the mode representation and the particle representation of the beam motion are identical. To describe fully 1012particles, one needs 1012modes, and vice versa. The detailed methods of analysis in the two approachesare different-the particle representation usually is conveniently treated in the time domain, while in the mode representation the frequency domain is more convenient-but, in principle, they necessarilygive the same final results. (p.272, chao)


%The total betatron phase shift between head and tail is the physical origin of the head tail instability. The head and the tail of the bunch oscillate therefore with a phase difference, which reduces to rigid-bunch oscillations only in the limit of zero chromaticity. A new (within-bunch) mode number m = ... , −1, 0, 1, ... , also called head–tail (or azimuthal) mode number, was introduced. This mode describes the number of betatron wavelengths (with sign) per synchrotron period. When computing the eigen modes of the coupled system one can see that there is also a freuqency shift. The Vlasov formalism used for the mode representation of the beam motion. 

%When studying the imepdance-induced instabilities it is convinient to move in the mode domain: the beam motion is described by a superposition of modes. % the beam motion is described by a superposition of modes
%allows the study of the beam motion in 


%\subsection{Active feedback?}
\section{Optics models for accelerators}\label{sec:optics_model_designs}
% or optics design
For the study of beam dynamics, it is essential to know the detailed arrangement of the magnets (position and strength) in the lattice, which will be referred to as optics. The optics also provide information on the twiss parameters and phase advances along the ring.

MAD-X~\cite{madx} is a code which is used extensively for the design and simulation of the accelerators at CERN. The official optics repositories of the CERN machines can be found in Ref.~\cite{cern_optics_repo}.

\textbf{SPS optics}\\
The studies presented in this thesis are performed for the nominal SPS optics for the LHC filling which are called Q26 optics as the integer part of the tune in both planes is 26. The model for the Q26 optics can be found in the official CERN repository~\cite{SPS_optics_repo} and will be referred to as the nominal SPS model in this thesis. The values of the optics parameters in what follows correspond to the model values unless stated otherwise.
\normalsize{\textbf{} % fixed with duck tape



\section{Tracking simulation codes}\label{sec:simualtion_codes}
In this section the two tracking simulation codes used in this thesis to study the noise-induced emittance growth are presented. Both codes are macroparticle tracking libraries that simulate the particle motion in the six-dimensional (6D) phase space $(x, x^\prime, y, y^\prime, z, \delta)$. The first code (PyHEADTAIL) performs tracking between interaction points around a circular accelerator at which the paticles receive kicks from magnetic elements or from collective effects. The second code (Sixtracklib), uses the detailed optics model of the machine for the tracking. In both cases, the tracking is performed with the use of transfer matrices.
% sympectic intgration. drift-kick approach

\subsection{PyHEADTAIL}\label{subsec:pyheadtail}

PyHEADTAIL~\cite{pyheadtail_repository} is an open-source 6D macroparticle tracking code, developed at CERN, which was originally designed to study collective effects in circular machines and to be easily extensible with custom elements. 
% https://indico.cern.ch/event/930271/contributions/3910265/attachments/2066139/3467770/30June2020_Simulations_emit_growthCC_RFnoise.pdf  
Details on its implementation and its features can be find in Refs.~\cite{pyheadtail_manual_adrian, pyheadtail_schenk}. Below the main steps of a simulation are listed: % according to the interest of this thesis.

% Thesis david: https://cds.cern.ch/record/2707064/files/CERN-THESIS-2019-272.pdf (p.45) and for the transverse matrix.

\begin{enumerate}
    \item \textbf{Machine initialisation:} The accelerator ring is split into a number of segments of equal length, after each of which there is an interaction point\footnote{It should be clarified that this is not an "interaction point" in the sense of collision between counter-circulating beams.} (IP). At the interaction points the macroparticles experience kicks from various accelerator components (feedback system, multipoles etc) or from collective effects such as the wakefields. The machine parameters, such as the circumference, the betatron and synchrotron tunes and the optics at the interaction points are defined. It should be noted that PyHEADTAIL uses smooth \mbox{approximation} which means that only a few segments are defined per turn. % which results to a very  rough optics model.
    \item \textbf{Bunch initialisation:}  A particle bunch is represented by a collection of macroparticles, each of which represents a clustered collection of physical particles. Each macroparticle is described by four transverse $(x, x^\prime, y, y^\prime)$ and two longitudinal coordinates $(z, \delta)$, a mass and an electric charge. For the studies presented in this thesis $10^{5}$ macroparticles are sufficient for an accurate representation of the bunch, unless it is stated otherwise. There are various distributions available. In this thesis the simulations are performed using a Gaussian distribution in transverse and longitudinal planes.
    \item \textbf{Transverse tracking}: In the transverse plane, the macroparticles are transported from one interaction point to another 
    (e.g. grom IP0 to IP1) following the matrix formalism of Eq.~\eqref{eq:matrix_formalism_intro}. The linear transfer matrices, $M$, introduced in Eq.~\eqref{eq:linear_transfer_matrix} take into account the optics parameters at the beginning and the end of the corresponding segment.
    %In the transverse plane, the macroparticles are transported from one interaction point to another 
    %(e.g. grom IP0 to IP1) by linear transfer matrices which take into account the optics parameters at the beginning and the end of the corresponding segment following Eq.~\eqref{eq:matrix_formalism_intro}.
    % For example, in the horizontal plane the transport of each macroparticle from IP1 to IP2 along the ring is given by Eq.~\eqref{eq:matrix_formalism_intro}:
    %\begin{equation}
    %    \binom{x_i}{x_i^\prime}_{\mathrm{IP0}} = M \binom{x_i}{x_i^\prime}_%{\mathrm{IP1}},
    %\end{equation}
    %where $i=1, ..., N$ with N being the number of macroparticles and the matrix $M$ is given by:
    %\begin{equation}\label{eq:linear_transfer_map}
    %    M = \begin{pmatrix}
    %        \sqrt{\beta_{x, \mathrm{IP0}}} & 0 \\ 
    %        -\frac{\alpha_{x, \mathrm{IP0}}}{\sqrt{\beta_{x,  \mathrm%{IP0} }}} & \frac{1}{\sqrt{\beta_{x, \mathrm{IP0}}}}
   %         \end{pmatrix} \begin{pmatrix}
   %             \cos(\Delta \mu_{x, \mathrm{IP0 \to IP1 }}) & \sin(\Delta \mu_%{x, \mathrm{IP0 \to IP1 }})\\
   %              - \sin(\Delta \mu_{x, \mathrm{IP0 \to IP1 }})  &\cos(\Delta \mu_{x, \mathrm{IP0 \to IP1 } }) 
   %             \end{pmatrix}  \begin{pmatrix}
   %                 \sqrt{\beta_{x, IP1}} & 0 \\ 
   %                 -\frac{\alpha_{x, IP1}}{\sqrt{\beta_{x, IP1}}} & \frac{1}%{\sqrt{\beta_{x, IP1}}}
   %                 \end{pmatrix} 
   % \end{equation}
    %where $\alpha_{x, \mathrm{IP0/IP1}}$ and $\beta{x, \mathrm{IP0/IP1}}$ are the optics paramters at the interaction points IP0 and IP1 respectively and $\Delta \mu_{x, \mathrm{IP0 \to IP1 }}$ the phase advance from IP0 to IP1. As smooth approximation is used, the phase advance equals: 
    The phase advance, for each segment equals:
    \begin{equation}\label{eq:phase_advance_smooth_approximation}
        \Delta \psi_{u, \mathrm{IP0 \to IP1 }} = Q_u \frac{L}{C},
    \end{equation}
    where $Q_u$ is the transverse betatron tune, $C$ the machine circumference and $L$ the length of the corresponding segment. It should be noted that if no detuning source is added (see next step) the matrix $M$ is the same for all particles.
    \item \textbf{Chromaticity and detuning with transverse amplitude:} The chromaticity (up to higher orders) and amplitude detuning are implemented as a change of the phase advance of each individual particle as follows (example for the horizontal plane):
    \begin{equation}\label{eq:change_phase_advance_detunign}
        \Delta \psi_{x, i \mathrm{IP0 \to IP1 }} = \Delta \psi_{x, \mathrm{IP0 \to IP1 }} + (\xi^{1}_x \delta_i + \alpha_{xx}J_{x, i} + \alpha_{xy}J_{y,i}) \frac{\Delta \psi_{x, \mathrm{IP0 \to IP1 }}}{2\pi Q_x}, 
    \end{equation}
    where $i=1, ..., N$ with $N$ being the number of macroparticles, $\Delta \psi_{x, \mathrm{IP0 \to IP1 }}$ is the phase advance for all macroparticles defined in the previous step, $\xi_x^{1}$ the horizontal chromaticity of first order (see Eq.~\eqref{eq:chroma_up_to_order_m} for $n=1$) normalised to the betatron tune, $\alpha_{xx}$ and $\alpha_{xy}$ are the detuning coefficients, while $J_x$ and $J_y$ are the horizontal and vertical actions of the macroparticle. Therefore, in the presence of detuning the elements of the $M$ matrix (introduced in Eq.~\eqref{eq:linear_transfer_matrix}) are different for every particle.

    \item \textbf{Longitudinal tracking:}  %In PyHEADTAIL the longitudinal particle dynamics are described with the longitudinal equations of motion introduced in Eq.~\eqref{eq:long_eq_motion_z} and ~\eqref{eq:long_eq_motion_delta}
    The longitudinal coordinates are advanced once per turn after solving numerically the equations of motion introduced in Eq.~\eqref{eq:long_eq_motion_z} and Eq.~\eqref{eq:long_eq_motion_delta}. The motion can be linear or not (non-linear RF system). The studies presented in this thesis use the linear longitudinal tracking.
    %  The solutions of the above equation is obtained numerically~\cite{pyheadtail_schenk}. 
    % "numerically", slide 37: https://www2.kek.jp/accl/legacy/seminar/file/PyHEADTAIL_PyECLOUD_2.pdf . Symplectic integration E. Forst and W. herr

    \item \textbf{Trasnverse impedance effects:} In PyHEADTAIL, wakefield kicks are used to implement the effect of the transverse impedance in the time domain. To improve the computational efficiency, the total impedance of the full machine is lumped at one of the interaction points along the ring and the kicks are applied to the macroparticles once per turn. Additionally, instead of computing the wakefield kicks from each particle to the rest individually, the bunch is divided into a number of slices longitudinally and the macroparticles in each slice recieve a wakefield kick generated by the preceding slices\footnote{This is valid in the ultrarelativistic scenario when no wakefield is generated in front of the bunch. The condition applies for the SPS experiments described in this thesis, performed at 270\,GeV.}~\cite{Salvant:1274254}. This is illustrated schematically in Fig.~\ref{fig:longitudinal_slicing_wakefields}. A large number of slices is required such as the wakes can be assumed constant within the slice. A high number of macroparticles is also needed in order to avoid statistical noise effects caused by undersampling~\cite{pyheadtail_manual_adrian}. For the studies presented, 500 slices are used over a range of three rms bunch lengths in both directions from the bunch center with the bunch being represented by $10^6$ macroparticles (instead of $10^5$ required for simulations without impedance effects).
    

    \begin{figure}[!ht]
        \centering
        \begin{subfigure}[t]{0.45\textwidth}
            \centering
            \includegraphics[width=1\textwidth]{images/Ch2/before_slicing.png}
            \caption{Without slicing}
            %\label{fig:add_label_here}
        \end{subfigure}
        \hfill
        \begin{subfigure}[t]{0.45\textwidth}
            \centering
            \includegraphics[width=1\textwidth]{images/Ch2/after_slicing.png}
            \caption{With slicing}
            %\label{fig:add_label_here}
        \end{subfigure}
        \hfill
         \caption{Longitudinal bunch slicing for the implementation of wakefield kicks in PyHEADTAIL. Without the slicing technique (left) the wake kicks on the red macroparticle are generated from all the green macroparticles resulting in computationally expensive simulations. Instead, when the bunch is sliced longitudinally (right) the wake kicks on the macroparticles in the red slice $i$ are generated by the macroparticles in the green slices $j$, decreasing significantly the computation time. The figures are courtesy of M. Schenk~\cite{pyheadtail_schenk}} % bunch passage
         \label{fig:longitudinal_slicing_wakefields}
     \end{figure}
        
    The wakefield kicks are computed using a convolution of the wake function with the moments of each particle. %p.39 michael schenk thesis
    The wake functions are available from detailed imepdance model of the machine which are obtained from a combination of theoretical computations and electromagnetic simulations and can be imported in PyHEADTAIL in form of tables. More details on the SPS impedance model are provided in Section~\ref{sec:sps_impedance_model}.
    
    \item \textbf{Data acquisition:} The updated macroparticle coordinates after each turn are available at IP0 for post-processing. Typically, $10^{5}$ turns are required for the noise-induced emittance growth simulations presented in this thesis. 
    
\end{enumerate}

%Last sentence from (LinearMap): https://github.com/PyCOMPLETE/PyHEADTAIL/blob/master/PyHEADTAIL/trackers/longitudinal_tracking.py
%or longutidanl equations of motion: slide 37 https://www2.kek.jp/accl/legacy/seminar/file/PyHEADTAIL_PyECLOUD_2.pdf

Figure~\ref{fig:pyheadtail_accelerator_model} shows a graphic representation of the accelerator model and the tracking procedure supporting the steps described above.
% Graph is created app.diagrams.net and is saved in goodle dirve.

\begin{figure}[!h]
    \centering         
    \includegraphics[width=0.6\textwidth]{images/Ch2/accelerator_model_graph_pyheadtail.png}
        \caption{Graphical representation of the accelerator model and tracking procedure in PyHEADTAIL (inspired by the graphs in Refs.~\cite{pyheadtail_schenk, inproceedings_ibs_pyheadtail}). In this example the ring is split in four segments seperated by the interaction points (IPs). Wakefield and multipole kicks are applied to the macroparticles in IP2 and IP4. The macroparticles are transported between the IPs by a linear transfer map (which can include detuning effects) in the transverse plane.  The longitudinal coordinates are updated once per turn.}
        \label{fig:pyheadtail_accelerator_model}
 \end{figure}


\subsection{Sixtracklib}\label{subsec:sixtracklib}
%Introduction to sixtrackib: https://indico.cern.ch/event/833895/contributions/3577803/attachments/1927226/3190636/intro_sixtracklib.pdf
% https://inspirehep.net/files/6273430c727ace3796a92d069f651ade

Sixtracklib~\cite{sixtracklib_repo} is a library for performing single charged particle simulations developed at CERN. It simulates the motion of the particles in the six-dimensional (6D) phase space. The individual trajectories are computed taking into account the interactions with all the magnetic elements in the machine using the detailed design optics model described in Section~\ref{sec:optics_model_designs}. The particles advance from one element to the other with transfer maps. Simulations with Sixtracklib are time efficient as the library can run on Graphical Processing Units (GPUs). Further details on Sixtracklib implementation and usage can be found in Ref.~\cite{sixtracklib_introduction} and Ref.~\cite{Schwinzerl:2781835}. 

Since Sixtracklib code was not used as extensively as PyHEADTAIL for the studies presented here, and was not modified or extended in any way where it was used, we do not give further description here, but refer the reader to the references mentioned above.

%Sixtracklib is described here in much less detail than PyHEADTAIL as its used less extensively for the studies in this thesis and additionally because the author didn't build and customize the machine and the beam dynamics effects like in PyHEADTAIL.

