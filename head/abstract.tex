% English abstract
\cleardoublepage
\chapter*{Abstract}
\markboth{Abstract}{Abstract}
\addcontentsline{toc}{chapter}{Abstract} % adds an entry to the table of contents
% put your text here

High-Luminosity LHC (HL-LHC) is the upgrade of the Large Hadron Collider (LHC) which aims to increase significantly the luminosity at the experiments in order to extend its potential for discoveries. HL-LHC will employ Crab Cavities in order to compensate the luminosity reduction coming from the crossing angle at the interaction points of ATLAS and CMS. 

Two prototype Crab Cavities have been installed in the Super Proton Synchrotron (SPS), prior to their installation to the LHC, in order to be tested for the first time with proton beams. An important point to consider is the undesired transverse emittance growth due to noise in their RF system. The experimental tests in the SPS showed that the effects of the beam tranvserse impedance play an important role on the emittance growth driven by Crab Cavity RF noise. This thesis, investigates this interplay with detailed tracking simulations and with experimental measurements. The implications for the HL-LHC are also dicussed.




%\textbf{Probably put to abstract}\\
%In 2018, two prototype Crab Cavities (CCs) were installed in the SPS to be tested for the first time with proton beams. A series of dedicated machine developemnt studies was carried out in order to validate their working principle and answer various beam dynamic questions. One of the operational issues that needed to be addressed concerned the expected emittance growth due to noise in their RF system, which is the main subject this thesis.  As mentioned in chapter~\ref{Ch:CC_noise_theory} a theoretical model had already been developed and validated by trackig simulations~\cite{PhysRevSTAB.18.101001}. 
%As a part of the first experiemental campaign with $\CC$s in SPS a dedicated experiment was conducted to benchmark these models with experimental data and confirm the analytical predictions. The objective of this chapter is to provide an overview of the machine setup for the $\CC$ experiements and introduce the instruments and methods used for measuring the beam parameters of interest for the emittance growth studies.



