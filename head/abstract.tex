% English abstract
\cleardoublepage
\chapter*{Abstract}
\markboth{Abstract}{Abstract}
\addcontentsline{toc}{chapter}{Abstract} % adds an entry to the table of contents
% put your text here

High-Luminosity LHC (HL-LHC) is the upgrade of the Large Hadron Collider (LHC) which aims to increase significantly the luminosity at the experiments in order to extend its potential for discoveries. HL-LHC will employ Crab Cavities in order to compensate the luminosity reduction coming from the crossing angle at the interaction points of ATLAS and CMS. The Crab Cavities are expected to result to some undesired emittance growth due to noise in their RF system which would result to luminosity loss. 

Two prototype Crab Cavities have been installed in the Super Proton Synchrotron (SPS), prior to their installation to the LHC, in order to be tested for the first time with proton beams.  The results from the first experiemental campaign that took place in 2018 showed that the me 


This work

experimental first beam dynamics with CCs

investigates the limitations of the model

revealed a new mechanism of emittance growth suppression from the beam transverse ipedance


investigate the interplay with impedance and amplitude dependent tune spread (experimetnally and in simulations)


its limitations, 

interplay with imepdance and amplitude dependent tune spread

%\lipsum[1-2]
test

Move here the paragraph why the studies are perfromed in the sps. conclusion or the introduction.

%\textbf{Probably put to abstract}\\
%In 2018, two prototype Crab Cavities (CCs) were installed in the SPS to be tested for the first time with proton beams. A series of dedicated machine developemnt studies was carried out in order to validate their working principle and answer various beam dynamic questions. One of the operational issues that needed to be addressed concerned the expected emittance growth due to noise in their RF system, which is the main subject this thesis.  As mentioned in chapter~\ref{Ch:CC_noise_theory} a theoretical model had already been developed and validated by trackig simulations~\cite{PhysRevSTAB.18.101001}. 
%As a part of the first experiemental campaign with $\CC$s in SPS a dedicated experiment was conducted to benchmark these models with experimental data and confirm the analytical predictions. The objective of this chapter is to provide an overview of the machine setup for the $\CC$ experiements and introduce the instruments and methods used for measuring the beam parameters of interest for the emittance growth studies.


The implementations for the HL-LHC will be dicussed.