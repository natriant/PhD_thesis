% English abstract
\cleardoublepage
\chapter*{Abstract}
\markboth{Abstract}{Abstract}
\thispagestyle{simple} % formatting the first page of each chapter
\addcontentsline{toc}{chapter}{Abstract} % adds an entry to the table of contents
% put your text here

\textit{"Studies of the emittance growth due to noise in the Crab Cavity RF systems"} by Natalia Triantafyllou.

The High-Luminosity LHC (HL-LHC) is the upgrade of the Large Hadron Collider (LHC) machine which aims to increase significantly the luminosity at the experiments in order to extend the potential for physics discoveries. Crab Cavities are a key component of the HL-LHC upgrade, as they will be deployed to mitigate the luminosity reduction induced by the crossing angle at the main LHC experiments (ATLAS and CMS). An important point to consider is the undesired transverse emittance growth due to noise in the Crab Cavities RF system, which can result in considerable loss of luminosity.

This thesis explored the mechanisms for emittance growth from Crab Cavity RF noise through numerical simulations and experimental measurements. The studies focused on the Super Proton Synchrotron (SPS) machine, in which two prototype Crab Cavities were installed in 2018, prior to their installation in the LHC, to be tested for the first time with proton beams. It was found that the beam transverse impedance plays an important role in the emittance growth driven by Crab Cavity RF noise. The measured emittance growth rates are much smaller than predicted from simulations and available theoretical models without including impedance effects. When impedance effects are included in the simulations, the results are in much better agreement with the experiment. The simulations including the machine impedance also demonstrate the dependence of the emittance growth suppression on the tune shift with amplitude, which is another feature consistent with experimental observations. The significance of these results and the implications for the HL-LHC are also discussed.



%Two prototype Crab Cavities have been installed in the Super Proton Synchrotron (SPS), prior to their installation in the LHC, in order to be tested for the first time with proton beams. An important point to consider is the undesired transverse emittance growth due to noise in their RF system. The experimental tests in the SPS showed that the effects of the beam transverse impedance play an important role in the emittance growth driven by Crab Cavity RF noise. This thesis investigates this interplay with detailed tracking simulations and experimental measurements. The implications for the HL-LHC are also discussed.


 


%\textbf{Probably put to abstract}\\
%In 2018, two prototype Crab Cavities (CCs) were installed in the SPS to be tested for the first time with proton beams. A series of dedicated machine developemnt studies was carried out in order to validate their working principle and answer various beam dynamic questions. One of the operational issues that needed to be addressed concerned the expected emittance growth due to noise in their RF system, which is the main subject this thesis.  As mentioned in chapter~\ref{Ch:CC_noise_theory} a theoretical model had already been developed and validated by trackig simulations~\cite{PhysRevSTAB.18.101001}. 
%As a part of the first experiemental campaign with $\CC$s in SPS a dedicated experiment was conducted to benchmark these models with experimental data and confirm the analytical predictions. The objective of this chapter is to provide an overview of the machine setup for the $\CC$ experiements and introduce the instruments and methods used for measuring the beam parameters of interest for the emittance growth studies.



