\nomenclature[01]{\(  \mathcal{L}  \)}{Instantaneous luminosity of a collider.}
\nomenclature[02]{\(f_\mathrm{rev}\), $\omega_\mathrm{rev}$}{Revolution frequency of the machine in [Hz] and in [rad/s] respectively.} 
\nomenclature[03]{\(\sigma_x\),  \(\sigma_y\)}{Horizontal and vertical rms beam size in [m].} 
\nomenclature[04]{$\sigma_z$, $\sigma_t$, $\sigma_\phi$}{rms bunch length in units of [m], [s] and [rad] respectively.} 
\nomenclature[05]{$C_0$}{Circumference of the accelerator.} 
\nomenclature[06]{$\mathbf{F}_L$}{Lorentz force vector.}
\nomenclature[07]{$\mathbf{E}$}{Electric field vector.}
\nomenclature[08]{$\mathbf{B}$}{Magnetic field vector.}
\nomenclature[09]{$\mathbf{v}$}{Velocity vector.}
\nomenclature[10]{$C$}{Circumference of an accelerator ring.}
\nomenclature[11]{$R$}{Radius of an accelerator ring.}
\nomenclature[12]{$E_0, p_0, v_0$}{Reference energy, momentum and velocity.}
\nomenclature[13]{$\beta_0, \gamma_0$}{Relativistic beta and gamma (Lorentz factor).}
\nomenclature[14]{$e$, $m_p$}{The proton charge and rest mass respectively.}
\nomenclature[15]{$\rho$}{Bending radius.}
\nomenclature[16]{$c$}{Speed of light in vaccum.}
\nomenclature[17]{$s$}{Location along the ring.}
\nomenclature[18]{$(x, x^\prime)$}{Horizontal co-ordinates: position and normalised  momentum to the momentum of the reference particle in units of [m] and [rad] respectively.} % or conjugate momentum, M. Schenl
\nomenclature[19]{$(y, y^\prime)$}{Vertical co-ordinates: position and normalised momentum to the momentum of the reference particle in units of [m] and [rad] respectively.} % or conjugate momentum
\nomenclature[20]{$(z,\delta)$}{Longitudinal co-ordinates: position in units of [m] and momentum offset (no units).} % or conjugate momentum
\nomenclature[21]{$(p_x, p_y, p_z)$}{Particle's momentum in the horizontal, vertical and longitudinal plane respectively.} 
\nomenclature[22]{$t$}{Time in [s].} 
\nomenclature[23]{$b_n$, $a_n$}{Normal and skew multipole coefficients.} 
\nomenclature[24]{$k_n$}{Normalised normal multipole coefficient.} 
\nomenclature[25]{$\psi_u(s)$}{Phase advance from the start of the ring, $s_0$, where $u=(x,y)$.} 
\nomenclature[26]{$\Delta \psi_u$}{Phase advance between two locations along the ring, with $u=(x,y)$.} 
\nomenclature[27]{$\alpha_u(s)$, $\beta_u(s)$, $\gamma_u(s)$}{Alpha, beta and gamma functions respectively or Twiss or Courant-Snyder parameters, with $u=(x,y,z)$.} 
\nomenclature[28]{$Q_x$, $Q_y$}{Horizontal and vertical betatron tunes.}
\nomenclature[29]{$Q_{x0}$, $Q_{y0}$}{Horizontal and vertical working points of an accelerator.} 
\nomenclature[30]{$J_x$, $J_y$}{Horizontal and vertical action.} 
\nomenclature[31]{$u_N$, $u^\prime_N$}{Normalised transverse co-ordinates, with $u=(x,y)$.} 
\nomenclature[32]{$\epsilon^{\mathrm{geom}}_x$, $\epsilon^{\mathrm{geom}}_y$}{Horizontal and vertical geometric emittance of the beam.} 
\nomenclature[33]{$\epsilon_x$, $\epsilon_y$}{Horizontal and vertical normalised emittance of the beam.} 
\nomenclature[34]{$D_x$, $D_y$}{Horizontal and vertical dispersion functions.} 
\nomenclature[35]{$Q_x^{(n)}$, $Q_y^{(n)}$}{Horizontal and vertical chromaticity of nth order.} 
\nomenclature[36]{$Q_x^{\prime}$, $Q_x^{\prime \prime}$}{Horizontal chromaticity first and second order respectively.} 
\nomenclature[37]{$Q_y^{\prime}$, $Q_y^{\prime \prime}$}{Vertical chromaticity first and second order respectively.} 
\nomenclature[38]{$\alpha_{xx}$, $\alpha_{yy}$, $\alpha_{xy}$}{Horizontal, vertical and cross-term detuning coefficients respectively in units of [1/m].} 
\nomenclature[39]{$T_\mathrm{rev}$}{Revolution period of an accelerator.}
\nomenclature[40]{$\phi_\mathrm{RF}$}{The phase of the main RF system of an accelerator.}
\nomenclature[41]{$\omega_\mathrm{RF}$}{The angular frequency of the main RF system of an accelerator.}
\nomenclature[42]{$h$}{Harmonic number.}
\nomenclature[43]{$\phi_s$}{The phase of the synchronous particle.}
\nomenclature[44]{$V_\mathrm{RF}$, $f_\mathrm{RF}$}{Voltage and frequency of the main RF system of an accelerator.}
\nomenclature[45]{$\alpha_p$}{Momentum compaction factor.}
\nomenclature[46]{$\eta_p$}{Phase slip factor.}
\nomenclature[47]{$\gamma_\mathrm{tr}$}{Transition energy.}
\nomenclature[48]{$Q_s$, $\omega_s$}{Synchrotron tune and angular synchrotron frequency respectively.}
\nomenclature[49]{$W^\mathrm{const}_x(z)$, $W^\mathrm{const}_y(z)$}{Horizontal and vertical constant wake functions.}
\nomenclature[50]{$W^\mathrm{dip}_x(z)$, $W^\mathrm{dip}_y(z)$}{Horizontal and vertical dipolar wake functions.}
\nomenclature[50]{$W^\mathrm{quad}_x(z)$, $W^\mathrm{quad}_y(z)$}{Horizontal and vertical quadrupolar wake functions.}
\nomenclature[51]{$Z_x(\omega)$, $Z_y(\omega)$}{Horizontal and vertical impedance.}
\nomenclature[52]{$\Delta \Omega_u ^{(l)} $}{Horizontal and vertical complex coherent frequency shift of headtail mode $l$, with $u=(x,y)$.}
\nomenclature[53]{$i$}{Imaginary unit.}
\nomenclature[54]{$V_\mathrm{0, CC}$}{Peak amplitude of the Crab Cavity voltage.}
\nomenclature[55]{$f_\mathrm{CC}$}{Frequency of the Crab Cavity.}
\nomenclature[56]{$\phi_\mathrm{CC}$}{Phase of the Crab Cavity.}
\nomenclature[57]{$\beta_{u, \mathrm{CC}}$, $\alpha_{u, \mathrm{CC}}$, $D_{u, \mathrm{CC}}$}{Transverse beta and alpha and dispersion functions at the location of Crab Cavity, with $u=(x,y)$.}
\nomenclature[58]{$\Delta A$}{Relative devitation from the nominal amplitude of the signal or amplitude noise (no units).}
\nomenclature[59]{$\Delta \phi$}{Devitation from the nominal phase of the signal or phase noise in units of [$\mathrm{rad^2}$].}
\nomenclature[60]{$S_{\Delta A}$, $S_{\Delta \phi}$ }{Power spectral density of amplitude and phase noise signal in units of [1/Hz] and [$\mathrm{rad^2/Hz}$], respectively.}
\nomenclature[61]{$I_n(x)$}{Modified Bessel function of the first kind.}
\nomenclature[62]{$\Gamma$}{Gamma function.}
\nomenclature[63]{$N_\mathrm{mp}$}{Number of macroparticles used in PyHEADTAIL or Sixtracklib simulations.}
\nomenclature[64]{$N_\mathrm{turns}$}{Number of turns used in PyHEADTAIL or Sixtracklib simulations.}
\nomenclature[65]{$q_x, q_y, q_s$}{Decimal part of the horizontal betatron tune, vertical betatron tune and synchrotron tune respectively.}
