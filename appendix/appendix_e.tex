\section{Reference Wire Scanner measurements}\label{sec:sps_transverse_beam_profiles}
\subsection{Example beam profiles}
Figures~\ref{fig:WS_example_profiles_H_2022} and~\ref{fig:WS_example_profiles_V_2022} show two example horizontal and vertical beam profile as obtained from the SPS.BWS.51637.H and SPS.BWS.41677.V instruments respectively during the experiment with CC1 in SPS in 2022. The data points from the IN (OUT) scan are shown with a blue (orange) color. 

The measured data points (light blue) are fitted with a four-parameter gaussian (orange) following the procedure discussed in Section~\ref{subsec:sps_ws} to obtain the beam size. Thereafter, the emittance values and their uncertainties are computed from Eqs.~\eqref{eq:emittance_from_WS} and~\eqref{eq:emittance_from_WS_uncertainty} respectively. The results of the fit are also shown in the plots. It is evident that the calculated uncertainties are two orders of magnitude smaller than the corresponding emittance values themselves. This is the case for all acquisitions. However, not all the profiles are displayed in this thesis for practical reasons.

\begin{figure}[htp]
    \centering
    \begin{subfigure}{.45\textwidth}
        \centering
        \includegraphics[width=.95\linewidth]{images/app_c/SPS.BWS.51637.H_IN_OUT_ 15_13_04.png}  
        \caption{Horizontal beam profile.}
        \label{fig:WS_example_profiles_H_2022}
    \end{subfigure}
    \begin{subfigure}{.45\textwidth}
        \centering
        \includegraphics[width=.95\linewidth]{images/app_c/41678.V_IN_OUT_ 16_28_55.png}  
        \caption{Vertical beam profile.}
        \label{fig:WS_example_profiles_V_2022}
    \end{subfigure}
    \caption{Transverse beam profiles as obtained from SPS.BWS.51637.H during the CC experiment in the SPS in 2022. The data points from the IN (OUT) scan are shown with blue (orange) color.}
    \label{fig:WS_example_profiles_H_V_2022}
 \end{figure}
 

%\begin{figure}[!h]
 %   \centering         
 %   \includegraphics[width=0.6\textwidth]{images/app_c/SPS.BWS.51637.H_IN_OUT_ 15_13_04.png}
 %       \caption{Horizontal beam profile as obtained from SPS.BWS.51637.H during the CC experiment in the SPS in 2022. The data points from the IN (OUT) scan are shown with blue (orange) color. }
%        \label{fig:WS_example_profiles_H_2022}
 %\end{figure}

 %\begin{figure}[!h]
 %   \centering         
 %   \includegraphics[width=0.6\textwidth]{images/app_c/41678.V_IN_OUT_ 16_28_55.png}
 %       \caption{Vertical beam profile as obtained from SPS.BWS.51637.H during the CC experiment in the SPS in 2022. The data points from the IN (OUT) scan are shown with blue (orange) color. }
 %       \label{fig:WS_example_profiles_V_2022}
 %\end{figure}

 \subsection{Emittance values from IN and OUT scan}\label{subsec:ex_emit_growth_in_vs_out}
 Figure~\ref{fig:WS_IN_vs_OUT_2022} shows the transverse emittance evolution acquired with  SPS.BWS.51637.H and SPS.BWS.41677.V for horizontal and vertical planes respectively during the experiment with CC noise on May 16, 2022. The emittance values acquired during the IN scan are shown on the left while the ones acquired during the OUT scan are shown on the right. 

It can be seen that the emittance values obtained from the OUT scan appear to be more fluctuated than the values from the IN scan. In this particular example, this is enhanced in the horizontal plane (blue). It is also clearly visible for the acquisitions during the first half of the coast.

% Path to data:/eos/user/n/natriant/2022/SPS_MDs_2022/cc_md_16May2022/roundA_online_analysis_ws/for_thesis/coast8
% Relevant slides: https://docs.google.com/presentation/d/1QIaQNfqVWaI8cHGGgb5eeS7c_jdUMuxLqD_poHrOtH0/edit#slide=id.g148272a84c8_2_0
 \begin{figure}[htp]
    \centering
    \begin{subfigure}{.45\textwidth}
        \centering
        \includegraphics[width=.95\linewidth]{images/app_c/ws_coast8_set1.png}  
        \caption{Emittance evolution from IN scan.}
        \label{fig:WS_example_profiles_H_2022}
    \end{subfigure}
    \begin{subfigure}{.45\textwidth}
        \centering
        \includegraphics[width=.95\linewidth]{images/app_c/ws_coast8_set2.png}  
        \caption{Emittance evolution from OUT scan.}
        \label{fig:WS_example_profiles_V_2022}
    \end{subfigure}
    \caption{Transverse beam profiles as obtained from SPS.BWS.51637.H during the CC experiment in the SPS in 2022. The data points from the IN (OUT) scan are shown with blue (orange) color.}
    \label{fig:WS_IN_vs_OUT_2022}
 \end{figure}


 \section{Transverse emittance growth measurements}\label{sec:emittance_growth_2022}




 \subsection{Experiment II: sensitivity of emittance growth to amplitude-dependent tune shift}\label{subsec:emittance_growth_2022_exper2}


% location: /eos/user/n/natriant/2022/SPS_MDs_2022/cc_md_16May2022/roundA_online_analysis_ws/for_thesis
% script to plot: /afs/cern.ch/work/n/natriant/public/SPS_MDs_2022/cc_md_16May2022/ws_measurements 
\begin{figure}[htp]
    \centering
    \begin{subfigure}{.45\textwidth}
        \centering
        \includegraphics[width=.95\linewidth]{images/Ch8/emit_vs_time_Set1_coast6.png}  
        \caption{$k_\mathrm{LOD}=+15 \ \mathrm{/m^{4}}$}
        \label{fig:cc_md_2022_coast6}
    \end{subfigure}
    \begin{subfigure}{.45\textwidth}
        \centering
        \includegraphics[width=.95\linewidth]{images/Ch8/emit_vs_time_Set1_coast7.png}  
        \caption{$k_\mathrm{LOD}=+10 \ \mathrm{/m^{4}}$}
        \label{fig:cc_md_2022_coast7}
    \end{subfigure}
    \begin{subfigure}{.45\textwidth}
        \centering
        \includegraphics[width=.95\linewidth]{images/Ch8/emit_vs_time_Set1_coast8.png}  
        \caption{$k_\mathrm{LOD}=+5 \ \mathrm{/m^{4}}$}
        \label{fig:cc_md_2022_coast8}
    \end{subfigure}
    \begin{subfigure}{.45\textwidth}
            \centering
            \includegraphics[width=.95\linewidth]{images/Ch8/emit_vs_time_Set1_coast9.png}  
            \caption{$k_\mathrm{LOD}=-5 \ \mathrm{/m^{4}}$}
            \label{fig:cc_md_2022_coast9}
    \end{subfigure}
    \caption{Horizontal (blue) and vertical (red) emittance evolution of a single bunch during the CC experiment on 16 May, 2022 driven by phase noise of -104.7\,dBc/Hz. The different octupole settings are displayed in the captions of each plot.}
    \label{fig:cc_md_2022_overview_plots_klod_scan}
 \end{figure}
 

 \section{Bunch length measurements}\label{sec:bunch_length_meas_2022}



 The bunch length measurements presented in this section were performed with the Wall Current monitor. The Wall Current monitor acquires the longitudinal bunch profiles and uses a Gaussian fit for the evaluation of the bunch length. Each point shown in the folllowing plots corresponds to the average bunch length value obtained from 100 consecutive acquisitions are  



 %https://accelconf.web.cern.ch/e08/papers/thpc144.pdf (from email communication with IVAN)
 
 AND USES A GAUSSIAN FIT FOR THE EVALUATION O
 
 once per turn



 \subsection{Experiment I: dependece of emittance growth on CC RF noise power}\label{subsec:2022_exp1_bunch_length}
 Figure~\ref{fig:cc_md_2022_overview_plots_noise_scan_bunch_length} illustrates the evolution of the rms bunch length measured in the SPS on May 16, 2022, for the four different levels of phase noise injected in the CC RF system increasing from the top left to bottom right. The Wall Current monitor acquires the longitudinal bunch profiles, from which the bunch length is computed, once per turn. In the figures below, each point corresponds to the average bunch lengths from the profiles over 100 turns while the error bars indicate the standard deviation between them. \textcolor{red}{Is it correct that is acquires profiles once per turn?}
 % Scripting plot: /eos/user/n/natriant/2022/SPS_MDs_2022/cc_md_16May2022/longitudinal_profiles/plot_bunch_length_for_thesis.ipynb
 \begin{figure}[htp]
     \centering
     \begin{subfigure}{.45\textwidth}
         \centering
         \includegraphics[width=.95\linewidth]{images/app_c/bunch_length_COAST_02.png}  
         \caption{-115.2\,dBc/Hz}
         %\label{fig:cc_md_2022_coast2}
     \end{subfigure}
     \begin{subfigure}{.45\textwidth}
         \centering
         \includegraphics[width=.95\linewidth]{images/app_c/bunch_length_COAST_03.png}  
         \caption{-109.5\,dBc/Hz}
         %\label{fig:cc_md_2022_coast3}
     \end{subfigure}
     \begin{subfigure}{.45\textwidth}
         \centering
         \includegraphics[width=.95\linewidth]{images/app_c/bunch_length_COAST_04.png}  
         \caption{-104.7\,dBc/Hz}
         %\label{fig:cc_md_2022_coast4}
     \end{subfigure}
     \begin{subfigure}{.45\textwidth}
             \centering
             \includegraphics[width=.95\linewidth]{images/app_c/bunch_length_COAST_05.png}  
             \caption{-100.1\,dBc/Hz}
             %\label{fig:cc_md_2022_coast5}
     \end{subfigure}
     \caption{Evolution of the bunch length during the CC Experiment I on May 16, 2022. The different phase noise levels injected in the RF system of CC1, are displayed at the captions of each plot.}
     \label{fig:cc_md_2022_overview_plots_noise_scan_bunch_length}
  \end{figure}
  


 \subsection{Experiment II: sensitivity of emittance growth to amplitude-dependent tune shift}

 % Scripting plot: /eos/user/n/natriant/2022/SPS_MDs_2022/cc_md_16May2022/longitudinal_profiles/plot_bunch_length_for_thesis.ipynb
  \begin{figure}[htp]
     \centering
     \begin{subfigure}{.45\textwidth}
         \centering
         \includegraphics[width=.95\linewidth]{images/app_c/bunch_length_COAST_06.png}  
         \caption{$k_\mathrm{LOD}=+15 \mathrm{/m^{4}}$}
         %\label{fig:cc_md_2022_coast6}
     \end{subfigure}
     \begin{subfigure}{.45\textwidth}
         \centering
         \includegraphics[width=.95\linewidth]{images/app_c/bunch_length_COAST_07.png}  
         \caption{$k_\mathrm{LOD}=+10  \mathrm{/m^{4}}$}
         %\label{fig:cc_md_2022_coast7}
     \end{subfigure}
     \begin{subfigure}{.45\textwidth}
         \centering
         \includegraphics[width=.95\linewidth]{images/app_c/bunch_length_COAST_08.png}  
         \caption{$k_\mathrm{LOD}=+5  \mathrm{/m^{4}}$}
         %\label{fig:cc_md_2022_coast8}
     \end{subfigure}
     \begin{subfigure}{.45\textwidth}
             \centering
             \includegraphics[width=.95\linewidth]{images/app_c/bunch_length_COAST_09.png}  
             \caption{$k_\mathrm{LOD}=-5 \mathrm{/m^{4}}$}
             %\label{fig:cc_md_2022_coast9}
     \end{subfigure}
     \caption{Evolution of the bunch length during the  CC Experiment II on May 16, 2022. The different octupole settings are displayed at the captions of each plot.}
     \label{fig:cc_md_2022_overview_plots_klod_scan_bunch_length}
  \end{figure}
 
 
 The average measured bunch length over all above coasts on May 16, 2022 (Experiments I and II) was found to be, 4$\sigma_t$=1.83\,ns. 
 
 
 

 \subsection{Experiment III: sensitivity of emittance growth to amplitude-dependent tune shift}\label{subsec:2022_exp3_bunch_length}

 % Plotting script: /eos/user/n/natriant/2022/SPS_MDs_2022/cc_md_12Sep2022/longitudinal_profiles/plot_bunch_length_for_thesis.ipynb
 % Figures location: /eos/user/n/natriant/2022/SPS_MDs_2022/cc_md_12Sep2022/longitudinal_profiles/figures_for_thesis
 \begin{figure}[htp]
    \centering
    \begin{subfigure}{.45\textwidth}
        \centering
        \includegraphics[width=.95\linewidth]{images/app_c/bunch_length_cc_md_sep_coast6.png}  
        \caption{$k_\mathrm{LOD}=-30 \mathrm{/m^{4}}$}
        %\label{fig:cc_md_2022_coast6}
    \end{subfigure}
    \begin{subfigure}{.45\textwidth}
        \centering
        \includegraphics[width=.95\linewidth]{images/app_c/bunch_length_cc_md_sep_coast7.png}  
        \caption{$k_\mathrm{LOD}=-20 \mathrm{/m^{4}}$}
        %\label{fig:cc_md_2022_coast7}
    \end{subfigure}
    \begin{subfigure}{.45\textwidth}
        \centering
        \includegraphics[width=.95\linewidth]{images/app_c/bunch_length_cc_md_sep_coast8.png}  
        \caption{$k_\mathrm{LOD}=-10 \mathrm{/m^{4}}$}
        %\label{fig:cc_md_2022_coast8}
    \end{subfigure}
    \begin{subfigure}{.45\textwidth}
            \centering
            \includegraphics[width=.95\linewidth]{images/app_c/bunch_length_cc_md_sep_coast9.png}  
            \caption{$k_\mathrm{LOD}=+30 \mathrm{/m^{4}}$}
            %\label{fig:cc_md_2022_coast9}
    \end{subfigure}
    \begin{subfigure}{.45\textwidth}
        \centering
        \includegraphics[width=.95\linewidth]{images/app_c/bunch_length_cc_md_sep_coast11.png}  
        \caption{$k_\mathrm{LOD}=+10  \mathrm{/m^{4}}$}
        %\label{fig:cc_md_2022_coast9}
\end{subfigure}
    \caption{Evolution of the bunch length during the CC Experiment III on September 12, 2022. The different octupole settings are displayed at the captions of each plot.}
    \label{fig:cc_md_sep_2022_overview_plots_klod_scan_bunch_length}
 \end{figure}


The average measured bunch length over all above coasts in 2022, was found to be, 4$\sigma_t$=1.77\,ns. 



 \section{Intensity measurements}\label{sec:intensity_meas_2022}

 The intensity measurements presented in the section were acquired using the 



 w. The longitudinal bunch profiles were acquired every turn using a wall
 The bunch length measurements presented in this section were performed with the Wall Current monitor. The Wall Current monitor acquires the longitudinal bunch profiles and uses a Gaussian fit for the evaluation of the bunch length. Each point shown in the folllowing plots corresponds to the average bunch length value obtained from 100 consecutive acquisitions are  


 \subsection{Experiment I: dependece of emittance growth on CC RF noise power}\label{subsec:2022_exp1_intensity}

 % Plotting script: /eos/user/n/natriant/2022/SPS_MDs_2022/cc_md_16May2022/BCTDC_intensity/plot_intensity_evolution.ipynb
 \begin{figure}[htp]
    \centering
    \begin{subfigure}{.45\textwidth}
        \centering
        \includegraphics[width=.95\linewidth]{images/app_e/intensity_cc_md_16May22_coast_2.png}  
        \caption{-115.2\,dBc/Hz}
        %\label{fig:cc_md_2022_coast2}
    \end{subfigure}
    \begin{subfigure}{.45\textwidth}
        \centering
        \includegraphics[width=.95\linewidth]{images/app_e/intensity_cc_md_16May22_coast_3.png}  
        \caption{-109.5\,dBc/Hz}
        %\label{fig:cc_md_2022_coast3}
    \end{subfigure}
    \begin{subfigure}{.45\textwidth}
        \centering
        \includegraphics[width=.95\linewidth]{images/app_e/intensity_cc_md_16May22_coast_4.png}  
        \caption{-104.7\,dBc/Hz}
        %\label{fig:cc_md_2022_coast4}
    \end{subfigure}
    \begin{subfigure}{.45\textwidth}
            \centering
            \includegraphics[width=.95\linewidth]{images/app_e/intensity_cc_md_16May22_coast_5.png}  
            \caption{-100.1\,dBc/Hz}
            %\label{fig:cc_md_2022_coast5}
    \end{subfigure}
    \caption{Evolution of the intensity during the CC Experiment I on May 16, 2022. The different phase noise levels injected in the RF system of CC1, are displayed at the captions of each plot.}
    \label{fig:cc_md_16_may_2022_intensity_overview_exper1}
 \end{figure}
 

 \subsection{Experiment II: sensitivity of emittance growth to amplitude-dependent tune shift}\label{subsec:2022_exp2_intensity}


 % Plotting script: /eos/user/n/natriant/2022/SPS_MDs_2022/cc_md_16May2022/BCTDC_intensity/plot_intensity_evolution.ipynb
\begin{figure}[htp]
     \centering
     \begin{subfigure}{.45\textwidth}
         \centering
         \includegraphics[width=.95\linewidth]{images/app_e/intensity_cc_md_16May22_coast_6.png}  
         \caption{$k_\mathrm{LOD}=+15 \mathrm{/m^{4}}$}
         %\label{fig:cc_md_2022_coast6}
     \end{subfigure}
     \begin{subfigure}{.45\textwidth}
         \centering
         \includegraphics[width=.95\linewidth]{images/app_e/intensity_cc_md_16May22_coast_7.png}  
         \caption{$k_\mathrm{LOD}=+10  \mathrm{/m^{4}}$}
         %\label{fig:cc_md_2022_coast7}
     \end{subfigure}
     \begin{subfigure}{.45\textwidth}
         \centering
         \includegraphics[width=.95\linewidth]{images/app_e/intensity_cc_md_16May22_coast_8.png}  
         \caption{$k_\mathrm{LOD}=+5  \mathrm{/m^{4}}$}
         %\label{fig:cc_md_2022_coast8}
     \end{subfigure}
     \begin{subfigure}{.45\textwidth}
             \centering
             \includegraphics[width=.95\linewidth]{images/app_e/intensity_cc_md_16May22_coast_9.png}  
             \caption{$k_\mathrm{LOD}=-5 \mathrm{/m^{4}}$}
             %\label{fig:cc_md_2022_coast9}
     \end{subfigure}
     \caption{Evolution of the intensity during the  CC Experiment II on May 16, 2022. The different octupole settings are displayed at the captions of each plot.}
     \label{fig:cc_md_2022_overview_plots_klod_scan_intensity_exper2}
\end{figure}


 The average measured intensity over all above coasts on May 16, 2022 (Experiment III) was found to be 2.9$\times 10^{10}$\,protons per bunch. 
 

 \subsection{Experiment III: sensitivity of emittance growth to amplitude-dependent tune shift}\label{subsec:2022_exp3_intensity}

 \begin{figure}[htp]
    \centering
    \begin{subfigure}{.45\textwidth}
        \centering
        \includegraphics[width=.95\linewidth]{images/app_e/intensity_cc_md_12Sep22_coast_6.png}  
        \caption{$k_\mathrm{LOD}=-30 \mathrm{/m^{4}}$}
        %\label{fig:cc_md_2022_coast6}
    \end{subfigure}
    \begin{subfigure}{.45\textwidth}
        \centering
        \includegraphics[width=.95\linewidth]{images/app_e/intensity_cc_md_12Sep22_coast_7.png}  
        \caption{$k_\mathrm{LOD}=-20 \mathrm{/m^{4}}$}
        %\label{fig:cc_md_2022_coast7}
    \end{subfigure}
    \begin{subfigure}{.45\textwidth}
        \centering
        \includegraphics[width=.95\linewidth]{images/app_e/intensity_cc_md_12Sep22_coast_8.png}  
        \caption{$k_\mathrm{LOD}=-10 \mathrm{/m^{4}}$}
        %\label{fig:cc_md_2022_coast8}
    \end{subfigure}
    \begin{subfigure}{.45\textwidth}
            \centering
            \includegraphics[width=.95\linewidth]{images/app_e/intensity_cc_md_12Sep22_coast_9.png}  
            \caption{$k_\mathrm{LOD}=+30 \mathrm{/m^{4}}$}
            %\label{fig:cc_md_2022_coast9}
    \end{subfigure}
    \begin{subfigure}{.45\textwidth}
        \centering
        \includegraphics[width=.95\linewidth]{images/app_e/intensity_cc_md_12Sep22_coast_11.png}  
        \caption{$k_\mathrm{LOD}=+10  \mathrm{/m^{4}}$}
        %\label{fig:cc_md_2022_coast9}
\end{subfigure}
    \caption{Evolution of the bunch length during the CC Experiment III on September 12, 2022. The different octupole settings are displayed at the captions of each plot.}
    \label{fig:cc_md_sep_2022_overview_plots_klod_scan_intensity}
 \end{figure}

 The average measured intensity over all above coasts on September 12, 2022 (Experiments III) was found to be 2.5$\times 10^{10}$\,protons per bunch. 